\begin{thebibliography}{52}


\bibitem{akopyan}
\begin{otherlanguage}{russian}
Акопян А. В.,
PL-аналог теоремы Нэша --- Кейпера.
\textit{Одиннадцатый конкурс Мёбиуса.} (2007),
\end{otherlanguage}
\href{http://www.moebiuscontest.ru/files/2007/akopyan.pdf}{www.moebiuscontest.ru}

\bibitem{akopyan-tarasov}
Akopyan, A. V.; 
Tarasov, A. S.,
A constructive proof of Kirszbraun's theorem. 
\textit{Math. Notes.} 84 (2008), no. 5--6, 725--728.

\bibitem{alexandrov} 
Alexandrov, A. D.,
\textit{Convex Polyhedra}
Springer Monographs in Mathematics. Berlin: Springer (2005).


\bibitem{alexander} 
Alexander, R., 
Lipschitzian mappings and total mean curvature of polyhedral surfaces. I.
\textit{Trans. Amer. Math. Soc.} 288 (1985), no. 2, 661--678.

\bibitem{arnold} 
Arnold, V. I., 
\textit{Arnold's problems.} 
Springer-Verlag, Berlin, 2004. 

\bibitem{bezdek-connelly} 
Bezdek, K.;  
Connelly, R., 
Pushing disks apart---the Kneser--Poulsen conjecture in the plane.  
\textit{J. Reine Angew. Math.}  553  (2002), 221--236.

\bibitem{burago-zalgaller-0}
\begin{otherlanguage}{russian}
Бураго, Ю. Д., 
В. А. Залгаллер. 
Реализация разверток в виде многогранников. \textit{Вестник ЛГУ}, 15, no. 7, (1960): 66--80.
\end{otherlanguage}

\bibitem{burago-zalgaller}
Burago, Yu. D.; 
Zalgaller, V. A., 
Isometric piecewise linear immersions of two-dimensional manifolds with polyhedral metrics into $\RR^3$.
\textit{St. Petersburg Math. J.}, 
7, (1996) no. 3, 369--385.

\bibitem{brehm} Brehm, U., 
Extensions of distance reducing mappings to piecewise congruent mappings on $\RR^m$.  
\textit{J. Geom.}  16  (1981), no. 2, 187--193.

\bibitem{BBI}  
Burago, D.;  
Burago, Yu.;  
Ivanov, S.,
\textit{A course in metric geometry.}
American Mathematical Soc., 2001.

\bibitem{DGK} 
Danzer, L.; 
Gr\"unbaum, B.; 
Klee, V.,
\textit{Helly's theorem and its relatives.} 
Convexity, Proc. Symp. Pure Math. 7, American Mathematical Society, pp. 101--179.

\bibitem{TF} 
Fuchs, D.; 
Tabachnikov, S.,
\textit{Mathematical omnibus.
Thirty lectures on classic mathematics.} 
American Mathematical Society, Providence, RI, 2007. 


\bibitem{gromov}  Gromov, M., 
\textit{Partial differential relations.} 
Ergebnisse der Mathematik und ihrer Grenzgebiete (3), 9. 
Springer-Verlag, Berlin, 1986.

\bibitem{hull} Hull, T. C.,
Solving cubics with creases: the work of Beloch and Lill, 
\textit{Amer. Math. Monthly} 118 (2011), no. 4, 307--315.

\bibitem{eliashberg-mishachev} Eliashberg, Y.;  Mishachev, N., \textit{Introduction to the \textit{h}-Principle.} 
American Mathematical Soc., 2002.

\bibitem{kirszbraun} Kirszbraun, M. D., 
\"Uber die zusammenziehende und Lipschitzsche Transformationen. 
\textit{Fund. Math.} 22 (1934) 77--108.

\bibitem{krat} Krat, S., 
\textit{Approximation Problems in Length Geometry,} 
Thesis, 2005, PSU.

\bibitem{lang-secrets} Lang, R. J., 
\textit{Origami Design Secrets: Mathematical Methods for an Ancient Art},
CRC Press, Boca Raton, FL, 2012.


\bibitem{lang} 
Montroll, J.;  
Lang, R. J., 
\textit{Origami Sea Life},
Dover Publications, 1991.

\bibitem{petrunin-ruble}  
\begin{otherlanguage}{russian}
Петрунин, А., 
Плоское оригами и длинный рубль.
\textit{Задачи Санкт-петербургской олимпиады школьников по матаматике}, 2008, 116--125; \texttt{arXiv:1004.0545}
\end{otherlanguage}

\bibitem{petrunin-inverse} Petrunin, A.,
On intrinsic isometries to Euclidean space.
\textit{St. Petersburg Mathematical Journal}, 22 (2011), 803--812.

\bibitem{petrunin-yashinsky-arXiv} 
Petrunin, A.; 
Yashinski, A.,
Lectures on piecewise distance preserving maps.
\texttt{arXiv:1405.6606}.

\bibitem{saraf}  Saraf, S., 
Acute and nonobtuse triangulations of polyhedral surfaces.
\textit{European Journal of Combinatorics.} 
30 (2009), Issue 4, Pages 833--840

\bibitem{tarasov} 
\begin{otherlanguage}{russian}
Тарасов, А. С.,
Решение задачи Арнольда «о мятом рубле».
\textit{Чебышевский сборник},  
5, (2004), выпуск 1, 174--187.
\end{otherlanguage}

\bibitem{yashenko} Yashenko, I.,
Make your dollar bigger now!!! 
\textit{Math. Intelligencer} 20 (1998), no. 2, 38--40

\bibitem{valentine} Valentine, F. A.,  
A Lipschitz Condition Preserving Extension for a Vector Function. \textit{American Journal of Mathematics}, (1945) 67 (1) 83--93.


\bibitem{zalgaller-PL} 
\begin{otherlanguage}{russian}
Залгаллер, В. А.,
Изометричекие вложения полиэдров,
\textit{Доклады АН СССР},
123 (1958) 599--601.
\end{otherlanguage}

\end{thebibliography}
