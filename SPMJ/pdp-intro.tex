\tableofcontents

\section*{Introduction}
\addcontentsline{toc}{section}{Introduction}
\addtocontents{toc}{\protect\begin{quote}}


These lectures were a part of the geometry course held during the Fall 2011 MASS Program at Penn State (\url{www.math.psu.edu/mass/}).
%???ALLAN-EDIT???
%The online vesion of these lectures \cite{petrunin-yashinsky-arXiv} also contains video illustrations,
%hints and solutions for most of the exercises and minimalistic preliminaries. 
The online version of these lectures \cite{petrunin-yashinsky-arXiv} also contains video illustrations,
hints and solutions for most of the exercises, and a minimalistic section covering preliminaries. 

The lectures discuss piecewise distance preserving maps from a 2-dimensional polyhedral space into the plane.  Roughly speaking, a polyhedral space is a space that is glued together out of triangles, for example the surface of a polyhedron.  If one imagines such a polyhedral space as a paper model, then a piecewise distance preserving map into the plane is essentially a way to fold the model so that it lays flat on a table. 

We only consider the 2-dimensional case to keep things easy to visualize.
However, 
%???REF???
most of the results admit generalizations to higher dimensions.
%???REF???  
These results are discussed in the Final Remarks, where proper credit and references are given.

\parbf{Acknowledgments.} 
We would like to thank
Arseniy Akopyan, 
Robert Lang, 
Alexei Tarasov
for their help.
Also we would like to thank all the students in our class
for their participation and true interest.

\parbf{Notations.}
All the necessary background material is discussed in the first three chapters of the book
``Metric Geometry'' by Burago--Burago--Ivanov \cite{BBI}.

Let us list the less standard conventions which we use in the lectures.
\begin{itemize}
\item The distance between two points $x$ and $y$ in a metric space $X$
will be denoted as 
$$|x-y|,
\ \ 
\text{or}\ \  |x-y|_X;$$ 
the latter notation is used if we need to emphasize that the distance is measured in the metric space $X$.

\item A metric space $X$ is called a \emph{length space}\index{length space} if for any two points $x,y\in X$ and any $\eps>0$, there is a curve $\alpha$ from $x$ to $y$ such that
$$\length \alpha<|x-y|_X+\eps.$$

\item \label{def:length-preserving}
Let  $f\:X\z\to Y$ be a continuous map between two metric spaces.
Then
\begin{itemize}
\item $f$ is called \emph{distance preserving} if
$$|f(x)-f(y)|_Y=|x-y|_X$$
for any pair of points $x,y\in X$;
\item $f$ is called \emph{distance non-expanding} if
$$|f(x)-f(y)|_Y\le|x-y|_X$$
for any pair of points $x,y\in X$;
\item $f$ is called \emph{length-preserving}\index{length-preserving map} if for any curve $\alpha$ in $X$,
we have 
$$\length\alpha=\length(f\circ\alpha).$$
\end{itemize}
\item A length space $P$ is called a \emph{polyhedral space}\index{polyhedral space}
if it admits a finite triangulation such that each simplex in $P$ is isometric to a simplex in Euclidean space.
\end{itemize}



\addtocontents{toc}{\protect\end{quote}}