\chapter{The space of sets}\label{chap:hausdorff}

In this chapter we introduce Hausdorff convergence,
a type of convergence for subsets of a metric space.
This is a simple idea which seriously changed the shape of geometry sometime ago.

As a motivating example,
I give a proof of isoperimetrical inequality.
This proof was given by Blaschke in \cite{blaschke};
it use Hausdorff convergence together with Steiners's 4-joint method. 
Historically, 
this was the first essential application of Hausdorff convergence.

The material of this chapter I learned from the book of Wilhelm Blaschke \cite{blaschke}.
This is a truly remarkable book which I recommend to read to everyone who can.
(It is available in German  
and  Russian; no English translations so far.)








\section{Steiner's 4-joint method}

\begin{thm}{Theorem}\label{thm:steiner}
Let $F$ be a plane figure bounded by a curve of length $1$ 
which has maximal area.
Then $F$ is a round disc. 
\end{thm}
 
Note that if we would know that such $F$ exists, then as a corollary we would obtain the 
isoperimetric inequality in the plane:

\begin{thm}{Theorem}\label{thm:isoperimeter}
Any closed simple curve in the plane of length $L$ bounds area at most $\tfrac1{4\cdot\pi}\cdot L^2$.
Moreover, in case of equality, the curve is a circle.
\end{thm}


\begin{wrapfigure}{r}{22mm}
\begin{lpic}[t(-4mm),b(-2mm),r(0mm),l(0mm)]{pics/cut(0.15)}
\end{lpic}
\end{wrapfigure}

The existence of $F$ will be proved later;
in this proof we use Hausdorff convergence constructed later in this chapter.

The following argument was found by Jacob Steiner in 1842.

\parit{Proof of Theorem \ref{thm:steiner}.}
Let $F$ be a maximal such figure.  We shall show that $F$ is a disc.  First note that $F$ is convex (i.e. the straight light segment connecting any two points in $F$ is also contained in $F$); 
otherwise one could make the perimeter of $F$ smaller and the area larger.

Take any point $P$ in the bounding curve of $F$ (further denoted by $\partial F$).
Consider a point $P'\in\partial F$ so that both arcs $\partial F$ of from $P$ to $P'$ have the same length (which has to be $\tfrac12$).

%\begin{wrapfigure}{r}{25mm}
%\begin{lpic}[t(-6mm),b(-2mm),r(0mm),l(3mm)]{pics/FF'(0.1)}
%\lbl[r]{7,80;$P$}
%\lbl[l]{190,80;$P'$}
%\lbl{90,40;$F_1$}
%\lbl[w]{90,140;$F_2$}
%\end{lpic}
%\end{wrapfigure}

Divide $F$ by the segment $[PP']$ into two parts $F_1$ and $F_2$.
Without loss of generality we can assume that $\area F_1\ge \area F_2$.
Let $F'_1$ be the reflection of $F_1$ in the line $(PP')$.
Set $F'=F_1\cup F_1'$



Note that 
$$\area F'=2\cdot\area F_1\ge \area F_1+\area F_2=\area F$$    

%\begin{wrapfigure}{r}{22mm}
%\begin{lpic}[t(-4mm),b(-1mm),r(0mm),l(0mm)]{pics/F'(0.1)}
%\lbl[r]{2,80;$P$}
%\lbl[l]{187,80;$P'$}
%\lbl[t]{49,4;$Q$}
%\lbl[b]{53,158;$Q'$}
%\end{lpic}
%\end{wrapfigure}

\noi and the perimeter of $F'$ is the same as perimeter of $F$.
Hence $F'$ also has maximal possible area and $\area F_1=\area F_2$.


Note that if $F$ is not a round disc then points $P$ and $P'$ can be chosen so that $F'$ is not a round disc.
Chose arbitrary point $Q\in \partial F'$ and let $Q'$ be the reflection of $Q$ in $(PP')$.


One can think of $F'$ as quadrilateral $PQP'Q'$ with a lune attached at each side.
Think about these lunes as being made of rigid material (say cut it from cardboard)
and imagine that at each vertex $P$, $Q$, $P'$, $Q'$ we have a joint;
so the quadrilateral $PQP'Q'$ can be moved continuously keeping its sides fixed.

Note that if $\angle PQP'\not=\tfrac\pi2$
then we can move this construction slightly and increase the area of the obtained figure.

Hence $\angle PQP'=\tfrac\pi2$ for any $Q\in \partial F'$.
It follows that $F'$ is a disc; hence $F$ is also a disc.
\qeds

As it was mentioned above, in order to prove isoperimetric inequality (Theorem~\ref{thm:isoperimeter})
one only has to show existence of an extremal object (the figure $F$ in Theorem~\ref{thm:steiner}) and then apply Steiner's argument.
One possible approach is to cook up a compact metric space out of plane figures and
show that volume and perimeter depend continuously (or semicontinuously) 
on the figure. 
Then existence of a maximal $F$ would follow, as a continuous function on a compact metric space attains a maximum.

The first step is to define a metric on the set of figures in the plane; 
this is done in the next section.


\section{Hausdorff metric}


Let $X$ be a metric space.
Given a subset $A\subset X$,
consider the distance function to $A$
$$\Dist_A: X \to [0,\infty)$$
defined as 
$$\Dist_A x
\df
\inf_{a\in A}\{\Dist_a x\}.$$

\begin{thm}{Definition}
Let $A$ and $B$ be two compact subsets of a metric space $X$.
Then the Hausdorff distance between $A$ and $B$ is defined as 
$$|A-B|_{\mathcal{H}(X)}
\df
\sup_{x\in X}|\Dist_Ax-\Dist_Bx|.
$$

\end{thm}
 

\begin{thm}{Exercise}\label{ex:R-hausdorff}
Let $A,B$ be two compact subsets of a metric space $X$.
Show that $|A-B|_{\mathcal{H}(X)}\le R$ if and only if 
$\Dist_Ab\le R$ for any $b\in B$
and 
$\Dist_Ba\le R$ for any $a\in A$.
\end{thm}


\begin{thm}{Exercise}\label{ex:H_X}
Show that the set of all nonempty compact subsets of a metric space $X$ equipped with the Hausdorff metric forms a metric space.

This new metric space will be denoted as $\mathcal{H}(X)$.
\end{thm}

\begin{thm}{Exercise}\label{ex:diam}
Let $X$ be a metric space.
Given a subset $A\subset X$ define its diameter as 
$$\diam A\df\sup_{a,b\in A} |a-b|.$$

Show that 
$$\diam\:\mathcal{H}(X)\to \RR$$ 
is a continuous function.
\end{thm}


\begin{thm}{Blaschke's compactness theorem}\label{thm:compact+Hausdorff}
Let $X$ be a metric space.
Then the space $\mathcal{H}(X)$ is compact if and only if $X$ is compact.
\end{thm}




\parit{Proof; ``only if'' part.}
Note that the map $\iota\:X\to \mathcal{H}(X)$, defined as $\iota\:x\mapsto\{x\}$
(i.e., point $x$ mapped to the one-point subset $\{x\}$ of $X$)
is distance preserving.
Thus $X$ is isometric to the set $\iota(X)$ in $\mathcal{H}(X)$.

Note that for a nonempty subset $A\subset X$, we have $\diam A=0$ if and only if $A$ is a one-point set.
Therefore, from Exercise~\ref{ex:diam}, it follows 
that $\iota(X)$ is closed in $\mathcal{H}(X)$.
Hence $\iota(X)$ is compact, as it is a closed subset of a compact space. 
Since $X$ is isometric to $\iota(X)$,
``only if'' part follows.
\qeds

To prove ``if'' part we will need the following two lemmas.%???CHANGE THE PROOF???

\begin{thm}{Lemma}\label{lem:decreasing-converges}
Let $K_1\supset K_2\supset\dots$ be a sequence of nonempty compact sets in a metric space $X$
then $K_\infty=\bigcap_n K_n$ is the Hausdorff limit of $K_n$;
i.e., $|K_\infty-K_n|_{\mathcal{H}(X)}\to0$ as $n\to\infty$.
\end{thm}

\parit{Proof.}
Note that $K_\infty$ is compact and nonempty.
If the assertion were false, 
then there is $\eps>0$ such that for each $n$ 
one can choose $x_n\in K_n$
such that $\Dist_{K_\infty}x_n\ge\eps$.
Note that $x_n\in K_1$ for each $n$.
Since $K_1$ is compact, 
there is 
a \emph{partial limit}\index{partial limit}\footnote{Partial limit is a limit of a subsequence.}
 $x_\infty$ of $x_n$.
Clearly $\Dist_{K_\infty}x_\infty\ge \eps$.

On the other hand, since $K_n$ is closed and $x_m\in K_n$ for $m\ge n$,
we get $x_\infty\in K_n$ for each $n$.
It follows that $x_\infty\in K_\infty$ and therefore $\Dist_{K_\infty}x_\infty=0$,
a contradiction.\qeds

The following statement was used in the previous proof.

\begin{thm}{Exercise}\label{ex:intersection-of-compacts}
Show that if $K_1\supset K_2\supset\dots$ is a nonempty compact subsets of a metric space, then 
$$K_\infty=\bigcap_{n=1}^\infty K_n$$ is nonempty and compact.  

Give an example that shows that this statement is false if we replace ``compact'' with ``closed''.
\end{thm}


\begin{thm}{Lemma}\label{lem:complete+Hausdorff}
If $X$ is a compact metric space then $\mathcal{H}(X)$
is complete.
\end{thm}

\parit{Proof.}
Let $(Q_n)$ be a Cauchy sequence in $\mathcal{H}(X)$.
Passing to a subsequence of $Q_n$ we may assume that 
$$|Q_n-Q_{n+1}|_{\mathcal{H}(X)}\le \tfrac1{10^n}\eqlbl{eq:eps=1/10}$$
for each $n$.

Set 
\begin{align*}
K_n&= \set{x\in X}{\Dist_{Q_n}(x)\le \tfrac1{10^n}}
\end{align*}
Clearly, $|Q_n-K_n|_{\mathcal{H}(X)}\le \tfrac1{10^n}$ and from \ref{eq:eps=1/10}, we get
$K_n\supset K_{n+1}$ 
for each $n$.
Set 
$$K_\infty=\bigcap_{n=1}^\infty K_n.$$
Applying Lemma \ref{lem:decreasing-converges},
we get that $|K_n-K_\infty|_{\mathcal{H}(X)}\to 0$ as $n\to\infty$.
Since $|Q_n-K_n|_{\mathcal{H}(X)}\le \tfrac1{10^n}$, we get $|Q_n-K_\infty|_{\mathcal{H}(X)}\to 0$ as $n\to\infty$.
Hence lemma follows.
\qeds

\parit{Proof of ``if'' part of Blaschke's compactness Theorem~\ref{thm:compact+Hausdorff}.}
According to Lemma~\ref{lem:complete+Hausdorff},
$\mathcal{H}(X)$ is complete.
So to show that $\mathcal{H}(X)$ is compact, it only remains to show that $\mathcal{H}(X)$ is totally bounded;
i.e., given $\eps>0$ there is a finite $\eps$-net in $\mathcal{H}(X)$.

Choose a finite $\eps$-net $A$ in $X$.
Denote by $\mathcal{A}$ the set of all subsets of $A$.
Note that  $\mathcal{A}$ is finite set in $\mathcal{H}(X)$.  We shall show that $\mathcal{A}$ is an $\eps$-net in $\mathcal{H}(X)$.

For each compact set $K\subset X$, consider the subset $K'$ of all points $a\in A$
such that $\Dist_K a\le \eps$.
Then $K' \in \mathcal{A}$ and $|K-K'|_{\mathcal{H}(X)}\le\eps$.
In other words $\mathcal{A}$ is a finite $\eps$-net in $\mathcal{H}(X)$.
\qeds











\section{Isoperimetric inequality: the end of proof}

Let $Q$ be a figure in the plane which is bounded by a closed curve.
Note that if $Q$ is not convex then the \emph{convex hull}\index{convex hull}%
\footnote{i.e., the minimal convex set which contains the given set}
 of $Q$ has bigger area and smaller perimeter.
Therefore it is sufficient to prove the isoperimetric inequality only for convex figures.
In other words, to prove Theorem~\ref{thm:isoperimeter},
it is sufficient to prove the following.

\begin{thm}{Theorem}\label{thm:isoperimeter-convex}
Any convex plane figure wth area $A$ is bounded by a curve of length at least
$\sqrt{4\cdot \pi\cdot A}$.
Moreover, in case of equality, the figure is a disc.
\end{thm}

Note that by Steiner's argument (\ref{thm:steiner}),
Theorem~\ref{thm:isoperimeter-convex} follows from the following.

\begin{thm}{Proposition}\label{prp:there_is_F}
There is a convex figure $F$ in the plane,
which maximize the area among all convex figures with perimeter $1$.
\end{thm}


First note that any convex figure with perimeter $1$ can be moved into a $1\times1$-square by parallel translation.
Therefore it is sufficient to show existence of $F$ which maximizes area among convex figures with perimeter $1$ in a given $1 \times 1$-square, which will be denoted as $\square$.

Note that $\square$ is compact.
According to Blaschke's compactness Theorem~\ref{thm:compact+Hausdorff}, 
$\mathcal{H}(\square)$ is compact.
Let us denote by $\mathcal{C}$ the subset of $\mathcal{H}(\square)$ formed by all convex sets.

\begin{thm}{Exercise}\label{ex:convex=closed-in-H}
Show that $\mathcal{C}$ is a closed subset of $\mathcal{H}(\square)$.
\end{thm}

Since $\mathcal{H}(\square)$ is compact,
the above exercise implies that $\mathcal{C}$ is also compact.

The area and perimeter define two functions on $\mathcal{C}$.
$$\area\:\mathcal{C}\to\RR\ \ \text{and}\ \ \per\:\mathcal{C}\to\RR.$$
In order to ensure existence of $F$ it remains to prove the following claim:

\begin{thm}{Proposition}\label{prop:area+per}
Both functions 
$$\area\:\mathcal{C}\to\RR\ \ \text{and}\ \ \per\:\mathcal{C}\to\RR$$
are continuous.

(In other words, for any Hausdorff converging sequence $K_n\to K_\infty$ of convex compact sets in $\square$,
we have
$$\area K_\infty=\lim_{n\to\infty}\area K_n
\ \ \text{and}\ \ 
\per K_\infty=\lim_{n\to\infty}\per K_n.)$$

\end{thm}

Indeed, since $\per\:\mathcal{C}\to\RR$ is continuous, 
we have that the figures in $\mathcal{C}$ with perimeter $1$ form a closed set, say $\mathcal{C}_1$.
Since  $\mathcal{C}$ is compact,  so is $\mathcal{C}_1$.
Then the restriction of $\area\:\mathcal{C}\to\RR$ to $\mathcal{C}_1$ admits a maximal value by Exercise~\ref{ex:EVT};
hence Proposition~\ref{prp:there_is_F} follows.

In the proof of Proposition~\ref{prop:area+per}, we will use the following two lemmas.

\begin{thm}{Lemma}\label{lem:perimeter}
Let $P$ and $Q$ be two convex compact figures in the plane.
Then
$$P\subset Q\ \ \text{and}\ \ \per P\le \per Q.$$ 

\end{thm}

\parit{Proof.}
First note that it is sufficient to prove the lemma only for convex polygons $P$ and $Q$.
Note that if $P$ is obtained from $Q$ by intersecting $Q$ with a half-plane then the inequality $\per P\le \per Q$ follows from the triangle inequality.

\begin{wrapfigure}{r}{33mm}
\begin{lpic}[t(-4mm),b(-2mm),r(0mm),l(0mm)]{pics/P-Q(0.15)}
\end{lpic}
\end{wrapfigure}


Now note that there is an increasing sequence of polygons 
$$P=P_0\subset P_1\subset\dots\subset P_n=Q$$
such that $P_{i-1}$ is intersection of $P_{i}$ with a half-plane.
Therefore 
\begin{align*}
\per P=\per P_0&\le\per P_1\le\dots
\\
\dots&\le\per P_n=\per Q
\end{align*}
and the lemma follows.
\qeds

\begin{thm}{Lemma}\label{lem:between-poly}
Let $Q$ be a compact convex plane figure 
and the interior of $Q$ contains the origin $0\in\RR^2$.
Then given $\eps>0$ there is a convex polygon $P\subset Q$ such that
the rescaled polygon
$$(1+\eps)\cdot P=\set{(1+\eps)\cdot x\in \RR^2}{x\in P}$$
contains $Q$.
\end{thm}

\begin{wrapfigure}[5]{r}{33mm}
\begin{lpic}[t(-20mm),b(-2mm),r(0mm),l(0mm)]{pics/P-F(0.15)}
\end{lpic}
\end{wrapfigure}

\parit{Proof.}
Assume $Q$ contains $\bar B_R(0)$ in its interior.

Fix small $\delta>0$  and consider the set $Z$ all points
$(\delta\cdot m,\delta\cdot n)\in Q$ such that $m$ and $n$ are integers.
Let $P$ be the convex hull of $Z$.
Since $Q$ is convex, we have $P\subset Q$.

Note that for any point $q\in Q$ there is a point $p\in P$ such that
$$|p-q|\le \sqrt{2}\cdot\delta.$$ 
Since $\delta$ is small, we can assume that $P\supset \bar B_R(0)$

In this case straightforward calculation show that the set 
$(1+\eps) \cdot P$
contains all points on distance $\eps\cdot R $ from $P$.
In particular, 
lemma holds if in addition $\delta<\eps\cdot \tfrac{R}{10}$.
\qeds

\begin{thm}{Exercise}\label{ex:calculations}
Perform the ``straightforward calculation'' in the above proof.
\end{thm}



\parit{Proof of Proposition~\ref{prop:area+per}.}
Assume $K_\infty$ is nondegenerate; i.e. $K_\infty$ contains a disc, say $B_R(z)$.
Without loss of generality, we may assume that $z$ is the origin $0\in\RR^2$;
so we can apply Lemma~\ref{lem:between-poly} to $K_\infty$.



Choose small $\eps>0$ and choose $P$ as in the Lemma~\ref{lem:between-poly}.
Note that there is $N$ such that
$$(1-2\cdot\eps)\cdot P\subset K_n\subset (1+2\cdot\eps)\cdot P$$ 
for all $n\ge N$.
It follows that%
\footnote{Here $$a\lege(1\pm\eps)\cdot b$$ means two inequalities: $$a\le(1+\eps)\cdot b\ \ \text{and}\ \  a\ge(1-\eps)\cdot b.$$
The sign ``$\lege$'' should be red as ``more-or-less'' or better as ``less-or-more''.} \index{$\lege$}
\begin{align*}
\area K_n&\lege(1\pm2\cdot\eps)^2\cdot \area P
\lege
\\
&\lege(1\pm2\cdot\eps)^4\cdot \area K_\infty.
\end{align*}

From Lemma~\ref{lem:perimeter}, we get 
\begin{align*}\per K_n
&\lege(1\pm2\cdot\eps)\cdot \per P\lege
\\
&\lege(1\pm2\cdot\eps)^2\cdot \per K_\infty.
\end{align*}

Hence the we get 
$$\area K_\infty=\lim_{n\to\infty}\area K_n
\ \ \text{and}\ \ 
\per K_\infty=\lim_{n\to\infty}\per K_n.$$ 

The following exercise finishes the proof.\qeds

\begin{thm}{Exercise}\label{ex:area+perim}
Prove that  
$$\area K_\infty=\lim_{n\to\infty}\area K_n
\ \ \text{and}\ \ 
\per K_\infty=\lim_{n\to\infty}\per K_n.$$
in case $K_\infty$ is degenerate $K_\infty$;
i.e., if $K_\infty$ is either one-point set or a segment.

(If $\ell$ is the length of a segment then its perimeter is defined to be $2\cdot\ell$.)
\end{thm}

\section{Comments}

It seems that Hausdorff convergence was first introduced by Felix Hausdorff in his book \cite{hausdorff} which appeared in print in 1914.
Blaschke's book appears two years later and it introduce the same concept without
a reference to Hausdorff.
It it therefore likely that Hausdorff convergence was known in mathematical folklore before these books,
but I would be very interested to know the real story.

The Hausdorff metric was only used to 
define convergence of compact subsets in a metric space
and any equivalent metric would have done the job.
In other words, the topology which the Hausdorff metric describes on the 
set of compact subsets is important, while the concrete metric is not.

One may wonder what will happen if we consider all subsets of the metric space $X$,
instead of just the compact sets.
In this case, distinct sets might have zero distance, 
say consider $X=\RR$
and sets $A=[0,1]$
and $B$ is the set rational point in $[0,1]$.
It suggests that one should only consider closed sets.
Further, the distance between closed subsets might be infinite. 
For example, consider $X=\RR$ and the subsets $A=[0,\infty)$ and $B=\{0\}$.
I.e., formally, the Hausdorff distance would not satisfy definition \ref{def:metric-space}.
Therefore, one would have to modify the definition of metric space to allow the distance between two points to be infinite,
or one could restrict the set of subsets further to bounded closed sets.
In the latter case, we get a nice metric space, say $\mathcal{H}'(X)$.
Blashke's compactness theorem still holds for $\mathcal{H}'(X)$, but the proof becomes slightly more technical. 


For unbounded closed sets,
there is a better modification of Hausdorff convergence.
Given a sequence of closed sets $A_n$, we say that it converges to a closed set $A_\infty$ 
if the sequence of functions $\Dist_{A_n}$ converge pointwise to $\Dist_{A_\infty}$.
This modification was introduced by 
Frol\'{\i}k in \cite{frolik}
and few years later by
Wijsman in \cite{wijsman}.

Note that in the above definition, if one changes \emph{pointwise convergence} to \emph{uniform convergence} then one obtains standard Hausdorff convergence.
For example the sequence of sets $A_n=[0,n]$ in $\RR$ diverges in the stadard Hausdorff sense,
but converges to $[0,\infty)$ in the sense of Hausdorff--Wijsman. 

Often one works with equivalence classes of subsets.
For example, consider the set of compact sets in metric space $X$ up to congruence.
Two subsets $A,B\subset X$ are congruent (briefly, $A\cong B$) if there is an isometry $\iota\:X\to X$ such that $\iota(A)=B$. 
In this case the Hausdorff metric can be used to define a metric on the congruemce classes.
For example we may talk about distance between sets up to congruence;
say denote by $\mathcal{H}(X)/{\cong}$ the set of congruence classes of compact subsets of $X$
and equip $\mathcal{H}(X)/{\cong}$ with the metric defined as
$$|[A]-[B]|_{\mathcal{H}(X)/{\cong}}=\inf_{\iota}|A-\iota(B)|_{\mathcal{H}(X)},\eqlbl{eq:hausdorff-upto-cong}$$
where the infimum is taken for all isometries $\iota\:X\to X$.

\begin{thm}{Exercise}\label{ex:hausdorff-upto-cong}
Show that \ref{eq:hausdorff-upto-cong} defines a metric on $\mathcal{H}(X)/{\cong}$.
\end{thm}




\section*{Exercises}

\begin{pr} Consider an $n$-gon $A$ which maximize area among all $n$-gons with side-lengths $a_1,a_2,\dots,a_n$. 
Show that $A$ is inscribed in a circle;
i.e., all vetexes of $A$ lie on a circle.
\end{pr}


\begin{pr}\label{pr:Hausdorff-bry}
Let $X$ and $Y$ be two compact subsets in $\RR^2$.
Assume $|X-Y|_{\mathcal{H}(\RR^2)}<\eps$, 
is it true that
$|\partial X-\partial Y|_{\mathcal{H}(\RR^2)}<\eps$,
where $\partial X$ denotes the boundary of $X$.

Does the converse holds; i.e. assume $X$ and $Y$ be two compact subsets in $\RR^2$
and $|\partial X-\partial Y|_{\mathcal{H}(\RR^2)}<\eps$. 
Is it true that $|X-Y|_{\mathcal{H}(\RR^2)}<\eps$?
\end{pr}

\begin{pr}\label{pr:Hausdorff-Conv}
Let $A$ and $B$ be compact subsets of $\RR^2$.
Show that 
$$ |\Conv A-\Conv B|_{\mathcal{H}(\RR^2)}\le |A- B|_{\mathcal{H}(\RR^2)},$$
where $\Conv A$ denotes the convex hull of $A$.
\end{pr}



\begin{pr}
Let $X$ be a compact metric space and $A, B\subset X$ be two finite $\eps$-nets in $X$.
Show that $|A-B|_{\mathcal{H}(X)}\le \eps$.
\end{pr}


\begin{pr}\label{pr:LebesgueMinimalProblem}
Recall that a \href{http://en.wikipedia.org/wiki/Curve_of_constant_width}{figure of constant width $a$} is a compact convex set in the plane such that the length of the orthogonal projection to any line is equal to $a$. 
\begin{enumerate}[(i)]
\item Show that any set $K$ in the plane with diameter $1$ is a subset of a figure  constant width 1.
\item Show that the area of any figure of constant width 1 is at least $1/100$.
\item Let $F$ be a compact set in the plane which can be presented as a union of figures of constant width 1.
Assume $\diam F>10000$.
Show that either $F$ can be presented as a union of two disjoint compact sets or $\area F>1$.
\item Let $\mathcal{Q}$ be the set of all compact sets in the plane, which can be presented as a union of figures of constant width 1.
Try to prove that $\area$ is a continuous function on $\mathcal{Q}$ (with respect to the Hausdorff metric).
\item Use all above to show existance of a solution for the \href{http://mathworld.wolfram.com/LebesgueMinimalProblem.html}{Lebesgue Minimal Problem};
i.e., show that there is a plane figure $F$ of least area  which is capable of covering any plane figure of unit  diameter.

Try to guess what is $F$.
\end{enumerate}
\end{pr}

\begin{pr}\label{pr:complete>compact}
Let $X$ be a complete metric space and $K_n$ be a sequence of compact sets 
which converges in the sence of Hausdorff.
Show that closure $Q$ of union $\bigcup_{n=1}^\infty K_n$ is compact.

Use this to show that in Lemma~\ref{lem:complete+Hausdorff} compactness of $X$ can be exchanged to completeness.
\end{pr}

\begin{pr}\label{pr:GH/area}
Let $K_1, K_2, \dots$ and $K_\infty$ be compact convex sets in $\RR^2$.
Assume that $K_\infty$ is nondegenerate;
i.e., it contains interior points.
Prove that $K_n\GHto K_\infty$ if and only if 
$$\area (K_n\backslash K_\infty)+\area (K_\infty\backslash K_n)\to 0.$$
\end{pr}






