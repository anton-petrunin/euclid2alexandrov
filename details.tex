\chapter{Details}

\section{Lebesgue's number}

The results in this section will not be used directly further in the lectures.
We present it only to give a connection to standard definition of comapct space.
In addition, a similar argument will be used once in the proof of Hopf--Rinow theorem (\ref{thm:Hopf-Rinow}).

\medskip

Let $X$ be a metric space, a collection of open subsets $\{U_\alpha\}_{\alpha\in\mathcal A}$ is called \emph{open cover}\index{open cover} of $X$ if 
$$X=\bigcup_{\alpha\in A}U_\alpha.$$


\begin{thm}{Lebesgue's number lemma}\label{lem:lebesgue-number}
Let $\{U_\alpha\}_{\alpha\in\mathcal A}$ be an open cover of a compact metric space $X$.
Then there is an $\eps>0$ (it is called a \emph{Lebesgue number}\index{Lebesgue number} of the cover)
such that for any $x\in X$ the ball $B_\eps(x)\subset U_\alpha$ for some $\alpha\in\mathcal{A}$.
\end{thm}

\parit{Proof.}
Given $x\in X$, denote by $\rho(x)$ the maximal value $R>0$ such that $B_R(x)\subset U_\alpha$ for some $\alpha\in\mathcal{A}$.
Clearly $\rho(x)>0$ for any $x\in X$.

Without loss of generality, we may assume $\rho(x)<\infty$ for one (and therefore any) $x\in X$.
Otherwise the conclusion of the lemma holds for arbitrary $\eps>0$.

Note  $$|\rho(x)-\rho(y)|\le |x-y|$$ for any $x,y\in X$;
in particular $\rho$ is continuous.
Then the conclusion of the lemma holds for 
$$\eps=\tfrac12\cdot\min_{x\in X}\{\rho(x)\}.$$
\qedsf


As a corollary of Lebesgue's number lemma, 
we obtain an alternative definition of compact metric space using open coverings.
This definition is the standard definition of compact spaces.
Since it use only the notion of open sets, it can be generalized to so called topological spaces.


\begin{thm}{Theorem}\label{thm:compact+covering}
A metric space $X$ is compact if and only if any open cover of $X$ contains a finite subcover.

I.e., for any open cover $\{U_\alpha\}_{\alpha\in\mathcal A}$ of $X$ there is a finite set 
$\{\alpha_1,\alpha_2,\dots,\alpha_n\}\subset \mathcal A$ such that 
$$X=\bigcup_{i=1}^nU_\alpha.$$
 
\end{thm}

\parit{Proof; ``if''-part.}
First let us show that $X$ is complete.
Assume contrary;
i.e., there is a Cauchy sequence $(x_n)$ which is not converging.
Set $r_n=\sup_{m\ge n}\{|x_n-x_m|_X\}$ and
$U_n=X\backslash \bar B_{r_n}(x_n)$.
Since $x_n$ does not converge, we have $$\bigcap_{n=1}\bar B_{r_n}(x_n)=\emptyset,$$
or equivalently $\{U_n\}_{n=1}^\infty$ is a cover of $X$.
On the other hand it is easy to see that any finite sub-collection of $\{U_n\}_{n=1}^\infty$ does not contain $x_n$ for all large $n$, a contradiction. 


Fix $\eps>0$ and consider cover of $X$ by open balls $\{B_\eps(x)\}_{x\in X}$.
Note that if $\{B_\eps(x_i)\}_{i=1}^n$ is a finite subcover then $\{x_1,x_2,\dots,x_n\}$ forms an $\eps$-net in $X$. Apply Theorem~\ref{thm:finite_pack=compact}.

\parit{``only if''-part.}
Let $\eps>0$ be a Lebesgue's number of the covering.
Choose a finite $\tfrac\eps2$-net $\{x_1,x_2,\dots,x_n\}$ of $X$.
Clearly 
$$\bigcup_{i=1}^n B_\eps(x_i)=X.\eqlbl{eq:balls-cover-X}$$
For each $x_i$ choose $U_{\alpha_i}$ such that $U_{\alpha_i}\supset B_\eps(x_n)$.
From \ref{eq:balls-cover-X}, 
$$\bigcup_{i=1}^n U_{\alpha_i}=X.$$
\qedsf

\section{Domain invariance}\label{sec:domain-invariance}

In the proof of Domain invariance theorem, we will use the following lemma in combinatoric topology.

\begin{wrapfigure}{r}{37mm}
\begin{lpic}[t(-21mm),b(-16mm),r(0mm),l(-3mm)]{pics/sperner(0.2)}
\end{lpic}
\end{wrapfigure}


\begin{thm}{Sperner's lemma}
Let $\mathcal{T}$ be a triangulation of $m$-dimensional simplex $\Delta$.
Assume that we color all the vertices of $\mathcal{T}$  in $m+1$ colors in such a way that
each vertex of $\Delta$ colored in different colors and the vertices of $\mathcal{T}$
which lie on a face $F$ of $\Delta$ is colored in the color of the vertices of $F$.

Then among the $m$-dimensional simplexes of $\mathcal{T}$ there is at least one which vertices are colored in all the $m+1$ different colors. 
\end{thm}

If you google Sperner's lemma you will find a dousen of nice proofs, 
and it is also fun to make one your-self.