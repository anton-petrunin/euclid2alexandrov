\chapter{The space of spaces}\label{chap:gromov-hausdorff}

Here we introduce so called Gromov--Hausdorff convergence for metric spaces.
This convergence was introduced by Gromov around 1980, published in \cite{gromov-polynomial-growth}.
Very soon this notion began to be used in all branches of geometry.
In fact today I have difficulty to understand 
how one could do geometry without this type of convergence.%
(Some types of convergences of metric spaces was considered before Gromov,
but they had lack of generality;
each type of convergence was desined to solve one particular problem.)



\section{Gromov--Hausdorff metric}

The goal of this section is to cook up a metric space out of metric spaces.
More precisely, we want to define the so called  Gromov--Hausdorff metric on the set of \emph{isometry classes} of compact metric spaces.
(Being isometric is an equivalence relation on the class of metric spaces, 
and an isometry class is an equivalence class with respect to this equivalence relation.)

The obtained metric space will be denoted as $\mathcal{M}$.
Given two metric spaces $X$ and $Y$,
denote by $[X]$ and $[Y]$ their isometry classes;
i.e., $X'\in [X]$ if and only if $X'\iso X$.
Pedantically, the Gromov--Hausdorff distance from $[X]$ 
to $[Y]$ should be denoted as $|[X]-[Y]|_{\mathcal{M}}$;
but we will often write it as $|X-Y|_{\mathcal{M}}$ and say (not quite correctly) 
``$|X-Y|_{\mathcal{M}}$ is the Gromov--Hausdorff distance from  $X$ 
to  $Y$''.
In other words, from now on, if I say ``metric space'',
you should guess from the context if I mean ``metric space'' 
or ``isometry class of this metric space'' (this is an abuse of notation).

Let us describe the idea behind
the definition. 
First, we want to define the metric on $\mathcal{M}$ so that the distance between subspaces in
the same metric space has to be no greater than the Hausdorff distance between
them. 
In other words, if two subspaces of the same space are close to each
other in the sense of Hausdorff distance in the ambient space, their isometry classes must be
close to each other in $\mathcal{M}$. 
Second, we want
the distance between isometric spaces to be zero. 
The Gromov--Hausdorff
distance is in fact the maximum distance satisfying these two requirements.


\begin{thm}{Definition}\label{def:GH}
Let $X$ and $Y$ be compact metric spaces. 
The Gromov--Hausdorff distance $|X-Y|_{\mathcal{M}}$
between them is defined by the following
relation.
 
Given  $r > 0$, we have that $|X-Y|_{\mathcal{M}} < r$ if and only if there exist a metric
space $Z$ and subspaces $X'$ and $Y'$ in $Z$ which are isometric to $X$ and $Y$
respectively and such that $|X'-Y'|_{\mathcal{H}(Z)} < r$. 
(Here $|X'-Y'|_{\mathcal{H}(Z)}$ denotes the Hausdorff distance between sets $X'$ and $Y'$ in $Z$.)
\end{thm}

In other words, $|X-Y|_{\mathcal{M}}$ is
the infimum of all $r>0$ for which the above $Z$, $X'$ and $Y'$ exist. 


We say that a sequence
of (isometry classes of) compact metric spaces $X_n$ 
\emph{converges in the sense of Gromov--Hausdorff}\index{converges in the sense of Gromov--Hausdorff} to the (isometry classes of)
compact metric space $X_\infty$ if $|X_n - X_\infty|_{\mathcal{M}} \to 0$ as $n\z\to\infty$;
in this case we write $X_n\GHto X_\infty$.



\begin{thm}{Theorem}\label{thm:GH-is-a-metric}
The set of isometry classes of compact metric spaces equipped with Gromov--Hausdorff metric forms a metric space.

This metric space will be denoted further as $\mathcal{M}$; named for ``metric space''.
\end{thm}

Before proving this theorem, we give couple of variations 
of the definition of Gromov--Hausdorff distance.

\subsection*{Metrics on disjoined union of \textit{X} and \textit{Y}}


Definition~\ref{def:GH} deals with a huge class of metric spaces,
namely, all metric spaces $Z$ that contain subspaces isometric to $X$ and $Y$.
It is possible to reduce this class to metrics on the disjoint unions of $X$ and $Y$. 
More precisely, 

\begin{thm}{Proposition}\label{prop:GH=X+Y}
The Gromov--Hausdorff distance between two compact metric spaces $X$
and $Y$ is the infimum of $r > 0$ such that there exists a metric
$|{*}-{*}|_W$ on the disjoint union $W=X\sqcup Y$ 
such that the restrictions of $|{*}-{*}|_W$ to $X$ and $Y$
coincide with $|{*}-{*}|_X$ and $|{*}-{*}|_Y$ 
and $|X-Y|_{\mathcal{H}(W)} < r$. 
\end{thm}


\parit{Proof.}
Identify $X\sqcup Y$ with $X'\cup Y' \subset Z$ 
(the notation
is from Definition~\ref{def:GH}). 

More formally, fix isometries $f\: X \to X'$ and
$g\: Y \to Y'$, then define the distance between $x \in X$ and $y \in Y$ by
$|x-y|_W = |f (x)- g(y)|_Z+\eps$ for small enough $\eps>0$.%
\footnote{We add $\eps$ to ensure that $d(x, y) > 0$ for any $x\in X$ and $y\in Y$;
so $|x-y|_W$ is indeed a metric.}
This yields a metric on $W=X\sqcup Y$ for which
$|X- Y|_{\mathcal{H}(W)} \z< r$.
\qeds
 



\subsection*{A definition with fixed \textit{Z}}

\begin{thm}{Proposition}\label{prop:GH-with-fixed-Z}
In the Definition~\ref{def:GH}, 
one can fix the space $Z$ once for all, by taking $Z=\mathcal{F}(\NN)$%
\footnote{i.e., the space of bounded infinite sequences.}.  
That is, 
$$|X-Y|_{\mathcal{M}} = \inf \{|X'-Y'|_{\mathcal{H}(\mathcal{F}(\NN))}\}$$ 
where the infimum is taken over all pairs of sets $X'$ and $Y'$ in $\mathcal{F}(\NN)$
which isometric to  $X$ and $Y$ correspondingly.  
\end{thm}

\parit{Proof.}
It is clear that $|X-Y|_{\mathcal{M}} \leq \inf \{|X'-Y'|_{\mathcal{H}(\mathcal{F}(\NN))}\}$.  
Let $W$ be an arbitrary metric space with the underlying set $X\sqcup Y$ as in the proof of Proposition~\ref{prop:GH=X+Y}.
Note $W$ is compact since it is union of two compact subsets $X,Y\subset W$.
According to Problem~\ref{pr:compact->F_N},
$W$ admits a distance preserving map to $\mathcal{F}(\NN)$.
So $\inf \{|X-Y)|_{\mathcal{H}(\mathcal{F}(\NN))}\} \leq |X-Y|_{\mathcal{H}(W)}$, and taking the infimum over all such $W$ gives $\inf \{|X-Y|_{\mathcal{H}(\mathcal{F}(\NN))} \}\leq |X-Y|_{\mathcal{M}}$.
\qeds

\begin{thm}{Exercise}\label{ex:euclid-isom}
Let $X,Y$ be two compact sets in the Euclidean plane $\RR^2$.
Show that $X$ is isometric to $Y$ if and only if there is an isometry $\iota\:\RR^2\to \RR^2$
which sends $X$ to $Y$.
\end{thm}

\begin{thm}{Exercise}\label{ex:mink-isom}
Find two isometric subsets $X,Y$ of $\mathcal{F}(\NN)$
such that there is no isometry $\iota\:\mathcal{F}(\NN)\to \mathcal{F}(\NN)$ 
which sends $X$ to $Y$.
\end{thm}

\section{Almost isometries}\label{sec:alm-isom}

\begin{thm}{Definition} Let $X$ and $Y$ be metric spaces and $\eps > 0$. 
A  map\footnote{possibly noncontinuous} $f\: X \z\to Y$ is called an $\eps$-isometry 
if 
$$|f(x)-f(x')|_Y\lege |x-x'|_X\pm\eps$$
for any $x,x'\in X$ 
and if $f(X)$ is an $\eps$-net in $Y$.
\end{thm}

\begin{thm}{Exercise}\label{ex:alm-isom:compositon}
Let $f\:X\to Y$ and $g\:Y\to Z$ be two $\eps$-isometries.
Show that $g\circ f\: X\to Z$ is a $(3\cdot\eps)$-isometry.
\end{thm}


\begin{thm}{Exercise}\label{ex:alm-isom:inverse}
 Assume $f\: X \z\to Y$ is an $\eps$-isometry.
Show that there is a $(3\cdot\eps)$-isometry 
$g\: Y\to X$.
\end{thm}

\begin{thm}{Exercise}\label{ex:GH=>eps-isom}
Assume $|X-Y|_{\mathcal{M}}<\eps$, show that there is a $(2\cdot\eps)$-isometry 
$f\: X\to Y$.
\end{thm}

\begin{thm}{Proposition}\label{prop:alm-isom=>GH}
Let $X$ and $Y$ be metric spaces 
and let $f\: X\to Y$ be an $\eps$-isometry.
Then $|X-Y|_{\mathcal{M}}\le 2\cdot \eps$.
\end{thm}

\parit{Proof.} Let us equip $W=X\sqcup Y$ with the metric defined the following way:
\begin{enumerate}
\item  For any $x,x'\in X$
$$|x-x'|_W=|x-x'|_X;$$
\item For any $y,y'\in Y$,
$$|y-y'|_W=|y-y'|_Y$$
\item For any $x\in X$ and $y\in Y$,
$$|x-y|_W=\eps+\inf_{x'\in X}\{|x-x'|_X+|f(x')-y|_Y\}.$$
\end{enumerate}

\begin{thm}{Exercise}\label{ex:alm-isom=>GH}
 Show $|{*}-{*}|_W$ is indeed a metric on $W$.
\end{thm}

Since $f(X)$ is an $\eps$-net in $Y$,
for any $y\in Y$ there is $x\in X$ such that $|f(x)-y|_Y\z\le\eps$;
therefore $|x-y|_W\le 2\cdot\eps$.
On the other hand for any $x\in X$, we have $|x-y|_W\le\eps$
for $y=f(x)\in Y$.

It follows that $$|X-Y|_{\mathcal{H}(W)}\le 2\cdot\eps.$$
Hence the result.
\qedsf

\section{Gromov--Hausdorff metric is a metric}

In this section we prove  Theorem~\ref{thm:GH-is-a-metric}.


Let $X$, $Y$ and $Z$ be arbitrary  compact metric spaces.
We need to check the following (see Definition~\ref{def:metric-space}).
\begin{enumerate}[{\it (i)}]
\item\label{GH-1} $|X-Y|_{\mathcal{M}}\ge 0$;
\item\label{GH-2} $|X-Y|_{\mathcal{M}}=0$ if and only if $X$ is isometric to $Y$;
\item\label{GH-3} $|X-Y|_{\mathcal{M}}=|Y-X|_{\mathcal{M}}$;
\item\label{GH-4} $|X-Y|_{\mathcal{M}}+|Y-Z|_{\mathcal{M}}\ge |X-Z|_{\mathcal{M}}$.
\end{enumerate}


Note that {\it (\ref{GH-1})}, {\it(\ref{GH-3})} and ``if''-part of {\it(\ref{GH-2})} follow directly from the definition of Gromov--Hausdorff metric (\ref{def:GH}).

\parit{Proof of (\ref{GH-4}).}
Choose arbitrary $a,b \in \mathbb{R}$ such that
$$a>|X-Y|_{\mathcal{M}}\ \ \text{and}\ \  b>|Y-Z|_{\mathcal{M}}.$$
Choose two metrics on $U=X\sqcup Y$ and $V=Y\sqcup Z$ so that
$|X-Y|_{\mathcal{H}(U)}<a$ and $|Y-Z|_{\mathcal{H}(V)}<b$ 
and the inclusions $X\hookrightarrow U$, $Y\hookrightarrow U$, $Y\hookrightarrow V$ and $Z\hookrightarrow V$ are distance preserving.

Consider the metric on $W=X\sqcup Z$ 
so that inclusions $X\hookrightarrow W$ and $Z\hookrightarrow W$ are distance preserving
and 
$$|x-z|_W=\inf_{y\in Y}\{|x-y|_U+|y-z|_V\}.$$
Note that $|{*}-{*}|_W$ is indeed a metric and 
$$d^W_H(X,Z)<a+b.$$
The last inequality holds for any $a>|X-Y|_{\mathcal{M}}$ and $b>|Y-Z|_{\mathcal{M}}$;
hence {\it (\ref{GH-4})} follows.
\qeds

\parit{Proof of ``only if''-part of (\ref{GH-2}).}
According to Exercise~\ref{ex:GH=>eps-isom},
for any sequence $\eps_n\to0^+$ there is a sequence of $\eps_n$-isometries 
$f_n\:X\to Y$.

Since $X$ is compact, 
we can choose a countable dense set
$S$ in $X$.
Use a diagonal procedure if necessary, to pass to a subsequence of $(f_n)$
such that for every $x \in S$ the sequence $(f_n(x))$ 
converges in $Y$. 
Consider the pointwise limit map  $f_\infty \: S \to Y$ defined by
 $$f_\infty(x) = \lim_{n\to\infty} f_n (x)$$ for every $x \in S$. 
Since $$|f_n (x)- f_n (x')|_Y\lege |x- x'|_X \pm\eps_n,$$ 
we have 
$$|f_\infty(x)-f_\infty (x')|_Y 
= \lim_{n\to\infty} |f_n(x)-f_n (x')|_Y 
= |x -x'|_X$$ for all
$x, x' \in S$; 
i.e., $f_\infty\:S\to Y$ is a distance-preserving map. 
Then $f_\infty$ can be extended to a distance-preserving map from all of $X$ to $Y$.
The later is done by setting 
$$f_\infty(x)=\lim_{n\to\infty} f_\infty(x_n)$$ 
for some (and therefore any) sequence of points $(x_n)$ in $S$
which converges to $x$ in $X$.
(Note that if $x_n\to x$ then $(x_n)$ is Cauchy.
Since $f_\infty$ is distance preserving, $y_n=f_\infty(x_n)$ is also a Cauchy sequence in $Y$;
therefore it converges.)

This way we obtain a distance preserving map $f_\infty\:X\to Y$. 
It remains to show that $f_\infty$ is surjective; i.e. $f_\infty(X)=Y$.

Note that in the same way we can obtain a distance preserving map $g_\infty\:Y\z\to X$.
If $f_\infty$ is not surjective, then neither is $f_\infty\circ g_\infty\:Y\to Y$.
So $f_\infty \circ g_\infty$ is a distance preserving map from a compact space to itself which is not an isometry.
The later contradicts Problem~\ref{pr:non-contracting=>isometry}. 
\qeds

\begin{thm}{Exercise}\label{ex:point-diam}
Let $P$ be a one-point metric space. 
Prove that 
$$|X-P |_{\mathcal{M}} = \frac{\diam X}2$$ for any compact metric space $X$.
\end{thm}

\begin{thm}{Exercise}\label{ex:d_GH-and-diam}
 Let $X$ and $Y$ be two compact metric spaces.
Prove that 
$$|\diam X - \diam Y |\le 2\cdot |X-Y|_{\mathcal{M}}.$$
In other words, $\diam$ is a $2$-Lipschitz function on $\mathcal{M}$.
\end{thm}

\begin{thm}{Exercise}\label{ex:pack-GH}
Assume $X_n$ be a sequence of compact metric spaces which converges to a compact metric space $X_\infty$
in the sense of Gromov--Hausdorff.
Show that for any $\eps>0$
$$\pack_\eps X_n\ge\pack_\eps X_\infty$$ 
for all large enough $n$.
In particular, $\pack_\eps$ is a lower semicontinuous function on $\mathcal{M}$.
\end{thm}


\section{Gromov--Hausdorff convegence}

%???Check, maybe we are happy with d'_{GH}???

The Gromov--Hausdorff metric defines Gromov--Hausdorff convegence
and this is the only thing it is good for.
In other words in all applications, we use only topology on $\mathcal{M}$
and we do not care about particular value of Gromov--Hausdorff distance between spaces.

In order to determine that a given sequence of metric spaces $(X_n)$ converges in the Gromov--Hausdorff sense to $X_\infty$, it is sufficient to estimate distances $|X_n-X_\infty|_{\mathcal{M}}$ and  check if $|X_n-X_\infty|_{\mathcal{M}}\to 0$.
This problem turns to be simpler than finding Gromov--Hausdorff distance between a particular pair of spaces.
The proposition below gives one way to do this.

\begin{thm}{Proposition}\label{prop:GH-e-isom}
A sequence of compact metric spaces $(X_n)$ converges to  $X_\infty$ in the sense of Gromov--Hausdorff if and only if there is a sequence $\eps_n\to0^+$
and an $\eps_n$-isometry $f_n\:X_n\to X_\infty$ for each $n$.
\end{thm}

\parit{Proof.} Follows from Propsition~\ref{prop:alm-isom=>GH} and Exercise~\ref{ex:GH=>eps-isom}
\qeds


\section{Gromov's compactness theorem}

The following theorem is analogous to Blaschke's compactness theorems (\ref{thm:compact+Hausdorff}).

\begin{thm}{Gromov's compactness theorem}\label{thm:gromov-compactness}%
Let $\mathcal{Q}$ be a closed subset of $\mathcal{M}$.
Then $\mathcal{Q}$ is compact if and only if there is a sequence of positive numbers $\eps_1,\eps_2,\dots$ such that $\eps_n\to 0$ and 
$$\pack_{\eps_n}X\le n\eqlbl{eq:pack<n}$$
for any space $X$ in $\mathcal{Q}$.
\end{thm}

\begin{thm}{Exercise}\label{pack<n;n>1}
Show that the conclusion of the theorem does not hold
if the inequality \ref{eq:pack<n} holds only for $n\ge 2$.
\end{thm}




\begin{thm}{Lemma}
$\mathcal{M}$ is complete.
\end{thm}

\parit{Proof.}
Let $(X_n)$ be a Cauchy sequence in $\mathcal{M}$.
Passing to a subsequence if necessary, 
we can assume that $|X_n-X_{n+1}|_{\mathcal{M}}<\tfrac1{2^n}$ for each $n$.
In particular, for each $n$ one can equip $W_n=X_n \sqcup X_{n+1}$ with a metric such that
inclusions $X_n\hookrightarrow W_n$ and $X_{n+1}\hookrightarrow W_n$ are distance preserving
and $$|X_n-X_{n+1}|_{\mathcal{H}(W_n)}\z<\tfrac1{2^n}$$
for each $n$.

Set $W$ to be the disjoint union of all $X_n$.
Let us equip $W$ with a metric defined the following way:
\begin{itemize}
\item for any fixed $n$ and any two points $x_n,x_n'\in X_n$ set
$$|x_n-x_n'|_W=|x_n-x_n'|_{X_n}$$
\item for any positive integers $m>n$ and any two points $x_n\in X_n$ and $x_m\in X_m$ set
$$|x_n-x_m|_W=\inf\left\{\sum_{i=n}^{m-1}|x_i-x_{i+1}|_{W_i}\right\},$$
where the infimum is taken for all sequences $x_i\in X_i$.
\end{itemize}

\begin{thm}{Exercise}
Check that this indeed defines a metric on $W$.
\end{thm}

Let $\bar W$ be the completion of $W$.
Note that $|X_m-X_n|<\tfrac1{2^{n-1}}$ if $m>n$.
Therefore the union of $X_1\cup X_2\cup\dots\cup X_n$ forms a $\tfrac1{2^{n-1}}$-net in $\bar W$.
Since each $X_i$ is compact, we get that $\bar W$ admits a compact $\eps$-net for any $\eps>0$.
According to Problem~\ref{pr:compact-net}, $\bar W$ is compact.

According to Blaschke's compactness theorem (\ref{thm:compact+Hausdorff}),
we can pass to a subsequence of $(X_n)$ which converge in $\mathcal{H}(\bar W)$ and therefore in $\mathcal{M}$.
\qeds

\parit{Proof of \ref{thm:gromov-compactness}; ``only if'' part.}
If there is no sequence $\eps_n\to0$ as described in the problem, then for a fixed fixed $\delta>0$
there is a sequence of spaces $X_n\in\mathcal{Q}$ such that $$\pack_\delta X_n\to\infty\ \ \text{as}\ \  n\to\infty.$$
Since $\mathcal{Q}$ is compact, 
this sequence has a partial limit say $X_\infty\in\mathcal{Q}$.
It is easy to see that $\pack_{\delta/10} X_\infty=\infty$;
the later contradicts Theorem~\ref{thm:finite_pack=compact}.

\parit{``If'' part.}
Let us fix the sequence $\eps_n\to 0$ as in the problem and consider the set $\hat{\mathcal{Q}}$ of all (isometry classes of all) metric spaces $X$ such that
$\pack_{\eps_n} X\le n$ for any $n$. 
According to Exercise~\ref{ex:pack-GH}, $\hat{\mathcal{Q}}$ is closed in $\mathcal{M}$.
Clearly $\mathcal{Q}\subset\hat{\mathcal{Q}}$.
Therefore it is sufficient to prove that $\hat{\mathcal{Q}}$ is compact.

Note that $\diam X\le \eps_1$ for any $X\in \hat{\mathcal{Q}}$.
Given positive integer $n$ consider set of all metric spaces $\mathcal{W}_n$
with number of points at most $n$ and diameter $\le \eps_1$.
Note that $\mathcal{W}_n$ is compact for each $n$.
Further a maximal $\eps_n$-packing of any $X\in\hat{\mathcal{Q}}$ forms a subspace from $\mathcal{W}_n$.
Therefore $\mathcal{W}_n\cap\hat{\mathcal{Q}}$ is a comapct $\eps_n$-net in  $\hat{\mathcal{Q}}$.
Problem~\ref{pr:compact-net} implies that $\hat{\mathcal{Q}}$ is compact.
\qeds



\section{Comments} 

Given two metric spaces $X$ and $Y$, we will write $X\preccurlyeq Y$ if there is a noncontracting map $f\:X\to Y$;
i.e., if 
$$ |x-x'|_X\le|f(x)-f(x')|_Y$$
for any $x,x'\in X$.

Further, given $\eps>0$, we will write $X\preccurlyeq Y+\eps$
if there is a map $f\:X\to Y$ such that 
$$|x-x'|_X\le|f(x)-f(x')|_Y+\eps$$
for any $x,x'\in X$.

Define 
$$d'_{GH}(X,Y)=\inf\set{\eps}{X\preccurlyeq Y+\eps\ \ \text{and}\ \ Y\preccurlyeq X+\eps}$$
It turns out that $d'_{GH}$ is a different metric on the set of isometry classes of compact metric spaces; i.e., in general $d'_{GH}(X,Y)\not=|X-Y|_{\mathcal{M}}$. 
However, these two metrics define the same topology on $\mathcal{M}$.
More precicely:

\begin{thm}{Proposition}\label{GH-po}
For any sequence of compact metric spaces $(X_n)$ and a compact metric space $X_\infty$,
we have
$$|X_n-X_\infty|_{\mathcal{M}}\to 0 \ \ \ \Leftrightarrow\ \ \ d'_{GH}(X_n,X_\infty)\to 0$$ 
as $n\to\infty$.
\end{thm}

We will not give a proof of this proposition. 
Likely, we will not use it further in the lectures, 
but it might help you to build intuition for Gromov--Hausdorff convergence.
If you want to prove it yourself look in the proof of Theorem~\ref{thm:GH-is-a-metric} 
and try to modify it using ideas from the proof of Problem~\ref{pr:non-contracting=>isometry}.

The Gromov--Hausdorff distance can be defined for arbitrary pair of metric space.
Therefore it is natural to ask why we only consider compact metric spaces.
First note the Gromov--Hausdorff distance from amy metric space $X$ 
to its completion $\bar X$ is zero.
Therfore if you want to end up in a metric space, it is better to consider only complete metric spaces.

Further, the distance between one-point-space and a metric spce with infinite diameter is infinite.
Therefore one has to either consider only bounded metric spaces (i.e., the spaces with finite diameter)
or relux the definition of metric space which allow metric to take infinite value.

Finally, the class of isometry classes of all bounded complete metric spaces forms a class, but not a set.
Hence again we have two choices: either relux the definition of metric space so its points will form a class, or restrict further the class of spaces for which the isometry classes will form a set.

\begin{thm}{Exercise}
Prove that isometry classes of compact metric spaces form a set. 
\end{thm}


\section*{Exercises}

\begin{pr}\label{pr:GH1}
Let $X=\{x,y,z\}$ be a three point subset of Euclidean plane with distances
$$|x-y|=|y-z|=|z-x|=1.$$
\begin{enumerate}[(i)]
\item Find the minimal Hausdorff distance from $X$ to a one-point subset of the plane.
\item Find the Gromov--Hausdorff distance from $X$ to the one-point metric space. 
\end{enumerate}
\end{pr}

\begin{pr}\label{pr:GH2}
Let $X$ and $Y$ be a compact metric spaces which have isometric $\eps$-nets.
Show that 
$$|X-Y|_{\mathcal{M}}\le 2\cdot\eps.$$
Is it always true that 
$$|X-Y|_{\mathcal{M}}\le \eps?$$
\end{pr}




\begin{pr}\label{pr:GH3}
Define the \emph{radius of a metric space}\index{radius of a metric space} $X$ as 
$$\rad X=\inf_x\left\{\sup_y\{|x-y|_X\}\right\}.$$
Equivalently, 
$$\rad X=\inf\set{R>0}{\text{there is}\ x\in X\  \text{such that}\ B_R(x)\supset X}.$$
 
\begin{enumerate}[(i)]
\item Show that for any compact metric space $X$ we have
$$\tfrac12\cdot\diam X\le \rad X\le \diam X.$$
\item Show that for any compact metric spaces $X,Y$ we have
$$|\rad X-\rad Y|\le 2\cdot |X-Y|_{\mathcal{M}}.$$
\end{enumerate}
\end{pr}

\begin{pr}\label{pr:F-X}
Let $X$ be a metric space.
If two compact sets $A, B$ in $X$ are isometric,
we will write $A\iso B$. 
Set
$$d(A,B)=\inf \set{|A'-B'|_{\mathcal{H}(X)}}{A'\iso A \ \text{and}\ B'\iso B}.$$
Note that if $X=\mathcal{F}(\NN)$ then according to Proposition~\ref{prop:GH-with-fixed-Z}, 
$d$ is a metric on $\mathcal{H}(X)/\iso$ (i.e., on the ``$\iso$''-equivalecne classes of $\mathcal{H}(X)$).

Show that it does not hold for arbitrary metric space $X$.
Understand the reason why it holds for $X=\mathcal{F}(\NN)$.
\end{pr}

\begin{pr}\label{pr:under}
Let $X$ be a comapact metric space.
Denote by $\Under(X)$ the set of all isometry classes of metric spaces $Y$ 
which admit a distance non-contracting map $Y\to X$.

\begin{subthm}{pr:under:if}
Show that $\Under(X)$ forms a compact set in $\mathcal{M}$.
\end{subthm}

\begin{subthm}{pr:under:only-if}
Show that for any compact set $K$ in $\mathcal{M}$ there is a compact space $X$
such that  $\Under(X)\supset K$.
\end{subthm}

\end{pr}

\begin{pr}\label{pr:GH-variation}
Consider the pairs $(X,A)$, where $X$ is a compact metric space and $A$ is a closed subset in $X$.
Two such pairs, say $(X,A)$ and $(X',A')$ will be called equivalent (briefly $(X,A)\sim(X',A')$)
if there is an isometry $\iota\:X\to X'$ such that $\iota(A)=A'$.

Modify the definition of Gromov--Hausdorff metric to construct a natural metric on the set of $\sim$-equivalence classes of the pairs $(X,A)$.
\end{pr}



