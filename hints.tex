\chapter{Hints and solutions}

\refstepcounter{section}
\section*{Chapter~\ref{chap:metr}}


\parbf{Exercise \ref{ex:Euclidean-is-metric}.}
Note that it is sufficient to show that for any two vectors $\bm{x}, \bm{y}\in\RR^n $
$$|\bm{x}+\bm{y}|\le |\bm{x}|+|\bm{y}|.$$
Take the square of the left and right parts, simplify, and google Cauchy--Bunyakovsky inequality.


\parbf{Exercise \ref{ex:packing=net}.}
Arguing by contadiction we may find a point which is of distance at least $\eps$ from any element of the $\eps$-packing.
This contradicts the maximality of the packing.


\parbf{Exercise \ref{ex:net=2packing}.} Assume $B$ is a $(2\cdot\eps)$-packing with $m$ elements.
Since $A$ is an $\eps$-net,
for each $b\in B$ we can choose an element $f(b)\in A$ such that $|f(b)-b|\le \eps$.
Note that 
$$|f(b)-f(b'|\ge |b-b'|-2\cdot\eps>0$$
if $b\ne b'$.
I.e., $f\:B\to A$ is injective and hence $m\le n$.


\parbf{Problem \ref{pr:non-contracting=>isometry}.}

\parit{First solution.}???


\parit{Second solution.}\footnote{This proof was suggested by Travis Morrison, one of the students in the MASS program 2011.} 
Given any pair of points $x,y\in X$, 
set $x_n=f^{n}(x)$ and $y_n\z=f^{n}(y)$.
Since $X$ is compact, one can choose an increasing sequence of integers $n_k$
such that both sequences $(x_{n_i})_{i=1}^\infty$ and $(y_{n_i})_{i=1}^\infty$
converge.
In particular, both of these sequences  are Cauchy;
i.e.,
\[
\dist{x_{n_i}}{x_{n_j}}{},\dist{y_{n_i}}{y_{n_j}}{}\to 0
\]
as $\min\{i,j\}\to\infty$.
Since $f$ is noncontracting, we get
\[
\dist{x}{x_{|n_i-n_j|}}{}\le \dist{x_{n_i}}{x_{n_j}}{}.
\]
It follows that  
there is a sequence $m_i\to\infty$ such that
\[
x_{m_i}\to x\ \ \text{and}\ \ y_{m_i}\to y\ \ \text{as}\ \ k\to\infty.
\eqlbl{eq:x_l->x}
\]

Let $\ell_n=\dist{x_n}{y_n}{}$.
Since $f$ is noncontracting, $(\ell_n)$ is a nondecreasing sequence.
On the other hand, 
from \ref{eq:x_l->x}, it follows that $\ell_{m_i}\to\dist{x}{y}{}=\ell_0$ as $i\to \infty$;
i.e., $(\ell_n)$ is a constant sequece.
In particular, $\ell_0=\ell_1$ for any $x$ and $y\in X$;
i.e., $f$ is distance preserving map.

Therefore $f(X)$ is isometric to $X$.
From \ref{eq:x_l->x}, we get that $f(X)$ is everywhere dense.
Since $X$ is compact, we get that $f(X)$ is closed, and hence $f(X)=X$.

\parbf{Problem \ref{pr:nonisometry}.}


\parbf{Problem \ref{pr:1,2,4,8}.} The answer is not unique.
The set $B=\{\tfrac12,4,8\}$ will do; $|A-B|_{\mathcal{H}(\RR)}=\tfrac12$.


\parbf{Problem \ref{ex:kuratowski}.} Use the triangle inequality to show that $K_x$ is distance non-expanding.
To show that $K_x$ is distance non-contracting, note that 
$$|z-y|=\Dist_z(y)-\Dist_y(y).$$


\parbf{Problem \ref{pr:compact->F_N}.} 
Let $X$ be a compact metric space.
First show that there is a countable everywhere dense set $\{x_1,x_2,\dots\}$ in $X$.
Then consider the map $f\:X\to \mathcal{F}(\NN)$ defined by $f\:x\mapsto (|x-x_1|,|x-x_2|,\dots)$.
The same argument as in Problem~\ref{ex:kuratowski} shows that the map is distance non-expanding.
To show that $K_x$ is distance non-contracting, note that 
$$|z-y|=\lim{x_{n_i}\to y}\Dist_z(x_{n_i})-\Dist_y(x_{n_i}).$$


\parbf{Problem \ref{pr:complition-kuratowski}.} Use Exercise~\ref{ex:close-complete}.


\parbf{Problem \ref{pr:almost-min}.} Assuming contrary, one can construct a sequence of points $x_n$ such that
$$\rho(x_{n+1})\le \tfrac{99}{100}\cdot\rho(x_n)\ \ \text{and}\ \ |x_{n+1}-x_n|<\rho(x_n).$$
Show that $x_n$ is a Cauchy sequence and look at its limit.


\parbf{Problem \ref{pr:compact-net}.} Let $Q$ be a compact $\eps$-net in $X$.
Applying Theorem~\ref{thm:finite_pack=compact}, choose a finite $\eps$-net $A$ in $Q$ and note that $A$ is a $2\cdot\eps$-net in $X$.
Then apply Theorem~\ref{thm:finite_pack=compact} again.

\refstepcounter{section}
\section*{Chapter~\ref{chap:hausdorff}}

\parbf{Exercise \ref{ex:R-hausdorff}.}
The ``only if'' part is trivial. 
To prove the ``if'' part, note that from the triange inequality,
we have 
$$\Dist_A\lege \Dist_B\pm R.$$


\parbf{Exercise \ref{ex:H_X}.}
The triangle inequality and symmetry are evident.
It only remains to show that $|A-B|_{\mathcal{H}(X)}=0$ implies $A=B$ (see Definition \ref{def:metric-space}).

If $|A-B|_{\mathcal{H}(X)}\z=0$, then the closure of $A$ contains $B$ and vise versa.
The sets $A$ and $B$ are compact and therefore closed (see page \pageref{compact=>closed}).
Hence $A=B$.



\parbf{Exercise \ref{ex:diam}.} Let $A$ and $B$ be two compact subsets of $X$.
Let $x,y\in A$ be a pair of points of maximal distance apart.

If $|A-B|_{\mathcal{H}(X)}\le\eps$ then there are points $x',y'\in B$ such that $|x-x'|, |y-y'|\le\eps$.
Hence 
$$|x-y|\le |x'-y'|+2\cdot\eps.$$ 
Therefore $\diam A\le \diam B+2\cdot\eps$.
Switching $A$ and $B$, we get 
$$|\diam B-\diam A|\le 2\cdot\eps.$$
In particular, $\diam$ is 2-Lipschitz functuon on $\mathcal{H}(X)$.

\parbf{Exercise \ref{ex:intersection-of-compacts}.} 
The sets $K_n$ are compact and therefore closed. 
Therefore their intresection $K_\infty$ is also closed.
Since $K_\infty\subset K_1$, we have that $K_\infty$ is compact.

To show that $K_\infty\not=\emptyset$, choose a point $x_n\in K_n$ and 
let $x_\infty$ be a partial limit of the sequence $x_n$.
For any fixed $m$, we have $x_n\in K_m$ for all large $n$.
Therefore $x_\infty \in K_m$ for any $m$;
i.e., $x_\infty\in K_\infty$.

For the last part of the exercise, one could take the subsets $K_n=[n,\infty)$ of real line.

\parbf{Exercise \ref{ex:convex=closed-in-H}.}Let $K_n$ be a sequence of convex compact figures
which converges to $K_\infty$ in the Hausdorff sense.
It is sufficient to show that for any two points $x_\infty,y_\infty\in K_\infty$,
its midpoint $z_\infty=\tfrac{x_\infty+y_\infty}{2}$ lies in $K_\infty$.

\begin{wrapfigure}{r}{30mm}
\begin{lpic}[t(-10mm),b(-10mm),r(0mm),l(0mm)]{pics/drop(0.15)}
\end{lpic}
\end{wrapfigure}

Clearly, we can choose points $x_n,y_n\in K_n$
such that $x_n\to x_\infty$ and $y_n\to y_\infty$.
Since $K_n$ is convex, we have $z_n=\tfrac{x_n+y_n}{2}\in K_n$.
Clearly, $z_n\to z_\infty$;
hence $z_\infty\in K_\infty$.


\parbf{Exercise \ref{ex:calculations}.} Given $x\in P$, consider the ``teardrop'' $D_x$ formed by the convex hull of $x$ and $\bar B_R(0)$.
Note that $D_x\subset P$ and $(1+\eps)\cdot D_x\supset \bar B_{R\cdot\eps}(x)$.


\parbf{Exercise \ref{ex:area+perim}.}
Prove that 
$$\per F\ge 2\cdot \diam F$$
for any convex figure $F$ in the plane.
Use this statement and Exercise~\ref{ex:diam}
in addition to the arguments for nondegenerate case.
 
\parbf{Exercise~\ref{ex:hausdorff-upto-cong}.}???


\parbf{Problem \ref{problem2}.}
Since $K$ is compact, the function $\Dist_x$ admits a minimum $K$.

Assume there are two distinct points of minimal distance, say $\bar x$ and $\bar x'$.
From convexity, their midpoint $z=\tfrac{\bar x+\bar x'}{2}$ lies in $K$.
Clearly $|x-z|<|x-\bar x|=|x-\bar x'|$,
a contradiction.

Given $x,y\in\RR^m$, we need to show that 
$$|\bar x-\bar y|\le|x-y|.\eqlbl{eq:|xy|=<|xy|}$$ 
In the case where $x,y\in K$,
we have $\bar x=x$ and $\bar y=y$, hence \ref{eq:|xy|=<|xy|} follows. 

If $x\in K$, $y\notin K$ then $\bar x=x$.
We may assume that $\bar y\ne\bar x$, otherwise \ref{eq:|xy|=<|xy|} is trivial.
Note that 
$$\angle \bar y\bar x x\ge \tfrac\pi2.\eqlbl{eq:angle>=pi/2}$$ 
If not, moving $\bar y$ along $[\bar y \bar x]$ would decrease the distance to $y$.
From convexity of $K$, we have $[\bar y \bar x]\subset K$, hence a contradiction.
Clearly, \ref{eq:angle>=pi/2} imples \ref{eq:|xy|=<|xy|}

It remains to consider the case $x, y\notin K$.
As above we may assume that $\bar y\ne\bar x$.
Further the same argument as above shows that 
$$\angle \bar y\bar x x,\angle \bar x\bar y y\ge \tfrac\pi2.\eqlbl{eq:angle,angle>=pi/2}$$
Straightforward calculations show that \ref{eq:angle,angle>=pi/2} imply \ref{eq:|xy|=<|xy|}.
The latter is also a partial case of the Arm Lemma (\ref{lem:arm}).


\parbf{Problem \ref{pr:Hausdorff-bry}.} Both statements do not hold. 
For the first one take $A$ to be the unit square in the plane and $B$ a finite $\eps$-net in $A$.
In this case, $|A-B|_{\mathcal{H}(\RR^2)}\le \eps$.  On the other hand, $\partial B=B$ and 
$|B-A|_{\mathcal{H}(\RR^2)}\ge \tfrac12-\eps$.

For the second part of problem, one can take the unit square $A$ and $B=\partial A$,
then $\partial B=B$.

(In fact one can construct an arbitrary number of compact sets with the same boundary.
Google ``Lakes of Wada'' to see such examples.)


\parbf{Problem \ref{pr:LebesgueMinimalProblem}.} %???
Given a set of points $X$, consider a maximal set $F$, which includes $X$ as a subset.
Let us show that $F$ is a figure of constant width.

Note that $F$ is an intersection of a family of unit discs; in particular $F$ is convex.
If $F$ lies between two lines of distance less than $1$, then
$F$ has to lie in an intersection of two discs tangent to these lines.
In this case, the center of one of these discs can be added to $F$ so that the diameter does not exceed $1$,
a contradiction.

Any figure of constant width contain a pair of points of distance $1$.
Then it contains the intersection of all unit discs which contain these two points.
The set is a digon with sides formed by arcs of unit circle;
the area of the digon is much bigger than $1/100$.

Assume $\diam F\ge 10000$.
We can assume that the points in $F$ which realize the diameter have the form $(0,0)$ and $(x,0)$.
If $F$ can not be broken into two parts, then one can find $50000$ points 
of the form $(2,y_1)$ $(4,y_2),\dots,(10000,y_{50000})$ in $F$.
Each lies in a figure of width $1$ and from above their total area is more that $50000/100$.
  

\parbf{Problem \ref{pr:complete>compact}.} Note that given $\eps>0$, the set
$$\bigcup_{i=1}^n K_i$$
forms an $\eps$-net in $Q$ for all large enough $n$.
Then apply Problem~\ref{pr:compact-net}.

To generalize Lemma~\ref{lem:complete+Hausdorff},
note that we can pass to the subspace $Q$ in $X$.

\refstepcounter{section}
\section*{Chapter~\ref{chap:gromov-hausdorff}}



\parbf{Exercise \ref{ex:euclid-isom}.}
???


\parbf{Exercise \ref{ex:mink-isom}.} First do the same for two two-point sets in the plane equipped with the Manhattan metric; it is defined on page \pageref{manhattan-metric}.  Note that an isometry must map midpoints to midpoints.


\parbf{Exercise \ref{ex:alm-isom:compositon}.}
From the definition of $\eps$-isometry,
we have
\begin{align*}
|g(f(x)) - g(f(x'))|_Z 
&\lege |f(x) - f(x')|_Y \pm \eps 
\lege
\\
&\lege |x - x'|_X \pm 2\cdot\eps.
\end{align*}
  
It remains to show that $(g \circ f)(X)$ is a $3\cdot\eps$-net in $Z$.  Given $z \in Z$, choose $y \in Y$ and subsequently $x \in X$ such that
$$|g(y) - z|_Z \le \eps, \quad |f(x) - y|_Y \le \eps.$$  Then
\begin{align*}
|g(f(x)) - z|_Z &\leq |g(f(x)) - g(y)|_Z + |g(y) - z|_Z\\
&\leq |f(x) - y|_Y + \eps + \eps\\
&\leq 3\cdot\eps.
\end{align*}


\parbf{Exercise \ref{ex:alm-isom:inverse}.}
If $f\: X \to Y$ is an $\eps$-isometry, then for any $y \in Y$, the set $\set{x \in X}{|f(x) - y| \le \eps}$ is nonempty.  By the Axiom of Choice, we can choose such an $x$ for each $y$, producing a function $g\: Y \to X$ with the property that
$$|f(g(y)) - y|_Y \le \eps, \quad \text{ for all } y \in Y.$$  
Then
\begin{align*}
|g(y) - g(y')|_X &< |f(g(y)) - f(g(y'))|_Y + \eps\\
&\le |y - y'|_Y + |f(g(y)) - y|_Y + |f(g(y')) - y'|_Y + \eps\\
&\le |y - y'|_Y + 3\cdot\eps,
\end{align*} and in a similar fashion,
\begin{align*}
|y - y'|_Y &\le |f(g(y)) - f(g(y'))|_Y + |f(g(y)) - y|_Y + |f(g(y')) - y'|_Y\\
&< |g(y) - g(y')|_X + 3\cdot\eps.
\end{align*}  
Finally,
\begin{align*}
|g(f(x)) - x|_X 
&< |f(g(f(x))) - f(x)|_Y + \eps 
\\
&\le 2\cdot\eps\end{align*}
shows that $g(Y)$ is a $2\cdot\eps$-net in $X$ (hence it is a $3\cdot\eps$-net in $X$.)


\parbf{Exercise \ref{ex:GH=>eps-isom}.} Embed $X$ and $Y$ in some space $Z$ such that $|X - Y|_{\mathcal{H}(Z)} < \eps.$  Given $x \in X$, the set $\set{y \in Y}{|x - y|_Z < \eps}$ is therefore nonempty.  We can then (using the Axiom of Choice) choose an element $y$ in this set and define $f(x) = y$.  
The function $f\: X \to Y$ has the property that 
$$|f(x) - x|_Z < \eps, \quad \text{ for all }x \in X.$$  
By the triangle inequality,
\begin{align*}
|f(x) - f(x')|_Y 
&\le |x - x'|_Z + |f(x) - x|_Z + |f(x') - x'|_Z 
\\
&< |x - x'|_X + 2\cdot\eps
\end{align*}
and similarly
\begin{align*}
|x - x'|_X 
&\le 
|f(x) - f(x')|_Z + |f(x) - x|_Z + |f(x') - x'|_Z 
\\
&< 
|f(x) - f(x')|_Y + 2\cdot\eps.
\end{align*}
Lastly, we show that $f(X)$ is a $2\cdot\eps$-net in $Y$.  Given $y \in Y$, there exists $x \in X$ with $|x - y|_Z < \eps$ because $|X - Y|_{\mathcal{H}(Z)} < \eps.$  Then
\begin{align*}
|f(x) - y|_Y 
&\le 
|f(x) - x|_Z + |x - y|_Z 
\\
&< 2\cdot\eps.
\end{align*}



\parbf{Exercise \ref{ex:alm-isom=>GH}.}

\parbf{Exercise \ref{ex:point-diam}.}
We need to prove two inequalities: $|X-P|_{\mathcal{M}} \le \tfrac{\diam X}2$ and $|X-P|_{\mathcal{M}} \ge \tfrac{\diam X}2$.

To prove the first one, we shall simply construct a metric on the set 
$$W=P\sqcup X=\{p\}\sqcup X$$ 
by setting $|p-x|_W=\tfrac{\diam X}2$ for any $x\in X$ and $|x-y|_W=|x-y|_X$ for any $x,y\in X$.
It is straightforward to check that this indeed defines a metric and $|P-X|_{\mathcal{H}(W)}= \tfrac{\diam X}2$.

To prove the second inequality, choose points $x,y\in X$ so that $|x-y|_X= \diam X$.
Assume $X$ is a subset of a space $Z$ such that $|\{p\}-X|_{\mathcal{H}(Z)}<r$.
Then $|p-x|,|p-y|<r$ and therefore $r>\tfrac{\diam X}2$.


\parbf{Exercise \ref{ex:d_GH-and-diam}.} Let $P$ be a one-point space.
According to Exercise \ref{ex:point-diam}, $|X-P|_{\mathcal{M}} = \tfrac{\diam X}2$.
From the triangle inequality,%???we do not have it yet???
 we have
$$||X-P|_{\mathcal{M}}-|X-P|_{\mathcal{M}}|\le |X-Y|_{\mathcal{M}},$$
hence, the result.


\parbf{Exercise \ref{ex:pack-GH}.} Let $x_1, \ldots , x_N \in X_\infty$ be a maximal $\eps$-packing.  If $|X_n-X_\infty|_{\mathcal{M}} < r$, then we can embed $X_n$ and $X_\infty$ in some space $Z$ such that there exist points $y_1, \ldots , y_N \in X_n$ with $|x_i - y_i|_Z < r$ for each $i$.  It follows from the triangle inequality that
$$|y_i - y_j| \ge |x_i - x_j| - |x_i - y_i| - |x_j - y_j| > |x_i - x_j| - 2r$$ for all $i,j$.  Since $|x_i - x_j| > \eps$ for all $i \neq j$, and since there are only finitely many $x_i$, there is a $\delta > 0$ such that $|x_i - x_j| > \eps + \delta$ for all $i \neq j$.  It follows from the above calculation that if $r < \delta / 2$, then $|y_i - y_j| > \eps$ for all $i \neq j$.  Hence, $y_1, \ldots , y_N$ is an $\eps$-packing in $X_n$ (not necessarily maximal.)  Thus
$$ \pack_\eps X_n \ge \pack_\eps X_\infty. $$


\parbf{Exercise \ref{pack<n;n>1}.}
Consider the set $\mathcal{Q}$ of all metric spaces with at most two points.

\parbf{Problem \ref{pr:GH1}.}
\begin{enumerate}[(i)]
\item The minimal Hausdorff distance is $1/\sqrt{3}$ and is achieved by taking the point to be the center of the equilateral triangle.
\item See Exercise \ref{ex:point-diam}.
\end{enumerate}


\parbf{Problem \ref{pr:GH2}.}
If $A \subseteq X$ and $B \subseteq Y$ are $\eps$-nets such that $A$ is isometric to $B$, then $|A - B|_{\mathcal{M}} = 0$.  Check that $|X - A|_{\mathcal{H}(X)} \le \eps$, which shows $|X - A|_{\mathcal{M}} \le \eps$ (and similarly $|Y - B|_{\mathcal{M}} \le \eps$.)  By the triangle inequality,
$$|X - Y|_{\mathcal{M}} \le |X - A|_{\mathcal{M}} + |A - B|_{\mathcal{M}} + |Y - B|_{\mathcal{M}} \le 2\cdot\eps.$$

??? For the second question...
% ??? I don't know the answer to the second question - Allan


\parbf{Problem \ref{pr:GH3}.}
\begin{enumerate}[(i)]
\item Since $B_{\diam X}(x) \supset X$ for any $x \in X$, we have $\rad X \le \diam X$.  If there exist $R > 0$ and $x \in X$ such that $B_R(x) \supset X$, then for any $y, z \in X$,
$$|y - z| \le |x - y| + |x - z| \le 2\cdot R.$$  
Taking the supremum over all $y,z \in X$ gives $\tfrac12 \cdot \diam X \le R$, and taking the infimum over all such $R$ gives $\tfrac12 \cdot \diam X \le \rad X.$
\item It suffices to show 
$\rad X \le \rad Y + 2 \cdot |X - Y|_{\mathcal{M}}$.  
Embed $X$ and $Y$ in a space $Z$ and let $D = |X - Y|_{\mathcal{H}(Z)}$.  
Let $y \in Y$ be such that $B_R(y) \supset Y$ for some $R > 0$.  Then there exists $x \in X$ such that $|x - y|_Z \le D$.  
For any other $x' \in X$, we can similarly find $y' \in Y$ with $|x' - y'|_Z \le D$, and then
$$|x - x'|_X \le |x - y|_Z + |y - y'|_Y + |x' - y'|_Z \le R + 2\cdot D.$$
In other words, $B_{R + 2\cdot D}(x) \supset X$ and consequently $\rad X \le R + 2\cdot D$.  
Taking the infimum over all such $R$ gives $\rad X \le \rad Y + 2\cdot D$.  
Taking the infimum over all such embeddings of $X$ and $Y$ into $Z$ yields 
$$\rad X \leq \rad Y + 2\cdot |X - Y|_{\mathcal{M}}.$$
\end{enumerate}

\parbf{Problem \ref{pr:F-X}.}
???

\parbf{Problem \ref{pr:under}.}
Part \ref{SHORT.pr:under:if} follows directly from 
Gromov's compactness theorem (\ref{thm:gromov-compactness}) 
and Theorem \ref{thm:finite_pack=compact:pack}.

\parit{\ref{SHORT.pr:under:only-if}).} Applying Gromov's compactness theorem (\ref{thm:gromov-compactness}), we get that there is a sequence $\eps_n\to0+$ such that if $[Z]\in K$
then $\pack_{\eps_n}Z\le n$ for each $n$ .

It remains to construct a compact space $X$ such that if $\pack_{\eps_n}Z\le n$ for each $n$ then there is a distance non-contracting map $Z\to X$.
As a set of points in $X$, take all infinite sequences of positive integers $(k_1,k_2,\dots)$
such that $k_n\le n$ for each $n$.
Given two such sequences $\bm{k}=(k_1,k_2,\dots)$ and $\bm{k}'=(k'_1,k'_2,\dots)$ set
$$|\bm{k}-\bm{k}'|_X=2\cdot\eps_{n-1},\eqlbl{eq:metric-on-X}$$
where $n$ is the smallest number such that $k_n\ne k_n'$.

To check that $X$ is compact, we apply diagonal procedure ...???

It remains to construct a distance non-contracting map $Z\to X$.
According to Exercise \ref{ex:packing=net},
we can choose an $\eps_n$-net $a_{n,1},a_{n,2},\dots,a_{m,n}$ in $Z$ with $m\le n$.
Given a point $z\in Z$, consider sequence $k_n$ of integers such that
$|a_{k_n,n}-z|<2\cdot\eps_n$.
Note that a sequence $\bm{k}=(k_1,k_2,\dots)$ describes a point in $X$.
From \ref{eq:metric-on-X}, 
it follows that the map $Z\to X$ defined as $z\mapsto \bm{k}$ 
is distance non-contracting.

\parbf{Problem \ref{pr:GH-variation}.}
???

\refstepcounter{section}
\section*{Chapter~\ref{chap:length}}

\parbf{Exercise \ref{ex:nonrectifiable-curve}.}

\parbf{Exercise \ref{ex:length=closed-in-GH}.}

\parbf{Exercise \ref{ex:length-is-not-homeo}.}
Consider the ``comb space" $X \subset \RR^2$ which is the union of the line segment from $(0,0)$ to $(1,0)$, the line segment from $(0,0)$ to $(0,1)$, and the line segment from $(1/n,0)$ to $(1/n,1)$ for each $n \in \NN$.  Any two points in $X$ can be connected by a curve of length at most $3$, so $\hat d$ is a finite metric.  However, any $d$-open ball about a point $(0,y)$ will be disconnected, whereas $\hat d$-open balls are always path-connected.


\parbf{Exercise \ref{ex:proper=>geodesic}.}


\parbf{Exercise \ref{ex:lc-not-complete}.}
The open interval $(0,1)$ is locally compact, but not complete.

\parbf{Exercise \ref{ex:no-angle}.}




\parbf{Problem \ref{pr:H(R^2)-length-space}.}
By (???) it suffices to show that for any two $X, Y \in \mathcal{H}(\RR^2)$ and any $\eps > 0$, there is $Z \in \mathcal{H}(\RR^2)$ such that
$$|X-Z|_{\mathcal{H}(\RR^2)} < \frac{1}{2}\cdot|X-Y|_{\mathcal{H}(\RR^2)} + \eps, 
\quad 
|Y-Z|_{\mathcal{H}(\RR^2)} < \frac{1}{2}\cdot|X-Y|_{\mathcal{H}(\RR^2)} + \eps.$$  
By Theorem \ref{thm:finite_pack=compact}, every compact set has a finite $\eps$-net.  
It is readily checked that the Hausdorff distance between a set and its $\eps$-net is less than $\eps$.  
So by the triangle inequality for the Hausdorff metric, it suffices to prove the above result in the case where $X$ and $Y$ are finite sets.

Suppose $X = \{x_1, \ldots , x_n\}$ and $Y = \{y_1, \ldots , y_m\}$.  
For every $x_i \in X$, choose $y \in Y$ of minimal distance to $x_i$ and let $a_i = \frac{x_i + y}{2}$ be their midpoint.  
Similarly, for every $y_j \in Y$, choose $x \in X$ of minimal distanct to $y_j$ and let $b_j = \frac{y_j + x}{2}$.  Let $A = \{a_1, \ldots , a_n\}$ and $B = \{b_1, \ldots , b_m\}$ and set $Z = A \cup B$.  Note that because $Z$ is a finite set, it is compact.

We will show that $|X - Z|_{\mathcal{H}(\RR^2)} \le \frac{1}{2}\cdot|X - Y|_{\mathcal{H}(\RR^2)}$ and by a similar argument, $|Y - Z|_{\mathcal{H}(\RR^2)} \le \frac{1}{2}\cdot|X - Y|_{\mathcal{H}(\RR^2)}$.  Let $R > 0$ be such that
$$\bigcup_{x \in X}B_R(x) \supset Y, \quad \bigcup_{y \in Y}B_R(y) \supset X.$$
It follows that
$$\bigcup_{z \in Z}B_{\frac{1}{2}\cdot R}(z) \supset X$$
because for every $x \in X$, there is $y \in Y$ and $z \in Z$ with $|x - z| = \frac{1}{2}\cdot|x - y|$, and moreover $y$ can be taken to be of minimal distance to $x$, so that $|x - y| \le R$.  
On the other hand, for every $z \in Z$, there is $x \in X$ and $y \in Y$ 
such that $|x - z| = \frac{1}{2}\cdot|x - y| \le \frac{1}{2}\cdot\cdot R$, again by minimality.  
Thus,
$$\bigcup_{x \in X}B_{\frac{1}{2}\cdot R}(x) \supset Z.$$  
So $|X - Z|_{\mathcal{H}(\RR^2)} \le R$, and by taking an infimum over all such $R$,
$$|X - Z|_{\mathcal{H}(\RR^2)} 
\le 
\tfrac{1}{2}\cdot|X - Y|_{\mathcal{H}(\RR^2)}.$$


\parbf{Problem \ref{pr:M-length-space}.}
Repeat the argument of Problem \ref{pr:H(R^2)-length-space} to show that $\mathcal{H}(\mathcal{F}(\NN))$ is a length space.  
The result then follows from Proposition \ref{prop:GH-with-fixed-Z}.

\parbf{Problem \ref{pr:H>GH-boundary}.}
Use the distance non-expanding map as in Problem~\ref{problem2}.

\parbf{Problem \ref{pr:trig-inq=>interval}.}

\parbf{Problem \ref{pr:1-st-var}.}

\parbf{Problem \ref{pr:adj-angles}.}
\refstepcounter{section}
\section*{Chapter~\ref{chap:triangulation}}

\parbf{Problem \ref{pr:1000}.}
Look at the natural triangulation of the boundary of convex polyhedra with all the vertices on the curve $t\mapsto(t,t^2,t^3,t^4)$ in $\RR^4$.

???

\parbf{Problem \ref{pr:tringulation-of-poly}.}
Let $k$ be the number of vertices of $P$.
Let us apply induction on $k$;
the base case $k=3$ is trivial.

Choose vertices $b$ of $P$ 
which minimize 
the $x$ coordinate.
Note that the polygon $P$ is convex at $b$.

Let $a$ and $c$ be the vertices of $P$
which are the right and left neighbors of $b$.
If the interior of segment $[ac]$ lies completely in $P$,
then after cutting the triangle $\trig abc$
from $P$ we get a polygon $P'$ with smaller number of vertices; 
applying the induction hypothesis,
we get a triangulation of $P'$ which together with $\trig abc$ gives a triangulation of $P$.

If the interior of $[ac]$ does not lie inside of $P$,
then triangle $\trig abc$ contains yet an other vertex of $P$.
Let $d$ be a vertex of $P$ in the triangle $\trig abc$
which lies on the maximal distance from the line $(ac)$.
Note that the interior of the segment $[bd]$ lies in the interior of $P$;
so $[bd]$ cuts $P$ into two polygons
say $Q$ and $R$ 
with smaller number of vertices in each.
It remains to apply the induction hypothesis of $Q$ and $R$.

\parit{Comment.}
Note that if $P$ is a convex polygon with vertices $a_1,a_2,\dots,a_k$ then cutting $P$ along the segments 
$[a_1a_3]$, $[a_1a_4],\dots,[a_1a_{k-1}$
gives a triangulation of $P$.

\parbf{Problem \ref{pr:tringulation-of-poly-3D}.}

\begin{wrapfigure}{r}{48mm}
\begin{lpic}[t(-9mm),b(-4mm),r(0mm),l(0mm)]{pics/acute-triangulation(0.75)}
\end{lpic}
\end{wrapfigure}

\parbf{Problem \ref{ex:acute-triangulation}.}
The solution should be clear from the picture.
In fact any 2-dimensional polyhedral space admits a triangulation all of which triangles are acute,
see \cite{saraf}.

\parbf{Problem \ref{pr:delaunay.triangulation}.}
Google Delaunay triangulation.

\parbf{Problem \ref{pr:polytope=local.cone}.}
Since $P$ is compact,
we can choose a finite cover of $P$ by open balls $B(x_i,r_i)$,
such that for each $i$ the ball $B(x_i,7\cdot r_i)$ forms a cone neighborhood of $x_i$.
Assume further that $i\in \{1,2,\dots,n\}$.

Given $z\in P$ consider the set 
$$F_z=\set{i\in \{1,2,\dots,n\}}{z\in B(x_i,r_i)}.$$
Set 
$$\square_z=\Conv\set{x_i}{i\in F_z}.$$
Note that it is sufficient to show that 
$$P=\bigcup_{z\in P}\square_z.$$

First let us use induction to show that 
\begin{clm}{}\label{clm:sqare.in.P}
 $\square_z\subset P$ 
for any $z\in P$.
\end{clm}
Without loss of generality we may assume that $F_z=\{1,2,\dots,k\}$ for some $k\le n$ 
and $r_1\le r_2\le\dots r_k$.
Set 
$$A_i=\Conv\set{x_j}{j\le i};$$
so $\square_z=A_k$.
Clearly, $A_1=\{x_1\}\subset P$ and
$$
\begin{aligned}
A_{i}&=\Conv(A_{i-1}\cup \{x_{i}\})
=
\\
&=
\set{t\cdot x_i+(1-t)\cdot y\in \RR^m}{y\in A_{i-1},
\ t\in[0,1]}
\end{aligned}
\eqlbl{eq:conv(A)}
$$
for any $i\le k$.

Note that 
$B_{3\cdot r_i}(x_i)\supset A_{i-1}$ for any $i\le k$.
From \ref{eq:conv(A)} we have
\begin{align*}
A_{i-1}\subset P
\ &\Rightarrow\ 
A_{i-1}\subset B_{3\cdot r_i}(x_i)\cap K_{x_i}
\ \Rightarrow\ 
\\
&\Rightarrow\ 
A_i\subset B_{3\cdot r_i}(x_i)\cap K_{x_i}
\ \Rightarrow\ 
A_i\subset P.
\end{align*}
Hence \ref{clm:sqare.in.P} follows.

It remains to show that 
$$P\subset \bigcup_{z\in P}\square_z.\eqlbl{eq:P.in.sqares}$$
Assume contrary.
Let $Q=\bigcup_{z\in P}\square_z$ and  $z\in P$ be a  point which maiximize the distance to 



It remains to show that for any $y\in P$ there is $z\in P$ such that 



For each $i$ consider funnction 
$f_i(z)=|x_iz|^2-r_i^2$.
We will call $f_i(z)$ \emph{power of} $z$ \emph{with respect to sphere of radius} $r_i$  \emph{centered} at $x_i$. 
Clearly $f_i(z)<0$ if and only if $z\in B(x_i,r_i)$. 
Consider corresponding Voronoi domain
$$V_i=\set{z\in \RR^m}{f_i(z)\le f_j(z)\ \text{for any}\ i}.$$
Note that each $V_i$ is an intersection of $n-1$ half-spaces
$$H_{i,j}=\set{z\in \RR^m}{f_i(z)\le f_j(z)}.$$

Set $f=\min_i f_i$;
it is a continuous function on $X$.
Note that $f<0$, in particular $V_i\subset B(x_i,r_i)$ for each $i$.

Note that if $z$ is a point of local maximum of $f$
then $\square_z\ni z$.
Indeed, assume contrary; 
i.e., $z\notin \square_z$.
Let $z^*\in \square_z$ be the closest point to $z$
(it exists by ???).
The 


We may assume that radii are chousen generically;
i.e. if $z\in\cap_{i\in Q}V_i$ for some index set $Q$ then the functions $\set{f_i}{i\in Q}$ are linearly independent in arbitrary neighborhood of $z$.

Consider nerve of covering $\{V_i\}$ of $X$;
it is an abstract simplicial complex $\mathcal S$
with set of vertexes in the index set of $x_i$ 
and an index subset $Q$ forms a simplex 
$\bigcap_{i\in Q}V_i\not=\emptyset$.

Notice that vertexes of any simplex $\Delta^k$ in $\mathcal S$ 
can be reordered as ${i_0},{i_1},\dots,{i_k}$ on such a way that $r_{i_0}\le r_{i_1}\le\dots\le r_{i_k}$.
Clearly $x_{i_n}\in B(x_{i_m},3\cdot r_{i_m})$ for all $n\le m$.
Let us construct a map $\Delta^k\to X$.
\begin{enumerate}
\item map $i_0\mapsto x_{i_0}$;
\item map $i_1\mapsto x_{i_1}$ and use cone structure in $B(x_{i_1},3\cdot r_{i_1})$ to extend it linearly to 1-simplex $i_0i_1$;
\item map $i_2\mapsto x_{i_2}$ and use cone structure in $B(x_{i_2},3\cdot r_{i_2})$ to extend it linearly to 2-simplex $i_0i_1i_2$;
\item and so on.
\end{enumerate}
It is straightforward to check that simplex with metric induced by this map is isometric to a simplex in Euclidean space.
Further, this map agree on intersections of different simplexes of $\mathcal S$, hence we get a map $\iota\:\mathcal S\to X$.

It only remains to show that $\iota(\mathcal S)=X$.
Assume contrary; 
i.e., $\Omega=X\backslash \iota(\mathcal S)\z\ne\emptyset$.
For each $x\in X$ choose a closest point $x^*\in \iota(\mathcal S)$.
Note that 
$$f(x)>f(x^*)$$
for all $x\in \Omega$ sufficiently close to $\iota(\mathcal S)$.
It follows that there is a point $x_0\in \Omega$,
of local maximum of $f$.

Let $Q$ be the a subset of the index set,
such that $i\in Q$ if and only if $V_i\ni x_0$;
denote by $\Delta$ the simplex corresponding to $Q$. 
Sinse $x_0$ is a maximum point of $f$,
we get $x_0\in \Delta$,
a contradiction.
\qeds

\refstepcounter{section}
\section*{Chapter~\ref{chap:pdp}}

\parbf{Exercise \ref{ex:voronoi-in-star}.}
First note that if $x\in V_i$ then any geodesic $[z_ix]$ lies in $V_i$.
Indeed, if $y\in [z_ix]$ then for any $j$ we have
\begin{align*}
|z_i-y|
&=|z_i-x|-|y-x|\le
\\
&\le |z_j-x|-|y-x|\le
\\
&\le 
|z_j-y|;
\end{align*}
i.e., $y\in V_i$.

Therefore if $V_i\not\subset S_i$
then there is a triangle $\Delta$ of the triangulation such that $z_i\in \Delta$
and $V_i$ contains a point $x$ on a side $e$ of $\Delta$
which does not contain $z_i$.

The smallest distance from $x$ to on of $z_j$ on $e$ is smaller than $\eps$.
Hence $|z_i-x|<\eps$???


\parbf{Exercise \ref{ex:PDP=>cont}.}

\parbf{Exercise \ref{ex:PDP+fixed-triang}.}




\parbf{Exercise \ref{ex:star-triangulation}.}

\parbf{Exercise \ref{ex:triangle-reflect}.}

\parbf{Exercise \ref{ex:a-in-Brehm}.}

\parbf{Exercise \ref{ex:akopyan-brehm}.}
Choose a sufficiently fine triangulation of $P$, 
say the diameter of each triangle is less than $\eps$.
If $\{a_1,a_2,\dots,a_n\}$ is the set of vertices for this triangulation,
take $b_i=f(a_i)$ and apply Brehm's extension theorem.
We obtain a map $f_\eps\:P\to\RR^m$ which coincides with $f$ on the set $\{a_1,a_2,\dots,a_n\}$.

Then the statement follows since the triangulation is fine 
and  both $f$ and $f_\eps$ are distance non-expanding.


\parbf{Exercise \ref{ex:tripod}.}

\parbf{Exercise \ref{LP=>short}.}

\parbf{Exercise \ref{ex:S->R--not-inj}.}

\parbf{Exercise \ref{ex:limit-above}.}


\parbf{Problem \ref{pr:zalgaller-acute}.}

\parbf{Problem \ref{pr:brehm}.}
Let $A$ be as in Brehm's theorem.
According to Problem~\ref{problem2}, 
there is a distance non-expanding map $h\:\RR^2\to A$
such that $h(a)=a$ for any $a\in A$.
Taking the composition $f\circ h$ we get the needed distance non-expanding map $\RR^2\to \RR^2$.

\begin{wrapfigure}{r}{48mm}
\begin{lpic}[t(-0mm),b(-0mm),r(0mm),l(0mm)]{pics/bigger-area(0.75)}
\lbl[t]{13,0;$a_1$}
\lbl[b]{3,21;$a_2$}
\lbl[t]{33,0;$a_3$}
\lbl[t]{42,0;$b_1$}
\lbl[b]{43,23;$b_2$}
\lbl[t]{65,0;$b_3$}
\lbl[]{16,6;$A$}
\lbl[]{48,7;$B$}
\end{lpic}
\end{wrapfigure}

\parbf{Problem \ref{pr:perimeter}.}
The second inequality does not hold in general.
This can be seen in the picture.

The first inequality follows easily from two famous theorems in discreet geometry,
one is Alexander's theorem \cite{alexander}
and the other is the 
Kneser--Poulsen conjecture which was solved in 2-dimensional case by Bezdek and Connelly in \cite{bezdek-connelly}.
(The reduction to each of these theorems takes one line, 
but might be not completely evident.)
I encourage you to read these papers, they are totally  beautiful.
You will be surprised to learn that to solve this 2-dimensional problem it is convenient to work in 4-dimensional space, the reason can be seen in Problem~\ref{pr:alexander}.
Here I present an other solution from \cite{petrunin-ruble},
it is not nearly as nice, 
but more elementary.

Given a finite collection of points $a_1,a_2,\dots,a_n$,
Set 
$$\ell(a_1,\dots, a_n)=\per\Conv(a_1,\dots, a_n).$$
Then $\per A\ge\per B$ can be written as
$$\ell(a_1,\dots, a_n)\ge \ell(b_1,\dots, b_n).$$
Applying Brehm's extension theorem (\ref{thm:brehm})
we get a piecewise distance preserving map $f\:A\to\RR^2$
such that $f(a_i)=b_i$ for each $i$.

Assume contrary;
i.e.,
$$\ell(a_1,\dots, a_n)< \ell(b_1,\dots, b_n).\eqlbl{eq:conta-a-b}$$
We can assume that 
$\{a_1,a_2,\dots,a_n\}$ and $f$ are chosen in such a way that 
\begin{clm}{}
\label{min-n} The number $n$ takes the minimal value for which \ref{eq:conta-a-b} can hold.
\end{clm}
\begin{clm}{}\label{max-l} If $x_1,\dots, x_n\in A$
and $y_i=f(x_i)$ then
$$\ell(y_1,\dots,y_n)-\ell(x_1,\dots,x_n)\le\ell(b_1,\dots, b_n)- \ell(a_1,\dots, a_n).$$

\end{clm}
\noi
To meet Condition~\ref{max-l}, 
one has to take instead of $a_1,\dots,a_n$ 
an $n$-point subset $x_1,\dots, x_n\in A$ 
for which the difference taking the maximal value.
It is possible since $A$ is compact.

Note that all $b_i$ are distinct vertices of $B$.
Indeed, assume $b_n$ lies inside or on side of $B$ or $b_n=b_i$ for some $i\not=n$.
Then
\begin{align*}
\ell(b_1,\dots,b_{n-1})&=\ell(b_1,\dots,b_{n-1},b_{n}),
\\
\ell(a_1,\dots,a_{n-1})&\le \ell(a_1,\dots,a_{n-1},a_{n}).
\end{align*}
This contradicts Condition~\ref{min-n}.

By $\mangle a_i$, we will denote the angle of $A$ at $a_i$, 
and by $\mangle b_i$ the angle of $B$ at $b_i$.
Let us show that
$$\mangle b_i\ge \mangle a_i\eqlbl{eq:a<b}$$
Move $a_i$ with unit speed inside $A$ along the bisector of the angle, 
then the value $\ell(a_1,a_2,\dots,a_{n})$ 
decrease with rate $2{\cdot}\cos\tfrac\alpha2$.
The point $b_i=f(a_i)$ will also move with unit speed;
it is not hard to see that at the value $\ell(b_1,\dots,b_{n})$ 
can not decrease faster than with  $2{\cdot}\cos\tfrac{\beta}2$.
By Condition~\ref{max-l}, 
the difference 
$$\ell(b_1,\dots,b_{n})-\ell(a_1,\dots,a_{n})$$
can not increase.
Therefore
$2{\cdot}\cos\tfrac{\mangle b_i}2
\ge
2{\cdot}\cos\tfrac{\mangle a_i}2;$
hence \ref{eq:a<b}.

Applying the theorem about sum of angles of $n$-gon to \ref{eq:a<b},
we get that all $a_i$ are vertices of $A$.
(We also get $\mangle b_i=\mangle a_i$, 
but we will not need it.)

We can assume that  $a_i$ are labeled in the cyclic order as they appear on the boundary of $A$.
In this case
$$\ell(a_1,\dots,a_{n})=|a_1-a_2|+\dots+|a_{n-1}-a_n|+|a_n-a_1|$$
Since  $|b_i-b_j|\le |a_i-a_j|$ for all $i$ and $j$,
we get
$$|b_1-b_2|+\dots+|b_n-b_1|\le|a_1-a_2|+\dots+|a_n-a_1|.$$

\begin{wrapfigure}{r}{45mm}
\begin{lpic}[t(-0mm),b(-0mm),r(0mm),l(0mm)]{pics/obkhvat-shorter(0.75)}
\lbl[tr]{6,17;$b_1$}
\lbl[b]{34,52;$b_2$}
\lbl[lt]{54,1;$b_3$}
\lbl[r]{1,35;$b_4$}
\lbl[t]{28,1;$b_5$}
\lbl[bl]{55,39;$b_6$}
\lbl[l]{59,17;$b_7$}
\end{lpic}
\caption*{red $\ge$ green $\ge$ black.}
\end{wrapfigure}

Finally, note that 
$$\ell(b_1,\dots,b_{n})\le|b_1-b_2|+\cdots+|b_n-b_1|.$$
In other words the $\ell$ of the vertices of a closed broken 
line can not exceed the total length of the broken line.
The later statement should be evident from the picture.

Therefore 
$$\ell(A_1',A_2',\dots,A_{n}')\le\ell(a_1,a_2,\dots,a_{n}),$$ 
a contradiction.

\parbf{Problem \ref{pr:alexander}.}
Here is an example of such curves.
$$
\alpha_i(t) = \left(\frac{a_i + b_i}{2} + 
\cos(\pi\cdot t)\cdot \frac{a_i - b_i}2,\  
\sin(\pi\cdot t)\cdot \frac{a_i - b_i}2\right). 
$$
It is straightforward to check that
$\ell_{i,j}$ are monotonic.

\refstepcounter{section}
\section*{Chapter~\ref{chap:flex}}

\parbf{Exercise \ref{ex:rigit/flex}.}

\parbf{Problem \ref{pr:trig-S2}.}

\parbf{Problem \ref{bricard-second}.}

\parbf{Problem \ref{pr:alexander-flex}.}
Apply the same construction as in the proof of Alexander's theorem (\ref{thm:alexander}).

\refstepcounter{section}
\section*{Chapter~\ref{chap:unique}}

\parbf{Exercise \ref{ex:poly+geod}.}

\parbf{Exercise \ref{ex:mnfl-combi}.}

\parbf{Exercise \ref{ex:sum=2pi}.} Let $k$, $l$ and $m$  be the number of vertices, edges, and triangles respectively in the triangulation $\mathcal{T}$. 
Use that sum of angles of in any triangle is $\pi$
together with Euler's formula 
$k-l+m=2$ and the identity $3\cdot m=2\cdot l$.

\parbf{Exercise \ref{ex:curv-is-nonneg}.} 
Let $K_p$ be the intersection of $K$ with a small sphere centered at $p$.
Note that $K_p$ is a convex spherical polygon.

Applying  the idea in the proof of Lemma~\ref{lem:perimeter},
we get that perimeter of $K_p$ is not bigger than 
perimeter of half-sphere.
Hence the statement follow.

\parbf{Problem \ref{pr:sum>1pi}.}

\parbf{Problem \ref{pr:aa+2}.}

\parbf{Problem \ref{pr:arm}.}

\parbf{Problem \ref{pr:tetrahedron}.}

\parbf{Problem \ref{pr:zalgaller}.}

\parbf{Problem \ref{pr:K-P-simmetry}.}


\refstepcounter{section}
\section*{Chapter~\ref{chap:exist}}

\parbf{Exercise \ref{ex:P,eps-delta}.}
Let us first show \ref{clm:surjective}.
Assume $P'\in\left] \mathbf{P}_k\right[$ 
and $|P'-P|_\mathcal{M}<\tfrac\delta2$,
let $f\:P'\to P$ be a $2\cdot\delta$-isometry;
it exists according to Exercise~\ref{ex:GH=>eps-isom}.
For $\eps$ and $\delta$ as in Exercise~\ref{ex:P,eps-delta},
we can label the vertices of $P'$ 
by $v_1'$, $v_2',\dots,v_k'$ so that 
$|f(v_i')-v_i|<\eps$.
In particular 
$$|v'_i-v'_j|_{P'}\lege|v_i-v_j|_{P}\pm 3\cdot\eps$$
for any $i$ and $j$.

Let us join a pair of vertices $(v_i',v_j')$ by a geodesic in $P'$
if the corresponding pair $(v_i,v_j)$ is connected by an edge 
in $\mathcal{T}$.

Notice that if $\delta$ is small enough
the constructed geodesics intersect only at the common end points. 
Indeed, assume contrary holds for arbitrary small $\eps>0$;
i.e., we have a sequence of polyhedral spaces 
$P_n\in\left]\mathbf{P}_k\right[$
such that $P_n\GHto P$ 
as $n\to\infty$.
Let $\alpha_n$ and $\beta_n$ be geodesics which at $p_n$ in $P_n$.
Let $p$ be a partial limit of $f_n(p_n)\in P$.
Note that $p$ lies on a geodesic between two pairs of vertices,
a contradiction. 


In follows that these geodesics appear as edges 
of a triangulation of $P'$.
It means that the constructed geodesics subdivide $P'$
in flat triangles;
so we obtain a triangulation $\mathcal{T}'$ 
of $P'$ 
with the same combinatorics as $\mathcal{T}$.


Note if $[P']\in\mathbf{P}_k$ 
is sufficiently close to $[P]$ then $P'$ can be triangulated by the same simplicial complex as $P$ in such a way that the lengths of all corresponding edges in $P$ and $P'$ are sufficiently close. 
???

Note that the length of the edges and triangulation completely describe the polyhedral space.
Thus we get a parametrization of a neighborhood of $[P]$ 
by an open set in $\RR^l$, where $l$ is the number of edges in the triangulation.
Recall that according to Euler's formula (\ref{thm:euler}), 
$l=3\cdot k-6$.
Hence the lemma follows.

Now assume $P$ is a realizable space.
Let $K$ be a convex polyhedron with the surface isometric to $P$.
By Lemmas ??? and ???, we can choose neighborhoods $W\ni \bar P$ in $\mathbf{P}_k$ 
and $V\ni \bar K$ in $\mathbf{K}_k$ which homeomorphic to open sets in $\RR^{2\cdot k-6}$.
Since $\Phi_k$ is continuous, we may assume that $\Phi_k(V)\subset W$.
Applying Domain invariance theorem, 
we get $\Phi_k(V)$ is open in $W$.
Hence $\Phi_k(V)$ contains a small ball around $\bar P$.
Hence the result.

\parbf{Problem \ref{}.}

\parbf{Problem \ref{}.}

\parbf{Problem \ref{}.}

\parbf{Problem \ref{}.}

\parbf{Problem \ref{}.}

\parbf{Problem \ref{}.}
\refstepcounter{section}
\section*{Chapter~\ref{chap:euclid}}

\parbf{Problem \ref{}.}

\parbf{Problem \ref{}.}

\parbf{Problem \ref{}.}

\parbf{Problem \ref{}.}

\parbf{Problem \ref{}.}

\parbf{Problem \ref{}.}
\refstepcounter{section}
\section*{Chapter~\ref{chap:cbb}}

\parbf{Problem \ref{}.}

\parbf{Problem \ref{}.}

\parbf{Problem \ref{}.}

\parbf{Problem \ref{}.}

\parbf{Problem \ref{}.}

\parbf{Problem \ref{}.}
\refstepcounter{section}
\section*{Chapter~\ref{chap:erdos}}

\parbf{Problem \ref{}.}

\parbf{Problem \ref{}.}

\parbf{Problem \ref{}.}

\parbf{Problem \ref{}.}

\parbf{Problem \ref{}.}

\parbf{Problem \ref{}.}
\refstepcounter{section}
\section*{Chapter~\ref{chap:cba}}

\parbf{Problem \ref{}.}

\parbf{Problem \ref{}.}

\parbf{Problem \ref{}.}

\parbf{Problem \ref{}.}

\parbf{Problem \ref{}.}

\parbf{Problem \ref{}.}

\refstepcounter{section}
\section*{Chapter~\ref{chap:collisions}}

\parbf{Problem \ref{}.}

\parbf{Problem \ref{}.}

\parbf{Problem \ref{}.}

\parbf{Problem \ref{}.}

\parbf{Problem \ref{}.}

\parbf{Problem \ref{}.}

\refstepcounter{section}
\section*{Chapter~\ref{chap:degressions}}