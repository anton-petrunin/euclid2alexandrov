\chapter*{Preface}
These lectures were a part of the geometry course held during the Fall 2011 Mathematics Advanced Study Semesters (MASS) Program at Penn State (\url{http://www.math.psu.edu/mass/}).

The lectures are meant to be accessible to advanced undergraduate and early graduate students in mathematics.  
We have placed a great emphasis on clarity and exposition, and we have included many exercises.  Hints and solutions for most of the exercises are provided in the end. 

The lectures discuss piecewise distance preserving maps from a 2-dimensional polyhedral space into the plane.  Roughly speaking, a polyhedral space is a space that is glued together out of triangles, for example the surface of a polyhedron.  If one imagines such a polyhedral space as a paper model, then a piecewise distance preserving map into the plane is essentially a way to fold the model so that it lays flat on a table. 
We have five lectures on the following topics:

\begin{itemize}
\item Zalgaller's folding theorem, which guarantees the existence of a piecewise distance preserving map from a 2-dimensional polyhedral space into the plane.  In other words, it is always possible to fold the paper model onto the table.
\item Brehm's extension theorem, which allows one to build piecewise distance preserving maps from a convex polygon into the plane with prescribed images on a finite subset of the polygon.
\item Akopyan's approximation theorem, which allows one to approximate  maps from a 2-dimensional polyhedral space into the plane by piecewise distance preserving maps.
\item Gromov's rumpling theorem, which shows the existence of a length-preserving map from the sphere into the plane; 
i.e., a map that preserves the lengths of all curves.
\item An entertaining problem of Arnold on paper folding, which asks if it is possible to fold a square in the plane so that the perimeter increases.
\end{itemize}

We only consider the 2-dimensional case to keep things easy to visualize.
However, 
most of the results admit generalizations to higher dimensions.
These results are discussed in the Final Remarks, where proper credit and references are given.

\parbf{Acknowledgments.} 
We would like to thank
Arseniy Akopyan, 
Robert Lang, 
Alexei Tarasov
for their help.
Also we would like to thank all the students in our class
for their participation and true interest.

\tableofcontents

\chapter*{Preliminaries}
\addcontentsline{toc}{chapter}{Preliminaries}
\addtocontents{toc}{\protect\begin{quote}}

This chapter serves as a quick review of the necessary background material.  It's primary purpose is to refresh the reader's memory and to familiarize the reader with our notation.  For a more in-depth account of this material, we recommend the first three chapters of the book
``Metric Geometry'' by Burago--Burago--Ivanov \cite{BBI}.


\section*{Metric spaces}
\addtocontents{toc}{Metric spaces.}

\begin{thm}{Definition}
\label{def:metric-space}
A metric space is a pair $(X,\Dist)$ where $X$ is a set and 
$$\Dist \: X \times X \rightarrow [0,\infty)$$
is a function satisfying

\begin{subthm}{def:metric-space:zero}
$\Dist(x,y) = 0$ if and only if $x=y$,
\end{subthm}

\begin{subthm}{}
$\Dist(x,y) = \Dist(y,x)$ for any $x, y \in X$,
\end{subthm}

\begin{subthm}{triangle-inq}
$\Dist(x,z)\le \Dist(x,y)+\Dist(y,z)$ for any $x, y, z \in X$.
\end{subthm}

\end{thm}

Condition (\ref{SHORT.triangle-inq}) is called the \emph{triangle inequality}\index{triangle inequality}.
An element $x \in X$ is called a \emph{point}\index{point} of the metric space $X$.
The function $\Dist\: X \times X \rightarrow [0,\infty)$ is called a \emph{metric}\index{metric}, 
and the non-negative value $\Dist(x,y)$ is called the \emph{distance}\index{distance} from $x$ to $y$.

Most of the time, it will be understood which metric we are using on a given set.
In such cases, we may refer to the \emph{metric space $X$}, instead of the \emph{metric space $(X,\Dist)$}.
We will often use the notation $|x-y|$ to denote\footnote{Be aware that this is just alternate notation for $\Dist(x,y)$, and $x - y$ itself has no meaning in a general metric space.} the distance $\Dist(x,y)$, and we shall occasionally write $|x-y|_X$ to emphasize that we are using the metric on the space $X$.
%Also, in this case, 
%the distance $\Dist(x,y)$ may be denoted simply as $|x-y|$ 
%or $|x-y|_X$, if we want to emphasize that the points $x$ and $y$ belong to the metric space $X$.  

An important example for us is $n$-dimensional Euclidean space $\RR^n$ with the standard metric
\[|x-y|_{\RR^n} = \sqrt{(x_1 - y_1)^2+\dots+(x_n - y_n)^2},\]
defined for the points $x = (x_1, \ldots , x_n)$ and $y = (y_1, \ldots , y_n)$ in $\RR^n$.

Any subset of a metric space is also a metric space, by restricting the original metric to the subset.  In this way, all subsets of Euclidean space, in particular convex polyhedra,
are metric spaces.
The unit sphere $\SS^n$, which is the set of all unit vectors in $\RR^{n+1}$, is also a metric space in this way.  
However, we shall be also interested in a different metric on $\SS^n$, the \emph{angle metric}\index{angle metric}.\label{angle-metric}
This metric can be formally defined as
\[ |x - y|_{\SS^n} = \arccos\langle x, y\rangle,\]  
where, $\langle x, y \rangle$ is the standard inner product of the two unit vectors $x, y \in \RR^{n+1}$.
The relationship between these two metrics on $\SS^n$ will be described below, in the discussion of induced length metrics.

\begin{thm}{Definition}
Let $X$ and $Y$ be metric spaces.
\begin{subthm}{}
A map $f\:X\to Y$ is \emph{distance non-expanding}\index{distance non-expanding map} if
$$|f(x)-f(x')|_Y\le |x-x'|_X$$
for any $x,x'\in X$.
\end{subthm}

\begin{subthm}{}
A map $f\:X\to Y$ is \emph{distance preserving}\index{distance preserving map} if
$$|f(x)-f(x')|_Y= |x-x'|_X$$
for any $x,x'\in X$.
\end{subthm}

\begin{subthm}{}
A distance preserving bijection $f\:X\to Y$ is called an \emph{isometry}\index{isometry}.
\end{subthm}

\begin{subthm}{}
The spaces $X$ and $Y$ are called \emph{isometric}\index{isometric spaces} (briefly $X\iso Y$)
 if there is an isometry  $f\:X\to Y$.
\end{subthm}

\end{thm}


\begin{thm}{Exercise}\label{ex:IsometriesOfR2}

\begin{subthm}{} \label{ex:IsometriesOfR2Uniqueness}
Prove that an isometry $\iota: \RR^2 \to \RR^2$ is uniquely determined by its effect on any three non-collinear points.
\end{subthm}

\begin{subthm}{}\label{ex:IsometriesOfR2Existence}
Show that if $x, y, z$ and  $x', y', z'$ are two collections of non-collinear points in $\RR^2$ satisfying
$$|x - y| = |x' - y'|, \quad |x - z| = |x' - z'|, \quad |y - z| = |y' - z'|,$$
then there is an isometry $\iota$ of $\RR^2$ such that
\[ \iota(x) = x', \quad \iota(y) = y', \quad \iota(z) = z'. \]  By part a), $\iota$ is unique.
\end{subthm}
\end{thm}

\section*{Calculus}
\addtocontents{toc}{Calculus.}

\begin{thm}{Definition}
 Let $X$ be a metric space.
A sequence of points $x_1, x_2, \ldots$ in $X$ is called \emph{convergent}\index{convergent}
if there is 
$x_\infty\in X$ such that $|x_\infty -x_n|\to 0$ as $n\to\infty$.  
That is, for every $\eps > 0$, there is a natural number $N$ such that for all $n \ge N$, we have $$|x_\infty-x_n| < \eps.$$

In this case we say that the sequence $(x_n)$ \emph{converges} to $x_\infty$, 
or $x_\infty$ is the \emph{limit} of the sequence $(x_n)$.
Notationally, we write $x_n\to x_\infty$ as $n\to\infty$
or $x_\infty=\lim_{n\to\infty} x_n$.
\end{thm}

\begin{thm}{Definition}
Let $X$ and $Y$ be metric spaces.
A map $f\:X\to Y$ is called continuous if for any convergent sequence $x_n\to x_\infty$ in $X$,
%the sequence $y_n\z=f(x_n)$ converges to $y_\infty=f(x_\infty)$ in $Y$.
we have $f(x_n) \to f(x_\infty)$ in $Y$.

Equivalently, $f\:X\to Y$ is continuous if for any $x\in X$ and any $\eps>0$,
there is $\delta>0$ such that 
$$|x-x'|_X<\delta\ \text{ implies }\ |f(x)-f(x')|_Y<\eps.$$
%$$|x-x'|_X<\delta\ \Rightarrow\ |f(x)-f(x')|_Y<\eps.$$

\end{thm}

\begin{thm}{Definition}
Let $X$ and $Y$ be metric spaces.
A continuous bijection $f\:X\to Y$ 
is called a \emph{homeomorphism}\index{homeomorphism} 
if its inverse $f^{-1}\:Y\z\to X$ is also continuous.

If there exists a homeomorphism $f\:X\to Y$,
we say that $X$ is \emph{homeomorphic}\index{homeomorphic} to $Y$,
or  $X$ and $Y$ are \emph{homeomorphic}.
\end{thm}
Notice that a distance non-expanding map is always continuous, and an isometry is an example of a homeomorphism.




\begin{thm}{Definition}
A subset $A$ of a metric space $X$ is called \emph{closed}\index{closed set} if whenever a sequence $(x_n)$ of points from $A$ converges in $X$, we have that $\lim_{n\to\infty} x_n \in A$.

A set $\Omega \subset X$ is called \emph{open}\index{open set} if the complement $X \setminus \Omega$ is a closed set.
Equivalently, $\Omega \subset X$ is open if for any $z\in \Omega$, 
there is $\eps>0$ such that if $|x-z|<\eps$, then $x \in \Omega$.
\end{thm}

Note that the intersection of an arbitrary family of closed sets is closed.
It follows that for any set $Q$ in a metric space $X$, there is a minimal closed set which contains $Q$.
This set is called the \emph{closure of $Q$}\index{closure} and is denoted $\Closure Q$.
The closure of $Q$ can be obtained as the intersection of all closed sets $A$ containing $Q$.\footnote{Notice that the whole space $X$ is a closed subset, hence there is at least one closed set which contains $Q$.}
The closure of $Q$ is also equal to the set of all limits of all sequences in $Q$.


Similarly, the union of an arbitrary family of open sets is open.
It follows that for any set $Q$ in a metric space, there is a maximal open set which is contained in $Q$.
This set is called the \emph{interior of $Q$}\index{interior} and is denoted as $\Interior Q$.
The interior of $Q$ can be obtained as the union of all open sets contained in $Q$.

The set-theoretic difference
$$\partial Q=\Closure Q\backslash \Interior Q$$
is called the \emph{boundary}\index{boundary} of $Q$.
Since the boundary of $Q$ depends on the space in which it is embedded, we may use the notation $\partial_X Q$ if we want to emphasize that $Q$ is a subset of the metric space $X$.
Notice that $\partial Q$ is a closed set.
One can show that a point $p$ is in $\partial Q$ if and only if for any $\eps>0$, there are points $q\in Q$ and $q'\notin Q$ such that $|p-q| < \eps$ and $|p-q'|<\eps$.

\section*{Curves}
\addtocontents{toc}{Curves.}

A \emph{real interval}\index{real interval} is a convex subset of $\RR$ which contains more than one point.
Examples include $(0,1)$, $(-\infty, 0]$, and $\RR$.

\begin{thm}{Definition}\label{def:curve}
A \emph{curve}\index{curve} is a continuous mapping $\alpha\:\II\to X$,
where $\II$ is a real interval and $X$ is a metric space. 

If $\II=[a,b]$ and $$\alpha(a)=p,\ \ \alpha(b)=q,$$
we say that $\alpha$ is a \emph{curve from $p$ to $q$}\index{curve from $p$ to $q$}.
\end{thm}

\begin{thm}{Definition}\label{def:length}
Let $\alpha\:\II\to X$ be a curve. Define the \emph{length}\index{length of curve} of $\alpha$ to be
\begin{align*}
\length \alpha
&= 
\sup \{|\alpha(t_0)-\alpha(t_1)|+|\alpha(t_1)-\alpha(t_2)|+\dots
\\
&\ \ \ \ \ \ \ \ \ \ \ \ \ \ \ \ \ \ \ \ \ \ \ \ \ \ \ \ \ \ \ \ \dots+|\alpha(t_{k-1})-\alpha(t_k)|\}, 
\end{align*}
where the supremum is taken over all positive integers $k$ and all sequences $t_0 < t_1 < \cdots < t_k$ in $\II$.

A curve is called \emph{rectifiable}\index{rectifiable curve} if its length is finite.
\end{thm}


\begin{thm}{Semicontinuity of length}\label{thm:length-semicont}
Length is a lower semi-continuous functional on the space of curves
$\alpha\:\II\to X$ with respect to point-wise convergence. 

In other words: assume that a sequence
of curves $\alpha_n\:\II\to X$ converges point-wise 
to a curve $\alpha_\infty\:\II\to X$;
i.e., for any fixed $t\in\II$, we have $\alpha_n(t)\to\alpha_\infty(t)$ as $n\to\infty$. 
Then 
$$\liminf_{n\to\infty} \length\alpha_n \ge \length\alpha_\infty.\eqlbl{eq:semicont-length}$$

\end{thm}

See \cite[Proposition 2.3.4.]{BBI} for a proof.

\begin{wrapfigure}{r}{32mm}
\begin{lpic}[t(-8mm),b(-3mm),r(0mm),l(0mm)]{pics/stairs(0.5)}
%\lbl[lb]{162,180;$\alpha_0$}
%\lbl[lb]{82,150;{\small $\alpha_1$}}
%\lbl[lb]{42,130;{\tiny $\alpha_2$}}
\end{lpic}
\end{wrapfigure}


Note that the inequality \ref{eq:semicont-length} might be strict.
For example, the diagonal $\alpha_\infty$ of unit square (gray in the picture)
can be  approximated by a sequence of stairs-like
polygonal curves $\alpha_n$
with sides parallel to the sides of the square (for example, $\alpha_6$ is black the picture.)
In this case
$\length\alpha_\infty=\sqrt{2}$
and $\length\alpha_n=2$ for all $n$.

%Note that applying triangle inequality to the definition of length, 
%we get that if $\gamma$ is a curve from $x$ to $y$ in a metric space $X$ 
%then
%$$\length \gamma\ge |x-y|_X.$$

By taking $t_0$ and $t_1$ to be the endpoints of $\II$ in the definition of length, it follows that
$$\length \alpha\ge |x-y|_X$$ whenever $\alpha$ is a curve in $X$ from $x$ to $y$.

\begin{thm}{Definition}\label{def:length-space}
A metric space $X$ is called a \emph{length space}\index{length space} if for any two points $x,y\in X$ and any $\eps>0$, there is a curve $\alpha$ from $x$ to $y$ such that
$$\length \alpha<|x-y|_X+\eps.$$

\end{thm}

Let $X$ be a metric space. 
Consider the function $\hat d\:X\z\times X\to\RR \cup \{\infty\}$ defined by
$$\hat d(x,y)\df\inf_\alpha\{\length\alpha\}$$
where the infimum is taken along all the curves $\alpha$ from $x$ to $y$
(if there is no such curve, then $\hat d(x,y)=\infty$.)

It is straightforward to see that $\hat d\:X\times X\to\RR$ satisfies all conditions of a metric, provided $\hat d(x,y)<\infty$ for all $x,y\in X$.
In this case, the metric $\hat d$ will be called the \emph{induced length metric}\index{induced length metric} of the metric $|{*}-{*}|_X$.  By construction, $(X, \hat d)$ is a length space.

For example, the angle metric on $\SS^n$ discussed on page \pageref{angle-metric}, 
is the induced length metric of the restriction of the Euclidean metric on $\RR^{n+1}$ to $\SS^n$.


\begin{thm}{Definition}\label{def:length-preserving}
A continuous map $f\:X\z\to Y$ between two metric spaces
is called \emph{length-preserving}\index{length-preserving map} if for any curve $\alpha\:\II\to X$,
we have 
$$\length\alpha=\length(f\circ\alpha).$$

\end{thm}

Note that since $f$ is continuous, the composition $f\circ\alpha$ is continuous, hence a curve in $Y$. %(see Definition~\ref{def:curve});
Therefore the above definition makes sense.


\begin{thm}{Exercise}\label{LP=>short}
Let $X$ and $Y$ be length spaces.
Show that 

\begin{enumerate}[a)]
\item\label{LP=>short:a} Any length-preserving map $f: X\to Y$
is also distance non-expanding.
\item\label{LP=>short:b} A distance non-expanding $f: X\to Y$ is length-preserving if 
for any two points $p$ and $q$ in $X$
and any curve $\alpha$ from $p$ to $q$, we have 
$$\length(f\circ\alpha)\ge |p-q|.$$
\end{enumerate}
\end{thm}

\begin{thm}{Definition}\label{def:geodesic}
 A curve $\alpha\:\II\to X$ is called a \emph{geodesic}\index{geodesic}%
\footnote{Formally our ``geodesic'' should be called ``unit-speed minimizing geodesic'',
and the term ``geodesic'' is reserved for curves which \emph{locally} satisfy the identity
$$|\alpha(t_0)-\alpha(t_1)|_X=\Const\cdot|t_0-t_1|$$
for some $\Const\ge 0$.}
 if it is a distance preserving map;
i.e., if 
$$|\alpha(t_0)-\alpha(t_1)|_X=|t_0-t_1|$$
for any $t_0,t_1\in\II$.
\end{thm}

If $\alpha$ is a geodesic from $p$ to $q$ then the image $\alpha(\II)$
will be also denoted as $[p,q]$.
Once we write $[p,q]$, we mean that there is at least one geodesic from $p$ to $q$ and we made a choice of one of them.

\section*{Polyhedral spaces}
\addtocontents{toc}{Polyhedral spaces.}

A subset $C \subseteq \RR^n$ is \emph{convex} if for any two points $x, y \in C$, the line segment connecting $x$ and $y$ lies entirely in $C$.
In other words,
\[(1-t)\cdot x + t\cdot y\in C\]
for any $t\in[0,1]$.

Given a subset $V\subset\RR^n$, 
the intersection of all convex sets containing $V$ is called 
the \emph{convex hull}\index{convex hull} of $V$,
and will be denoted by $\Conv V$.

The convex hull of a finite subset of $\RR^n$ is called a \emph{convex polyhedron}\index{convex polyhedron}.  A convex polyhedron is a metric space under the metric it inherits as a subset of $\RR^n$.

\parbf{Simplices.}
Assume $V=\{v_0,\dots,v_m\}$ is a finite subset of $\RR^n$
such that the $m$ vectors 
$$v_1-v_0,\  v_2-v_0,\ \dots,\ v_m-v_0$$ 
are linearly independent.
Then the convex hull $\Delta^m=\Conv V$
is called an \emph{$m$-dimensional (Euclidean) simplex}\index{simplex}.

So,
a 0-dimensional simplex is a one-point set; 
a 1-dimensional simplex is a line segment;
a 2-dimensional simplex is a triangle;
a 3-dimensional simplex is a tetrahedron.

If $\Delta^m$ is as above,
then the convex hull of any $(k+1)$-point subset of $\{v_0,\dots,v_m\}$ is a $k$-dimensional simplex, which will be called a \emph{face}\index{face} of $\Delta^m$.


\parbf{Barycentric coordinates.}
Let $\Delta^m=\Conv\{v_0,\dots,v_m\}$ be an $m$-dimensional simplex.
Note that $x\in \Delta^m$ if and only if 
$$x=\lambda_0\cdot v_0+\dots+\lambda_m\cdot v_m$$
for some (necessary unique) array of non-negative real numbers $\lambda_0,\z\dots,\lambda_m$ such that
\[\lambda_0+\dots+\lambda_m=1.\]
In this case, the real array $(\lambda_0,\dots,\lambda_m)$ will be called the \emph{barycentric coordinates}\index{barycentric coordinates} of the point $x$.

\parbf{Simplicial complexes.}
A \emph{simplicial complex}\index{simplicial complex} is defined as a finite collection $\mathcal{K}$
of simplices in $\RR^n$ that satisfies the following conditions:
\begin{itemize}
\item Any face of a simplex from $\mathcal{K}$ is also in $\mathcal{K}$.
\item The intersection of any two simplices $\Delta_1$ and $\Delta_2\in \mathcal{K}$ is either the empty set or it 
is a face of both $\Delta_1$ and $\Delta_2$.
\end{itemize}

The dimension  of simplicial complex $\mathcal{K}$
(briefly $\dim \mathcal{K}$)
is defined as the maximal dimension of all of its simplices.

For example, a $1$-dimensional simplicial complex,
also called a \emph{graph},
is a finite collection of points 
with a collection of non-crossing edges connecting some of these points.  
An example of a $2$-dimensional simplicial complex is the surface of a tetrahedron in $\RR^3$, or more general the surface of any polyhedron in $\RR^3$.  
Two disjoint $2$-simplices with a single segment connecting a vertex from each simplex gives an other example of a $2$-dimensional simplicial complex.
Note that in the last example 
not every simplex forms a face in a simplex of maximal dimension.

We say that a point $x$ belongs to simplicial complex $\mathcal{K}$ if it belongs to one of its simplices.
The set of all points of  $\mathcal{K}$ is called \emph{underlying set} of $\mathcal{K}$,
which will be denoted by $|\mathcal{K}|$.  Since $|\mathcal{K}| \subseteq \RR^n$, $|\mathcal{K}|$ is naturally a metric space.

A metric space $X$ is called a \emph{topological polytope}
if there is a simplicial complex $\mathcal{K}$
and a homeomorphism
$f\:|\mathcal{K}|\to X$.
In this case the complex $\mathcal{K}$ and the homeomorphism are called a
\emph{triangulation}\index{triangulation}\footnote{The term is a bit misleading, as a triangulation may contain simplices of dimension larger than $2$.} 
of $X$.
The images of simplicies of $\mathcal{K}$ in $X$ will be called the simplices of the triangulation\footnote{Since the image of straight lines under a homeomorphism are curves, the simplices in a topological polytope may not ``look like" Euclidean simplices.}.

\begin{thm}{Definition}\label{def:poly}
A length space $P$ is called a \emph{polyhedral space}\index{polyhedral space}
if it admits a triangulation such that each simplex in $P$ is isometric to a simplex in Euclidean space.
\end{thm}

When we refer to a triangulation of a polyhedral space in the future, we will always mean a triangulation as in the above definition.

To construct an example of polyhedral space,
one may take the underlying set of any simplicial complex in $\RR^n$
and equip it with the induced length metric.%
\footnote{In fact any polyhedral space is  isometric to one of these examples; see Exercise~\ref{ex:zalgalle+embedding}.}

Given a triangulation of a polyhedral space, 
we can consider the associated \emph{barycentric coordinates}\index{barycentric coordinates}.
If $\{v_1,\dots,v_n\}$ is the set of all vertices of the triangulation,
then any point $x$ is described uniquely by $n$ numbers $\lambda_1,\dots,\lambda_n$
such that $\lambda_i\ge 0$ for all $i$,
$\lambda_1+\dots+\lambda_n=1$,
and there is a simplex $\Delta$ in the triangulation such that
$\lambda_i\ne 0$ if and only if $v_i$ is a vertex of $\Delta$.
Indeed, assume $\Delta$ be the minimal simplex which contains $x$.
Since $\Delta$ is isometric to a Euclidean simplex,
we can use barycentric coordinates described above and take $\lambda_i=0$
if $v_i$ is not a vertex of $\Delta$.

The \emph{dimension of polyhedral space}\index{dimension of polyhedral space} is defined as the maximum dimension of the simplices in its triangulation. (One can show that this value does not depend on the choice of triangulation.)

\begin{thm}{Exercise}
Prove that a convex polygon $A$ in $\RR^2$ with the subspace metric is a 2-dimensional polyhedral space.  Moreover, show that for any triangulation of $A$ as in the definition of polyhedral space, the simplices of $A$ are Euclidean simplices.
\end{thm}

Two other important examples of polyhedral spaces for us will be
a convex polyhedron in $\RR^3$,
as well as the boundary of a convex polyhedron equipped with its induced length metric.
In order to prove that these spaces are indeed polyhedral, 
one might triangulate these spaces by hands. 
Along the same lines one can also prove the following
 characterization of polyhedral spaces:

\begin{thm}{Theorem}
A length space $P$ is a polyhedral space if it can be covered by finite number of subsets $M_1,\dots, M_n$
such that each $M_i$, as well as every intersection of a subcollection of $M_i$'s, is isometric to a convex polyhedron.  
\end{thm}


%\parbf{Piecewise ``awesome'' maps.}
%We will use terms ``piecewise distance preserving map'', 
%``piecewise linear map'' and so on. 
%We will define it separately each time but here we need to explain the general usage of word piecewise on one example.
%
%Assume $f$ is a map defined on polyhedral space $P$.
%We say that $f$ ``piecewise awesome''
%if there is a triangulation of $P$ such that the restriction of $f$ to each simplex in the triangulation is  ``awesome''.



\addtocontents{toc}{\protect\end{quote}}
