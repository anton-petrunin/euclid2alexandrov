\chapter{Metric spaces}\label{chap:metr}
\section{Definitions}

\begin{thm}{Definition}
\label{def:metric-space}
A metric space is a pair $(X,\Dist)$ where $X$ is a set and 
$$\Dist \: X \times X \rightarrow [0,\infty)$$
is a function such that 

\begin{subthm}{def:metric-space:zero}
$\Dist(x,y) = 0$ if and only if $x=y$;
\end{subthm}

\begin{subthm}{}
$\Dist(x,y) = \Dist(y,x)$ for any $x, y \in X$;
\end{subthm}

\begin{subthm}{triangle-inq}
$\Dist(x,z)\le \Dist(x,y)+\Dist(y,z)$ for any $x, y, z \in X$.
\end{subthm}

\end{thm}

The condition (\ref{SHORT.triangle-inq}) above is called \emph{triangle inequality}.
The set $X$ is called \emph{underlying set}\index{underlying set} of the metric space
and its elements are called \emph{points} of the metric space.
The function $\Dist\: X \times X \rightarrow [0,\infty)$ is called a \emph{metric}\index{metric}, 
the value $\Dist(x,y)\ge 0$ is called the \emph{distance}\index{distance} from $x$ to $y$.



\parbf{Examples:}
\begin{itemize}
\item Discrete metric. For any set $X$, the discrete metric on $X$ is defined by 
$$\Dist(x,y)=\left[
\begin{aligned}
&1&&\text{if}&x&\not=y
\\
&0&&\text{if}&x&=y
\end{aligned}
\right.$$

\item Euclidean space. 
The set is formed by arrays of $n$ real numbers
$$\bm{x}=(x_1,x_2,\dots,x_n)\in\RR^n$$
and the distance function defined as 
$$\Dist(\bm{x},\bm{y})
\df
\lVert\bm{x}-\bm{y}\rVert_2,$$
where 
$$\lVert\bm{x}\rVert_2
\df
\sqrt{x^2_1+x^2_2+\dots+x_n^2}.
$$
\end{itemize}

\begin{thm}{Exercise}\label{ex:Euclidean-is-metric}
Show that Euclidean space is a metric space.
\end{thm}

\begin{itemize}
\item \label{manhattan-metric} Manhattan metric on the plane. The set is formed by all pairs $\bm{x}\z=(x_1,x_2)\in\RR^2$ and the metric is defined as 
$$\Dist(\bm{x},\bm{y})
\df
\lVert\bm{x}-\bm{y}\rVert_1,$$
where 
$$\lVert\bm{x}\rVert_1
\df
|x_1|+|x_2|.
$$
\item Space of functions with sup-norm.
Given a set $X$,
consider the set $\mathcal{F}(X)$ of all bounded functions 
$f\: X\to\RR$ with the metric defined as
$$\Dist(f,g)
\df
\lVert f-g\rVert_\infty,$$
where
$$\lVert f\rVert_\infty
\df 
\sup_{x\in X}|f(x)|.
$$
\item Subspaces. Given an arbitrary subset $A\subset X$ of a metric space $(X,\Dist)$,
one can give $A$ the metric defined as the restriction of $\Dist$ to $A\times A\z\subset X\times X$. 
In this situation, $(A,\Dist|_{A\times A})$ is called a \emph{subspace}\index{subspace} of $(X,\Dist)$.
\end{itemize}

\subsection*{Notation for distance}

The distance $\Dist(x,y)$ between $x$ and $y$ in a metric space $X$ will be further also denoted as 
$$|x-y|=|x-y|_X=\Dist_x y=\Dist_y x=\Dist(x,y).$$
We will write $|x-y|_X$ to emphasize that the points $x$ and $y$ belong to the metric space $X$
and the notation $\Dist_x$ is used when we need to consider the distance to the point $x$ as a function $\Dist_x\:X\to\RR$.

It should be noted that the expression $x-y$ for two points in a metric space makes no sense
and $|x-y|$ should be read as \emph{distance from $x$ to $y$}\index{distance}.

\subsection*{Isometries}

\begin{thm}{Definition}
Let $X$ and $Y$ be metric spaces.
\begin{subthm}{}
A map $f\:X\to Y$ is \emph{distance preserving}\index{distance preserving map} if
$$|f(x)-f(x')|_Y=|x-x'|_X$$
for any $x,x'\in X$.
\end{subthm}

\begin{subthm}{}
A distance preserving bijection $f\:X\to Y$ is called an \emph{isometry}\index{isometry}.
\end{subthm}

\begin{subthm}{}
The spaces $X$ and $Y$ are called \emph{isometric}\index{isometric spaces} (briefly $X\iso Y$)
 if there is an isometry  $f\:X\to Y$.
\end{subthm}

\end{thm}

Note that 
\begin{itemize}
\item ``$\iso$'' is an equivalence relation on the class of metric spaces.
\item Distance preserving map is necessarily injective.
\item The existence of a distance preserving map $X\to Y$ is equivalent to the existence of subset of $Y$ which is isometric to $X$.
\end{itemize}

\section{Calculus in metric spaces}

\subsection*{Convergence and continuity}

\begin{thm}{Definition}
 Let $X$ be a metric space.
A sequence of points $x_1, x_2, \ldots$ in $X$ is called \emph{convergent}\index{convergent}
if there is 
$x_\infty\in X$ such that $|x_\infty -x_n|\to 0$ as $n\to\infty$.  
That is, for every $\eps > 0$, there is a natural number $N$ such that for all $n \geq N$, we have $$|x_\infty-x_n| < \eps.$$

In this case we say that the sequence $(x_n)$ converges to $x_\infty$ 
or $x_\infty$ is the limit of the sequence $(x_n)$
and write 
$x_\infty=\lim_{n\to\infty} x_n$
or $x_n\to x_\infty$ as $n\to\infty$.
\end{thm}

Note that any converging sequence has unique limit.

\begin{thm}{Definition}
Let $X$ and $Y$ be metric spaces.
A map $f\:X\to Y$ is called continuous if for any convergent sequence $x_n\to x_\infty$ in $X$,
the sequence $y_n\z=f(x_n)$ converges to $y_\infty=f(x_\infty)$ in $Y$.

Equivalently, $f\:X\to Y$ is continuous if for any $x\in X$ and any $\eps>0$
there is $\delta>0$ such that 
$$|x-x'|_X<\delta\ \Rightarrow\ |f(x)-f(x')|_Y<\eps.$$

\end{thm}

It is not hard to see that two definitions of continuity given above are equivalent;
try to prove it.

\subsection*{Open and closed sets}

\begin{thm}{Definition}
A subset $A$ of a metric space $X$ is called \emph{closed}\index{closed set} if whenever a sequence $(x_n)$ of points from $A$ converges in $X$, we have that $\lim_{n\to\infty} x_n \in A$.

A set $\Omega \subset X$ is called \emph{open}\index{open set} if the complement $X \setminus \Omega$ is a closed set.
Equivalently, $\Omega \subset X$ is open if for any $z\in \Omega$ 
there is $\eps>0$ such that the \emph{ball}\index{ball}
$$B_\eps(z)=\set{x\in X}{|x-z|<\eps}$$
is contained in $\Omega$.
\end{thm}


The proof of equivalence of the two definitions of an open set given above is left to the reader.

Note that whole space as well as the empty set are both open and closed at the same time.
Also, the half-closed interval $[0,1)$ is neither open nor closed as a subset of the real line.


\subsection*{Closure, interior and boundary}

Note that the intersection of an arbitrary number of closed sets is closed.
It follows that for any set $Q$ in a metric space $X$, there is a minimal closed set which contains $Q$;
this set is called the \emph{closure of $Q$}\index{closure} and is denoted as $\Closure Q$.
The closure of $Q$ can be obtained as the intersection of all closed sets $A\supset Q$.
It is also equal to the set of all limit points for all sequences in $Q$.


Similarly, the union of an arbitrary number of open sets is open.
It follows that for any set $Q$ in a metric space, there is a maximal open set which is contained in $Q$;
this set is called the \emph{interior of $Q$}\index{interior} and is denoted as $\Interior Q$.
The interior of $Q$ can be obtained as the union of all open sets in $Q$.
It is also equal to the set of all point $p\in Q$ such that $B_\eps(p)\subset Q$ for some $\eps>0$.

Further the set-theoretic difference
$$\Closure Q\backslash \Interior Q$$
is called the \emph{boundary}\index{boundary} of $Q$ and is denoted as $\partial Q$ or $\partial_X Q$ if we want to emphasise that $Q$ is a subset of $X$.
Clearly, $\partial Q$ is a closed set.
A point $p$ lies in the boundary of $Q$ if and only if for any $\eps>0$, there are points $q\in Q$ and $q'\notin Q$ such that $|p-q|,|p-q'|<\eps$.

\section{Completeness}

\begin{thm}{Definition}
Let $X$ be a metric space.
A sequence of points $x_1, x_2, x_3, \ldots$ in $X$ is called \emph{Cauchy}\index{Cauchy sequence}, 
if for every $\eps>0$, 
there is an integer $N$ such that 
$$|x_m - x_n| < \varepsilon,$$
for any $m, n > N$.
\end{thm}

\begin{thm}{Definition}
A metric space $X$ is called \emph{complete}\index{complete space} if any Cauchy sequence in $X$ is convergent.
\end{thm}

\parbf{Examples:}
\begin{itemize}
\item The real line as well as Euclidean space are complete.
\item The subspace of real line formed by the rational numbers is not complete.
\end{itemize}

\begin{thm}{Exercise}\label{ex:close-complete}
Let $X$ be a complete metric space and $A\subset X$,
then the subspace formed by $A$ is complete if and only if $A$ is closed.
\end{thm}



\subsection*{Completion}

For any metric space $X$, 
there is a canonical construction of a complete metric space $\bar X$, 
which contains $X$.
The space $\bar X$ is called the \emph{completion}\index{completion} of $X$ and it is constructed as 
a set of equivalence classes of Cauchy sequences in $X$. 

For any two Cauchy sequences $\bm{x}=(x_n)$ and $\bm{y}=(y_n)$ in $X$, 
we may define their distance as
$$|\bm{x}-\bm{y}| = \lim_{n\to\infty} |x_n-y_n|.$$
This limit exists because the space of real numbers is complete. 

The defined function $|\bm{x}-\bm{y}|$ satisfies all axioms of metric but \ref{def:metric-space:zero};
i.e., two different Cauchy sequences may have the distance $0$ 
(the functions of that type are called \emph{pseudometrics}\index{pseudometric}). 
But ``having distance 0'' is an equivalence relation on the set of all Cauchy sequences, and the set of equivalence classes, which we denote $\bar X$, is a metric space called the \emph{completion} of $X$.  

The original space is embedded in this space via the identification of an element $x$ of $X$ with the equivalence class of the sequence with constant value $x$.  
This defines a distance preserving map $X\to \bar X$, as required. 

It is staightforward to verify all claims made about $\bar X$ and prove that the distance function given is well-defined and is indeed a metric on $\bar X$.


\section{Compactness}

\begin{thm}{Definition}
A metric space $X$ is \emph{compact}\index{compact space} if any sequence of points in $X$ contains a convergent subsequence.

If a subset $A$ in a metric space forms a compact subspace then we say that $A$ is a \emph{compact subset}\index{compact subset} of $X$.
\end{thm}


The proofs of the following properties are left to the reader;
these proofs can be also found in many books,
for example in \cite{rudin}
(which is an excellent book).

\parbf{Properties:} %??? remove those we do not use???
\begin{itemize}
\item (Heine--Borel theorem.) A subset of Euclidean space is compact if and only if it is both closed and bounded.
\item Any closed subset of a compact space is compact.
\item\label{compact=>closed} Any compact subset of a metric space is closed.
\item Any compact metric space is complete.
\item The Cartesian product of two compact spaces $X\times Y$ equipped with the metric 
$$|(x_0,y_0)-(x_1,y_1)|\df \max\{|x_0-x_1|,|y_0-y_1|\}$$
is compact.
\item \label{ex:compact-image}
If $f\: X \to Y$ is continuous map between metric spaces and $X$ is compact, then the image $f(X)$ is a compact subset of $Y$.
\item\label{ex:compact-homeo}
If $f\: X \to Y$ is a continuous bijection between metric spaces and $X$ is compact, then the inverse map $f^{-1}\: Y \to X$ is continuous.
\item\label{ex:EVT}
(Extreme Value Theorem) Prove that if $X$ is a compact metric space, then any continuous function $f\: X \to \mathbb{R}$ attains a global maximum value at some point of $X$.  That is, there exists $x \in X$ such that $f(y) \leq f(x)$ for all $y \in X$.
(Here, it is understood that the metric we are considering on $\mathbb{R}$ is the usual Euclidean metric.)
\end{itemize}

\begin{thm}{Definition}
A subset $A$ of a metric space $X$ is called \emph{bounded}\index{bounded set} if for one (and therefore any) point $x$ there is a constant $D<\infty$ such that 
$$|x-a|_X\le D$$ for any $a\in A$.
\end{thm}


\begin{thm}{Definition}\label{def:proper}
A metric space $X$ is called \emph{proper}\index{proper space} if any bounded, 
closed set in $X$ is compact.
\end{thm}



\subsection*{Nets and packings}

\begin{thm}{Definition}
Let $X$ be a metric space and let $\eps>0$.
A set $A\subset X$ is called an \emph{$\eps$-net of $X$}\index{net} if for any point $x\in X$ there is a point $a\in A$
such that $|x-a|\le\eps$.
\end{thm}

\begin{thm}{Definition}
Let $X$ be a metric space and let $\eps>0$.
A set of points $x_1,x_2,\dots,x_n\in X$ such that $|x_i-x_j|>\eps$ for all $i\not=j$ is called an \emph{$\eps$-packing}\index{packing}.

The supremum of all integers $n$ for which 
there is a $\eps$-packing $x_1,x_2,\dots,x_n\z\in X$ 
is denoted as $\pack_\eps X$.
($\pack_\eps$ takes integer value or $\infty$.)

If $n=\pack_\eps X$ is finite,
then an $\eps$-packing $x_1,x_2,\dots,x_n\in X$ 
is called a \emph{maximal $\eps$-packing}\index{maximal $\eps$-packing}.
\end{thm}

\begin{thm}{Exercise}\label{ex:packing=net}
Any maximal $\eps$-packing is an $\eps$-net.
\end{thm}

\begin{thm}{Exercise}\label{ex:net=2packing}
Let $\{a_1,a_2,\dots,a_n\}$ be a finite $\eps$-net in a metric space $X$.
Show that $\pack_{2\cdot\eps}X\le n$.
\end{thm}

\begin{thm}{Theorem}\label{thm:finite_pack=compact}
Let $X$ be a complete metric space.
Then the following conditions are equivalent:

\begin{subthm}{thm:finite_pack=compact:compact}
$X$ is compact;
\end{subthm}

\begin{subthm}{thm:finite_pack=compact:pack}
$\pack_\eps X$ is finite for any $\eps>0$;
\end{subthm}

\begin{subthm}{thm:finite_pack=compact:net}
$X$ is \emph{totally bounded}\index{totally bounded space};
i.e., for any $\eps>0$ there is a finite $\eps$-net in $X$.
\end{subthm}

\end{thm}

\parit{Proof; (\ref{SHORT.thm:finite_pack=compact:compact})$\Rightarrow$(\ref{SHORT.thm:finite_pack=compact:net}).}
Assume $X$ has no finite $\eps$-net.
Then for any point array $z_1,z_2,\dots,z_{n-1}$ in $X$
there is a $z_n\in X$
such that $|z_i-z_n|>\eps$ for any $i<n$.

Applying the above statement inductively, we can construct an infinite sequence $(z_n)$ in $X$
such that $|z_i-z_j|>\eps$ for all $i\not=j$.
Therefore $(z_n)$ has no convergent subsequence,
a contradiction.

\parit{(\ref{SHORT.thm:finite_pack=compact:net})$\Leftrightarrow$(\ref{SHORT.thm:finite_pack=compact:pack}).}
Follows from exercises \ref{ex:net=2packing} and \ref{ex:packing=net}.

\parit{(\ref{SHORT.thm:finite_pack=compact:net})$\Rightarrow$(\ref{SHORT.thm:finite_pack=compact:compact}).}
Set $\eps_k=\frac1{2^k}$.
For each $k$, 
let $\{z_{1,k},z_{2,k},\dots,z_{n_k,k}\}$ be an $\eps_k$-net.
Note that for each $k$, the collection of balls
$$\bar B_{\eps_k}(z_{i,k})=\set{x\in X}{|x-z_i|\le \eps_k}$$
cover all of $X$.

To show $X$ is compact, given an infinite sequence $x_1,x_2,\dots$ in $X$, we must find a convergent subsequence.
We shall apply a diagonal process to choose a subsequence
$x_{n_1},x_{n_2},\dots$  with the following property:
\textit{for each $k$ there is $i_k$ such that
$x_{n_m}\z\in\bar  B_{\eps_k}(z_{i_k,k})$ for all $m\ge k$}.

For the first step, note that since the finite collection of balls $\bar B_{\eps_1}(z_{i,1})$ cover $X$, there must be an index $i_1$ such that $\bar B_{\eps_1}(z_{i_1,1})$ contains infinitely many terms of our sequence $(x_n)$.  Let $F_1 = \bar B_{\eps_1}(z_{i_1,1})$. Choose an integer $n_1$ such that $x_{n_1} \in F_1$.

For the second step, since the balls $\bar B_{\eps_2}(z_{i,2})$ cover $X$, in particular they cover $F_1$.  Since $F_1$ contains infinitely many points of our sequence $(x_n)$, there must be an index $i_2$ such that $F_1 \cap \bar B_{\eps_2}(z_{i_2,2})$ also contains infinitely many of the $x_n$.  Let $F_2 = F_1 \cap \bar B_{\eps_2}(z_{i_2,2})$ and choose $n_2 > n_1$ such that $x_{n_2} \in F_2$.

Proceeding inductively at the $k$-th step, we know that the balls $\bar B_{\eps_k}(z_{i,k})$ cover $X$ and hence $F_{k-1}$, which contains infinitely many terms of our sequence $(x_n)$.  So there must exist $i_k$ so that $F_{k-1} \cap \bar B_{\eps_k}(z_{i_k,k})$ also contains infinitely many of the $x_n$.  Let $F_k = F_{k-1} \cap \bar B_{\eps_k}(z_{i_k,k})$ and choose $n_k > n_{k-1}$ with $x_{n_k} \in F_k$.

Note that $F_{k+1} \subseteq F_k$, so that for all $m \geq k$, $x_{n_m} \in F_k \subseteq B_{\eps_k}(z_{i_k,k})$ as desired.  It follows that $(x_{n_k})$ is a Cauchy sequence.
Since $X$ is complete, 
$(x_{n_k})$ converges, and we have proven that $X$ is compact.
\qeds

\section{Topology}

\begin{thm}{Definition}
Let $X$ and $Y$ be metric spaces.
A map $f\:X\to Y$ is called a \emph{homeomorphism}\index{homeomorphism}
if it is a continuous bijection and the inverse $f^{-1}\:Y\to X$ is also continuous.

Two metric spaces $X$ and $Y$ are called \emph{homeomorphic}\index{homeomorphic spaces} (briefly $X\hom Y$) if there is homeomorphism $f\:X\to Y$.
\end{thm}

It is straightforward to check that $\hom$ is an equivalence relation.

Every isometry is a homeomorphism,
but not every homeomorphism is an isometry.
The homeomorphism class of metric space is much larger than its isometry class.

\begin{thm}{Exercise}\label{ex:cont+biject}
Give an example of a continuous bijection between metric spaces that is not a homeomorphism.
\end{thm}

Let $\rho$ and $\rho'$ be two metrics on the same set $X$.
We say that $\rho$ is \emph{equivalent} to $\rho'$ if the identity map 
$(X,\rho)\to (X,\rho')$ is a homeomorphism.
In other words, $\rho$ is equivalent to $\rho'$ if and only if
for any sequence of points $(x_n)$ and a point $x_\infty$ in $X$, we have
$$\rho(x_n,x_\infty)\to 0\ \ \Leftrightarrow\ \ \rho'(x_n,x_\infty)\to 0.$$

The class of equivalent metrics on a fixed set $X$
is described completely by all open subsets of $X$;
i.e., a metric $\rho$ is equivalent to a metric $\rho'$
if and only if any set which is open with respect to the metric $\rho$ is open with respect to the
metric $\rho'$ and visa versa.
\begin{thm}{Exercise}
Prove the last statement. 
\end{thm}

The set of open sets in $X$ is called the \emph{topology}\index{topology} of $X$.
A topology has to satisfy the following axioms:
\begin{enumerate}
\item the union of an arbitrary number of open sets is open,
\item the intersection of a finite number of open sets is open
\item the empty set and $X$ are both open sets
\end{enumerate}
A set with chosen topology is called \emph{topological space}.

Many of concepts described above can be formulated in terms of toplogy.
For example a map $f\: X\to Y$ is continuous if and only if for any open set $U\subset Y$,
the preimage $f^{-1}(U)$ is an open set in $X$.
Compactness also can be described entirely in topological terms;
see Theorem~\ref{thm:compact+covering}.
On the other hand, completeness is not a topological property,
for example the space $[0,1)$ is homeomorphic%
\footnote{say $x\mapsto \tfrac x{1-x}$ is a homeomorphism}
to $[0,\infty)$, but $[0,1)$ is not complete whereas $[0,\infty)$ is.

Given a topological space $X$, 
we may say ``consider a metric $\rho$ on $X$'' 
meaning that $d$ is a metric which describe the topology of $X$.
In the case that $X$ is a metric space,
we mean that $\rho$ is equivalent to the original metric on $X$. 

\begin{thm}{Definition}
A map $f\:X\to Y$ is called an \emph{embedding}\index{embedding}
if $f$ gives a homeomorphism of $X$ to its image $f(X)$ in $Y$.
\end{thm}

Note that any distance preserving map is an embedding,
but the converse is not true. 

\parbf{Remarks.}
These definitions are a starting point of so called \emph{point-set topology};
but this is as far as we need to go in this direction.
We will only use the fact that any metric space is naturally a topological space.

A topological space which can appear this way is called \emph{metrizable}%
\index{metrizable topological space}.
Not every topological space is metrizable;
i.e., one can find a set $X$ with a collection of \textit{open sets}
which satisfies above 3 axioms, but there is no metric on $X$ for which these are the open sets; 
we will not consider such monsters in these lectures.




\section*{Exercises}

\begin{pr}\label{pr:nonisometry}
Give an example of a metric space $X$ with a distance preserving map $f\:X\z\to X$ which is not a bijection. 
\end{pr}

\begin{pr}\label{pr:non-contracting=>isometry}
Let $K$  be a compact metric space and
$f\:K\to K$ be a non-contracting map;
i.e., 
$$|f(x)-f(y)|_K\ge |x-y|_K$$
for any $x,y\in K$.
Prove that $f$ is an isometry.
\end{pr}



\begin{pr}\label{ex:kuratowski}
Let $X$ be a metric space; fix $x\in X$.
Show that the map 
$$K_x\: X\to\mathcal{F}(X)$$ 
defined by 
$$K_x(z) = \Dist_z-\Dist_x$$
is distance preserving;
i.e. 
$$|K_x(y)-K_x(z)|_{\mathcal{F}(X)}=|y-z|_X$$
for any $y,z\in X$.
\end{pr}

{\it The map $K_x\: X\to\mathcal{F}(X)$ is called the \emph{Kuratowski embedding with base $x$}\index{Kuratowski embedding}.

If $X$ has bounded diameter, i.e., if there is a constant $D<\infty$,
such that $|z-y|\le D$ for any $z,y\in X$,
then $\Dist_x$ is bounded for any $x\in X$
and the map 
$$K\:X\to\mathcal{F}(X)$$ given by $$K(x) = \Dist_x $$ 
is also distance preserving.
The latter map is also called the \emph{Kuratowski embedding}\index{Kuratowski embedding}.}

\begin{pr}\label{pr:compact->F_N}
Show that any compact space is isometric to a subset of $\mathcal{F}(\NN)$;
i.e., the space of bounded sequences with the metric induced by sup-norm.
\end{pr}

\begin{pr}\label{pr:complition-kuratowski}
Show that completion of any metric space is isometric to the closure of its image under Kuratowski embedding.
\end{pr}

\begin{pr}\label{pr:almost-min}
Let $X$ be a complete metric space and $\rho\:X\to \RR$ be a continuous positive function.

Show that there is $x\in X$ such that
$$\rho(y)>\tfrac{99}{100}\cdot\rho(x)$$ 
for any $y\in B_{\rho(x)}(x)$.
\end{pr}

\begin{pr}\label{pr:compact-net}
Show that a complete metric space $X$ is compact if for any $\eps>0$ there is a compact $\eps$-net in $X$.  
\end{pr}


