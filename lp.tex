\chapter{Length-preserving maps}\label{chap:lp}

\section{Length-preserving map from sphere to the plane}

Let $X$ and $Y$ be metric spaces.
Recall that a continuous map $f\:X\to Y$
is called \emph{length-preserving}\index{length-preserving map} if for any curve $\alpha\:\II\to X$,
we have 
$$\length\alpha=\length(f\circ\alpha).$$
Note that since $f$ is continuous, the composition $f\circ\alpha$ is a curve in $Y$ (see Definition~\ref{def:curve});
therefore the above definition makes sense.

\begin{thm}{Exercise}\label{LP=>short}
Show that if $X$ is a complete length space then any length-preserving map $X\to Y$
is also a distance nonexpanding.
\end{thm}

Here is the main theorem of this chapter.

\begin{thm}{Theorem}\label{thm:S2->R2}
There is a length-preserving map $\SS^2\to\RR^2$.
\end{thm}


The existence of a length-preserving map $\SS^2\to\RR^2$ might look counterintuitive.
The following exercise shows that such a map $f$ has to crease on an everywhere dense set in $\SS^2$.

\begin{thm}{Exercise}\label{ex:S->R--not-inj}
Let $f\:\SS^2\to\RR^2$ be a length-preserving map. 
Show that the restriction of $f$ to any open subset of $\SS^2$ cannot be injective. 
\end{thm}

\parit{The plan of the proof.}
 We will use Problem~\ref{pr:H>GH-boundary}
to construct a sequence of 2-dimensional polyhedral spaces $P_n$ which converge to $\SS^2$
in the sense of Gromov--Hausdorff. 
The sequence $P_n$ will be chosen in such a way that there are surjective distance nonexpanding maps $\phi_n\:P_n\to P_{n-1}$.

The map $f_\eps$ will be constructed as the \emph{limit} of a sequence of piecewise distance preserving maps $f_n\:P_n\to\RR^2$ which are automatically length-preserving.
(We will eventually have to make rigorous what we mean by limit here.  Note that one cannot talk about the limit of the maps $f_n$ in the usual sense because each $f_n$ has a different domain.)

The maps $f_n$ will be constructed recursively;
to construct $f_n$
we will apply Akopyan's approximation theorem (\ref{thm:approx}) to the composition $f_{n-1}\circ\phi_n\:P_n\z\to\RR^2$.
Some extra care will be needed to ensure that the limit of $f_n$  will be a length-preserving map that is $\eps$-close to $f$.



\parbf{Remarks.}
From Exercise~\ref{LP=>short},
it follows that if we have a converging sequence of  length-preserving map $f_n\:X\to Y$ then their limit $f_\infty\:X\to Y$ is distance nonexpanding,
but in general, $f_\infty$ is not  length-preserving.

The existence of some piecewise distance preserving maps $P_n\to\RR^2$ is provided by Zalgaller's mapping theorem (\ref{thm:zalgaller}).
But these maps cannot be used to prove theorem \ref{thm:S2->R2}
since the maps constructed in the proof of Zalgaller's mapping theorem converge to the map which sends all of $\SS^2$ to one point.
The latter follows from the observation below.


\begin{thm}{Observation}
Let $P$ be a $2$-dimensional polyhedral space and $f\:P\z\to\RR^2$ be the map constructed in the proof of Zalgaller's mapping theorem for the collection of vertices $\{v_1,v_2,\dots,v_n\}\in P$ then 
$$\diam f(P)\le\max_{x\in P}\min_{v_i}\{|x-v_i|_P\}.$$

\end{thm}


\parit{Proof of Theorem~\ref{thm:S2->R2}.}
Consider a nested
sequence $K_0\subset K_1\subset \dots$ of convex polyhedra in $\RR^3$ whose union is a unit ball.
Clearly $K_n$ converges to the unit ball in the Hausdorff sense (compare Lemma~\ref{lem:decreasing-converges}).

Denote by $P_n$ the boundary of $K_n$ equipped with the induced length metric. %??? surface???
According to Exercise~\ref{ex:bry-is-poly}, $P_n$ is a polyhedral space for each $n$.
By Problem~\ref{pr:H>GH-boundary}, $P_n$ converges to $\SS^2$ in the Gromov--Hausdorff sense.

The same construction as in Problem~\ref{problem2} 
gives a distance nonexpanding map $\phi_n\:P_n\to P_{n-1}$ for each $n$.
It is straightforward to see that $\phi_n$ are piecewise linear;
i.e., there are triangulations of $P_n$ and $P_{n-1}$ such that for any simplex $\Delta$ in $P_n$, the image $\phi_n(\Delta)$ lies in a simplex $\Delta'$ of $P_{n-1}$ and the restriction $\phi |_\Delta$ is linear (if written with in the barycentric coordinates of $\Delta$ and $\Delta'$). 


Note that for any point $x\in \SS^2$, there is unique sequence of points $x_n\in P_n$ such that $x_n\to x$ as $n\to\infty$ and $\phi_n(x_n)=x_{n-1}$ for all $n\ge 1$.
To show existence of this sequence, fix any sequence $z_n\in P_n$ such that $z_n\to x$.
Consider the double sequence $y_{n,m}\in P_n$ defined for $n\le m$ such that $y_{n,n}=z_n$ and
$\phi_n(y_{n,m})= y_{n-1,m}$ for all $1\le n$.
Then set 
$$x_n=\lim_{m\to\infty} y_{n,m}.$$

\begin{thm}{Exercise}\label{ex:limit-above}
Show that the limit above is defined.
\end{thm}

If $x_n\to x\in \SS^2$ be a sequence as above,
define $\psi_n\:\SS^2\to P_n$ as $\psi_n(x)=x_n$.
Clearly $\psi_n\:\SS^2\to P_n$ is distance nonexpanding, 
$\psi_{n-1}=\phi_n\circ\psi_{n}$ for each $n\ge 1$
and for any $p$ and $q\in\SS^2$ we have
$$|p_n-q_n|_{P_n}\to |p-q|_{\SS^2}\ \text{as}\  n\to\infty\eqlbl{eq:pq-to-pq}$$
where $p_n=\psi_n(p)$ and $q_n=\psi_n(q)$.%
\footnote{Such sequences of spaces and maps 
$$\dots\xto{\phi_{2}}X_2\xto{\phi_{2}} X_1\xto{\phi_{1}}X_0$$
appear in all branches of mathematics. 
They are called \emph{inverse systems}\index{inverse systems}.
A sequence $(x_0, x_1, x_2, \ldots)$ such that $x_n \in X_n$ and $\phi_n(x_n)=x_{n-1}$ could be considered as a point in a new space $X$ which is called the \emph{inverse limit}\index{inverse limit} of the system and is denoted $X=\varprojlim X_n$. 
Given $x=(x_0,x_1,\dots )\in X$, the evaluation maps $\psi_n: X \to X_n$ given by $\psi_n(x)=x_n$ are called \emph{projections}\index{projections}.}
In particular $\psi_n\:\SS^2\to P_n$ is $\delta_n$-isometry for some sequence $\delta_n\to0+$.

\parit{Recursive construction of the maps $f_n\:P_n\to\RR^2$.}
Assume we have a piecewise distance preserving map $f_{n-1}\: P_{n-1}\z\to \RR^2$.
The composition $f_{n-1}\circ\phi_n\: P_n\z\to\RR^2$ is a piecewise linear and distance nonexpanding. 
Thus given $\eps_{n-1}>0$ we can apply Theorem~\ref{thm:approx} to construct a piecewise distance preserving map $f_n\:P_n\to\RR^2$ which is $\eps_{n-1}$-close to $f_{n-1}\circ\phi_n$.

Denote by $M(n)$ the number of triangles in a triangulation for $f_n$;
i.e., a triangulation of $P_n$ such that $f_n$ is distance preserving on each of its triangles. 
Set 
$$\eps_{n}=\frac{\eps_{n-1}}{10\cdot M(n)}.\eqlbl{eq:eps-n}$$
This way we recursively define the construction of $f_n$.
It goes as follows: 
\begin{enumerate}
\item Choose an arbitrary $\eps_0>0$ and take a piecewise distance preserving map $f_0\:P_0\to\RR^2$ which is provided by Zalgaller's mapping theorem (\ref{thm:zalgaller}).
\item Use $\phi_1$, $f_0$ and $\eps_0$ to construct $f_1$.
\item Use $f_1$ to construct $\eps_1$.
\item Use $\phi_2$, $f_1$ and $\eps_1$ to construct $f_2$.
\item Use $f_2$ to construct $\eps_2$.
\item $\dots$
\end{enumerate}
%???This procedure is similar to walking on stairs: you have to make right step, which makes possible to do left step, which makes possible to do right step and so on???
 

\medskip



It remains to prove the following claim:

\begin{clm}{}
The sequence of maps $f_n\circ\psi_n\:\SS^2\to \RR^2$ converges to a length-preserving map $f\:\SS^2\to\RR^2$. 
\end{clm}

Since $\eps_{n}$ decays faster than $\tfrac{\eps_0}{10^n}$, 
the sequences $(f_n\circ\psi_n)(x)$ are Cauchy,
for any fixed $x$;
so the limit map $f(x)$ is defined.
Since all $f_n\circ\psi_n$ are distance nonexpanding, we have that so is $f\:\SS^2\to\RR^2$.  Note that this implies $$\length (f\circ\alpha) \leq \length \alpha$$ for any curve $\alpha$ in $\SS^2$.

For the remaining part of proof we will need the following definition.

\begin{thm}{Definition}
Let $X$ be a metric space and $\alpha\:[0,1]\to X$
be a curve.
Set 
$$\ell_k(\alpha)
\df
\sup\set{\sum_{i=1}^k |\alpha(t_i)-\alpha(t_{i-1})|_X}{0=t_0<t_1<\dots<t_k=1}.$$

\end{thm}

Note that for any fixed $\alpha$,
the sequence $\ell_k(\alpha)$ is non-decreasing in $k$ and
$$
\length\alpha=\lim_{k\to\infty}\ell_k(\alpha).$$

\medskip

Assume $f$ is not length preserving.
Since $f$ is distance nonexpanding, it follows that for some curve 
$\gamma$ in $\SS^2$ we have 
$$\length(f\circ\gamma)<\length \gamma.$$
Together with the definition of length (\ref{def:length}),
the latter implies that on $\gamma$ one can choose two points, say $p$ and $q$, such that
$$\length (f\circ\alpha)<|p-q|_{\SS^2},\eqlbl{eq:length<|p-q|}$$
where $\alpha$ denotes the part of $\gamma$ from $p$ to $q$.

Set $p_n=\psi_n(p)$ and $q_n=\psi_n(q)$ as above and let $\alpha_n$ be a curve from $p_n$ to $q_n$.
Note that one can exchange $\alpha_n$ for a shorter curve, say $\beta_n$ which goes from $p_n$ to $q_n$ and whose image in any triangle of the triangulation of $P_n$ is a line segment.
It follows that $f_n\circ\beta_n$ is a broken line with at most $M(n)$ edges.
Hence
$$\ell_{M(n)}(f_n\circ\beta_n)=\length\beta_n\le \length\alpha_n.$$ 
Therefore
$$|p_n-q_n|_{P_n}=\inf \{\ell_{M(n)}(f_n\circ\alpha_n)\},\eqlbl{eq:pn-qn}$$
where the infimum is taken for all curves $\alpha_n$ from $p_n$ to $q_n$ in $P_n$.  

By the recursive construction of the maps, we have that
$$|(f_n\circ\psi_n)(x)-f(x)|<\eps_n$$
for any $x\in \SS^2$ and each $n$.
From \ref{eq:eps-n}, it follows that
$$|\ell_{M(n)}(f\circ\alpha)-\ell_{M(n)}(f_n\circ\psi_n\circ\alpha)|< \eps_{n-1}.\eqlbl{eq:inq-ell-k}$$
Note that \ref{eq:pq-to-pq} and \ref{eq:length<|p-q|} implies that
$$\eps_{n-1}<|p_n-q_n|-\length \alpha$$ 
for all large $n$,
the latter contradicts \ref{eq:inq-ell-k} and \ref{eq:pn-qn}.
\qeds


\section{Lirical degression: \textit{h}-priciple}

The Theorem \ref{thm:S2->R2} admits the following generalization.

\begin{thm}{Gromov's theorem}\label{thm:gromov}
Let $M$ be an $m$-dimaensional Riemannian manifold%
\footnote{If you do not know what is Riemannian manifold, think that $M$ is the smooth hypersurface in $\RR^{m+1}$ equipped with induced length metric.}%
.
Then any distance nonexpanding map 
$$f\:M\to\RR^m$$ 
can be approximated by length-preserving maps $M\to\RR^m$.

More precisely, given $\eps>0$ there is a length-preserving map $f_\eps\:M\to\RR^m$
such that 
$$|f_\eps(x)-f(x)|<\eps$$
for any $x\in M$.
\end{thm}

The proof of this theorem can be given along the same lines as the proof of Theorem \ref{thm:S2->R2},
but it require couple more technical steps.

In modern geometry, one often face so called \textit{local to global} problem.
That means that you know the behavior of some geometrical structure (metric or map or anything)
locally, i.e. in a small neigborhood of each point and you want to know whether thare are nontrivial consequences of this property to on big scale.
For example,
let $\square$ be the unit sqare in $\RR^2$;
the class of piecewise distance preserving (PDP) maps $\square\to\RR^2$ 
can be described locally; 
i.e., if the map is piecewise distance preserving in a neighborhood of any point then it is also piecewise distance preserving.
It is obvious that any such map is a distance non-expanding
and from Brahm's theorem it follows that any distance non-expanding map can be approxiamted by piecewise distance preserving maps.
In other words, for  PDP the answer is ``NO'';
i.e.,  PDP has no nontrivial consequences to the global behavior of the map.

In fact the answer ``NO'' is the most common.
There is a technology named ``\textit{h}-principle'' which makes possible to prove a ``NO''-answer for other local structures.
\textit{h}-Principle is not a theorem, it is a property which often holds for underdetermined partial differential equations.
There a number of methods to prove \textit{h}-principle for given equasion, including the one which is described in the proof of Theorem \ref{thm:S2->R2}%
\footnote{One may think of ``lenght-preserving maps'' as about weak solutions of a particular partial differential equasion.}%
.
Once the \textit{h}-principle is proved for the equasion, 
it becomes very easy to check existance of a solution.
The Gromov's theorem is one of the simplest example.
Other examples include 
\begin{itemize}
\item Nash--Kuiper theorem, which in particular implies existance of $C^1$-smooth lenght-preserving map $\SS^2\to\RR^3$ with arbitrary small diameter of image.
\item Smale's sphere eversion paradox, which states that there is a continuous one partameter family of smooth immersions $f_t\:\SS^2\to \RR^3$, $t\in[0,1]$ such that $f_0\:\SS^2\to \RR^3$ is the standard inclusion and $f_1(x)=-f_0(x)$ for all $x\in \SS^2$.
In fact combining the technique of Smale and Nash--Kuiper, one can make sphere eversion $f_t$ in the class $C^1$-smooth length-preserving maps. 
\end{itemize}
 
For further reading we sugest the comprehancive and reader-friendly introduction to \textit{h}-principle by  Eliazhberg and Mishachev \cite{eliashberg-mishachev}.




