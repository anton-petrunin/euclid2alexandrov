


In the proof we will \emph{approximate} $\SS^2$ by a sequence of polyhedral spaces $P_n$.
For each polyhedral space w 

\parit{The plan of the proof.}
We will use the following exercise 
to construct a sequence of 2-dimensional polyhedral spaces $P_n$ with surjective distance non-expanding maps $\phi_n\:P_n\to P_{n-1}$
and $\psi_n\:\SS^2\to P_n$ such that
$\psi_{n-1}=\phi_n\circ\psi_n$ for each $n$ 
and 
$$|\psi_n(x)-\psi_n(y)|_{P_n}\to |x-y|_{\SS^2}$$
for any $x,y\in \SS^2$.%
\footnote{Such sequences of spaces and maps 
$$\dots\xto{\phi_{2}}X_2\xto{\phi_{2}} X_1\xto{\phi_{1}}X_0$$
appear in all branches of mathematics. 
They are called \emph{inverse systems}\index{inverse systems}.
A sequence $(x_0, x_1, x_2, \ldots)$ such that $x_n \in X_n$ and $\phi_n(x_n)=x_{n-1}$ could be considered as a point in a new space $X$ which is called the \emph{inverse limit}\index{inverse limit} of the system and is denoted $X=\varprojlim X_n$. 
Given $x=(x_0,x_1,\dots )\in X$, the evaluation maps $\psi_n: X \to X_n$ given by $\psi_n(x)=x_n$ are called \emph{projections}.}



Further, we construct piecewise distance preserving maps $f_n\:P_n\z\to\RR^2$.
To construct $f_n$ we apply Akopyan's approximation theorem (\ref{thm:approx}) to the composition $f_{n-1}\circ\phi_n\:P_n\z\to\RR^2$.

The needed length-preserving map $f$ is  the limit of the compositions 
$$f_n\circ\psi_n\:\SS^2\z\to\RR^2.$$

\parbf{Remarks.}
From Exercise~\ref{LP=>short},
it follows that $f$ is distance non-expanding.
Some extra care will be needed to ensure that the  $f$ is actually length-preserving.

The existence of some piecewise distance preserving maps $P_n\to\RR^2$ is provided by Zalgaller's mapping theorem (\ref{thm:zalgaller}).
However if $f_n$ is the map constructed in the proof of Zalgaller's theorem for polyhedral space $P_n$
then diameter of the image of $f_n$ can not exceed the maximal diameter of triangle in a triangulation of $P_n$.
Therefore, the compositions $f_n\circ\psi_n$ 
might only converge to a \emph{constant map};
i.e., the map which sends everything to one point.



















From \ref{eq:w-on-edge}, in the latter case, we can assume that the points $x$ and $y$ lie on the different edges of $\mathcal{T}$.
It follows that in this case $x$ and $y$ have to lie near one vertex of $\mathcal{T}$.

\medskip

The constructed map $w$ almost satisfies the needed conditions;
it only fails to meet \ref{clm:global} for $x$ and $y$ 
near one of the vertex of $\mathcal{T}$.
We now make a modification of the above construction to fix this issue.

\parit{Correction to the construction of $w$.} Let us construct $w$ as above
 so that in addition it satisfies the following property:
\begin{clm}{}\label{cond:ends}
There is $\lambda>0$ such that 
if
$$|v-x|_P=|v-y|_P \leq \lambda$$
 for two points on $1$-skeleton $x$ and $y$ and a vertex $v$ of $\mathcal{T}$ then
$$w(x)=w(y).$$

\end{clm}
Once such a $w$ is  constructed, $|v-x|_P,|v-y|_P\le \lambda$
 would  imply 
\begin{align*}
 |w(x)-w(y)|_{\RR^2}
&\le \bigl||v-x|_P-|v-y|_P\bigr|\le
\\
&\le|x-y|_P
\end{align*}
for $\tfrac\theta\delta\ll\lambda$ the same way as in \ref{eq:w-on-edge}.

To make such a modification,
fix a small $\lambda > 0$.
Denote by $\mathcal{T}^1$ the 1-skeleton of $\mathcal{T}$.
Consider the map $q_\lambda\:\mathcal{T}^1\to\mathcal{T}^1$ which sends each edge to itself 
in such a way that any point of distance $\le\lambda$ to a vertex maps to this vertex 
and the map is linear on the remaining part of edge.
For each $\lambda>0$ we can find minimal $\mu\ge 0$ such that 
$(1-\mu)\cdot f\circ q_\lambda$ is distance nonexpanding.

It can be shown that $\mu\to0$ as $\lambda\to 0$.
Note that it is sufficienct to show this for three sides of a fixed triangle in $\mathcal{T}$.
The latter can be done by straightforward calculations.

Choose sufficiently large $n$, so $\tfrac{\theta(n)}\delta\ll\lambda$
and apply the zigzag procedure to each of the three portions of each edge separately in such a way that all the $\lambda$-long intervals starting at one vertex are mapped in an identical way.
The constructed $w$ meets condition~\ref{cond:ends}.










\begin{thm}{Exercise}\label{ex:PDP+fixed-triang}
Construct a $2$-dimensional polyhedral space $P$ 
with a fixed triangulation which does not admit 
a map $f\:P\to\RR^2$ which is distance preserving on each simplex.
\end{thm}





Given a point $v$ in a simplicial complex,
the minimal subcomplex containing a small neigborhood of $v$ is called \emph{star of $v$}.



\begin{thm}{Exercise}
Check the proof and make sure that the condition \ref{star-condition} is indeed sufficient.
\end{thm}


\begin{thm}{Exercise}\label{ex:star-triangulation}
Show that one can add a finite number of points on the edges of $P$ triangulation so that the condition \ref{star-condition} holds.
\end{thm}

These exercises finish the proof of two dimensional case.

\parit{Higher dimensions.}
First let me give one more definition.

\begin{thm}{Definition}
 Let $\Sigma$ be a metric space with diameter $\le \pi$. 
A metric space $K$ is called a Euclidean cone over $\Sigma$
if its underlying set 
is the quotient $\Sigma\times [0,\infty)/{\sim}$, 
where $(x,0)\sim(y,0)$ for any $x,y\in \Sigma$
and the metric is defined by the cosine rule;
i.e. 
$$|(x,a)-(y,b)|^2_K=a^2+b^2-2\cdot a\cdot b\cdot\cos|x-y|_\Sigma.$$

\end{thm}

The general case can be proved by induction on $m$. 
First, one has to choose a finite set of points $\{w_1,w_2,\dots,w_n\}$
so that the condition \ref{star-condition} holds.
The proof of existence of this set is not hard but tedious;
see a comment in the solution of Exercise~\ref{ex:star-triangulation}.

Now let us glue any pair of points $x$ and $y\in P$
if for some point $z\in V_i\cap V_j$, 
$x$ lies on the geodesic $[w_iz]$,
$y$ lies on the geodesic $[w_jz]$
and $|w_i-x|_P=|w_j-y|_P$.

It is not hard to check that these gluing maps are piecewise distance preseving.
Thus we get a piecewise distance preseving map from $P$ to a new  polyhedral space,
say $P'$.
Note that $P'$ can be viewed as a star shaped subset of a Euclidean cone,
 say $K$.

Therefore, it is sufficient to construct a  piecewise distance preseving map $K\z\to\RR^m$.
The latter is equivalent to finding a piecewise distance preserving map from any $(m-1)$-dimensional spherical polyhedral space%
\footnote{It is defined the same way as Euclidean polyhedral space, but using spherical simplices.}
 to $\SS^{m-1}$.

For a spherical polyhedral space,
one can repeat the same construction once the cone is exchanged to so called 
\emph{spherical suspension}\index{spherical suspension} defined below.
On each step we get a dimension reduction, and the $\dim P=1$ case can be proved the same way for spherical polyhedral space

\begin{thm}{Definition}
 Let $\Sigma$ be a metric space with diameter $\le \pi$. 
A metric space $\Lambda$ is called a spherical suspension over $\Sigma$
if its underlying set 
is the quotient $\Sigma\times [0,\pi]/{\sim}$, 
where $(x,0)\sim(y,0)$ and $(x,\pi)\sim(y,\pi)$ for any $x,y\in \Sigma$
and the metric is defined by the spherical cosine rule;
i.e. 
$$\cos|(x,a)-(y,b)|_\Lambda
=\cos a\cdot \cos b
+\sin a\cdot\sin b\cdot\cos|x-y|_\Sigma.$$

\end{thm}











We start with the proof of 1-dimensional case.
Further we give a regurus proof of Zalgaller's mapping theorem in dimension 2 
assuming in addition that polyhedral space admits a triangulation by acute triangles.
After that we give two hints;
the first leads to the proof in general 2-dimensional case
and the secont leads the general $m$-dimensional case.


\parit{Proof; case $\dim P=1$.}
Choose a triangulation of $P$;
in addition to  the vertices (the $0$-dimensional simplices), 
it contains a finite number of edges ($1$-dimensional simplices).

Let $\{v_1,v_2,\dots,v_n\}$ be the set of vertices
 of this triangulation.
Consider the map $f\:P\to\RR$
$$f(x)\df\min_i\{|x-v_i|_P\}.$$
Bisecting each edge, we get a new triangulation of $P$
and $f$ is distance preserving on each simplex of 
the new triangulation.

Indeed, label the edges of the triangulation as
$\ell_1,\ell_2,\dots,\ell_k$.
According to the definition of polyhedral space (\ref{def:poly}),
each edge $\ell_j$ in the triangulation admits an isometry
$\iota_j\:[0,a]\to \ell_j$ for some $a \in \RR$.
Then 
$$f\circ\iota_j(x)=\min\{|x|,|a-x|\}.$$
Therefore $f$ is distance preserving on each half of $\ell_j$.
\qeds






\parbf{Abstract finite simplicial complex.}
To describe a simplicial complex $\mathcal{K}$, it is sufficient to list the set of its vertices 
$\mathcal{V}=\{v_1,v_2,\dots,v_n\}$ and list all the subsets $\mathcal{S}$ of $\mathcal{V}$ which appear as vertices of single simplex in $\mathcal{K}$.
I.e., $W=\{w_0,w_1,\dots,w_k\}\z\in \mathcal{S}$ if and only if 
there is a simplex in $\mathcal{K}$ with vertices $\{w_0,w_1,\dots,w_k\}$.
The above condition imply that if $W\in \mathcal{S}$ then any subset $W'\subset W$ is also in $S$.

Any point $x$ in $\ushort{\mathcal{K}}$ can be described using barycentric coordinates 
$(\lambda_1,\lambda_2,\dots,\lambda_n)$ in a way such the set 
$$\set{v_i\in\mathcal{V}}{\lambda_i>0}$$
belongs to $\mathcal{S}$.

The latter gives a metric on $\ushort{\mathcal{K}}$ which is completely determined by the combinatorial structure described above.
If $(\lambda_1,\lambda_2,\dots,\lambda_n)$ and $(\mu_1,\mu_2,\dots,\mu_n)$
are barycentric coordinates of $x$ and $y$ in $\ushort{\mathcal{K}}$ we set
$$|x-y|_{\ushort{\mathcal{K}}}=\max_{i\in\{1,2,\dots,n\}}\{|\lambda_i-\mu_i|\}.$$
This metric differ from the one described above, but it defines the same convergence.



\parbf{Locally finite simplicial complex.}
A few times we will need notion of \emph{locally finite simplicial complex}\index{simplicial complex!locally finite simplicial complex}.
Instead of giving the definition, let us describe the necessary modification in the definition of abstract simplicial complex.

In the above construction, one can take $\mathcal{V}$ to be an infinite set and 
$\mathcal{S}$ to be any collection of finite subsets such that
\begin{itemize}
\item if $W\in \mathcal{S}$ then any subset $W'\subset W$ is also in $\mathcal{S}$;
\item For any $v\in \mathcal{V}$, there are only finitely many $W \in \mathcal{S}$ such that $v \in W$.
\end{itemize}
Then one can identify a point of simplicial complex with a nonnegative function
$\lambda\:\mathcal{V}\to\RR$ such that the set
$$\set{v\in\mathcal{V}}{\lambda(v)>0}$$
belongs to $\mathcal{S}$
and 
$$\sum_{v\in \mathcal{V}}\lambda(v)=1.$$
(Formally this sum has an infinite number of terms,
but only finitely many of these terms differ from $0$.)
The obtained set of points can be then equipped with the metric 
$$|\lambda-\mu|=\sup_{v\in\mathcal{V}}\{|\lambda(v)-\mu(v)|\}.$$
In particular we may talk about convergence of points in a locally finite simplicial complex.

\begin{thm}{Exercise}
Show that a compact locally finite simplicial complex has to be finite. 
\end{thm}







Given $K\in \left]\mathbf{K}_k\right[$,
consider its surface $P$.
According to ???, $P\in \left]\mathbf{P}_k\right[$.
Note that congruent polyhedra have isometric boundaries;
therefore we can consider map, say $\Phi_k\:\mathbf{K}_k\to\mathbf{P}_k$,
which sends $[K]$ to $[P]$.
Note that 
\begin{itemize}
\item A polyhedral space $P\in\left]\mathbf{P}_k\right[$ 
is realizable if and only if it 
lies in the image of $\Phi_k$.
\item According to Alexandrov uniqueness theorem $\Phi_k$ is injective.
\item According to Exercise~\ref{pr:K-P-simmetry}, $K\in \left]\mathbf{K}_k\right[$ has no symmetry if and only if $\Phi_k(K)\in\left]\mathbf{P}_k\right[$ has no symmetry. 
\end{itemize}

Finally,
the Alexandrov's existence theorem can be reformulated the following way.

\begin{thm}{Reformulation of Alexandrov's existence theorem}
For each integer $k\ge 3$, the map 
$$\Phi_k\:\mathbf{K}_k\to \mathbf{P}_k$$ 
is surjective.
\end{thm}

This reformulation will be proved by induction on $k$.











\subsection*{Braking the symmetry}

\parit{Whati is the problem?}
Note that conclusion of lemmas \ref{lem:P-mnld-nosimm} and \ref{lem:M-mnld-nosimm}
does not hold if one removes ``no simmetry'' condition.
For eaxample, the generic convex plane quadralieraral $Q$ 
has one nontrivial symmetry, the reflection in its plane.
The tetrahedra (possibly degenerate to quadralitra) near $Q$ 
can be parametrized by $6$ coordinates
$(x_2,x_3,y_3,x_4,y_4,z_4)$,
but due to the simmetry we may assume that $z_4\ge 0$.
This way we get a parametrization of a neigborhood of $[Q]$ in $\mathbf{K}_4$ 
by $(-\eps,\eps)^5\times [0,\eps)$.

The same thing happens in a neighborhood of symmetric metric in  $\mathbf{P}_4$.

For the neigborhood $(-\eps,\eps)^5\times [0,\eps)$ an analog of Domain Invariance theorem does not hold and that explain the difficulty which we face for the space with symmetry.

Two find a walk arround, we need to brake symmetry of the space.
this can be done by introducing extra structure on the convex polyhedron,
for example we may fix orientation and yet choose ordered three points.
This way we obtain a new a bigger set formed by finer equivalece classes say $\mathbf{K}_k^\#$.
Further we need to equip $\mathbf{K}_k^\#$ with a metric which takes into account the introduced extra struture.
Say ???

Now in  the new space an analog of Lemma \ref{lem:P-mnld-nosimm} holds in a neghborhood of any point.

A similar procedure has to be done with the space $\mathbf{P}_k$.
Again, we may choose ordered triple of vertices in a metric and its orientation;
this way we obtain an othe set of equivalence classes say $\mathbf{P}_k^\#$
and come up with a way to introduce metric on $\mathbf{P}_k^\#$ which takes into account this extra struture so that an analog of Lemma \ref{lem:M-mnld-nosimm} holds in a neghborhood of any point.

The surface of equipped polyheron $K$ appears to be an equipped metric.
Thus we define a map $\Phi_k^\#\:\mathbf{K}_k^\#\to \mathbf{P}_k^\#$.
In addition we get natural \emph{forgetting} maps $\mathbf{K}_k^\#\to\mathbf{K}_k$ and  $\mathbf{P}_k^\#\to\mathbf{P}_k$;
these maps simply forget the introduces extra structure in both cases.
Both maps ??? and ??? are continuous and they also send open sets to open sets
in addition the following diagram commutes.

???

The argument as above prove that show that $\Phi_k^\#(\mathbf{K}_k^\#)$ is open in $\mathbf{P}_k^\#$, which implies that $\Phi_k(\mathbf{K}_k)$ is open in $\mathbf{P}_k$.

 












\section{Notations and conventions.}





Let $K\mathbf{K}_k$ and $P$ be its surface.
According to ??? $P\in \mathbf{K}_k$.
Let us define the map $\Phi_k\:\mathcal{K}_k\to \mathcal{P}_k$
so that $\Phi_k\:[K]\mapsto [P]$;
note that according to ??? the map is well defined.





In this section we construct few metric spaces and maps between them which will be used in a reformulation of the Alexandrov's existence theorem which we will be proved later.

Further in this section we fix an integer $k\ge 3$.

\parbf{$\mathbf{S}_k$ and $\mathbf{S}_k'$.} 
Consider the set of all $k$-point subsets in $\RR^3$ equipped with Hausdorff metric.
The obtained metric space will be denoted as $\mathbf{S}_k$.
The space $\mathbf{S}_k$ is a subspace of $\mathcal{H}(\RR^3)$.
Note that $\mathbf{S}_k$ is not closed in $\mathcal{H}(\RR^3)$;
the closure of $\mathbf{S}_k$ in $\mathcal{H}(\RR^3)$ contains all $\mathbf{S}_m$ for $m\le k$.

We say that $k$-point subset $Q\subset \RR^3$ \emph{has a simmetry} if there is an isometry $\iota\:\RR^3\to\RR^3$ which is distinct from indentity
such that $\iota(Q)=Q$.
If there is no such isometry, we say that  $Q$ has \emph{no simmetry}.
The subset of $\mathbf{S}_k$ formed by $k$-point subset of $\RR^3$ with no simmetry will be denoted as $\mathbf{S}_k'$

Note that any $2$-point subset as well as any $3$-point subset in $\RR^3$ have a simmetry;
i.e., $\mathbf{S}_2'=\mathbf{S}_3'=\emptyset$.
On the other hand, for any $k\ge 4$,
a $k$-point subset in \emph{general position} has no simmetry.

\parbf{$\mathbf{K}_k$ and $\mathbf{K}_k'$.}
Denote by $\mathbf{K}_k$ the space of all convex polyhedron in $\RR^3$ which has exactly $k$ vetices.
Given a convex polyhedron $K$ denote by $\Verts K$ the set of vetices of $K$.
Let us equip $\mathbf{K}_k$ with the following metric
$$|K-K'|_{\mathbf{K}_k}=|\Verts K-\Verts K'|_{\mathcal{H}(\RR^3)}.$$
According to Problem~\ref{pr:Hausdorff-Conv},
$$|K-K'|_{\mathbf{K}_k}\ge |K-K'|_{\mathcal{H}(\RR^3)}.$$

\begin{thm}{Exercise}
These two metrics define the same topology on $\mathbf{K}_k$;
i.e., if $K_\infty, K_1,K_2,\dots\in \mathbf{K}_k$ then
$$|K_n-K_\infty|_{\mathbf{K}_k}\to 0\ \iff \ |K_n-K_\infty|_{\mathcal{H}(\RR^3)}\to 0.$$

\end{thm}



Consider a polyhedron $K$ with $k-1$ vertices,
mark one point $v$ on its surface, choose a curve $v_t$, $t\ge 0$ 
which starts at $v$ 
and goes outside of $K$ and consider one parameter family of convex polyhdrons
$K_t=\Conv(K\cup \{p_t\})$.
If $v\notin\Verts K$ then $K_t\in\mathbf{K}_k$ for all sufficiently small $t>0$ and $K_t\to K$ as $t\to 0$.

In particular $\mathbf{K}_k$ is not complete 
and its completion contains all convex polyhdrons $K\in K_{k-1}$ 
with marked point $v\notin \Verts K$ on its surface.
Such spaces will be called ???
(Note that if $\mathbf{K}_k$ would be equipped with Hausdorff metric then its completion would be different;
it wold contain all $\mathbf{K}_m$ for $m\le k$.)


 



Let $K\in \mathbf{K}_k$, denote by $\{w_1,w_2,\dots,w_k\}$ the set of its verices.
Consider the map $\Upsilon_k\:\mathbf{K}_k\to \mathbf{S}_k$ defined as $K\mapsto \{w_1,w_2,\dots,w_k\}$. 

\begin{thm}{Exerise}
Show that $\Upsilon_k$ is injective. 
\end{thm}

\begin{thm}{Exerise}
Show that $\Upsilon_k$ is continuous. 
\end{thm}

\begin{thm}{Exerise}
Show that $\Upsilon_k(\mathbf{K}_k)$ is open in $\mathbf{S}_k$. 
\end{thm}

We say that a convex polhedron $K$ \emph{has a simmetry} if there is an isometry $\iota\:\RR^3\to\RR^3$ which is distinct from indentity
such that $\iota(K)=K$.
If there is no such isometry, we say that  $K$ has \emph{no simmetry}.
The subset of $\mathbf{K}_k$ formed by polhedra with no simmetry will be denoted as $\mathbf{K}_k'$.
Assume $K\in \mathbf{K}_k$;
clearly 
$$K\in   \mathbf{K}_k'\ \iff\ \Upsilon_k(K)\in \mathbf{S}_k'.$$

\parbf{$\bm{\mathbf{P}_k}$, $\bm{\mathbf{P}_k'}$ and $\bm{\Phi_k\:\mathbf{K}_k\to\mathbf{P}_k}$.}
Given a polyhedral space $P$,
denote by $[P]$ its isometry class;
thus $[P]=[Q]$ if and only if $P\iso Q$.
Let us denote by $\mathbf{P}_k$ the set of isometry classes of all polyhedral all polyhdral spaces homeomorphic to $\SS^2$  
with non-negative curvature 
and exactly $k$ vertices (recal that \textit{vertex} means \textit{point with strictly positive curvature}).

Given a polyhderal space $P$, 
denote by $\Verts P$ the set of vertices of  $P$.
Since $\Verts P$ is completely defined by $P$,
we can think of $P$ as a pair $(P,\Verts P)$, a compact metric space and its subset.

We need to equip $\mathbf{P}_k$ with a metric,
which is slightly mere sensitive than Gromov--Hausdorff metric.
This is described for the pairs $(P,\Verts P)$ in the Exercise~\ref{pr:GH-variation}.
For two polyhedral spaces $P$ and $P$,
we write $|[P]-[P']|_{\mathbf{P}_k}\le r$
if there is a metric  space $Z$ 
and distance preserving maps
$\iota\:P\to Z$ and $\iota'\:P'\to Z$
such that $|\iota(P)-\iota'(P')|_{\mathcal{H}(Z)}\le r$
and $|\iota(\Verts P)-\iota'(\Verts P')|_{\mathcal{H}(Z)}\le r$.

Note that two polyhedral spaces $P$ and $P'$ are considered to be close in $\mathbf{P}_k$
if there is an $\eps$-isometry $\iota\: P\to P'$
such that $\iota(A)$ is close to $A'$ in the sense of Hausdorff
as the subsets of $P'$.

Note that Gromov--Hausdorff metric and the metric described above 
give the same topology on $\mathbf{P}_k$.

\begin{thm}{Exercise}
Prove the last statement.
Namely, show that if $P_1,P_2,\dots$ and $P_\infty$ 
are polyhedral spaces with
exactly $k$-vertices
and 
$|P_n-P_\infty|_\mathcal{M}\to 0$ 
if and only if
$|P_n-P_\infty|_{\mathbf{P}_k}\to 0$
\end{thm}

Note that the completions of $\mathbf{P}_k$ 
taken for these two metrics are different.
Denote by $\bar{\mathbf{P}}_k$ the completion of the obtained metric space.









Elements in $\bar{\mathbf{P}}_k\backslash \mathbf{P}_k$
can be visualized as isometry classes of polyhedral spaces with 
smaller number of vertices and maybe with some marked points --- the limit points of the vertices with curvature convering to zero.
In particular, any polyhdral space with $k-1$ vertices
and one marked point is a limit point of a seqence of ...
To construct such a sequence choose any triangulation of ...
so with marked point as a vertex and one triangle wit ... 

The constructed metric
and Hausdorff metric
both define the same topology on $\mathbf{P}_k$,
but they have different completions.

To see the last statement note that the completion of consider a polyhedral metric with $k-1$ vetrices



Note that the space $\mathbf{P}_k$ is not complete.
As usual we denote by $\bar{\mathbf{P}}_k$ the completion of $\mathbf{P}_k$.
Note that $\bar{\mathbf{P}}_k$???


Let $K\in \mathbf{K}_k$,
denote by $P$ the surface of $K$.
Note that $P$ has exactly $k$ vertices;
i.e.,  $[P]$ belongs to $\mathbf{P}_k$.
Hence we obtain a map $\Phi_k\:\mathbf{K}_k\to \mathbf{P}_k$, defined as $K\mapsto [P]$.
Acording to Problem~\ref{pr:H>GH-boundary}, the map $\Phi_k$ is continuous.

We say that a polyhedral space $P$ \emph{has a simmetry} if there is an isometry $\iota\:P\to P$ which is distinct from indentity.
If there is no such isometry, we say that  $P$ has \emph{no simmetry}.
The subset of $\mathbf{P}_k$ formed by polhedra with no simmetry will be denoted as $\mathbf{P}_k'$.

According to Alexandrov's uniqueness theorem,
$$\Phi_k(K)=\Phi_k(K')\ \iff\ K\cong K'$$
for any $K,K'\in \mathbf{K}_k$.
According to Problem~\ref{pr:K-P-simmetry}, a convex polyhedron with no simmetry has surface with no simmetry; i.e., 
$$K\in   \mathbf{K}_k'\ \iff\ \Phi_k(K)\in \mathbf{P}_k'$$
for any $K\in \mathbf{K}_k$.

\begin{thm}{Exercise}
Let $X$ be a compact metric space.
Then for any $\eps>0$ there is $\delta>0$ such that 
if $f_0,_1\:Y\to X$ be two $\delta$-isometries then 
 there is an isometry $\iota\:X\to X$ such that
$|f_0(y)-\iota\circ f_1(y)|_X<\eps$ for any $y\in Y$.

In particular, if $X$ has no simmetry than 
$|f_0(y)-f_1(y)|_X<\eps$ for any $y\in Y$.
\end{thm}













\parit{Case of triagular faces and no symmetry.}
We first consider the ``generic'' case.
We assume  the $P$ has \emph{no symmetry};
i.e., the only isometry $\iota\: P\to P$ is the identyty map.
In addition we assume that the polyhedron $K$ has only triangular faces.

In this case $P$ can be triangulated by faces of $K$.
The edges of this triangulation are formed by unique geodesics between vertices of $P$.
Note that once we know the edge-lengths, 
we can reconstruct each triangle of the triangulation
and glue them according to the rule to obtain $P$.
In particular, these edge-lengths completely describe $P$.

Let us change the edge-lengths slightly 
and repeat the above construction.
We will obtain an other polyhderal space  say $P'$.
Note that $P'$ is still homeomorphic to the sphere.
Further, the total angle at each vertex of $P'$
is nearly the same as in $P$.
In particular, $P'\in\in\mathbf{P}_k$ has non-negative curvature.

On the other hand, if $[P']$ is sufficiently close to $[P]$
then $P'$ can be obtained as the result of the above construction.
It is sufficient to make a correspondance between vetexes of $P'$ and $P$
and triangulate $P'$ cutting it by geodesics bewteen ??? 
  
Let us start with two preliminary lemmas











\begin{thm}{Exercise}
Let $P\in\in \mathbf{P}_k$ and $v_1,v_2,\dots,v_k$ be the vertices of $P$.
Assume $P$ admits a triangulation $\mathcal{T}$ with vertices only at $v_i$.
Then for any $\eps>0$ there is $\delta>0$ such that if 
$|P-P'|_\mathcal{M}<\delta$ for some
$P'\in \mathbf{P}_k$ then
there is a homeomorphism $P\to P'$ which sends vertex to vertex and 
each triangle to a flat triangle in $P'$
and the difference between the length of edge ???
\end{thm}




much closer than any other pair of vertices of $P_1$ and $\omega_{p_1}+\omega_{q_1}<2\cdot \pi$
where $\omega_{p_1}$ and $\omega_{q_1}$ denote the curvatures of $P_1$ at $p_1$ and $q_1$.

More precisely, there is a deformation from $P_0$ to $P_1$, such that 
\begin{subthm}{}
$P_t\in\in \mathbf{P}_{k}$ for all $t$.
\end{subthm}

\begin{subthm}{lem:k>=4:pq}
It we denote by $p_1,q_1$
the vertices of $P_1$
corresponding to the vertices 
$p_0,q_0\z\in P_0$
then the distance $|p_1-q_1|_{P_1}$ is at least 10 times smaller than the distance between any other pair of vertices of $P_1$
and
$\omega_{p_1}+\omega_{q_1}<2\cdot \pi$
where $\omega_{p_1}$ and $\omega_{q_1}$ denote the curvatures of $P_1$ at $p_1$ and $q_1$.
\end{subthm}









\begin{thm}{Lemma}
Let $k\ge 4$ and $P\in\in \mathbf{P}_k$ and $p,q$ be two vertices of $P$ with curvatures $\omega_p$ and $\omega_q$.
Assume $\omega_p+\omega_q<2\cdot\pi$ and 
the distance $|p-q|_P$ is at least 10 times smaller than the distance between any other pair of vertices of $P$
then there is a deformation $P_t$ of $P$ such that
$P_t\in\in\mathbf{P}_k$ for $t<1$,
$P_1\in\in\mathbf{P}_{k-1}$.

More over if $W_t$ is a deformation 
such that $W_t\in\in \mathbf{P}_{k-1}$ 
then 
\end{thm}




\parbf{Observations.}
\begin{enumerate}
\item\label{obs:k-1} In the patching construction, we are free to choose the one parameter families of points $p_t$ and $q_t$ satisfying \ref{eq:<pqq_t}.
Note that if $\omega_p+\omega_q<2\cdot \pi$
then one can choose this family so that $p_1=q_1$;
in this case the quadrilateral $[\hat p p_1 q_1 \hat q]$
degenerates to a triangle and $P_1$ has $k-1$ vertices.

\item\label{obs:curvature} If $\omega_p+\omega_q\ge 2\cdot \pi$ then $p_t$ and $q_t$ stay apart and their curvatures 
$\omega_{p_1}$ and $\omega_{q_1}$ in $P_1$ 
can made arbitrary as far as $\omega_{p_1}, \omega_{q_1}>0$ and
$$
\omega_{p_1}+\omega_{q_1}
=
\omega_{p}+\omega_{q}.
$$ 
\item\label{obs:recursion} 

\item\label{obs:k}
Finally notice that if $P_t$ is a deformation of polyhedral spaces such that $P_t\in\in \mathbf{P}_{k-1}$
then one can apply an inverse of the patching construction to get a deformation $P'_t$ such that $P_t'\in\in \mathbf{P}_{k}$

To do this, choose arbitrary vertex $v_0$ in $P_0$,
and denote by $v_t$ be the corresponding vertex in $P_t$.
For each $t\in[0,1]$,
choose two points $x_t$ and $y_t$ such that
\begin{enumerate}
\item Both $x_t$ and $y_t$ depend continuously on $t$;
i.e., ???
\item Both $x_t$ and $y_t$
are sufficiently close to $v_t$;
say the distances $|v_t-x_t|_{P_t}$ and $|v_t-y_t|_{P_t}$ are at least twice smaller than the distance from $v_t$ to any other vertex of $P_t$.
\item the (necessary unique) geodesics 
$[v_tx_t]$ and $[v_ty_t]$ cut at $v$ two equal angles.   
\end{enumerate}
The points it the points satisfy the above properties then for each $t$ there are two geodesics from $x_t$ to $y_t$. 
One can cut out a neighborhood of $v$ 
along these geodesics.
in the remaining part of $P$,
glue the corresponding points 
\end{enumerate}
















Now choose 
\begin{itemize}
\item Fist we construct a one-parameter family of polyhedral spaces $P_t$ such that 
for each $t\in [0,1]$ the space $P_t$
has non-negative curvature and in the process of deformation of metric 
two vertices are cumming together and then spread apart again;
in particular the number of vertices 
of $P_t$ is either $n$ or $n-1$ for all $t\in[0,1]$.
In the proof of this part we use the induction hypothesis;
we deform construct a deformation of metrics on $P_0$ so that two vertices come together and becoming one vertex. 
Denote by $Q_0$ the obtained polyhedral space. 
The same way we deform $P_1$ into a polyhdral space $Q_1$ with $n-1$ vertices.
Finally we apply the induction hypothesis to deform $Q_0$ into $Q_1$.
Joining these three deformations we get a deformation from $P_0$ to $Q_0$ then to $Q_1$ and finally arrive to $P_1$.

\item Second we deform the obtained deformation so that in all intrmidiate times the metric has exactly $n$ vertices.
\end{itemize}







\begin{thm}{Lemma}\label{lem:path-to-(k-1)}
Assume $k\ge 4$. 
Then given  $P\in \mathbf{P}_k$, there is a family of polyhedral spaces $P_t$ with $t\in[0,t_{\max}]$
which is continuous with respect to the Gromov--Hausdorff metric 
and such that 
\begin{itemize}
\item $P_0=P$;
\item $P_t\in\mathbf{P}_k$ for $t<t_{\max}$;
\item $P_{t_{\max}}\in \mathbf{P}_{k-1}$.
\end{itemize}
\end{thm}

In other words, we can change $P$ continuously in $\mathbf{P}_k$ so that exactly one of the vertices disappear; i.e., its curvature will approach $0$ as $t\to t_{\max}$ and so the limit metric will lie in $\mathbf{P}_{k-1}$. 

Note that the induction hypothesis is that
$\Phi_{k-1}(\text{\bf P}_{k-1})=\mathbf{P}_{k-1}$,
and this lemma makes it possible to apply it.

Assume $P=P_0$ is
a non-negatively curved polyhedral space which is a manifold
and $p,q$ are two vertices in $P_0$. 
Let us describe a construction
which produce one-parameter family $P_t$,
of polyhedral spaces which is continuous 
in the sense of Gromov--Hausdorff;
i.e., for any $t$ in some real interval 
we construct a polyhedral space $P_t$ 
and
$$P_t\GHto P_{t_0}$$
once $t\to t_0$.



Now let us collect the corollaries from the above 
construction.

Assume that $k$, the number of vertices in $P$, is $\ge 5$.
Since the sum of all curvatures in $P$ is $4\cdot\pi$,
we always can choose $p$ and $q$ so that $\omega_p+\omega_q\le\tfrac8k\cdot\pi$.
In this case the above construction gives a non-negatively curved space for $t\le \tfrac{\omega_p}{2}$.
At the moment $t_0=\tfrac{\omega_p}{2}$,
the metric of $P$ has $k-1$ vertices;
the vertex $p$ degenerates to a point and the curvature of $p$ and $q$ is eaten by the new vertex $z$ in $P_{t_0}$.

According to the induction hypothesis,
$P_{t_0}$ is realizable;
i.e., there is a convex polyhedron $K$ with the surface isometric to $P_{t_0}$.

Our next aim is to show that 
for all $t$ close enough to $t_0$,
the space $P_t$ is realizable.

Let us triangulate each face of $K$,
we can do it in such a way that


The above construction can be applied recursively;
i.e.,
\begin{enumerate}
\item We may start with $P_0$, mark there two vertices $p$ and $q$, construct a family of polyhedral metrics $P_t$.
\item Stop at some time $t_0$, choose an other parir of vertices in $P_{t_0}$ and apply the construction again. 
\end{enumerate}

 


If 
$$\omega_p+\omega_q<2\cdot\pi,\eqlbl{eq:2curv<2pi}$$ 
the family $P_\alpha$ for $\alpha\in [0,\tfrac{\omega_p}2]$ satisfies the Lemma.

Recall that $p$ and $q$ are the vertices in $P$ with minimal possible curvature.
Since the sum of all curvatures in $P$ is $4\cdot\pi$,
we always have $\omega_p+\omega_q<2\cdot\pi$.
Therefore \ref{eq:2curv<2pi} holds for $k\ge 5$.

In the remaining case $k=4$, \ref{eq:2curv<2pi} does not hold if and only if curvature at each vertex is $\pi$.
In this case, we choose an arbitrary pair of vertices and apply the same construction as above to move our metric continuously into a metric such that the curvature at one vertex is strictly less than $\pi$.
For the obtained polyhedral space \ref{eq:2curv<2pi} holds.

In the latter case our family $P_t$ is joined from two families described above for two choices of pairs of vertices.
\qeds

Apply the lemma and set $Q=P_{t_{\max}}$.
Note that by the induction hypothesis, we can present $Q$ as the surface of a convex polyhedron with $k-1$ vertices, say $K$.

Mark a point $p$ on $K$ (the same notation as in the proof of the lemma).
The point $p$ lies on the side or face of some triangle of $\mathcal T$.
Note that if we move $p$ outside of $K$ a bit
then the convex hull $p$ and $K$ will form a convex polyhedron, say $K'$, 
with exactly $k$ vertices.
Clearly we can choose $K'$ to be arbitrarily close to $K$ and therefore its surface, say $Q'$ will be arbitrarily close to $Q$.
Clearly $Q'$ is realizable.
Therefore
it is sufficient to connect $Q'$ to some $P_t$ by a continuous family in $\mathbf{P}_k$.

Subdivide $\mathcal T$ to make $p$ one of its vertices.

Let $\ell_1,\ell_2,\dots,\ell_{3\cdot k-6}$ be the lengths of the edges of $\mathcal T$.
Let $\ell_1',\ell_2',\dots,\ell_{3\cdot k-6}'$ be a collection of real numbers, 
such that $\ell_i'$ is $\eps$-close to $\ell_i$ for each $i$ and some small enough $\eps>0$.
We can construct triangles in $\mathcal T$ with the side-lengths $\ell_i'$ instead of $\ell_i$;
denote by $Q(\ell_1',\ell_2',\dots,\ell_{3\cdot k-6}')$ the constructed polyhedral metric. 
If $\eps>0$ is small enough, triangle inequalities still hold in each triangle and the curvature of each vertex except $p$ is still positive.

\begin{wrapfigure}{r}{34mm}
\begin{lpic}[t(-5mm),b(0mm),r(-13mm),l(-15mm)]{pics/subgraph(0.3)}
\lbl[lb]{135,130;$\Gamma_f$}
\lbl{110,50;$\Gamma^-_f$}
\lbl[rb]{99,160;$q$}
\lbl{105,200;$V$}
\end{lpic}
\end{wrapfigure}

Without loss of generality, we may assume that $\ell_1$ is the length of the edge opposite from $p$ in some triangle.
Note that increasing $\ell_1$ while leaving all the remaining edges fixed decreases the curvature at $p$.

Thus the set $(\ell_1',\ell_2',\dots,\ell_{3\cdot k-6}')$ for which the curvature at $p$ is zero
is formed by a graph $\ell_1= f(\ell_2',\dots,\ell_{3\cdot k-6}')$ for some continuous real-valued function $f$ defined in a small neighborhood of $(\ell_2,\dots,\ell_{3\cdot k-6})\in\RR^{3\cdot k-7}$.
Above this graph curvature at $p$ is negative and below is positive.
Thus the condition (\ref{SHORT.lem:mapping:connect}) boils down to the following lemma.

\begin{thm}{Lemma}
Let $U\subset \RR^n$ be an open subset and $f\:U\to\RR$ be a continuous function.
Consider the graph and the open hypo-graph of $f$
$$
\Gamma_f
=
\set{(x,y)\in U\times \RR }{y=f(x)}
$$
$$
\Gamma^-_f
=
\set{(x,y)\in U\times \RR }{y<f(x)}
$$
Then any point $q\in\Gamma_f$ has an arbitrarily small neighborhood $V$ 
such that $V\cap \Gamma^-_f$ is connected.
\end{thm}


\begin{thm}{Exercise}
Look at the picture and
try to reconstruct the proof of this lemma.
\end{thm}










Note that from Alexandrov's theorem and Problem~\ref{pr:H>GH-boundary} we have the following:

\begin{thm}{Theorem}\label{thm:pre-body}
Let $P_\infty$ be a metric space.
Then $P_\infty$ is isometric to the surface of convex body in $\RR^3$
if and only if $P_\infty$ can be presented as Gromov--Hausdorff limit of a sequence of non-negatively curved polyhedral spaces $P_n$ homeomorphic to $\SS^2$.
\end{thm}

Assume you have  a metric space $P_\infty$ and you want to know if it is isometric to the surface of a convex body in $\RR^3$.
The above theorem can be applied easily if the answer is ``yes'';
it is seems to be harder to use this theorem to give a ``no'' answer to the question.
It turns out that non-negative curvature can be defined for general metric spaces
and this new definition will make it equally easy to answer either ``yes'' or ``no'' to the question.



The proof is using 
Theorem~\ref{thm:pre-body}, 
Proposition~\ref{prop:GH-lim(CBB)} and the following two propositions:

\begin{thm}{Proposition}\label{prop:poly-in-cbb}
Any non-negatively curved polyhedral space homeomorphic to%
\footnote{Instead of $\SS^2$, we could say homeomorphic to a two-dimensional manifold.} 
$\SS^2$ is a $\CBB{}{0}$ space. 
\end{thm}











From Proposition~\ref{prop:angk=<}, we get the following.

\begin{thm}{Corollary}\label{cor:angk-decres}
Let $X\in\CBB{}{0}$.
Given a hinge $\hinge p x y$ in $X$, consider the function of two arguments
$$\alpha\:(|p-\bar x|,|p-\bar y|)\mapsto\angk0 p {\bar x}{\bar y}$$
where 
$\bar x\in\left ]p x\right]$ and $\bar y\in\left]p y\right]$.

Then $\alpha(s,t)$ is nonincreasing in both arguments.
\end{thm}

From Corollary~\ref{cor:angk-decres}, we get the following.

\begin{thm}{Proposition}\label{prop:angle-mono}
Let $X\in\CBB{}{0}$.
Then for any hinge $\hinge p x y$ in $X$, the angle $\mangle\hinge p x y$ is defined.
\end{thm}











Since increasing one side of a planar triangle makes the opposite side bigger,
we get the following.

\begin{thm}{Proposition}\label{prop:angk=<}
The inequality \ref{eq:def:cbb} can be rewritten in the following way
\[
\angk0 x p z
\ge
\angk0 x y z.
\]

\end{thm}
















\section{Alexandrov's existence theorem}


\parit{Proof.}
A polyhedral space $P$ will be called \emph{realizable}\index{realizable polyhedral space}
if it is isometric to the surface of a convex polyhedron in $\RR^3$.
We need to show that
any non-negatively curved polyhedral space homeomorphic to sphere is realizable.

The proof is by induction on $k$,
where $k$ denotes the number of points in $P$ with positive curvature.
From now on, these points will be called \emph{vertices}\index{vertex} of $P$.

Clearly $k$ is finite (see Exercise~\ref{ex:non0curv}).
Further, $k\ge 3$ for any $P$.
Indeed, according to  Exercise~\ref{ex:sum=2pi}, 
the sum of the curvatures of $P$ is $4\cdot\pi$ 
and the other hand the curvature of each point in $P$ is strictly less than $2\cdot\pi$.


\parit{Base case; $k=3$.}
Assume that $P$ has exactly three vertices, say $u$, $v$, and $w$. 
It is sufficient to show that 
\begin{clm}{}\label{clm:doule-trig}
$P$ is isometric to a
doubling of a planar triangle.
i.e., the surface of this triangle in $\RR^3$.
\end{clm}

Choose geodesics  $[uv]$, $[vw]$ and $[wu]$ between each pair of points.
According to Exercise~\ref{ex:poly+geod}
these geodesics do not intersect each other at the interior points. 

Cut $P$ along the geodesics $[uv]$, $[vw]$ and $[wu]$.
As a result we get two triangles $\triangle$ and $\triangle'$.
Since all the remaining points have curvature $0$%???
,
both $\triangle$ and $\triangle'$ are planar
and 
they are congruent since they have the same side lengths.
Hence \ref{clm:doule-trig} follows.

\parit{Few definitions.} %%%%??? we have P\in M_k and K \in P_k --- should we find better letters???
Given a polyhedral space $P$ homeomorphic to $\SS^2$,
denote by $[P]$ its isometry class;
thus $[P]=[Q]$ if and only if $P\iso Q$.
Let us denote by $\text{\bf M}_k$ the set of isometry classes of all polyhedral metrics on $\SS^2$  with non-negative curvature and exactly $k$ vertices (recal that \textit{vertex} means \textit{point with strictly positive curvature}).
Equip $\text{\bf M}_k$ with Gromov--Hausdorff metric.

Given a convex polyhedron $K$ in $\RR^3$ denote by $[K]$ its congruence class;
thus $[K]=[L]$ if and only if $K\cong L$; i.e., if there is an isometry of $\RR^3$ which sends $K$ to $L$.
Denote by $\text{\bf P}_k$ the set of congruence classes of convex polyhedra in $\RR^3$
with exactly $k$ vertices.
Let us equip $\text{\bf P}_k$ with a metric.
Let $K$ be  a convex polyhedron with exactly $k$ vertices,
say $\{w_1,w_2,\dots,w_k\}$.
Clearly the set $W=\{w_1,w_2,\dots,w_k\}$ completely describe $K$ and vise versa.
Given an other convex polyhedron $K'$ with vetexes $W'=\{w'_1,w'_2,\dots,w'_k\}$
set $\Dist([K],[K'])$
to be \emph{Hausdorff metric up to isometry of $\RR^3$} between their sets of vertices $W$ and $W'$;
i.e.,
$$\Dist([K],[K'])=\inf_{\iota} |W-\iota(W')|_{\mathcal{H}(\RR^3)}.$$

Note that $\text{\bf P}_k$ can be considered as a subspace of the following metric space:
as the set we take the set of isometry classes of $k$-point sets in $\RR^3$ 
and the metric is  Hausdorff metric up to isometry of $\RR^3$.


Given a convex polyhedron $K$ with exactly $k$ vertices, consider its surface $P$.
According to Exercise~\ref{ex:curv-is-nonneg}, $P$ also has exactly $k$ vertices.
This way we construct a map $\Phi_k\:\text{\bf P}_k\to\text{\bf M}_k$
defined as $\Phi_k\:[K]\mapsto [P]$; i.e.,
$\Phi_k$ sends \textit{the congruence class of $K$} to \textit{the isometry class of $P$}.
Note that according to Problem~\ref{pr:H>GH-boundary}, $\Phi_k$ is a continuous map.

Clearly $P$ is realizable if and only if $[P]\in\Phi_k(\text{\bf P}_k)$.
Note that by Alexandrov's uniqueness theorem (\ref{thm:alexandrov-uni'}), $\Phi_k$ is injective.

To prove Alexandrov's existence theorem, 
it is sufficient to show that 
\begin{clm}{}\label{clm:Phi-sur}
$\Phi_k\:\text{\bf P}_k\to \text{\bf M}_k$ is surjective.
\end{clm}

\parit{Step.} The Claim~\ref{clm:Phi-sur} will be proved by induction.
The base case $k=3$ is already proved. 
To prove the induction step it is sufficient to prove the following three lemmas.
Their proofs will be given in the following sections.

Let us say that a polyhedral space $P$ has no simmetry if the identity map is the only isometry $P\to P$. 

\begin{thm}{Closed lemma}\label{lem:Phi(P)-closed}
$\Phi_k(\text{\bf P}_k)$ is a closed subset of $\text{\bf M}_k$.
\end{thm}

\begin{thm}{Open lemma}\label{lem:Phi(P)-open}
$\Phi_k(\text{\bf P}_k)$ is an open subset of $\text{\bf M}_k$.
\end{thm}


\begin{thm}{Connecting lemma}\label{lem:Phi(P)-connect}
Assume $\Phi_{k-1}\:\text{\bf M}_{k-1}\to \text{\bf P}_{k-1}$ is surjective.

Then given $[P]\in \text{\bf M}_{k}$,
there is one parameter family of polyhedral spaces $P_t$, $t\in [0,1]$
such that
$P_0=P$, $P_1$ is realizable and the map 
$t\mapsto [P_t]$ forms a curve in $\text{\bf M}_{k}$
\end{thm}

Let us finish the step of induction modulo these lemmas.

By induction hypothsis we can assume that 
$\Phi_{k-1}\:\text{\bf M}_{k-1}\to \text{\bf P}_{k-1}$ is surjective.
Assume there is a nonrealizable polyhedral space in $\text{\bf M}_{k}$.
Applying Connecting lemma (\ref{lem:Phi(P)-connect}), 
we get a one parameter family of polyhedral spaces in $P_t$, $t\in [0,1]$
such that
$P_0=P$, $P_1$ is realizable and the map 
$t\mapsto [P_t]$ forms a curve in $\text{\bf M}_{k}$.
Let $t_0$ be the supremum of all values $t\in [0,1]$
such that $P_t$ is not realizable.

From Closed lemma it follows that $P_{t_0}$ is realizable.
Further from Open lemma, it follows that $P_t$ is realizable for all $t$ sufficiently close to $t_0$.
This contradicts the way we defined $t_0$.
\qeds

\section{Proof of Closed lemma (\ref{lem:Phi(P)-closed})}

\parit{Proof.}
Let $P_1,P_2,\dots$ be a sequence of realizable polyhedral spaces.
Assume $P_n$ converges in the sense of Gromov--Hausdorff to $P_\infty$;
we need to show that $P_\infty$ is realizable.

Denote by $K_n$ a convex polyhedron in $\RR^3$ with surface isometric to $P_n$.
Without loss of generality we my assume that each $K_n$ contains the origin $0\in\RR^3$.

Since $P_n$ converges 
and diameter changes continuously with respect to Gromov--Hausdorff metric (see Exercise~\ref{ex:d_GH-and-diam}), 
the diameters of $P_n$ are uniformly bounded;
i.e., $\diam P_n\le D$ for some fixed $D$.
Note that 
$\diam K_n\le \diam P_n$
for each $n$.
In particular $\diam K_n\le D$ for any $n$.

Without loss of generality we may assume that each $K_n$ contains the origin $0\in\RR^3$.
In this case each $K_n$ lie in a closed ball of radius $D$ centered at the origin.
Pass to a convergent subsequence of $K_n$ in the Hausdorff metric;
it exists according to Blaschke's compactness theorem (\ref{thm:compact+Hausdorff}).

It is easy to see that the limit set $K_\infty$ is a convex polyhedron.
According to  Problem~\ref{pr:H>GH-boundary},
the surface of $K_\infty$ is isometric to $P_\infty$.
\qeds
















The following lemma implies that without loss of generality we may assume that $f$ is smooth.

\begin{thm}{Lemma}
Any distance nonexpanding map $f\:M\to\RR^m$ can be approximated by a smooth distance nonexpanding map.
\end{thm}

The needed approximation can be constructed by taking convolution of $f$.
(If you do not know what is convolution look in the book of ???.)


\parbf{Comments on the proof of Gromov's theorem.}
The above argument can be also applied to proof our partial case of Gromov's theorem (\ref{thm:gromov}).
One only needs to take $\eps_0$ sufficiently small,
choose the sequence $P_n$ so that $P_0$ is sufficiently closet to $\SS^2$
and construct a piecewise distance preserving map $f_0\:P_0\to \RR^2$ 
so that the composition $f_0\circ \phi_0$ is close enough to $f$.

According to the theorem on approximation (\ref{thm:approx}),
in order to construct $f_0$ it is sufficient to construct a piecewise linear 
distance nonexpanding map $h_0\:P_0\to \RR^2$ so that the composition $h_0\circ \phi_0$ is close enough to $f$.

A piecewise linear map $P_0\to \RR^2$ is uniquely determined by images of all the vertices
in a given triangulation. 
So given a vertex $v$, one can choose a value $h_0(v)$ in the set $(f\circ\psi^{-1})(v)$ and extend $h_0$ piecewise linearly to $P$.

In general the obtained map is not distance nonexpanding; see Problem~\ref{pr:long-triangle}.
It turns out however that if $f$ is smooth then given $\eps>0$ one can choose $P_0$ close enough to $\SS^2$
so the map $(1-\eps)\cdot h_0$ will be distance nonexpanding.
If $\eps$ is small $(1-\eps)\cdot h_0$ will be still close enough to $f$.

It remains to prove show the following lemma.



To prove this lemma, extend $f$ to $\RR^3$, say the following way 
$$\bar f(x)= f(\tfrac{x}{|x|}).$$
(The value $\bar f(0)$ is undefined, but we can still take the integral below.)
Choose a mollifier, i.e., a sufficiently smooth nonnegative function $w\:\RR^3\to \RR$
such that $w(x)=0$ if $|x|>\eps$ and $\int_{\RR^3}w=1$.
Consider convolution 
$$(\bar f*w)(x)
\df \int_{\RR^3} \bar f(y)\cdot w(x-y)\cdot dy$$
It is straightforward to check that the restriction of $\bar f*w$ to $\SS^2$ is distance nonexpanding 
and that for small $\eps$ it has to be close enough to $f$.










\begin{wrapfigure}{r}{35mm}
\begin{lpic}[t(-5mm),b(-10mm),r(0mm),l(0mm)]{pics/blind-zone(0.17)}
\lbl[b]{150,200;\color{white}{blind}}
\lbl[t]{150,190;\color{white}{zone}}
\lbl[t]{98,131;$a_n$}
\lbl[]{60,130;$\Omega$}
\end{lpic}
\end{wrapfigure}

 $\partial_{A}\Omega\subset \partial_{\RR^2}\Omega$.
Therefore it remains t
If $\Omega$ is unbounded then using property $(*)$ for $h$,
one can divide $\partial\Omega$ into finite number of rays and segments so that the restriction of $h$ to each is distance preserving.
We can apply the construction above to define $f$ on the part of $\Omega$ which is mareked blue on the picture.
But in addition we will have a ``blind zone'' formed by the angle (or few angles)  with vertex at $a_n$ which lie completely in $\Omega$.
Note that (the continuous extension of) $f$ maps each of the boundary rays 
by rotations around $a_n$.

Since $f$ is distance nonexpanding, the angle between the images of these rays 
is smaller than the corresponding angle of the blind zone.
It is then easy to extend $f$ as a piecewise distance preserving map into the blind zone
by folding the angle along one ray.























\parit{Proof.} First do the following exercise.

\begin{thm}{Exercise}\label{ex:alm-isom=>GH}
 Show that the following defines a metric on $W=X\sqcup Y$
\begin{enumerate}
\item  For any $x,x'\in X$
$$|x-x'|_W=|x-x'|_X;$$
\item For any $y,y'\in Y$,
$$|y-y'|_W=|y-y'|_Y$$
\item For any $x\in X$ and $y\in Y$,
$$|x-y|_W=\eps+\inf_{x'\in X}\{|x-x'|_X+|f(x')-y|_Y\}.$$
\end{enumerate}
\end{thm}

Since $f(X)$ is an $\eps$-net in $Y$,
for any $y\in Y$ there is $x\in X$ such that $|f(x)-y|_Y\z\le\eps$;
therefore $|x-y|_W\le 2\cdot\eps$.
On the other hand for any $x\in X$, we have $|x-y|_W\le\eps$
for $y=f(x)\in Y$.

It follows that the Hausdorff distance $d_{H}^W(X,Y)$ between $X$ and $Y$ in $W$ is at most $2\cdot\eps$.
Therefore
$$d_{GH}(X,Y)\le d_{H}^W(X,Y)\le 2\cdot\eps.$$
\qedsf













\section{Gromov--Hausdorff metric}

The goal of this section is to cook up a metric space out of metric spaces.
More precisely, we want to define the so called  Gromov--Hausdorff metric on the set of \emph{isometry classes} of compact metric spaces.
(Being isometric is an equivalence relation on the class of metric spaces, 
and an isometry class is an equivalence class with respect to this equivalence relation.)

Given two metric spaces, the Gromov--Hausdorff distance from the isometry class of $X$ 
to the isometry class of $Y$ will be denoted as $d_{GH}(X,Y)$;
but we will often say (not quite correctly) 
``$d_{GH}(X,Y)$ is the Gromov--Hausdorff distance from  $X$ 
to  $Y$''.
As you will see further, $d_{GH}(X,Y)=0$ if and only if $X$ is isometric to $Y$.
In other words, from now on, if I say ``metric space'',
you should guess from the context if I mean ``metric space'' 
or ``isometry class of this metric space'' (this is an abuse of notation).

Let us describe the idea behind
the definition. 
First, the Gromov--Hausdorff distance between subspaces in
the same metric space has to be no greater than the Hausdorff distance between
them. 
In other words, if two subspaces of the same space are close to each
other in the sense of Hausdorff distance in the ambient space, they must be
close to each other as abstract metric spaces. 
Second, we want
the distance between isometric spaces to be zero. 
The Gromov--Hausdorff
distance is in fact the maximum distance satisfying these two requirements.


\begin{thm}{Definition}\label{def:GH}
Let $X$ and $Y$ be compact metric spaces. 
The Gromov--Hausdorff $d_{GH}(X, Y )$
distance between them is defined by the following
relation.
 
Given  $r > 0$, we have that $d_{GH}(X, Y ) < r$ if and only if there exist a metric
space $Z$ and subspaces $X'$ and $Y'$ of it which are isometric to $X$ and $Y$
respectively and such that $|X'-Y'|_{\mathcal{H}(Z)} r$. 
(Here $|X'-Y'|_{\mathcal{H}(Z)}$ denotes the Hausdorff distance between sets $X'$ and $Y'$ in $Z$.)
\end{thm}

In other words, $d_{GH} (X, Y )$ is
the infimum of all $r>0$ for which the above $Z$, $X'$ and $Y'$ exist. 


We say that a sequence
of (isometry classes of) compact metric spaces $X_n$ 
\emph{converges in the sense of Gromov--Hausdorff}\index{converges in the sense of Gromov--Hausdorff} to the (isometry classes of)
compact metric space $X_\infty$ if $d_{GH}(X_n , X_\infty) \to 0$ as $n\z\to\infty$.



\begin{thm}{Theorem}\label{thm:GH-is-a-metric}
The set of isometry classes of compact metric spaces equipped with Gromov--Hausdorff metric forms a metric space.

This metric space will be denoted further as $\mathcal{M}$; named for ``metric space''.
\end{thm}

Before proving this theorem, we give couple of variations 
of the definition of Gromov--Hausdorff distance.

\subsection*{Metrics on disjoined union of \textit{X} and \textit{Y}}


Definition~\ref{def:GH} deals with a huge class of metric spaces,
namely, all metric spaces $Z$ that contain subspaces isometric to $X$ and $Y$.
It is possible to reduce this class to metrics on the disjoint unions of $X$ and $Y$. 
More precisely, 

\begin{thm}{Proposition}\label{prop:GH=X+Y}
The Gromov--Hausdorff distance between two compact metric spaces $X$
and $Y$ is the infimum of $r > 0$ such that there exists a metric
$|{*}-{*}|_W$ on the disjoint union $W=X\sqcup Y$ 
such that the restrictions of $|{*}-{*}|_W$ to $X$ and $Y$
coincide with $|{*}-{*}|_X$ and $|{*}-{*}|_Y$ 
and $|X-Y|_{\mathcal{H}(W)}  < r$. 
\end{thm}


\parit{Proof.}
Identify $X\sqcup Y$ with $X'\cup Y' \subset Z$ 
(the notation
is from Definition~\ref{def:GH}). 

More formally, fix isometries $f\: X \to X'$ and
$g\: Y \to Y'$, then define the distance between $x \in X$ and $y \in Y$ by
$|x-y|_W = |f (x)- g(y)|_Z+\eps$ for small enough $\eps>0$.%
\footnote{We add $\eps$ to ensure that $d(x, y) > 0$ for any $x\in X$ and $y\in Y$;
so $|x-y|_W$ is indeed a metric.}
This yields a metric on $W=X\sqcup Y$ for which
$|X-Y|_{\mathcal{H}(W)} < r$.
\qeds
 

\begin{thm}{Exercise}\label{ex:point-diam}
Let $P$ be a one-point metric space. 
Prove that 
$$d_{GH} (X, P ) = \frac{\diam X}2$$ for any compact metric space $X$.
\end{thm}

\begin{thm}{Exercise}\label{ex:d_GH-and-diam}
 Let $X$ and $Y$ be two compact metric spaces.
Prove that 
$$|\diam X - \diam Y |\le 2\cdot d_{GH} (X, Y ).$$
In other words, $\diam$ is a $2$-Lipschitz function on $\mathcal{M}$.
\end{thm}

\begin{thm}{Exercise}\label{ex:pack-GH}
Assume $X_n$ be a sequence of compact metric spaces which converges to a compact metric space $X_\infty$
in the sense of Gromov--Hausdorff.
Show that for any $\eps>0$
$$\pack_\eps X_n\ge\pack_\eps X_\infty$$ 
for all large enough $n$.
In particular, $\pack_\eps$ is a lower semicontinuous function on $\mathcal{M}$.
\end{thm}

\subsection*{A definition with fixed \textit{Z}}

\begin{thm}{Proposition}\label{prop:GH-with-fixed-Z}
In the Definition~\ref{def:GH}, 
one can fix the space $Z$ once for all, by taking $Z=\mathcal{F}(\NN)$%
\footnote{i.e., the space of bounded infinite sequences.}.  
That is, 
$$d_{GH}(X,Y) = \inf \{|X'-Y'|_{\mathcal{H}(\mathcal{F}(\NN))}\}$$ 
where the infimum is taken over all pairs of sets $X'$ and $Y'$ in $\mathcal{F}(\NN)$
which isometric to  $X$ and $Y$ correspondingly.  
\end{thm}

\parit{Proof.}
It is clear that $d_{GH}(X,Y) \leq \inf \{|X'-Y'|_{\mathcal{H}(\mathcal{F}_\NN)}\}$.  
Let $W$ be an arbitrary metric space with the underlying set $X\sqcup Y$ as in the proof of Proposition~\ref{prop:GH=X+Y}.
Note $W$ is compact since it is union of two compact subsets $X,Y\subset W$.
According to Problem~\ref{pr:compact->F_N},
$W$ admits a distance preserving map to $Z=\mathcal{F}_\NN$.
So $\inf \{|X-Y|_{\mathcal{H}(\mathcal{F}_\NN)}\} \leq |X-Y|_{\mathcal{H}(W)}$, and taking the infimum over all such $W$ gives $\inf \{|X-Y_{\mathcal{H}(\mathcal{F}_\NN)}\} \leq d_{GH}(X,Y)$.
\qeds

\begin{thm}{Exercise}\label{ex:euclid-isom}
Let $X,Y$ be two compact sets in the Euclidean plane $\RR^2$.
Show that $X$ is isometric to $Y$ if and only if there is an isometry $\iota\:\RR^2\to \RR^2$
which sends $X$ to $Y$.
\end{thm}

\begin{thm}{Exercise}\label{ex:mink-isom}
Find two isometric subsets $X,Y$ of $\mathcal{F}_\NN$
such that there is no isometry $\iota\:\mathcal{F}_\NN\to \mathcal{F}_\NN$ 
which sends $X$ to $Y$.
\end{thm}














\section{Gromov--Hausdorff metric is a metric}

In this section we prove  Theorem~\ref{thm:GH-is-a-metric}.


Let $X$, $Y$ and $Z$ be arbitrary  compact metric spaces.
We need to check the following (see Definition~\ref{def:metric-space}).
\begin{enumerate}[{\it (i)}]
\item\label{GH-1} $d_{GH}(X,Y)\ge 0$;
\item\label{GH-2} $d_{GH}(X,Y)=0$ if and only if $X$ is isometric to $Y$;
\item\label{GH-3} $d_{GH}(X,Y)=d_{GH}(Y,X)$;
\item\label{GH-4} $d_{GH}(X,Y)+d_{GH}(Y,Z)\ge d_{GH}(X,Z)$.
\end{enumerate}


Note that {\it (\ref{GH-1})}, {\it(\ref{GH-3})} and ``if''-part of {\it(\ref{GH-2})} follow directly from the definition of Gromov--Hausdorff metric (\ref{def:GH}).

\parit{Proof of (\ref{GH-4}).}
Choose arbitrary $a,b \in \mathbb{R}$ such that
$$a>d_{GH}(X,Y)\ \ \text{and}\ \  b>d_{GH}(Y,Z).$$
Choose two metrics on $U=X\sqcup Y$ and $V=Y\sqcup Z$ so that
$|X-Y|_{\mathcal{H}(U)}<a$ and $|Y-Z|_{\mathcal{H}(V)}<b$ 
and the inclusions $X\to U$, $Y\to U$, $Y\to V$ and $Z\to V$ are distance preserving.

Consider the metric on $W=X\sqcup Z$ 
so that inclusions $X\to W$ and $Z\to W$ are distance preserving
and 
$$|x-z|_W=\inf_{y\in Y}\{|x-y|_U+|y-z|_V\}.$$
Note that $|{*}-{*}|_W$ is a metric and 
$$d^W_H(X,Z)<a+b.$$
The last inequality holds for any $a>d_{GH}(X,Y)$ and $b>d_{GH}(Y,Z)$;
hence {\it (\ref{GH-4})}.
\qeds

\parit{Proof of ``only if''-part of (\ref{GH-2}).}
According to Exercise~\ref{ex:GH=>eps-isom},
for any sequence $\eps_n\to0^+$ there is a sequence of $\eps_n$-isometries 
$f_n\:X\to Y$.

Since $X$ is compact, 
we can choose a countable dense set
$S$ in $X$.
Use a diagonal procedure if necessary, to pass to a subsequence of $(f_n)$
such that for every $x \in S$ the sequence $(f_n(x))$ 
converges in $Y$. 
Consider the pointwise limit map  $f_\infty \: S \to Y$ defined by
 $$f_\infty(x) = \lim_{n\to\infty} f_n (x)$$ for every $x \in S$. 
Since $$|f_n (x)- f_n (x')|_Y\lege |x- x'|_X \pm\eps_n,$$ 
we have 
$$|f_\infty(x)-f_\infty (x')|_Y 
= \lim_{n\to\infty} |f_n(x)-f_n (x')|_Y 
= |x -x'|_X$$ for all
$x, x' \in S$; 
i.e., $f_\infty\:S\to Y$ is a distance-preserving map. 
Then $f_\infty$ can be extended to a distance-preserving map from all of $X$ to $Y$.
The later is done by setting 
$$f_\infty(x)=\lim_{n\to\infty} f_\infty(x_n)$$ 
for some (and therefore any) sequence of points $x_n$ in $S$
which converges to $x$ in $X$.
(Note that if $x_n\to x$ then $(x_n)$ is Cauchy.
Since $f_\infty$ is distance preserving, $y_n=f_\infty(x_n)$ is also a Cauchy sequence in $Y$;
therefore it converges.)

This way we obtain a distance preserving map $f_\infty\:X\to Y$. 
It remains to show that $f_\infty$ is surjective; i.e. $f_\infty(X)=Y$.

Note that in the same way we can obtain a distance preserving map $g_\infty\:Y\to X$.
If $f_\infty$ is not surjective, then neither is $f_\infty\circ g_\infty\:Y\to Y$.
So $f_\infty \circ g_\infty$ is a distance preserving map from a compact space to itself which is not an isometry.
The later contradicts Problem~\ref{pr:non-contracting=>isometry}. 
\qeds

