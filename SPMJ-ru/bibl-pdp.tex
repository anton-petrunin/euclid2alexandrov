\begin{thebibliography}{52}


\bibitem{akopyan}
\begin{otherlanguage}{russian}
Акопян А. В.,
PL-аналог теоремы Нэша --- Кейпера.
\textit{Одиннадцатый конкурс Мёбиуса.} (2007),
\end{otherlanguage}
\href{http://www.moebiuscontest.ru/files/2007/akopyan.pdf}{www.moebiuscontest.ru}

\bibitem{akopyan-tarasov}
Акопян, А. В.;   
Тарасов, А. С.,
Конструктивное доказательство теоремы Киршбрауна
\textit{Матем. заметки}, 84:5 (2008),  781--784


\bibitem{alexander} 
Alexander, R., 
Lipschitzian mappings and total mean curvature of polyhedral surfaces. I.
\textit{Trans. Amer. Math. Soc.} 288 (1985), no. 2, 661--678.

\bibitem{alexandrov} Александров А.Д. 
Выпуклые многогранники, 
Государственное издательство технико-теоретической литературы, 1950


\bibitem{arnold} 
Арнольд, В. И. \textit{Задачи Арнольда.} — Фазис, 2000.

\bibitem{bezdek-connelly} 
Bezdek, K.;  
Connelly, R., 
Pushing disks apart---the Kneser--Poulsen conjecture in the plane.  
\textit{J. Reine Angew. Math.}  553  (2002), 221--236.

\bibitem{brehm} Brehm, U., 
Extensions of distance reducing mappings to piecewise congruent mappings on $\RR^m$.  
\textit{J. Geom.}  16  (1981), no. 2, 187--193.

\bibitem{BBI} 
Бураго Д.Ю., 
Бураго Ю.Д., 
Иванов С.В. 
\textit{Курс метрической геометрии.}
Москва-Ижевск: Институт компьютерных исследований, 2004. 

\bibitem{burago-zalgaller-0}
Бураго, Ю. Д., 
В. А. Залгаллер. Реализация разверток в виде многогранников. \textit{Вестник ЛГУ}, 15, no. 7, (1960): 66--80.

\bibitem{burago-zalgaller}
Бураго, Ю. Д.;  
Залгаллер, В. А.,
Изометрические кусочно-линейные погружения двумерных многообразий с полиэдральной метрикой в $\RR^3$.
\textit{Алгебра и анализ}, 7:3 (1995),  76--95

\bibitem{valentine} Valentine, F. A.,  
A Lipschitz Condition Preserving Extension for a Vector Function. \textit{American Journal of Mathematics}, (1945) 67 (1) 83--93.

\bibitem{gromov}  Громов М. \textit{Дифференциальные соотношения с частными производными.} М. Мир 1990.

\bibitem{DGK} 
Данцер Л., 
Грюнбаум Б., 
Кли В., 
\textit{Теорема Хелли и ее применения}, М., изд-во «Мир», 1968. 

\bibitem{zalgaller-PL} 
\begin{otherlanguage}{russian}
Залгаллер, В. А.,
Изометричекие вложения полиэдров,
\textit{Доклады АН СССР},
123 (1958) 599--601.

\bibitem{kirszbraun} Kirszbraun, M. D., 
\"Uber die zusammenziehende und Lipschitzsche Transformationen. 
\textit{Fund. Math.} 22 (1934) 77--108.

\bibitem{krat} Krat, S., 
\textit{Approximation Problems in Length Geometry,} 
Thesis, 2005, PSU.

\bibitem{lang-secrets} Lang, R. J., 
\textit{Origami Design Secrets: Mathematical Methods for an Ancient Art},
CRC Press, Boca Raton, FL, 2012.

\bibitem{eliashberg-mishachev} Мишачев Н.M., Элиашберг Я.М., \textit{Введение в \textit{h}-принцип.} 
МЦНМО, 2004.


\bibitem{lang} 
Montroll, J.;  
Lang, R. J., 
\textit{Origami Sea Life},
Dover Publications, 1991.

\bibitem{petrunin-origami}  
Петрунин, А., 
Плоское оригами и построения. 
\textit{Квант}, 2008, № 1, 38--40

\bibitem{petrunin-ruble}  
\begin{otherlanguage}{russian}
Петрунин, А., 
Плоское оригами и длинный рубль.
\textit{Задачи Санкт-петербургской олимпиады школьников по матаматике}, 2008, 116--125; \textit{arXiv:1004.0545}
\end{otherlanguage}

\bibitem{petrunin-inverse} 
Петрунин, А.,
Внутренние изометрии в евклидово пространство.
\textit{Алгебра и анализ}, 22:5 (2010),  140--153.

\bibitem{petrunin-yashinsky-arXiv} 
Petrunin, A.; 
Yashinski, A.,
Lectures on piecewise distance preserving maps.
\texttt{arXiv:1405.6606}.

\bibitem{saraf}  Saraf, S., 
Acute and nonobtuse triangulations of polyhedral surfaces.
\textit{European Journal of Combinatorics.} 
30 (2009), Issue 4, Pages 833--840

\bibitem{TF} 
Табачников С.Л., Фукс Д.Б.,
\textit{Математический дивертисмент.} 
МЦНМО, 2011. 

\bibitem{tarasov} 
\begin{otherlanguage}{russian}
Тарасов, А. С.,
Решение задачи Арнольда «о мятом рубле».
\textit{Чебышевский сборник},  
5, (2004), выпуск 1, 174--187.
\end{otherlanguage}

\bibitem{hull} Hull, T. C.,
Solving cubics with creases: the work of Beloch and Lill, 
\textit{Amer. Math. Monthly} 118 (2011), no. 4, 307--315.

\bibitem{yashenko} Yashenko, I.,
Make your dollar bigger now!!! 
\textit{Math. Intelligencer} 20 (1998), no. 2, 38--40


\end{otherlanguage}

\end{thebibliography}
