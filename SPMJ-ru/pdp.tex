\section{Теорема Залгаллера}
\addtocontents{toc}{\protect\begin{quote}}
\addtocontents{toc}{Любое полиэдральное пространство допускает кусочно изометрическое отображение в евклидово пространство той же размерности.}

Пусть $P$ есть полиэдральное пространство.
Отображение $f\:P\to\RR^n$ называется \emph{кусочно изометрическим} если $P$ допускает триангуляцию такую, что 
для каждого её симплекса $\Delta$, 
сужение $f|_\Delta$ изометрично.

\begin{thm}{Упражнение}\label{ex:PDP=>cont}
Докажите, что любое кусочно изометрическое отображение непрерывно, сохраняет длины и следовательно короткое.
\end{thm}

Следующее утверждение может показаться очевидным,
попробуйте тем не менее дать формальное доказательство. 

\begin{thm}{Упражнение}\label{ex:n=<m}
Предположим, что
$m$-мерное полиэдральное пространство допускает кусочно изометрическое отображение в $\RR^n$.
Докажите, что $n\ge m$.
\end{thm}

Обратное утверждение к этому упражнению оказывается так же верно. 
Другими словами размерность является единственным препятствием к существованию кусочно изометрического отображения в евклидово пространство.
Следующая теорема утверждает, что это верно при $m=2$.

\begin{thm}{Теорема Залгаллера}\label{thm:zalgaller}
Любое двумерное полиэдральное пространство допускает кусочно изометрическое отображение в плоскость.
\end{thm}

Попробуйте склеить бумажную модель двумерного полиэдрального пространства, 
скажем поверхности куба, и разгладить её на столе.
По сути это и есть конструкция кусочно изометрического отображения.
Чтобы приблизиться к настоящему определению, 
нужно представить, что слои бумаги способны проходить друг сквозь друга.
Теорема Залгаллера говорит, что такое «разглаживание» всегда возможно.
  
Следующее упражнение показывает, что при таком разглаживании 
может потребоваться сгибать треугольники данной триангуляции,
см. также упражнения~\ref{ex:black-and-white} и \ref{pr:6-4-3-4}.

\begin{thm}{Упражнение}\label{pdp-for-tetrahedron}
Пусть $\Delta$ есть тетраэдр в $\RR^3$
и пусть $\partial\Delta$ обозначает его границу с индуцированной внутренней метрикой;  
это полиэдральное пространство склеенное из четырёх треугольников, граней $\Delta$.

Покажите что $\partial\Delta$ не допускает отображения в плоскость 
которое изометрично на каждой из четырёх граней $\Delta$.

Опишите явно кусочно изометрическое отображение
$$f\:\partial\Delta\to\RR^2$$
которое изометрично на двух из четырёх гранях $\Delta$. 
(Другие две грани должны быть подразбиты на меньшие треугольники.)
\end{thm}

Мы приводим два похожих доказательства теоремы Залгаллера: 
второе доказательство полное, первое не совсем.
В первом мы используем следующее утверждение без доказательства.


\begin{clm}{}\label{clm:acute-triangulation}
Произвольное двумерное полиэдральное пространство
допускает триангуляцию остроугольными треугольниками.
\end{clm}

\begin{thm}{Упражнение}\label{ex:acute-triangulation}
Покажите, что любой треугольник допускает триангуляцию остроугольными треугольниками. 
\end{thm}

Доказательство \ref{clm:acute-triangulation} дано в \cite{saraf},
см. также \cite{burago-zalgaller-0}.
Заметьте, что утверждение \ref{clm:acute-triangulation} не следует сразу из упражнения \ref{ex:acute-triangulation}.
В доказательстве \ref{clm:acute-triangulation},
кроме подразбиения каждого треугольника триангуляции 
надо следить за тем чтобы эти подразбиения подразбиения 
подходили друг-другу на рёбрах.


\parit{Доказательство с использованием \ref{clm:acute-triangulation}.}
Согласно \ref{clm:acute-triangulation},
$P$ можно триангулировать остроугольными треугольниками.
Обозначим эту триангуляцию $\mathcal{T}_0$.
Покрасим все вершины $\mathcal{T}_0$ белым
и обозначим их $\{w_1,\dots,w_k\}$.

Для каждой вершины $w_i$, рассмотрим её \emph{область Ворноного} $V_i$,
то есть подмножество
$$V_i=\set{x\in P}{|x-w_i|\le|x-w_j|\ \text{для любого}\ j}.$$
Обозначим через $S(w_i)$ \emph{звезду} вершины $w_i$,
то есть объединение всех симплексов в $\mathcal{T}_0$ содержащих $w_i$.

\begin{wrapfigure}[12]{r}{48mm}
\begin{lpic}[t(-5mm),b(-3mm),r(0mm),l(0mm)]{pics/zalgaller(1)}
\end{lpic}
\caption*{Области Ворноного\\ внутри треугольника.}
\end{wrapfigure}
\label{pic:voron}

Поскольку остроугольный треугольник содержит центр описанной окружности, область Вороного вершины треугольника не может содержать точек на его противоположной стороне.  
Отсюда следует, что 
$$V_i\subset S(w_i)$$ 
для любого $i$.
В частности, для любой точки $x\in V_i$, 
существует единственная кратчайшая $[w_i,x]$,
это прямолинейный отрезок в одном из треугольников или на ребре $\mathcal{T}_0$.

Заметим, что внутри каждого треугольника $\mathcal{T}_0$, 
есть единственная точка в которой сходится три области Вороного это центр описанной окружности
и в середине каждой стороны треугольника сходятся две области.
Разделим каждое ребро в $\mathcal{T}_0$ попалам
и разобьём каждый треугольник на $6$ треугольников как показано на рисунке (только сплошные линии).

Таким образом мы получаем новую триангуляцию $\mathcal{T}_1$. 
Покрасим все новые вершины в $\mathcal{T}_1$ чёрным.
Заметим, что
\begin{enumerate}
\item Каждая область $V_i$ есть объединение всех треугольников и рёбер в $\mathcal{T}_1$ с одной вершиной в $w_i$.
\item У каждого треугольника в $\mathcal{T}_1$ одна вершина белая и две чёрных.
\item\label{prop:cong-pairs} 
Пары треугольников в $\mathcal{T}_1$ с двумя общими чёрными вершинами равны.
\end{enumerate}

Для точки $x\in P$,
рассмотрим величину
$$\rho(x)=\min_i\{|w_i-x|\}.$$
Заметим, что если $x \in V_i$ то $\rho(x) = |w_i - x|$.
Далее, если $x\in V_i$, 
обозначим через $\theta_i(x)$ минимальный угол между кратчайшей $[w_i,x]$ 
и ребром в $\mathcal{T}_1$ выходящим из $w_i$.

Согласно свойству~\ref{prop:cong-pairs},  
если $x\in V_i\cap V_j$ то $\theta_i(x)=\theta_j(x)$.
Иными словами, функция $\theta$ определённая как
$$\theta(x) = \theta_i(x)\ \ \text{для}\ \  x \in V_i$$ 
определена и непрерывна в дополнении $P\backslash\{w_1,\dots,w_n\}$.

Теперь мы готовы описать отображение $f\:P\to\RR^2$ используя полярные координаты на плоскости.  
Положим
$f(w_i)=0$ для любого $i$
и  
$$f(x)=(\rho(x),\theta(x))$$ 
для $x \in P\backslash\{w_1,\dots,w_n\}$.

Подразбиение каждого треугольника вдоль биссектрисы при белой вершине (пунктирные линии на картинке выше)
даёт суперновую триангуляцию  $\mathcal{T}_2$.
По построению, $f$ изометрично на каждом треугольнике $\mathcal{T}_2$.
\qeds


Теперь мы изменим предыдущее доказательство
так чтобы оно не опиралось на
утверждение \ref{clm:acute-triangulation}.
Заметим, что включение $V_i\subset S(w_i)$ было единственным свойством остроугольной триангуляции которым мы пользовались.
С одной стороны, это включение не выполняется для произвольной триангуляции.
Пример можно построить подклеив треугольник к длинной стороне тупоугольного треугольника.
С другой стороны,
как будет видно в доказательстве ниже, 
увеличив число областей Вороного,
можно добиться того, что включения $V_i\subset S(w_i)$ выполняются без требования остроугольности.

\begin{wrapfigure}{r}{56mm}
\begin{lpic}[t(-6mm),b(-0mm),r(0mm),l(0mm)]{pics/voronoi(1)}
\end{lpic}
\caption*{Треугольник $\Delta$ в $\mathcal{T}_0$
с белыми точками пересечениями с областями Вороного.}
\end{wrapfigure}

\parit{Полное доказательство.}
Выберем произвольную триангуляцию $\mathcal{T}_0$ полиэдрального пространства $P$.
Мы построим новые триангуляции $\mathcal{T}_1$ и $\mathcal{T}_2$
подразбивая $\mathcal{T}_0$
и определим отображение $f\:P\to\RR^2$ 
которое изометрическое на каждом треугольнике в $\mathcal{T}_2$.
Вершины триангуляции $\mathcal{T}_1$ 
будут раскрашены в белый и чёрный цвета 
так что каждый треугольник в  $\mathcal{T}_1$ 
будет иметь две чёрных и одну белую вершину.

Сначала опишем множество белых вершин.

Выберем $\eps>0$.
Покрасим белым все вершины в $\mathcal{T}_0$
а также все точки на сторонах треугольников в $\mathcal{T}_0$
на расстоянии кратным $\eps$ до ближайшего конца.
Таким образом мы покрасили конечное число точек,
обозначим их через $w_1,\dots,w_k$.

Пусть $V_i$ есть область Вороного $w_i$, 
то есть
\[V_i=\set{x\in P}{|x-w_i|\le|x-w_j|\ \text{для любого}\ j}.\]
Обозначим через $S(w_i)$ звезду $w_i$ в $\mathcal{T}_0$, 
то есть объединение всех симплексов в $\mathcal{T}_0$ содержащих $w_i$.
Согласно следующему упражнению,
для достаточно малого  $\eps>0$,
вклучение
$V_i\subset S(w_i)$ выполняется при всех $i$.


\begin{thm}{Упражнение}\label{ex:voronoi-in-star}
Пусть $\ell$ есть минимальная длина ребра в $\mathcal{T}_0$
и $\alpha$ есть минимальный угол во всех треугольниках $\mathcal{T}_0$.
Предположим $\eps<\tfrac{\ell\cdot \alpha}{100}$.
Покажите, что 
то $V_i\subset S(w_i)$ для каждого $i$.
\end{thm}

Зафиксируем треугольник $\Delta$ в $\mathcal{T}_0$.
Заметим, что если $w_i\in \Delta$ то пересечение $V_i\cap \Delta$ есть выпуклый многоугольник.
Последнее утверждение следует из того, 
что для любой пары белых точек  $w_i,w_j\in\Delta$,
неравенство 
$$|x-w_i|\le |x-w_j|$$ 
описывает множество точек $x \in \Delta$
с одной стороны от срединного перпендикуляра к $w_i$ и $w_j$.
Покрасим чёрным все вершины всех многоугольников  $V_i\cap \Delta$ если они ещё не покрашены белым.

В случае если $\mathcal{T}_0$ 
сдержит ребро, которое не является стороной треугольника,
то покрасим чёрным его середину.  

Наша следующая цель описать триангуляцию $\mathcal{T}_1$.  
Чёрные и белые точки описанные выше станут вершинами в $\mathcal{T}_1$.
Белая вершина $w_i$ соединяется ребром с каждой чёрной вершиной $V_i$.
Пара чёрных вершин $b$ и $b'$ соединены ребром если они являются соседними вершинами в $V_i\cap \Delta$ 
для некоторого $\Delta$ и $w_i\in\Delta$.
В этом случае $w_i$, $b$ и $b'$ также образуют треугольник в $\mathcal{T}_1$.  
Отметим что каждое чёрно-чёрное ребро является стороной двух равных треугольников в $\mathcal{T}_1$ с различными белыми вершинами.

Остаток доказательства не отличается от предыдущего.
Пусть
$$\rho(x)=\min_i\{|w_i-x|_P\}.$$
Определим $\theta(x)$ для $x\in V_i$ как минимальный угол между кратчайшей $[w_i,x]$ и любым ребром 
в $\mathcal{T}_1$ выходящим из $w_i$.
Далее определим отображение $f\:P\to \RR^2$ 
так что 
$f(w_i)=0$ для любого $i$ 
и
$f(x)=(\rho(x),\theta(x))$ в полярных координатах.

Подразбивая каждый треугольник в $\mathcal{T}_1$
на два вдоль биссектрисы из белой вершины получаем суперновую триангуляцию $\mathcal{T}_2$.
По построению, отображение  $f$ изометрическое на каждом треугольнике $\mathcal{T}_2$.
\qeds

Следующее упражнение легко следует из теоремы Залгаллера

\begin{thm}{Продвинутое упражнение}\label{ex:zalgalle+embedding}
Любое двумерное полиэдральное пространство изометрично подлежащему пространству некоторого симплициального комплекса  в $\RR^n$, 
с индуцированной внутренней метрикой. 
\end{thm}

В заключении ещё одно развлекательное упражнение.

\begin{thm}{Упражнение}\label{ex:black-and-white}
Пусть $\mathcal{T}$ есть триангуляция выпуклого многоугольника $Q$
и каждый треугольник в $\mathcal{T}$ покрашен в чёрный или белый.
Докажите, что следующие два условия эквивалентны.
\begin{enumerate}[a)]
\item Существует кусочно изометрическое отображение
$Q\to\RR^2$ для этой триангуляции которое сохранят ориентацию
на каждом белом треугольнике и обращает ориентацию на каждом чёрном.

\item Сумма чёрных углов вокруг каждой вершины в $\mathcal{T}$ 
которая лежит во внутренности $Q$
равна $0$, $\pi$ или $2\cdot\pi$.
\end{enumerate}

\end{thm}

\addtocontents{toc}{\protect\end{quote}}
\section{Теорема Брэма о продолжении}
\addtocontents{toc}{\protect\begin{quote}}
\addtocontents{toc}{Любое короткое отображение из конечного множества точек евклидова пространства в то же евклидово пространство можно продолжить 
на всё пространство до кусочно изометрического отображения.}


\begin{thm}{Теорема Брэма о продолжении}\label{thm:brehm}
Пусть $a_1,\dots,a_n$ и $b_1,\dots,b_n$ 
две коллекции точек на плоскости таких, что 
$$|a_i-a_j|\ge |b_i-b_j|$$
для всех $i$ и $j$,
и $A$ есть выпуклый многоугольник содержащий все точки $a_1,\z\dots,a_n$.
Тогда существует кусочно изометрическое отображение  $f\:A\z\to \RR^2$
такое, что
$f(a_i)\z=b_i$ для всех $i$.
\end{thm}

Другими словами, 
если $F=\{a_1,\dots,a_n\}$ есть конечное множество в выпуклом многоугольнике $A$, 
тогда любое короткое отображение $\phi\: F \to \RR^2$ 
продолжается до кусочно изометрического отображения $f\: A \to \RR^2$.

\parit{Доказательство.}
Доказательство проводится индукцией по $n$.

База $n=1$:
мы можем взять отображение
$$f(x) = x + (b_1 - a_1),$$ 
которое сохраняет расстояния между любыми точками $A$.

Применим предположение индукции к последним $n-1$ парам точек. 
Мы получаем кусочно изометрическое отображение $h\:A\z\to \RR^2$ 
такое, что $h(a_i)\z=b_i$ для всех $i>1$.
Отображение $h$ будет использовано далее при построении нужного нам отображения $f\:A\to \RR^2$.

Рассмотрим множество
$$\Omega=\set{x\in A}{|a_1-x|<|b_1-h(x)|}.$$
Можно предположить, что $a_1 \in \Omega$;
иначе $h(a_1)= b_1$ и значит мы могли бы взять $f = h$.  

Докажем следующее утверждение.

\begin{clm}{}\label{clm:star-shaped}
Множество $\Omega$ звёздно относительно $a_1$.
То есть, если 
$x\in \Omega$ то отрезок $[a_1,x]$ лежит в $\Omega$. 
\end{clm}

Действительно, если $y\in [a_1,x]$ тогда 
$$|a_1-y|+|y-x|=|a_1-x|.
$$
Поскольку $x\in\Omega$, мы получаем
$$|a_1-x| < |b_1-h(x)|.
$$
Поскольку $h$ короткое (см. Упражнение~\ref{ex:PDP=>cont}),
получаем что
$$|h(x)-h(y)|\le |x-y|.
$$
По неравенству треугольника,
\begin{align*}
|a_1-y| &= |a_1-x| - |x-y|
<
\\
&< |b_1-h(x)| - |h(x)-h(y)|
\le
\\
&\le
|b_1-h(y)|. 
\end{align*}
То есть $y\in\Omega$,
отсюда \ref{clm:star-shaped}.

\medskip

\begin{center}
\begin{lpic}[t(-0mm),b(-0mm),r(0mm),l(0mm)]{pics/brehm-new(1)}
\lbl{55,55;\Large{$A$}}
\lbl[lb]{45,35;$E_i$}
\lbl[]{38,33;$T_i$}
\lbl[tr]{29,28;{\color{white}$a_1$}}
\lbl[]{15,28;{\color{white}\Large{$\Omega$}}}
\lbl[b]{10,36,25;\Large{$\partial_A\Omega$}}
\lbl[b]{-1,15,90;\Large{$Z$}}
\lbl[b]{15,10;{\color{white}слепая}}
\lbl[t]{15,9;{\color{white}зона}}
\end{lpic}
\end{center}

Напомним, что $\partial_A\Omega$ обозначает границу $\Omega$ относительно $A$.  
Относительная граница $\partial_A\Omega$ может отличается от $\partial_{\RR^2}\Omega$.
Заметим, что
$$|a_1-x|=|b_1-h(x)|
\eqlbl{eq:|ax|=|ah(x)|}$$
для любого $x\in\partial_A\Omega$.
Чтобы это увидеть, следует рассмотреть последовательность точек в $\Omega$ 
которая сходится к $x$ 
и другую последовательность точек в $A\backslash \Omega$ которая сходится к $x$, и далее воспользоваться непрерывностью $h$ (см. Упражнение~\ref{ex:PDP=>cont}).

Далее заметим следующее:
\begin{clm}{}\label{clm:broken-line}
Относительная граница $\partial_A\Omega$ образована конечным числом отрезков
$E_1,\z\dots, E_k$, которые пересекаются только в общих концах. 
Более того, $h$ изометрично на каждом таком отрезке.
\end{clm}

Действительно, зафиксируем триангуляцию $A$ такую, что $h$ изометрично на каждом треугольнике.
Пусть $x\in \partial_A\Omega$.
Выберем треугольник $\Delta\ni x$ в триангуляции такой, что  $\Delta\cap\Omega\ne\emptyset$.
Поскольку $h$ изометрично на $\Delta$, 
сужение $h|_\Delta$ можно продолжить однозначно до изометрии 
$\iota\:\RR^2\z\to\RR^2$.

Введём обозначение $b_1'=\iota^{-1}(b_1)$.
Так как $\iota$ изометрия и $\iota|_\Delta = h|_\Delta$,
имеем
\[|b_1'-x|=|b_1-h(x)|\] 
для любого $x\in\Delta$.

Заметим, что $a_1\ne b_1'$.
В противном случае, 
мы бы получили 
$$|a_1-x|=|b_1'-x|=|b_1-h(x)|$$ для любого $x\in\Delta$, 
последнее противоречит тому, что $\Delta\cap\Omega=\emptyset$.

Обозначим через $\ell_\Delta$ срединный перпендикуляр к отрезку $[a_1, b_1']$.
Пусть  $x\in \Delta$.
Заметим, что 
$x\in\Omega$ тогда и только тогда, когда $x$ и $a_1$ лежат с одной стороны от $\ell_\Delta$.
Таким образом $\partial_A\Omega$ есть объединение отрезков $\Delta\cap\ell_\Delta$ для всех таких $\Delta$. 
Поскольку число треугольников конечно,
получаем \ref{clm:broken-line}.

Для каждого ребра $E_i$ в  $\partial_A\Omega$, 
рассмотрим треугольник $T_i$ с вершиной $a_1$ и основанием $E_i$.
Условие \ref{eq:|ax|=|ah(x)|} влечёт существование изометрии плоскости $\iota_i$ такой, что $\iota_i(a_1)= b_1$
и  $\iota_i(x)=h(x)$ для любой точки $x\in E_i$.

Определим
$f(x)=h(x)$ для любого $x\notin\Omega$
и $f(x)=\iota_i(x)$ для каждого $i$ и любого $x\in T_i$.
Таким образом мы определили $f$ на $A\backslash \Omega$ 
и на всех отрезках с одним концом в $a_1$ а другим на $\partial_A\Omega$.

Заметим, что если $\partial_{A}\Omega=\partial_{\RR^2}\Omega$,
то мы определили $f$ на всём $A$.
Если $Z\z=\partial_{\RR^2}\Omega\backslash\partial_{A}\Omega$ не пусто, то точки лежащие на отрезках между $a_1$ и точками на $Z$ 
образуют «слепую зону» --- это подмножество $A$ на котором остаётся доопределить $f$.

Заметим, что замыкание слепой зоны можно представить как объединение конечного числа выпуклых многоугольников,
скажем $Q_1,\dots, Q_m$, 
которые пересекаются только в общей вершине $a_1$.
Каждый многоугольник $Q_i$ ограничен ломанной лежащей в замыкании $Z$ 
и двух отрезков из $a_1$ к концам ломанной. 

Пока что, отображение $f$ определено только на двух сторонах при $a_1$ у каждого $Q_i$.
По построению $f$ короткое и на каждой из этих двух сторон оно является изометрическим.
Из упражнения ниже следует, что 
$f$ можно продолжить на весь $Q_i$ сохраняя его кусочно изометрическим.
\qeds

\begin{thm}{Упражнение}\label{ex:triangle-reflect}
Пусть $Q=[a_1x_1\dots x_k]$ есть выпуклый многоугольник и $b_1$, $y_1$, $y_k$ некоторые точки на плоскости.
Предположим, что 
\begin{align*}
|b_1-y_1|&=|a_1-x_1|,&
|b_1-y_k|&=|a_1-x_k|,&
|y_1-y_k|&\le|x_1-x_k|.
\end{align*}
Тогда существует кусочно изометрическое отображение 
$f\:Q\to \RR^2$ такое, что  $f(x_1)\z=y_1$, $f(x_k)= y_k$ и $f(a_1)= b_1$.
\end{thm}

Мы заканчиваем лекцию дополнительными упражнениями.

\begin{thm}{Упражнение}\label{pr:perimeter}
Пусть $a_1,\dots,a_n$ и $b_1,\dots,b_n$ 
есть две коллекции точек на плоскости такие, что
$$|a_i-a_j|\ge |b_i-b_j|$$
для всех $i$ и $j$.
Пусть $A$ 
и $B$ их выпуклые оболочки.
Докажите, что 
$$\per (A)\ge \per (B),$$
где $\per (A)$ обозначает периметр  $A$.

Верно ли что всегда выполняется
$$\area (A)\ge \area (B),$$
где $\area (A)$ обозначает площадь  $A$.

\end{thm}

Следующее упражнение есть двумерный случай теоремы Александера 
\cite{alexander}.
Оно имеет очень простое решение, но при этом играет заметную роль в комбинаторной геометрии,
отличный пример статья Бездека и Конелли \cite{bezdek-connelly}.

\begin{thm}{Продвинутое упражнение}\label{pr:alexander}
Пусть $a_1,\dots,a_n$ и $b_1,\dots,b_n$ 
есть два набора точек на плоскости $\RR^2$.
Рассмотрим $\RR^4$ как $\RR^2 \times \RR^2$, 
Давайте думать про $\RR^2$ как про координатную плоскость  $\RR^2 \times \{0\}$ в $\RR^4=\RR^2 \times \RR^2$.

Постройте набор кривых  $\alpha_i\:[0,1]\to \RR^4$ 
таких, что
$\alpha_i(0)=a_i=(a_i,0)$, 
$\alpha_i(1)=b_i=(b_i,0)$ для всех $i$
и функция $\ell_{i,j}(t)=|\alpha_i(t)-\alpha_j(t)|$ 
монотонна (то есть возрастающая, убывающая или постоянная) для всей $i$ и $j$.
\end{thm}

\begin{thm}{Упражнение}\label{pr:brehm}
Используйте теорему Брэма чтобы доказать терему Киршбрауна, 
сформулированную ниже, для случая когда множество $Q$ конечно.
\end{thm}



\begin{thm}{Теорема Киршбрауна}
Пусть $Q$ есть произвольное подмножество плоскости и $f\:Q\to\RR^2$ есть короткое отображение.
Тогда $f$ допускает  короткое продолжение на всю плоскость.

Иными словами, существует короткое отображение $F\:\RR^2\to\RR^2$ 
такое, что сужение $F$ на $Q$ совпадает с $f$.
\end{thm}



\addtocontents{toc}{\protect\end{quote}}
\section{Теорема Акопяна об аппроксимации}
\addtocontents{toc}{\protect\begin{quote}}
\addtocontents{toc}{Любое короткое отображение из полиэдрального пространства в евклидово пространство той же размерности можно аппроксимировать кусочно изометрическими отображениями.}

Пусть $P$ есть полиэдральное пространство.
Отображение $h\:P\to\RR^n$ называется 
\emph{кусочно линейным}\label{page:piecewise linear map}
если существует триангуляция $P$ такая, что 
сужение $h$ на любой её симплекс $\Delta$ является линейным отображением;
то есть если $v_0,\dots,v_k$ вершины $\Delta$,
тогда для любого $x\in \Delta$ выполняется равенство
\[h(x)
=
\lambda_0\cdot h(v_0)+\dots+\lambda_k\cdot h(v_k),\]
гре $(\lambda_0,\dots,\lambda_k)$ барицентрические координаты точки  $x$.

\begin{thm}{Упражнение}\label{ex:PDPisPL}
Покажите, что любое кусочно изометрическое отображение из двумерного полиэдрального пространства на плоскость является кусочно линейным.
\end{thm}

Наша задача выяснить когда кусочно линейное отображение можно аппроксимировать кусочно изометрическим. 
Поскольку все кусочно изометрические отображения короткие, 
такая аппроксимация может существовать только для коротких отображений.
Следующая теорема утверждает, 
что это условие является достаточным.

\begin{thm}{Теорема Акопяна}\label{thm:approx}
Пусть $P$ есть двумерное полиэдральное пространство.
Тогда любое короткое кусочно линейное отображение $h\:P\z\to\RR^2$
допускает аппроксимацию кусочно изометрическими отображениями.

Более точно, для любого $\eps>0$ существует кусочно изометрическое отображение  $f\:P\to\RR^2$
такое, что
$$|f(x)-h(x)|<\eps$$
для любого $x\in P$.
\end{thm}

Заметим, что теорема Залгаллера (\ref{thm:zalgaller}) есть частный случай теоремы Акопяна.
Чтобы это увидеть рассмотрим отображение $h\:P\to\RR^2$,
которое отображает всё $P$ в одну точку плоскости.  
Заметим, что $h$ кусочно линейно. 
Значит мы можем применить к $h$ теорему Акопяна, 
полученное отображение  $f$ даёт пример кусочно изометрического отображения требуемого в теореме Залгаллера.

Отображение в доказательстве теоремы Акопяна строится менее явно чем в доказательстве теоремы Залгаллера,
оно использует теорему Брэма и
не использует теорему Залгаллера.

\begin{thm}{Упражнение}\label{ex:akopyan-brehm}
Сведите теорему Акопяна в случая когда $P$ есть выпуклый многоугольник к теореме Брэма (\ref{thm:brehm}).
\end{thm}

В доказательстве мы тринагулируем $P$ 
и применяем теорему Брэма к каждому треугольнику. 
При этом приходится заботиться чтобы отображения для треугольников согласовались на их общих сторонах.


\begin{wrapfigure}{r}{43mm}
\begin{lpic}[t(-3mm),b(-2mm),r(0mm),l(6mm)]{pics/zig-zag(1)}
\lbl[r]{1,46;$h(z_0)$}
\lbl[r]{8,35;$h(z_1)$}
\lbl[r]{11,29;$h(x)$}
\lbl{18,18,-57;$\dots$}
\lbl[r]{28,1;$h(z_n)$}
\lbl[bl]{17,37;$w_n(x)$}
\end{lpic}
\end{wrapfigure}

Для этого мы применяем следующее \emph{ступенчатое построение}.
Оно даёт кусочно изометрическое отображение которое близко к данному короткому отображению на отрезке прямой.  
Для этого построения нам нужно зафиксировать единичный вектор $e$ на плоскости.  
Выбор $e$ сам по себе не важен, 
но один и тот же вектор должен будет использоваться в каждом ступенчатом построении доказательства ниже.

\parit{Ступенчатое построение.} 
Пусть $E$ есть отрезок прямой
и $h\:E\to\RR^2$ есть короткое линейное отображение.
Пусть 
\[\ell = \length(E)\ \ \text{и}\ \  \ell' = \length(h(E)).\]  
Поскольку $h$ короткое, 
получаем $\ell' \z\le \ell$.

Выберем положительное целое $n$, 
и разделим $E$ на $n$ равных интервалов,
обозначим через $z_0,\dots,z_n$ их концы.

Заметим, что образ $h(E)$ есть отрезок прямой или точка.
В первом случае, выберем единичный вектор $u$ ортогональный к $h(E)$;
во втором положим $u=e$.  

Для $x\in E$, 
положим 
\begin{align*}
s_n(x)&=\min_i\{|z_i-x|\},
\\
w_n(x)&=k\cdot s_n(x)\cdot u+h(x),
\end{align*}
где
$k=\sqrt{1-(\ell'/\ell)^2}$. 
Если подразбить $E$ всеми $z_i$ 
и серединами каждого отрезка $[z_{i-1},z_i]$, 
тогда $w_n$ изометрично на каждом интервале этого разбиения.  
В частности, 
$w_n$ кусочно изометрично.
Более того, так как $k \le 1$ и $s_n(x) \le \tfrac{\ell}{2\cdot n}$ имеем
$$|w_n(x)-h(x)|\le\tfrac{\ell}{2\cdot n}$$
для любого $x\in E$.

Отображение $w_n$ и есть результат \emph{$n$-ступенчатого построения} применённого к $h$.

\medskip

Зафиксируем триангуляцию $\mathcal{T}$ пространства $P$
и обозначим через $\mathcal{T}^1$ её одномерный остов.
Снабдим $\mathcal{T}^1$ индуцированной внутренней метрикой.
Заметим, что $\mathcal{T}^1$ является одномерным полиэдральным пространством.

Следующее утверждение является основным техническим шагом в доказательстве.


\begin{thm}{Утверждение}\label{clm:main-step} 
Пусть $\mathcal{T}$ есть триангуляция двумерного полиэдрального пространства $P$,
$\mathcal{T}^1$ есть её одномерный остов
и пусть $h\:\mathcal{T}^1\to \RR^2$
есть кусочно линейное отображение
такое, что
$$|h(x)-h(y)|_{\RR^2}\le |x-y|_P$$
для любых $x,y\in \mathcal{T}^1$.
Тогда для любого $\eps > 0$, 
существует кусочно изометрическое отображение $w\:\mathcal{T}^1\to \RR^2$ такое, что
$$|w(x)-w(y)|_{\RR^2}\le |x-y|_P$$ 
для любых $x,y \in \mathcal{T}^1$ и
$|w(x) - h(x)| < \eps$
для всех $x \in \mathcal{T}^1$.
\end{thm}

\parit{Доказательство.}
Сначала мы проведём доказательство используя следующее дополнительное предположение на $h$:

\begin{clm}{}\label{clm:delta-condition}
Для некоторого $\delta>0$ неравенство
\[|h(x)-h(y)|_{\RR^2}\le(1-\delta)\cdot|x-y|_P\]
выполняется при всех $x,y\in \mathcal{T}^1$
и
$$h(v)=h(x)$$ 
для любой вершины $v$ в $\mathcal{T}^1$
и любой точки $x\in\mathcal{T}^1$ такой, что  $|v-x|_P\le\delta$.
\end{clm}


Пусть $\mathcal{S}$ 
обозначает подразбиение $\mathcal{T}^1$
такое, что $h$ линейно на каждом ребре $\mathcal{S}$.
Подразбивив ещё $\mathcal{S}$, мы можем добиться того, что каждое ребро  $\mathcal{S}$ 
исходящее из вершины в $\mathcal{T}^1$ 
имеет длину $\delta$. 
(Для последнего подразбиения, 
надо предположить, что $\delta$ в \ref{clm:delta-condition} достаточно мало.)

Обозначим через $\ell$ максимальную длину рёбер в $\mathcal{T}^1$.  
Для каждого ребра в $\mathcal{S}$, 
применим 
$n$-ступенчатое построение.
Поскольку ступенчатое построение не сдвигает концы интервала,
мы получаем кусочно изометрическое отображение 
$w_n\:\mathcal{T}^1\to\RR^2$ такое, что
$$|w_n(x) - h(x)| \le \tfrac{\ell}{2 \cdot n}
\eqlbl{eq:wn=f}$$ 
для всех $x \in \mathcal{T}^1$.

Мы должны показать, что если $n$ достаточно велико, 
то неравенство 
$$|w_n(x) - w_n(y)|_{\RR^2} \le |x - y|_P$$ 
выполняется для всех $x,y \in \mathcal{T}^1$.
Отметим, что
\begin{clm}{}\label{eq:w-on-edge}
$|w_n(x)-w_n(y)|\le |x-y|_P$
если $x$ и $y$ лежат на одном ребре в $\mathcal{T}^1$.
\end{clm}
\noi
Последнее утверждение следует поскольку $|x-y|_P = |x-y|_{E}$ 
если $x$ и $y$ лежат на одном ребре $E$ в $\mathcal{T}^1$, 
и того, что $w_n$ короткое на $E$.

Из \ref{clm:delta-condition} и \ref{eq:wn=f},  
мы видим, что 
\begin{align*}
&|w_n(x) - w_n(y)|_{\RR^2}\le\\ 
&\qquad \qquad \le |w_n(x) - h(x)|_{\RR^2} ~+~ |h(x) - h(y)|_{\RR^2} ~+~ |h(y) - w_n(y)|_{\RR^2}\le\\
&\qquad \qquad \le |x-y|_P+\left(\tfrac{\ell}{n} - \delta\cdot|x-y|_P\right)
\end{align*}
для любых $x$ и $y$ в $\mathcal{T}^1$.

Предположим, что $|w_n(x)-w_n(y)|_{\RR^2}> |x-y|_P$ для некоторых $x,y \in \mathcal{T}^1$. 
Из вышесказанного, получаем $|x - y|_P < \tfrac{\ell}{n\cdot \delta}$,
а значит 
\[|x-y|_P<\tfrac{C}{n}.
\eqlbl{eq:x-near-y}\] 
для некоторой постоянной $C$ независящей от $x$ и $y$.
Таким образом, \ref{eq:w-on-edge} и \ref{eq:x-near-y} 
влечёт следующее.

\begin{clm}{}\label{clm:near-vertex}
При достаточно большом%
\footnote{Величина $n$ не зависит от $x$ и $y$.  
Чтобы гарантировать то, что $x$ и $y$ 
лежат на смежных рёбрах, надо чтобы величина $C/n$ была меньше минимального расстояния между точками на непересекающихся рёбрах.  
Чтобы гарантировать то, что $x$ и $y$ лежат на расстоянии не меньшем чем $\delta$ от $v$, 
надо выбрать $n$ достаточно большим в зависимости от минимального угла в треугольниках триангуляции.  
В обоих случаях, для нахождения $n$ мы пользуемся тем, что триангуляция $\mathcal{T}$ конечна.} 
$n$, 
если $|w_n(x)-w_n(y)|> |x-y|_P$ тогда обе точки $x$ и $y$ 
лежат на разных рёбрах с общей вершиной, скажем $v$ в $\mathcal{T}^1$,
и
\[|x-v|_{P},\  |y-v|_{P}\le\delta.\] 
\end{clm} 
Пусть $x$, $y$ и $v$ как в \ref{clm:near-vertex}.
Отметим точку $x'$ на том же ребре, что и $y$
такую, что $|v-x'|_P=|v-x|_P$.
Заметим, что
$w_n(x')=w_n(x)$.
Последнее верно так как по условию \ref{clm:delta-condition} и построению триангуляции $\mathcal{S}$,
обе точки $x$ и $x'$ лежат на рёбрах длины $\delta$ из $v$
и $h$ отображает оба этих ребра в одну точку.  
Таким образом получаем 
\begin{align*}
|w_n(x)-w_n(y)|_{\RR^2}&=|w_n(x')-w_n(y)|_{\RR^2}\le
\\
&\le |x'-y|_P=
\\
&=\bigl||x-v|_P-|y-v|_P\bigr|\le
\\
&\le|x-y|_P.
\end{align*}

То есть, мы доказали, что при достаточно большом $n$, 
неравенство 
$$|w_n(x)-w_n(y)|_{\RR^2}\le |x-y|_P$$
выполняется для всех пар точек $x,y\in\mathcal{T}^1$.
Мы можем дополнительно предположить, что $\tfrac{\ell}{2\cdot n} < \eps$.
Введём обозначение $w = w_n$ для такого $n$.  
В этом случае, из \ref{eq:wn=f}
получаем, что 
$$|w(x) - h(x)| < \eps$$ 
для любого $x \in \mathcal{T}^1$.
Так мы закончили доказательство утверждения~\ref{clm:main-step} 
в предположении \ref{clm:delta-condition}.

\begin{wrapfigure}[14]{r}{45mm}
\begin{lpic}[t(-7mm),b(0mm),r(0mm),l(0mm)]{pics/q-graph(1)}
\lbl[b]{6.5,8;$\delta$}
\lbl[b]{39,8;$\delta$}
\end{lpic}
\caption*{Граф $q_\delta$ на ребре.}
\end{wrapfigure}

Остаётся показать, что любое $h\:\mathcal{T}^1\to \RR^2$
как в утверждении~\ref{clm:main-step}
допускает аппроксимацию отображениями удовлетворяющими дополнительно \ref{clm:delta-condition} и применить построение выше для этой аппроксимации.

Для малого $\delta > 0$ (скажем меньше половины наименьшего ребра), 
рассмотрим отображение
$$q_\delta\:\mathcal{T}^1\to\mathcal{T}^1$$ 
которое отображает $\delta$-окрестность каждой вершины  $\mathcal{T}^1$ в эту вершину и линейно растягивает остаток ребра
(смотри граф).

Обозначим через $L_\delta$ наилучшую постоянную Липшица для $q_\delta$;
то есть минимальное число такое, что 
\[|q_\delta(x)-q_\delta(y)|_P\le L_\delta\cdot|x-y|_P\] 
для всех $x,y \in \mathcal{T}^1$.
Заметим, что $L_\delta\z\to 1$ при $\delta\to 0^+$.
Тогда отображение
\[h_\delta
\df
\tfrac{1-\delta}{L_\delta}\cdot (h\circ q_\delta)\]
кусочно линейно и удовлетворяет условию \ref{clm:delta-condition}.
Более того, величину $\delta$ можно выбрать на столько малой, что
\[|h_\delta(x) - h(x)| < \tfrac{\eps}{2}\] 
для всех $x \in \mathcal{T}^1$.

Из первой части доказательства следует существование кусочно изометрического отображения
$w\: \mathcal{T}^1 \to \RR^2$ такого, что
$$|w(x) - h_\delta(x)| < \tfrac{\eps}{2}, \qquad |w(x) - w(y)|_{\RR^2} \le |x - y|_P$$
для всех $x, y \in \mathcal{T}^1$.  
По неравенству треугольника,
$$|w(x) - h(x)| < \eps$$ для всех $x \in \mathcal{T}^1$.
\qeds

\parit{Доказательство \ref{thm:approx}.}
Выберем мелкую триангуляцию $\mathcal{T}$
пространства $P$,
скажем такую, что диаметр каждого треугольника меньше $\tfrac\eps{3}$.
Пусть $\mathcal{T}^1$ обозначает одномерный остов $\mathcal{T}$.
Согласно утверждению~\ref{clm:main-step}, 
существует кусочно изометрическое отображение $w\:\mathcal{T}^1\to\RR^2$ такое, что
$$|w(x)-h(x)|_{\RR^2} < \tfrac\eps{3}$$
для любого $x\in \mathcal{T}^1$ и
$$|w(x)-w(y)|_{\RR^2}\le|x-y|_P$$
для любых $x$ и $y$ в $\mathcal{T}^1$.

Далее, для того чтобы продлить отображение $w$
на всё $P$,
мы воспользуемся теоремой Брэма.
Пусть $\mathcal{S}$ есть подразбиение $\mathcal{T}^1$
такое, что $w$ изометрично на каждом ребре $\mathcal{S}$.
Зафиксируем треугольник $\Delta$ в $\mathcal{T}$.
Обозначим через $a_1,\dots, a_n$ вершины $\mathcal{S}$ 
на границе $\Delta$.
Определим $b_i=w(a_i)$ для каждого $i$.
Применяя теорему Брэма,
мы получаем кусочно изометрическое отображение  $f_\Delta\: \Delta \to \RR^2$.

Поскольку $w$ изометрично на каждом ребре $\mathcal{S}$,
отображения $f_\Delta$ и $w$ совпадают на границе $\partial\Delta$.  
Отсюда набор отображений $\{f_\Delta\}$ для всех треугольников $\Delta$  из $\mathcal{T}$ согласован и из них
склеивается одно кусочно изометрическое отображение $f\:P\to\RR^2$.

Остаётся доказать, 
что $f$ удовлетворяет условиям теоремы.  
Пусть $x \in P$ произвольная точка и пусть 
$y$ есть точка на ребре треугольника $\mathcal{T}$ который содержит $x$.  
По условию выбора $\mathcal{T}$, имеем $|x - y| < \tfrac{\eps}{3}$.  
Следовательно
\begin{align*}
|f(x) - h(x)| &\le |f(x) - w(y)| + |w(y) - h(y)| + |h(y) - h(x)|=\\
&= |f(x) - f(y)| + |w(y) - h(y)| + |h(y) - h(x)|\le\\
&\le 2\cdot |x - y| + |w(y) - h(y)|<\\
& < \eps,
\end{align*} 
Поскольку $w(y) = f(y)$ и оба отображения $f$ и $h$ короткие.
\qeds

Несведущий читатель может предположить, 
что отображение $f$ в теореме Акопяна может быть построено так, что оно совпадает с $h$
на данном конечном множестве точек.
Следующее упражнение показывает, что в общем случае это невозможно.

\begin{thm}{Упражнение}\label{ex:tripod}
Рассмотрим следующие 5 точек в $\RR^3$:
\begin{align*}
o=(0,0,0),
\,
p=(0,0,1),
\,
a=(2,0,0),
\,
b=(-1,2,0),
\,
c=(-1,-2,0)
\end{align*}

\begin{wrapfigure}{r}{37mm}
\begin{lpic}[t(-0mm),b(-0mm),r(0mm),l(0mm)]{pics/3-pod(1)}
\lbl[t]{15,10;$o$}
\lbl[b]{15,23.5;$p$}
\lbl[t]{35,10;$a$}
\lbl[rb]{7,23;$b$}
\lbl[rb]{0,2;$c$}
\end{lpic}
\end{wrapfigure}

Пусть $P$ есть «тренога», полиэдральное пространство полученное склеиванием трёх треугольников $\triangle opa$, $\triangle opb$ и $\triangle opc$ в $\RR^3$,
и снабжённое внутренней метрикой.

Отметим, что сужение координатной проекции $\pi(x,y,z) = (x,y,0)$ на $P$
является коротким и кусочно линейным,
при этом
\begin{align*}
\pi(o)&=\pi(p)=o,&\pi(a)&=a,&\pi(b)&=b,&\pi(c)&=c.
\end{align*}

Покажите, что не существует кусочно изометрического отображения  $f\:P\to \RR^2=\RR^2\times \{0\}$
такого, что $f(a)=a$, $f(b)=b$ и $f(c)=c$.
\end{thm}



\addtocontents{toc}{\protect\end{quote}}
\section{Теорема Громова о сплющивании}\label{sec:S^2->R^2}
\addtocontents{toc}{\protect\begin{quote}}
\addtocontents{toc}{Существует отображение из сферы на плоскость сохраняющее длины кривых.}

Напомним, что $\SS^2$ обозначает единичную сферу в $\RR^3$
с индуцированной внутренней метрикой.  

\begin{thm}{Теорема}\label{thm:S2->R2}
Существует отображение $f\:\SS^2\to\RR^2$ сохраняющее длины.
\end{thm}

Заметим, что складки такого отображения $f$ 
должны образовывать всюду плотное множество в $\SS^2$.
Более точно,  
сужение $f$ на любое открытое множество $\SS^2$ не может быть инъективным.%
\footnote{В самом деле, если $f$ инъективно и сохраняет длины на открытом множестве $U \z\subset \SS^2$, 
тогда сужение $f$ на $U$ 
отображает кратчайшие в отрезки прямых (в доказательстве используется \emph{теорема об инвариантности области}, см. например II \S9 в \cite{alexandrov}). 
Далее легко видеть, что сужение $f$ на $U$ 
локально изометрично.}

В доказательстве теоремы мы будем пользоваться следующим упражнением.

\begin{thm}{Упражнение}\label{problem2}
Пусть $K$ выпуклый многогранник в $\RR^3$.
Докажите, что для любой точки $x$ в $\RR^3$ существует единственная точка $\bar x\in K$ которая минимизирует расстояние $|x-\bar x|$.
Более того,
отображение 
$$\phi\: \RR^3 \to K, \qquad \phi(x) = \bar x$$ 
является коротким. 
\end{thm}

\parit{Доказательство теоремы~\ref{thm:S2->R2}.}
Выберем возрастающую последовательность
$K_0\subset K_1\subset \dots$ выпуклых многогранников в $\RR^3$ 
такую, что их объединение является открытым единичным шаром.
Обозначим через $P_n = \partial K_n$ поверхность $K_n$, 
с индуцированной внутренней метрикой.
Заметим, что $P_n$ является двумерным полиэдральным пространством для каждого $n$.

Пусть $\phi_n$ обозначает проекцию на $K_{n}$, 
как в упражнении~\ref{problem2}.
Напомним, что $\phi_n$ короткое. 
Поскольку $K_{n} \subset K_{n+1}$, получаем  $\phi_n(P_{n+1}) = P_{n}$.
Заметим, что можно триангулировать $P_n$ и $P_{n+1}$ 
таким образом, что сужение $\phi_n$ на любой симплекс $P_{n+1}$ 
есть ортогональная проекция на некоторый симплекс в $P_{n}$.
В частности сужение $\phi_n$ на $P_{n+1}$ кусочно линейно%
\footnote{См. определение на странице~\pageref{page:piecewise linear map}.}
и короткое по отношению к внутренним метрикам на $P_{n+1}$ и $P_n$.

Мы утверждаем, что для любой точки $x\in \SS^2$, 
существует единственная последовательность точек $x_n\in P_n$ 
такая, что $x_n\to x$ при $n\to\infty$ и $\phi_n(x_{n+1})=x_n$ для всех $n$.  

Единственность следует из того, что все отображения  $\phi_n$ короткие.
Чтобы доказать существование, зафиксируем последовательность $z_n\in P_n$ такую, что $z_n\to x$.
Рассмотрим двойную последовательность $y_{n,m}\in P_n$, определённую при $n\le m$, такую, что $y_{n,n}=z_n$ и
$y_{n,m} \z= \phi_{n}(y_{n+1,m})$ при $0 \le n < m$.
Далее положим
$$x_n=\lim_{m\to\infty} y_{n,m}.$$

\begin{thm}{Упражнение}\label{ex:limit-above}
Докажите, что предел выше определён 
и $\phi_n(x_{n+1})\z=x_n$ для любого $n$.  
Далее докажите, что $x_n \to x$ при $n \to \infty$.
\end{thm}

Если $x_n\in P_n$ и $x\in \SS^2$ как выше,
определим отображение $\psi_n\:\SS^2\to P_n$ как $\psi_n(x)\z=x_n$.
Заметим, что $\psi_n$ короткое,
$\psi_n\z=\phi_n\circ\psi_{n+1}$ при всех $n$,
и если $p_n=\psi_n(p)$ и $q_n=\psi_n(q)$ для $p, q\in\SS^2$,
то
$$|p_n-q_n|_{P_n}\to |p-q|_{\SS^2}\ \text{при}\  n\to\infty.
\eqlbl{eq:pq-to-pq}$$

Нужное нам отображение $f\: \SS^2 \to \RR^2$ будет построено как «предел» некоторой последовательности $f_n\: P_n \to \RR^2$ кусочно изометрических отображений.  
Эта последовательность получается рекурсивным применением теоремы Акопяна так, чтобы выполнялось неравенство
$$|f_{n+1}(x) - f_n(\phi_n(x))| < \eps_n
\eqlbl{eq:eps-close}$$ 
для рекурсивно выбранной последовательности $(\eps_n)$ 
положительных чисел быстро стремящихся к $0$.

\parit{Рекурсивное построение $f_n\:P_n\to\RR^2$ и $\eps_n$.}
Предположим нам даны кусочно изометрическое отображение  $f_n\: P_n\z\to \RR^2$ и $\eps_n$.
Заметим, что 
$$f_n\circ\phi_n\: P_{n+1}\z\to\RR^2$$ кусочно линейное и короткое. 
Таким образом, применив теорему Акопяна (\ref{thm:approx}) 
получает кусочно изометрическое отображение $f_{n+1}\:P_{n+1}\to\RR^2$ которое $\eps_n$-близко к $f_n\circ\phi_n$ в смысле \ref{eq:eps-close}.

Обозначим через $M(n+1)$ число треугольников в триангуляции $P_{n+1}$ для которой $f_{n+1}$ изометрично на каждом треугольнике. 
Пусть
$$\eps_{n+1}=\frac{\eps_n}{2\cdot M(n+1)}.
\eqlbl{eq:eps-n}$$

Далее мы опишем рекурсивное построение последовательностей $(f_n)$ и $(\eps_n)$: 
\begin{enumerate}
\item Выберем $\eps_0>0$ и кусочно изометрическое отображение $f_0\:P_0\z\to\RR^2$.
Существование $f_0$ следует например из теоремы Залгаллера (\ref{thm:zalgaller}).
\item По $\phi_0$, $f_0$ и $\eps_0$ построим $f_1$.
\item По $f_1$ вычислим $\eps_1$.
\item По $\phi_1$, $f_1$ и $\eps_1$ построим $f_2$.
\item По $f_2$ вычислим $\eps_2$.
\item и так далее.%
\footnote{Построение этих двух последовательностей похоже на ходьбу: 
шаг правой (построение $f_1$) 
даёт возможность сделать шаг левой (вычисление $\eps_1$),
он в свою очередь даёт возможность сделать шаг правой (построение $f_2$) 
и так далее.}
\end{enumerate}

\medskip

Остаётся доказать следующее

\begin{clm}{}\label{clm:length-preserving}
Последовательность отображений  $f_n\circ\psi_n\:\SS^2\to \RR^2$ сходится к отображению $f\:\SS^2\to\RR^2$ сохраняющему длины. 
\end{clm}

Заметим, что $\eps_{n}<\tfrac{\eps_0}{2^n}$.
Отсюда,
для любого $x\in\SS^2$, последовательность
 $(f_n\circ\psi_n)(x) \in \RR^2$ является фундаментальной и в частности сходится.
Определим $f\: \SS^2 \to \RR^2$ как 
$$f(x) = \lim_{n \to \infty} (f_n\circ\psi_n)(x).$$
По построению $f_n$ имеем
$$|(f_n\circ\psi_n)(x)-f(x)|<\eps_n$$
для любых $x\in \SS^2$ и $n$.
Поскольку каждое $f_n\circ\psi_n$ короткое, 
$f$ также короткое.

Остаётся показать, что отображение $f\: \SS^2 \to \RR^2$ сохраняет длины.
Для этого достаточно доказать, что 
$$\length(f\circ\alpha) \ge |p-q|_{\SS^2}\eqlbl{length>=dist}$$
для любой кривой $\alpha$ между точками $p,q \in \SS^2$.
Для окончания доказательства нам потребуется ещё одно определение;
его следует воспринимать как аналог длины кривой.

\begin{thm}{Определение}
Пусть $\alpha\:[a,b]\z\to X$
кривая в метрическом пространстве $X$.
Положим
$$\ell_k(\alpha)
\df
\sup\set{\sum_{i=1}^k |\alpha(t_i)-\alpha(t_{i-1})|_X}{a=t_0<t_1<\dots<t_k=b}.$$

\end{thm}

Заметим что для любой кривой выполняется следующее
\begin{align*}
\ell_1(\alpha)&\le\ell_2(\alpha)\le \ell_3(\alpha)\le\dots,
\\
\ell_k(\alpha)&\to\length(\alpha)\ \ \text{при}\ \ k\to\infty,
\\
\ell_k(\alpha)&\le\length(\alpha) \ \text{для любого}\ \ k.
\end{align*}
Более того, если
\[\ell_k(\alpha)=\length(\alpha)\] 
то $\alpha$ 
есть ломанная составленная из не более чем $k$ кратчайших. 

Следующее упражнение говорит, что если две кривые $\alpha$ и $\beta$ достаточно близки то $\ell_k(\alpha)\approx\ell_k(\beta)$.
Отметим, что в той же ситуации, разница
$\length(\alpha)$ и $\length(\beta)$ может быть велика.

\begin{thm}{Упражнение}\label{ex:klength-approx}
Пусть $\alpha, \beta\: \II \to X$ суть кривые такие, что
\[|\alpha(t) - \beta(t)|_X < \eps\] для любого $t \in \II$.  
Покажите, что
\[ |\ell_k(\alpha) - \ell_k(\beta)| 
\le 2\cdot k\cdot \eps.\]

\end{thm}

Мы возвращаемся к доказательству \ref{length>=dist}.
Определим $p_n=\psi_n(p)$ и $q_n\z=\psi_n(q)$. 
Пусть $\beta$ 
есть произвольная кривая идущая от $p_n$ до $q_n$ в $P_n$.
Заметим, что можно найти кривую $\gamma$ от $p_n$ до $q_n$
с той же длиной или меньше, 
что пересечение $\gamma$ с любым треугольником в триангуляции $P_n$ есть отрезок прямой, 
и более того, концы всех таких отрезков лежат на $\beta$ в том же порядке.
Отсюда следует, что $f_n\circ\gamma$ есть ломанная линия на плоскости
с не более чем $M(n)$ звеньями.
Её вершины обозначим в порядке следования как 
$$f_n(p_n) = z_0, z_1, \dots , z_k = f_n(q_n).$$

Напомним, что  $k\le M(n)$
и каждая точка $z_i$ лежит на кривой $f_n \circ \beta$.
Таким образом
$$
\begin{aligned}
|p_n-q_n|_{P_n}
&\le \length(\gamma)=
\\
&= \ell_{M(n)}(f_n\circ\gamma)=
\\
&=|z_0-z_1|+\dots+|z_{k-1}-z_{k}|\le
\\
&\le\ell_{M(n)}(f_n\circ\beta);
\end{aligned}
\eqlbl{eq:pn-qn}
$$

Выберем кривую $\alpha$ от $p$ до $q$ в $\SS^2$.
Согласно упражнению~\ref{ex:klength-approx} 
и \ref{eq:eps-n},
для всех $n$ имеем
$$|\ell_{M(n)}(f\circ\alpha)-\ell_{M(n)}(f_n\circ\psi_n\circ\alpha)| 
\le 2\cdot M(n)\cdot \eps_n = \eps_{n-1}.
\eqlbl{eq:inq-ell-k}$$
По данному $\eps > 0$, 
можно выбрать $n$ на столько большым, 
что 
$\eps_{n-1} \le \tfrac{\eps}{2}$ и
$$|p-q|_{\SS^2} - |p_n - q_n|_{P_n} \le \tfrac{\eps}{2},$$ 
последнего можно добиться используя \ref{eq:pq-to-pq}.
Применив \ref{eq:pn-qn} для $\beta=\psi_n \circ \alpha$ 
и \ref{eq:inq-ell-k}, мы видим
\begin{align*}
\length(f\circ \alpha) &\ge \ell_{M(n)}(f\circ \alpha)\ge\\
&\ge \ell_{M(n)}(f_n \circ \psi_n \circ \alpha) - \eps_{n-1}\ge\\
&\ge |p_n - q_n|_{P_n} - \eps_{n-1}\ge\\
&\ge |p-q|_{\SS^2} - \tfrac{\eps}{2} - \eps_{n-1}\ge\\
&\ge |p-q|_{\SS^2} - \eps.
\end{align*}
Поскольку $\eps > 0 $ произвольно, получаем
$$\length(f\circ \alpha) \ge |p-q|_{\SS^2}.$$
Отсюда следует \ref{length>=dist}.
\qeds

\addtocontents{toc}{\protect\end{quote}}
\section{Задача Арнольда о мятом рубле}\label{sec:arnold}
\addtocontents{toc}{\protect\begin{quote}}
\addtocontents{toc}{Возможно ли сложить квадрат на плоскости так чтобы периметр полученной фигуры оказался больше периметра исходного квадрата?}

Это лекция полностью развлекательная.
Здесь мы обсудим следующую задачу поставленную Арнольдом \cite[Задача 1956-1]{arnold}.

\begin{thm}{Задача}
Можно ли сложить квадратный лист бумаги на плоскости так, 
что периметр полученной фигуры превысил бы периметр исходного листа?
\end{thm}

\begin{wrapfigure}{r}{55mm}
\noi\begin{lpic}[t(-8mm),b(-5mm),r(0mm),l(0mm)]{pics/skladka(0.6)}
\lbl{25,28;$M$}
\lbl{75,28;$M'$}
\lbl[r]{23,55;$q$}
\lbl[r]{87,55;$q$}
\end{lpic}
\end{wrapfigure}

Ответ на эту задачу зависит от того, что понимать под словом «сложить».
 
Под «складыванием» можно например понимать последовательность следующих «перегибаний»: возьмём многоугольник $M$, выберем прямую $q$, секущую $M$, перегнём $M$ вдоль $q$, при этом части $M$ по разные стороны от $q$ «склеятся» в новый многоугольник $M'$ (см. рисунок).

Как следует из упражнения, такими операциями увеличить периметр нельзя.

\begin{thm}{Упражнение}
Докажите, что периметр $M'$ не превосходит периметр $M$. 
(При решении этого упражнения обратите внимание, что в общем случае прямая $q$ может пересекаться с $M$ по нескольким отрезкам).
\end{thm}

При помощи этой операции нельзя отогнуть часть листа, которая лежит на другом. Например, невозможно провести последовательность «перегиба» и «отгиба» как на рисунке.
\begin{center}
\begin{lpic}[t(0mm),b(0mm),r(0mm),l(0mm)]{pics/otgib(0.7)}
\end{lpic}
\end{center}
Вторая операция увеличивает периметр, хотя он остаётся всё ещё меньше периметра исходного квадрата.
Если под «складыванием»  понимать последовательность таких перегибаний и отгибаний, то ответ на задачу до сих пор никто не знает (хотя, возможно, никто и не хочет узнать).

\parbf{Японский журавлик.}
Под складыванием можно понимать ещё более общую операцию: 
представьте себе, что мы разметили складки заранее и сгибаем лист целиком, так, что
все области, на которые разбит лист складками, остаются плоскими, а перегибание происходит только по складкам.

В такой постановке правильный ответ --- «можно».
Удивительнее то, что возможность немного увеличить периметр видна в построении одной 
из самых древних фигурок оригами --- традиционного японского журавлика%
\footnote{Первая дошедшая до нас книга по оригами «Сембацуру ориката» была издана в 1797 году. 
Вся эта книга посвящена только складыванию журавлика.
Без сомнения, эта фигурка оригами гораздо древнее этой книги.}.
Про то, что оригамисты знали ответ, первый математик узнал только в 1998 году.

\begin{figure}[h]
\ \ \ \ \ \ \ \ 
\includegraphics[scale=0.34]{pics/zh-2}
\hfill
\includegraphics[scale=0.18]{pics/razv-zh}
\ \ \ \ \ \ \ \ 
\end{figure}

Журавлика складывают из специальной заготовки, 
которая имеет четыре больших конца и ещё один короткий. 
Из двух концов делаются два крыла, а два другие «утоньшаются», 
из них получаются хвост и шея.
Если же эту «операцию утоньшения» применить дважды к каждому большому концу, 
то заготовку можно будет выложить на плоскости так, 
что периметр превысит периметр исходного квадрата.

\begin{center}
\begin{lpic}[t(0mm),b(0mm),r(0mm),l(0mm)]{pics/zagotovka(0.27)}
{\large
\lbl{230,170,-10;$\longrightarrow$}
\lbl{295,105,-20;$\longrightarrow$}
}
{\tiny
\lbl{119,119;$1$}
\lbl{134,110;$2$}
\lbl{147,110;$3$}
\lbl{160,110;$4$}
\lbl{170,110;$5$}
\lbl{180,110;$6$}
\lbl{190,110;$7$}
\lbl{200,110;$8$}

\lbl{117,100;$16$}
\lbl{134,100;$15$}
\lbl{147,100;$14$}
\lbl{160,100;$13$}
\lbl{170,100;$12$}
\lbl{180,100;$11$}
\lbl{190,100;$10$}
\lbl{200,100;$9$}

%\lbl{80;$1$}
\lbl{110,134;$2$}
\lbl{110,147;$3$}
\lbl{110,160;$4$}
\lbl{110,171;$5$}
\lbl{110,181;$6$}
\lbl{110,191;$7$}
\lbl{110,201;$8$}

\lbl{98,120;$16$}
\lbl{98,134;$15$}
\lbl{98,147;$14$}
\lbl{98,160;$13$}
\lbl{98,171;$12$}
\lbl{98,181;$11$}
\lbl{98,191;$10$}
\lbl{98,201;$9$}

\lbl{90,110;$17$}
\lbl{74,110;$18$}
\lbl{60,110;$19$}
\lbl{48,110;$20$}
\lbl{37,110;$21$}
\lbl{26.5,110;$22$}
\lbl{17,110;$23$}
\lbl{7,110;$24$}

\lbl{74,100;$31$}
\lbl{60,100;$30$}
\lbl{48,100;$29$}
\lbl{37,100;$28$}
\lbl{27,100;$27$}
\lbl{17,100;$26$}
\lbl{7,100;$25$}

\lbl{110,88;$17$}
\lbl{110,74;$18$}
\lbl{110,59;$19$}
\lbl{110,48;$20$}
\lbl{110,38;$21$}
\lbl{110,28;$22$}
\lbl{110,18;$23$}
\lbl{110,8;$24$}

\lbl{88,88;$32$}

\lbl{98,74;$31$}
\lbl{98,59;$30$}
\lbl{98,48;$29$}
\lbl{98,38;$28$}
\lbl{98,28;$27$}
\lbl{98,18;$26$}
\lbl{98,8;$25$}
}
\end{lpic}
\end{center}
На рисунке в центре --- сложенная заготовка журавлика с дважды заострёнными концами. 
Справа --- заготовка журавлика с расправленными концами.
Слева --- сетка складок; углы между всеми складками при вершинах квадрата равны; на каждом треугольнике указан номер его слоя в заготовке; складки, которые добавляются при расправлении концов помечены точечным пунктиром.

В полученной фигуре периметр увеличивается всего на пол процента, 
и сделать её руками не просто, хотя возможно. 
Следует найти очень тонкую бумагу, так как количество слоёв после 
расправления концов станет равно восьмидесяти.
Подобный пример описан в книге Лэнга \cite{lang-secrets}.

Предположим наш квадрат единичный.
Если  
операцию утоньшения провести много раз то обход каждой из четырёх игл займёт чуть меньше $1$, 
а обход короткого тупого конца займёт около $(\sqrt2-1)$, 
именно за счёт этого последнего конца мы получаем прибавку к периметру.
То есть таким способом можно добиться увеличения периметра почти на $(\sqrt2-1)\approx\tfrac25$.

\medskip

Способ увеличения периметра, описанные выше, обладают одним недостатком --- во время складывания необходимо одновременно сгибать лист в огромном числе мест. 
Следующий пример основан на идее Ивана Ященко \cite{yashenko},
он также даёт лишь незначительное увеличение периметра, но состоит в последовательном применении одного и того же простого действия. 
Если повторить его достаточно много раз, то получим фигуру сколь угодно близкую к фигуре справа, у неё периметр слегка больше чем у квадрата.
Каждая итерация добавляет пару слоёв около вогнутого угла;
общее число слоёв в этой модели гораздо больше чем в заготовке журавлика.

\begin{center}
\begin{lpic}[t(0mm),b(0mm),r(0mm),l(0mm)]{pics/yaschenko(0.54)}
\end{lpic}
\end{center}


\parbf{Ёжик и расчёска.}
Оказывается, что периметр сложенной фигуры можно сделать произвольно большим.
Это можно увидеть из построения другой фигурки оригами, так называемого «морского ежа» слева на картинке ниже.
Эта фигурка описана в книжке Монтролла и Лэнга \cite{lang}.

\begin{center}
\ \ \ \ 
\includegraphics[scale=0.28]{pics/urch}
\hfill
\includegraphics[scale=0.55]{pics/taras}
\ \ \ \ 
\end{center}

Не зная ни про ёжика, 
ни про журавлика, 
Алексей Тарасов описал другое решение  в \cite{tarasov}, 
сложив фигурку, которую мы назовём \emph{расчёской Тарасова}, 
она  изображена справа на картинке.
Он подошёл к задаче как математик, привёл точную формулировку и показал, что в процессе подобного складывания можно обойтись без растягиваний и сжиманий листа (приёмы, которыми оригамисты часто пользуются).
Интернет-версия этих лекций  \cite{petrunin-yashinsky-arXiv}
включает 3 видео иллюстрации предоставленных Тарасовым описывающие это построение.

\parbf{Оригами в четырёхмерном пространстве.} 
«Складывание» можно определить как кусочно изометрическое отображение
из квадрата в плоскость.
Это определение более общо, чем все рассмотренные выше.
Следующее упражнение показывает, что такое отображение не всегда возможно реализовать с помощью бумажной модели.

\begin{thm}{Упражнение}\label{pr:6-4-3-4}
Рассмотрим разбиение квадрата $\square$ как на картинке ниже,
это часть регулярного замощения плоскости на квадраты правильные шестиугольники и треугольники. 

Покажите, что существует отображение $f\:\square\to\RR^2$ 
которое изометрично на каждом многоугольнике замощения 
и которое обращает ориентацию только у серых многоугольников.%
\footnote{Неформально говоря, нужно согнуть вдоль каждого ребра этого замощения.} 

Покажите, что отображение $f$ невозможно реализовать  с помощью бумажной модели.%
\footnote{Более точно, надо представить, что плоскость лежит в пространстве $\RR^3$, 
и нам нужно доказать, что $f$ не допускает аппроксимации непрерывными инъективными отображениями  $\square\to \RR^3$.}
\end{thm}

\begin{wrapfigure}{r}{41mm}
\begin{lpic}[t(-6mm),b(-2mm),r(0mm),l(0mm)]{pics/3464(0.4)}
\end{lpic}
\end{wrapfigure}

Препятствия описанные в упражнении исчезают в $\RR^4$;
то есть, про кусочно изометрическое отображение квадрата на плоскость
можно думать как про сложение бумаги в четырёхмерном пространстве.
Более того, можно на самом деле сложить квадрат в $\RR^4$
в заданное кусочно изометрическое отображение.

Более точно, 

\begin{clm}{}\label{alexander-2}
Существует непрерывное однопараметрическое семейство кусочно изометрических отображений $f_t\:\square\to\RR^4$, $t\in[0,1]$ с фиксированной триангуляцией, скажем $\mathcal{T}$,
такое, что
\begin{itemize}
\item $f_0$ есть изометрическое отображение из $\square$ в координатную плоскость $\RR^2\times \{0\}$ в $\RR^4=\RR^2\times\RR^2$,
\item отображение $f_1$ есть данное кусочно изометрическое отображение из $\square$ в ту же координатную плоскость,
\item отображение $f_t$ инъективно при всех $t\ne1$.
\end{itemize}
\end{clm}

Доказательство последнего утверждения основано на упражнении~\ref{pr:alexander}.

\parit{Доказательство \ref{alexander-2}.}
Пусть $a_1,\dots,a_k$ 
есть вершины $\mathcal{T}$
и $b_1,\dots,b_k$ обозначает их образы при кусочно изометрическом отображении.
Определим $f_t(a_i)\z\in\RR^2\times\RR^2$ как  
$$f_t(a_i)= \left(\frac{a_i + b_i}{2} + 
\cos(\pi\cdot t)\cdot \frac{a_i - b_i}2,\  
\sin(\pi\cdot t)\cdot \frac{a_i - b_i}2\right);
$$
таким образом $f_0(a_i)=(a_i,0)$ и $f_1(a_i)=(b_i,0)$ для любого $i$.
Отображение $f_t$ можно продолжить линейно на каждый треугольник в $\mathcal{T}$.
Несложные вычисления показывают, что функция
 $\ell_{i,j}(t)=|f_t(a_i)-f_t(a_j)|$ монотонна по $t$;
в частности, если $|a_i-a_j|=|b_i-b_j|$ 
то $\ell_{i,j}(t)$ постоянна.
Отсюда следует, что $f_t$ кусочно изометрично.

Далее нетрудно проверить, что для любых $x,y\in\square$ есть постоянные $p$ и $q$ такие, что 
$$|f_t(x)-f_t(y)|^2=p-q\cdot\cos(\pi\cdot t).$$
Таким образом, если $x\ne y$ тогда $|f_t(x)-f_t(y)|>0$ для любого $t\ne1$.
Другими словами, $f_t$ инъективно при $t\ne1$.\qeds

Из вышесказанного видно, 
что возможность увеличить периметр складыванием квадрата в $\RR^4$,
следует из теоремы Брэма.
Нам достаточно построить короткое отображение $f$ 
из квадрата на плоскость такое, что периметр образа достаточно велик.
После этого, применив теорему Брэма для достаточно плотного конечного множества точек в квадрате,
мы получим кусочно изометрическое отображение $h$ произвольно близкое к $f$.
В частности мы можем добиться того, что периметр образа  $f(\square)$ 
был достаточно велик.

Нужное для этого отображение $h$ можно построить следующим образом:
Выберем достаточно большое целое $n$ 
и разделим единичный квадрат $\square$ на $n^2$ квадратиков со стороной  $\tfrac{1}{n}$.
Обозначим через $d(x)$ расстояние от точки $x\in\square$ 
до границы квадратика содержащего $x$. 
Функция $d\:\square\to\RR$ принимает значения в $[0,\tfrac{1}{2{\cdot}n}]$.
Далее, пронумеруем квадратики целыми числами от $1$ до $n^2$.  
Для точки $x\in \square$
обозначим через $i(x)$ номер квадратика содержащего $x$ (если их несколько, пусть $i(x)$ будет минимальным).

\begin{wrapfigure}{r}{40mm}
\begin{lpic}[t(-5mm),b(-0mm),r(0mm),l(0mm)]{pics/square(1)}
\lbl[b]{27,11.6;$\xto{\ f\ }$}
\end{lpic}
\end{wrapfigure}

Выберем $n^2$ различных единичных векторов 
$u_1,\dots,u_{n^2}$.
Рассмотрим отображение $f\:\square\to\RR^2$ 
определённое как
$$f(x)=d(x)\cdot u_{i(x)}.$$
Легко видеть что $f$ короткое.
Образ $f(\square)$ состоит из $n^2$ отрезков длины $\tfrac{1}{2{\cdot}n}$ исходящих из одной точки.
Соответственно периметр $f(\square)$ равен 
$\tfrac{2{\cdot}n^2}{2{\cdot} n}= n$.
Значит, взяв $n$ достаточно большим, 
можно сделать периметр образа $f(\square)$ произвольно большим.

(На рисунке показан случай $n=4$.  
В этом случае периметр $f(\square)$ равен периметру $\square$.  
Для $n>4$ периметр становится больше.)

\addtocontents{toc}{\protect\end{quote}}
\section*{Заключительные замечания}\label{sec:pdp-comments}
\addcontentsline{toc}{section}{Заключительные замечания}
\markboth{\MakeUppercase{Заключительные замечания}}{}


\parbf{Теорема Залгаллера.}
Теорема Залгаллера выполняется в любой размерности: 
произвольное $m$-мерное полиэдральное пространство $P$ 
допускает кусочно изометрическое отображение в $\RR^m$.

В \cite{zalgaller-PL}, Залгаллер доказал утверждение при $m\le 4$.
Трюк описанный во втором доказательстве дал возможность обобщить это доказательство на все размерности.
Этот трюк был предложен Крат в её диссертации \cite{krat}.

В старших размерностях, симплекс можно назвать остроугольным если он содежит центр описанной сферы.
Такое впечатление, что об остроугольных триангуляций в старших размерностях
ничего не известно, 
хотя их её существование в общем случае выглядит неправдоподобным.

\parbf{Теорема Брэма о продолжении.}
Эта теорема была доказана Брэмом в \cite{brehm}
и передоказана независимо Акопяном и Тарасовым в \cite{akopyan-tarasov}.
Оба доказательства строятся на той же идее.

Теорема Брэма выполняется во всех размерностях и доказывается тем же методом.

\parbf{Теорема Киршбрауна.}
Эта теорема была доказана Киршбрауном в его диссертации, защищённой в 1930. 
Несколько лет спустя он опубликовал этот результат в \cite{kirszbraun}.
Независимо тот же результат доказал Валентайн \cite{valentine}.

Книжка Данцера, Грюнбаума и Кли \cite{DGK} 
(превосходно написанная) 
приводит доказательство этой теоремы основанное на теореме Хелли о пересечении выпуклых тел в евклидовом пространстве.

\parbf{Теорема Акпяна об аппроксимации.}
Эта теорема также обобщается на старшие размерности.
Более того, условие что $f$ кусочно линейно можно обойти.
Для этого доказывается, что короткое отображение из полиэдрального пространства в $\RR^m$
допускает аппроксимацию короткими кусочно линейными отображениями.

Двумерный случай этой теоремы был доказана Крат в её диссертации \cite{krat}.
В \cite{akopyan},
Акопян заметил, что теорема Брэма о продолжении 
упрощает доказательство и также даёт возможность обобщить его на старшие размерности.

Гораздо раньше
аналогичные вопросы рассматривали Бураго и Залгаллер.
Они доказали, что любое кусочно линейное вложение полидральной поверхности в $\RR^3$ 
можно аппроксимировать кусочно изометрическим вложением,
см. \cite{burago-zalgaller-0} и \cite{burago-zalgaller}. 


\parbf{Сплющивание сферы.}
Теорема \ref{thm:S2->R2} 
допускает следующее обобщение,
которое может быть доказано тем же способом.

\begin{thm}{Теорема Громова о сплющивании}
Пусть $M$ есть $m$-мерное риманово многообразие.
Тогда любое короткое отображение $f\:M\to \RR^m$ 
может быть аппроксимировано отображениями сохраняющими длины.
Более точно, для любого $\eps>0$ существует отображение $f_\eps\:M\to\RR^m$
сохраняющее длины и такое, что 
$$|f_\eps(x)-f(x)|<\eps$$
для любого $x\in M$.
\end{thm}

В свою очередь, последний результат является
частным случаем теоремы в книжке Громова \cite[2.4.11]{gromov}.
Доказательство приведённое нами отличается от громовского,
оно основано на доказательстве в \cite{petrunin-inverse}.
Это доказательство даёт возможность строить отображения сохраняющие дины в $\RR^m$
из более экстравагантных пространств, 
в частности из многообразий с субримановой метрикой.

Неформально говоря, 
теорема Громова утверждает, 
что у класса отображений  сохраняющих длины «нет нетривиальных глобальных свойств».
Это пример типичного вопроса современной геометрии:
выяснить какие глобальные свойства вытекают из данных локальных свойств.
В нашем случае, 
«сохранение длины» играет роль «локального» свойства
и единственное «глобальное» следствие то что такое отображение короткое.

Для такого типа вопросов «от локального к глобальному»,
ответ «нет нетривиальных глобальных свойств»
появляется чаще других, 
но это не значит, что такой ответ легко доказывается.
Для таких доказательств существует технология изобретённая Громовым, 
это так называемый «гомотопический принцип» или «\textit{h}-принцип».
\textit{H}-принцип --- 
это свойство которое часто выполняется для локальных геометрических структур.
Есть несколько методов доказательства \textit{h}-принципа, 
включая тот, который был использован в доказательстве теоремы \ref{thm:S2->R2}.%
\footnote{Обычно \textit{h}-принцип формулируется для уравнений в частных производных, 
но о «отображениях сохраняющих длины» можно думать как о слабых решениях определённого уравнения в частных производных.}
В дополнении к теореме Громова о сплющивании, 
вот ещё несколько примеров таких утверждений.
\begin{itemize}
\item \emph{Теорема выворачивании конуса} утверждает, 
что существует непрерывное однопараметрическое семейство гладких функций $f_t(x,y)$, $t\in[0,1]$, без критических точек в $\RR^2\backslash\{0\}$, 
такие, что
$f_0(x,y)=-\sqrt{x^2+y^2}$ и $f_{1}(x,y)=\sqrt{x^2+y^2}$.
Доказательство приведено в книжке Табачникова и Фукса \cite[Лекция 27]{TF} (хорошая книжка, читайте целиком).
\item \emph{Теорема Нэша — Кейпера}, влечёт существование $C^1$-гладкого отображения $\SS^2\to\RR^3$ сохраняющего длины чей образ имеет произвольно малый диаметр.
\item \emph{Теорема Смейла о выворачивании сферы}, 
утверждает что есть непрерывное однопараметрическое семейство гладких отображений $f_t\:\SS^2\z\to \RR^3$, $t\in[0,1]$ 
таких, что $f_0\:\SS^2\to \RR^3$ 
стандартное вложение и $f_1(x)=-f_0(x)$ для всех $x\in \SS^2$.
\item Объединив техники доказательств теорем Смейла и Нэша --- Кейпера, 
можно построить выворачивание сферы $f_t$ в классе $C^1$-гладких отображений сохраняющих длины. 
\end{itemize}
 
Для ознакомления с \textit{h}-принципом 
мы рекомендуем книжку Элиашберга и Мишачёва \cite{eliashberg-mishachev}.

\parbf{Оригами.} 
Аспекты оригами связанные с построениями обсуждаются в статье Хулла \cite{hull},
есть также статья на русском первого автора \cite{petrunin-origami}.
Математические аспекты с сложением бумаги вдоль кривых обсуждаются в книжке Табачникова и Фукса \cite[Лекция 15]{TF}.