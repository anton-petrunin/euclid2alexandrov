\tableofcontents
\newpage
\section*{Введение}
\addcontentsline{toc}{section}{Введение}
\addtocontents{toc}{\protect\begin{quote}}

Это часть лекций прочитанных 
на MASS-программе (\url{www.math.psu.edu/mass/}) осенью 2011-го года.
Интернет версия этих лекций \cite{petrunin-yashinsky-arXiv} также содержит видео иллюстрации,
подсказки к упражнениям и минималистское введение.

Лекции обсуждают кусочно изометрические отображения из двумерных полиэдральных пространств на плоскость.
Можно думать, что полиэдральное пространство склеено из бумажных треугольников, 
как например поверхность многогранника.  
Тогда кусочно изометрическое отображение на плоскость
есть способ разгладить эту модель на столе. 

Чтобы упростить изложение, 
мы рассматриваем только двумерный случай.
Тем не менее, б\'{о}льшая часть результатов обобщаются на старшие размерности.
Эти результаты обсуждаются в заключительных замечаниях,
там же даны необходимые ссылки.


\parbf{Благодарности.}
Мы выражаем наши благодарности 
Арсению Акопяну,
Роберту Лэнгу,
Алексею Тарасову
за помощь.
Также мы хотим поблагодарить наших студентов за искренний интерес.

\parbf{Обозначения.}
Первых трёх глав книжки «Метрическая геометрия» \cite{BBI}
достаточно для понимания этих лекций.

Ниже приведён список не вполне стандартных соглашений используемые в лекциях.
\begin{itemize}
\item Расстояние между точками $x$ и $y$ в метрическом пространстве $X$
будет обозначаться 
$$|x-y|,
\ \ 
\text{или}\ \  |x-y|_X.$$ 

\item $\length(\alpha)$ обозначает длину кривой $\alpha$.

\item Метрическое пространство $X$ называется \emph{внутренним}
если для любых двух точек $x,y\in X$ и любого $\eps>0$, 
существует кривая $\alpha$ от $x$ до $y$ такая, что
$$\length(\alpha)<|x-y|_X+\eps.$$
\item \label{def:length-preserving}
Пусть $f\:X\z\to Y$ есть непрерывное отображение между метрическими пространствами, 
тогда
\begin{itemize}
\item $f$ называется \emph{изометрическим} если
$$|f(x)-f(y)|_Y=|x-y|_X$$
для любой пары точек $x,y\in X$;
\item $f$ называется \emph{коротким} если
$$|f(x)-f(y)|_Y\le|x-y|_X$$
для любой пары точек $x,y\in X$;
\item $f$ \emph{сохраняет длины} если для любой кривой $\alpha$ в $X$ выполняется равенство
$$\length(\alpha)=\length(f\circ\alpha).$$
\end{itemize}

\item Метрическое пространство $P$ называется \emph{полиэдральным}
если оно допускает конечную триангуляцию, 
такую что каждый её симплекс изометричен симплексу в евклидовом пространстве.
\end{itemize}



\addtocontents{toc}{\protect\end{quote}}