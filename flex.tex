\chapter{Flexible polyhedra}\label{chap:flex}
%ready



In this chapter we construct few examples of flexible polyhedra (with self intersections and without).
These examples are not used directly further in the lectures, 
but they might help to build right intuition in forcomming two chapters.


\section{Flexible vs. rigid}

Consider a simplicial complex $\mathcal{T}$ which is homeomorphic to the sphere $\SS^2$.
We are interested in maps $f\:\mathcal{T}\to\RR^3$ which are linear on each triangle.
(We assume that the image of each triangle in $\mathcal{T}$ is not degenerate.

Let $k$, $l$ and $m$  be the number of vertices, edges, and triangles 
in $\mathcal{T}$.
Label the vertices of $\mathcal{T}$ as $\{v_1,v_2,\dots,v_k\}$.
Note that the map $f$ is completely determined by values 
$$w_i=f(v_i)=(x_i,y_i,z_i)\in\RR^3.$$

Consider the following phisical model described by the map $f$.
If a pair $(v_i,v_j)$ is connected by an edge in $\mathcal{T}$,
let us connect the vertices $w_i$ and $w_j$ by a rigid bar 
and connect all the bars coming from one vertex $w_i$ with joint-hinge.
If the obtained model admits a motion different from an isometry of Euclidean space,
then $f$ is said to be \emph{flexible}\index{flexible};
if not, then $f$ is \emph{rigid}\index{rigid}.

One can reformulate the flexibility/rigidity the following way.
Consider polyhedral metric on $\SS^2$ which makes the restriction of $f$
to each triangle to be distance preserving.
This metric will be called \textit{the metric induced by $f$}.
Note that the induced metric is completely determined by the lengths of images of all edges of $\mathcal{T}$.
Without loss of generality, we may assume that $v_1$, $v_2$ and $v_3$ 
are vertexes of one triangle in $\mathcal{T}$.
The map is called flexible if there is a nontrivial continuous one-parameter family of maps $f_t$,
such that $f_0=f$ and $f_t(v_1)$, $f_t(v_2)$, $f_t(v_3)$ do not depend on $t$;
if there is no such $f_t$
then $f$ is called rigid.

\parbf{Counting constrains.}
Without loss of generality, we may assume that $w_1$ is the origin of $\RR^3$;
$w_2$ lies on $x$-axis 
and $w_3$ lies in the $xy$-plane.
In other words, 
\begin{align*}
w_1&=(0,0,0),& 
w_2&=(x_2,0,0),&
w_3&=(x_3,y_3,0)
\end{align*}
Thus, up to isometry of $\RR^3$,
the map $f$ is completely described by
$3\cdot k- 6$ numbers
$$(x_2),(x_3,y_3),(x_4,y_4,z_4),\dots,(x_k,y_k,z_k).\eqlbl{eq:xyz-flex}$$

Now let us count number of constraints.
We want to preserve the length of each bar;
i.e., if vertices $v_i$ and $v_j$ are connected by an edge in $\mathcal{T}$ then 
$$|w_i-w_j|_{\RR^3}=a_{ij},
\eqlbl{eq:edge}
$$
where $a_{ij}$ is the length of this edge.
(The equation \ref{eq:edge} is quadratic if written in the variables $x_i$, $y_i$, $z_i$, $x_j$, $y_j$ and $z_j$.)
All together $f$ has to satisfy system of $l$ equations of the form \ref{eq:edge}.

According to Euler's formula, we get
$$k-l+m=\chi(\SS^2)=2.$$
Further each edge appears as a side in exactly two triangles and each triangle has three sides;
i.e., we have
$3\cdot m=2\cdot l$.
Therefore
$$l=3\cdot k-6.$$
Thus, the number of equations ($l$) coincides with the number of parameters ($3\cdot k-6$) 
in \ref{eq:xyz-flex}.

\begin{wrapfigure}{r}{48mm}
\begin{lpic}[t(-30mm),b(-30mm),r(0mm),l(-3mm)]{pics/no-cauchy(0.25)}
\end{lpic}
\end{wrapfigure}

Untill we understand our equasions better, this coincidence does not mean much.
But assuming that the equasions are independed (in fact this is the case), it means that $f$ is rigid for ``generic'' choice of points $w_i$.  

\parbf{Example.} 
The system \ref{eq:edge} describes all the positions of $k$ points with given distances between the chousen $l$ pairs.
For example on the picture you see two configurations of $5$ points which have the same distances between $9$ pairs marked by the edges. 
Obvoisly these configurations could not be obtained one from the other by rigid move of $\RR^3$.

On the other hand each of these two maps is rigid.
In particular, it is impossible to get from one these polyhedra to the other one by a continuous deformation which keeps each edge rigid. 

\begin{thm}{Exercise}\label{ex:rigit/flex}
Prove the rigidity of these two configurations. 
\end{thm}






\section{Bricard octhedra}

In this section we construct the first nontrivial example of \emph{flexible polyhedral map}\index{flexible polyhedral map}.
These so called Bricard's octahedra are the piecewise linear maps $f\:\SS^2\z\to\RR^3$
with triangulation as in the surface of octahedra.
The map  have self-intersections,
i.e., $f$ fails to be embedding.
You may look at the animated image of \textattachfile[color=0 0 1,mimetype=picture/gif]{pics/Br1-anim.gif}{Bricard's octahedron}.

\parbf{Construction.} 
The polyhedral space $P$ is glued out of 8 triangles in the same way as the usual octahedron.
The triangulation, say $\mathcal{T}$, is the same as in an octahedron; it has 8 triangles, 12 edges and 6 vertices.
We contruct Bricard's octahedra such that $P$ isometric to the surface of convex centrally simmetric octahedron, 
say $K$ which is not non-regular.

The flexible map $f\:P\to\RR^3$ is isometric on each face,
but it does not map $P$ to the surface of $K$.
The 6 vertices of $K$ will be denoted as $x,y,z,x',y',z'$,
any pair of these vertices is connected by an edge, with exception for the 3 pairs $(x,x')$, $(y,y')$ and $(z,z')$.
Since $K$ is centrally simmetric,
the midpoints of the line segments $[xx']$, $[yy']$ and $[zz']$ coincide.
In particular, the points $x$, $x'$, $y$ and $y'$ lie on one plane, say $\Pi$.

The flexible map is obtained by reflecting $z'$ along with the 4 edges coming from $z'$ through $\Pi$.
The new map is distance preserving on each of the 8 triangles of $\mathcal{T}$.

\parbf{Why it is flexible.} Note first that the part glued out of 4 triangles with vertex $z$ is flexible.
This is true if each angle $\angle xzy$, $\angle xzy'$, $\angle x'zy$, $\angle x'zy'$ is strictly larger than the sum of the remaining three angles.
In particular, if the spherical quadrilateral $Q$ with vertices formed by unit vectors in the directions of rays $[zx)$, $[zy)$, $[zx')$ and $[zy')$ is not degenerate.
Note that flexibility of $Q$ in the sphere is equivalent to flexibility of our 4 triangles in $\RR^3$.
I.e., we can fix vertices $x$, $y$ and $z$ and move $x'$ and $y'$ along nontrivial curves  $x'(t)$, $y'(t)$ 
such that
 $x'(0)=x'$, $y'(0)=y'$
and all of the distances 
\begin{align*}
|z-x'(t)|,&&|z-y'(t)|,&&
|y-x'(t)|, &&|x'(t)-y'(t)|,&&|y'(t)-x|
\end{align*}
stay constant while $t$ changes. 

Set $z'(t)$ to be the rotation of $z$ by angle $\pi$ around the line passing through the midpoints of line segments $[x\,x'(t)]$ and  $[y\,y'(t)]$. 
For $t=0$, the midpoints of $[x\,x'(0)]$ and  $[y\,y'(0)]$ coincide;
in this case $z'$ is the rotation of $z$ by angle $\pi$ around the line passing through the midpoint of $[x\,x'(0)]$ and perpendicular to the plane containing $x, y, x'(0), y'(0)$.
In this case we have 
\begin{align*}
|z'(t)-x|&=|z-x'(t)|,
&
|z'(t)-y|&=|z-y'(t)|, 
\\
|z'(t)-x'(t)|&=|z-x|,
&
|z'(t)-y'(t)|&=|z-y|.
\end{align*}

Therfore
\begin{align*}
|z'(t)-x|,&&|z'(t)-y|, &&|z'(t)-x'(t)|,&&|z'(t)-y'(t)|
\end{align*}
stay constant while $t$ changes.


\section{Connelly sphere} 

Here we use Bricard's octahedra to produce a flexible polyhedral embedding which we call \emph{Connelly sphere}\index{Connelly sphere}.
It uses 3 Bricard's octahedra and gives a flexible embedding of sphere for a triangulation with 24 vertices.

We suggest to do the following exexcise before going into construction.

\begin{thm}{Exercise} 
Print \pageref{bricard}, 
cut the figure and glue the marked sides. 
You obtain 
 6 out of 8 triangles in a Bricard's octahedra.

Move this models, folding only along marked lines to make another flat polygon (not the square which you started with).
Imagine the missing edge and understand that its length does not change while you move the model.
\end{thm}

Start with a Bricard's octahedron glued from two squares as in the exercise.
The identified sides are decorated the same way.
One the picture below, you see ``upper'' and ``lower'' sides%
\footnote{These sides on the same level, we call one upper and one lower to distiguish them.}%
, both are squares and the triangulation is marked by solid lines. 
The dashed line is a line of self-intersection after a small deformation.

\begin{wrapfigure}{r}{58mm}
\begin{lpic}[t(-30mm),b(-30mm),r(0mm),l(-3mm)]{pics/connelly1(0.3)}
\end{lpic}
\end{wrapfigure}
\label{connelly1}



Remove 3 big triangles from the upper side and exchange each by 3 triangles with common vertex quite a bit above of the corresponding triangle.
This way you exchange 3 triangles to 9 and add 3 extra vertices.
Do the same for the lower side, but choose the vertex quite a bit below the corresponding triangle.
The next picture shows how it will look from above, form below and a side view.
The new model is still flexible 
and for small deformations its self intersections appear only at the spots marked by red.

\begin{center}
\begin{lpic}[t(-45mm),b(-45mm),r(0mm),l(-3mm)]{pics/connelly2(0.4)}
\end{lpic}
\end{center}

Consider one of the red spots on the upper side.
Cover it by two small triangles say $\Delta_1$ and $\Delta_2$ joined along the edge.
Construct a Bricard's octahedron as in the exercise with $\Delta_1$ and $\Delta_2$ as the missing faces.
Remove $\Delta_1$ and $\Delta_2$ from the model and glue instead the $6$ faces of the Bricard's octahedron.
This new model is still flexible.
For the right choice of the small triangles and the Bricard's octahedron, 
we can get rid of this self-intersection near this red spot.

If the same is done to the other red spot, we get a flexible embedding. 


\section{Comments}

The Bricard's octahedra were discoverd by Raoul Bricard in 1897;
in fact he classified all the flexible octahedra.
This classification includes the flexible octahedra constructed in Problem~\ref{bricard-second}.

The Connelly sphere was discovered by  Robert Connelly in 1977, see \cite{connelly-flex}.
The number of vertecies in this eaxample can be easely reduced to 20.
An other construction due to Klaus Steffen reduces the number of vertices to 9.
It is unknown if this number can be reduced further,
but maybe no one wants to know.  

In the late 1970s Connelly and Sullivan formulated so called \emph{Bellows conjecture} stating that
flexible polyhedron ``can not be used as bellows''.
More precisely that the volume of a flexible polyhedron is invariant under flexing. 
This conjecture was proved for by Sabitov in 1996 \cite{sabitov}.

There is also so called \emph{Strong belloes conjecture} which states that if $K$ and $K'$ be the regions bounded by a flexible polyhedron before and after flexing then $K$ and $K'$ are scissors congruet; i.e., one can cut $K$ into polhedons and rearange them into form of $K'$.
This conjecture is still open.

The flexibility and rigidity were extensively studied in the smooth category.
A $C^\infty$-smooth embedding $f_0\:\SS^2\to \RR^3$,
is called \emph{flexible}\index{flexible smooth surface} if there is one parameter family of $C^\infty$-smooth embeddings $f_t$, $t\in[0,1]$,
which is continuous in $t$ such that the which give the same induced metric on $\SS^2$,
but such that $f_t$ can not be presented as composition of isometry $\iota_t\:\RR^3\to \RR^3$ and $f_0$;
otherwise the embedding $f_0$ is called \emph{rigid}\index{rigid smooth surface}.
The existence of flexible smooth embedding is unknown by now;
the same applies for  flexible smooth immersions which could be defined the same way.





\section*{Exercises}


\begin{pr}\label{pr:trig-S2}
How many combinatorically different%
\footnote{Two triangulations $\mathcal{T}$ and $\mathcal{T}'$ are the same combinatorially if there is a bijection $f: V \to V'$ between their vertex sets such that \begin{enumerate}[i)] \item there is an edge from $v$ to $w$ in $\mathcal{T}$ if and only if there is an edge from $f(v)$ to $f(w)$ in $\mathcal{T}'$; \item there is a triangle connecting $u,v,w$ in $\mathcal{T}$ if and only if there is a triangle connecting $f(u), f(v), f(w)$ in $\mathcal{T}'$. \end{enumerate}}
triangulations of $\SS^2$ with 5 vertices are there?
Show that any map for corresponding simplicial complex(es) such that no 4 vertices lie in one plane is(are) rigid. 
\end{pr}

\begin{pr}\label{bricard-second}
Let $A=\{x,y,x',y'\}\in\RR^3$ be a set of distinct points such that
$|x-y|=|y-x'|=|x'-y'|=|y'-x|$.
Show that there are $4$ distinct isometries (including identity)
which send $A$ to it-self.

Use these isometires to construct 3 distinct flexible octahedra with vertices $x,y,x',y'$
and a given vertex $z$ in general position. 
\end{pr}

\begin{pr}\label{pr:alexander-flex}
Let $f,g\:\SS^2\to \RR^3$ be two maps which are linear on a each triangle of a a triangulation $\mathcal T$ of $\SS^2$.
Assume that the induced metrics on $\SS^2$ for $f$ and $g$ coincide.  
Let us consider $\RR^3$ as a subspace in $\RR^6$.
Show that there is a continuous one-parameter family of maps $h_t\:\SS^2\to \RR^6$, $t\in [0,1]$ such that 
for any $t$, the map $h_t$ is linear on each triangle of $\mathcal T$, the metric induced on $\SS^2$ by $h_t$ does not depend on $t$ 
and $f=h_0$, $g=f_1$. 
\end{pr}


\newpage

\begin{center}
\begin{lpic}[t(-0mm),b(-0mm),r(0mm),l(-0mm)]{pics/bricard(0.40)}
\end{lpic}
\end{center}
\label{bricard}




