\chapter{An application to crystallography}\label{chap:erdos}

Here we apply spaces with non-negative curvature in the sence of Alexandrov 
to crystallography.
The proof relies on the following problem in discrete geometry.

\section{Erd\H os problem}

\begin{thm}{Problem}
Assume $x_1,x_2,\dots,x_m$ is a collection of points in $n$-dimensional Euclidean space such that 
$\angle\hinge{x_i}{x_j}{x_k}\le \tfrac\pi2$ for any distinct $i$, $j$ and $k$. 
Show that $m\le 2^n$ and moreover, if $m= 2^n$ then the $x_i$ form the set of vertices of a right parallelepiped.
\end{thm}

This problem was posed by Erd\H os and solved by Danzer and Gr\"unbaum.

\parit{Proof.}
Let $K$ be the convex hull of $x_1,x_2,\dots,x_m$.
Without loss of generality we may assume that $K$ is a non-degenerate convex polyhedron.
(Otherwise, instead of $\RR^n$ take the minimal subspace which contain $K$; its dimension has to be $<n$.)

First let us prove the following

\begin{clm}{}\label{clm:=<pi/2}
$\mangle\hinge{x_i}{v}{w}\le \tfrac\pi2$ for each $i$ and any $v,w\in K$.
\end{clm}

Indeed, assume contrary.
For fixed $x_i$ and $v$,
the set of points $H_v$ containing $x_i$ and all $w\in\RR^n$ such that $\mangle\hinge{x_i}{v}{w}\le \tfrac\pi2$
is a half-space.
Thus, if $\mangle\hinge{x_i}{v}{w}>\tfrac\pi2$ for some $w\in K$ then
 $K\not\subset H_v$ and so $x_j \notin H_v$ for some $j$; i.e., $\mangle\hinge{x_i}{v}{x_j}> \tfrac\pi2$ for some $j$.
Repeating the same argument for $x_j$ instead of $v$, we get that 
$\mangle\hinge{x_i}{x_k}{x_j}> \tfrac\pi2$ for some $j$ and $k$, a contradiction.

For each $x_i$ denote by $K_i$ the dilation of $K$ with center $x_i$ and coefficient $\tfrac12$.

\begin{clm}{}\label{clm:K_inK_j}
For any $i\ne j$, the polyhedra $K_i$ and $K_j$ have no common interior points.
In particular, $\vol (K_i\cap K_j)=0$.
\end{clm}

Assume there is an interior point $v$ of $K_i\cap K_j$.
Without loss of generality, we can assume that $|v-x_i|\ne |v-x_j|$. 
Then there are points $y_i, y_j\in K$ such that
$v$ is the midpoint of $[x_iy_i]$ and $[x_jy_j]$.
Hence $x_iy_jy_ix_j$ is a parallelogram,
therefore $\mangle\hinge{x_i}{x_j}{y_j}+\mangle\hinge{x_j}{x_i}{y_i}=\pi$.
From \ref{clm:=<pi/2} we get that $\mangle\hinge{x_i}{x_j}{y_j}=\mangle\hinge{x_j}{x_i}{y_i}=\tfrac\pi2$.
I.e., $x_iy_jy_ix_j$ is a rectangle and therefore $|v-x_i|= |v-x_j|$, a contradiction.

Clearly $\vol K_i=\tfrac1{2^n}\cdot\vol K$ and $K_i\subset K$ for each $i$.
From \ref{clm:K_inK_j}, we get
$$\sum_{i=1}^m\vol K_i\le \vol K.$$
Hence the result.
\qeds


\section{Isometric actions}

Let $X$ be a metric space.
Denote by $\Isom X$ the set of all isometries of $X$.
Let $G$ be a nonempty subset of $\Isom X$ such that the following condition\footnote{In the language of group theory, $\Isom X$ is a group and $G$ is a subgroup of $\Isom X$.} holds:
\medskip
\begin{itemize}
\item \textit{Given two isometries $f,g\in G$
 the composition $f\circ g$ as well as the inverse $f^{-1}$ are in $G$.}
\end{itemize}
\medskip
In this case we say that the \emph{group} $G$ \emph{acts on}\index{isometric group action} $X$ \emph{by isometries}.
For example, one can take $G=\Isom X$ or $G=\{\id_X\}$.

In this case, given $g\in G$ and $x\in X$ the $g$-image of $x$ will denoted by $g\cdot x$, that is, $g\cdot x = g(x)$.
Consider the relation ``$\sim$'' on $X$ such that $x\sim y$ 
if and only if there if there is $g\in G$ such that $g\cdot x=y$.
Given $x\in X$, the $G$-orbit of $x$ is defined as
$$G\cdot x=\set{g\cdot x}{g\in G.}$$
Note that $G\cdot x=G\cdot y$ if and only if $x\sim y$;
i.e., $\sim$ is an equivalence relation and $G$-orbits form the $\sim$-equivalence classes.

The set of $G$-orbits of $X$ is denoted by $X/G$. 
Denote by $\pi\:X\to X/G$ the natural surjective map, $\pi\:x\mapsto G\cdot x$.

In the case that every $G$-orbit is a closed subset of $X$, 
then the set of orbits $X/G$ can be equipped with the following metric
\begin{align*}
|\pi(x)-\pi(y)|_{X/G}
&\df\inf_{g\in G}\{|x-g\cdot y|_X\}.
\end{align*}


\begin{thm}{Exercise}
Check that $|{*}-{*}|_{X/G}$ is a metric on $X/G$.
\end{thm}

The set $X/G$ equipped with this metric is called the quotient space;
for the quotient space, we keep the same notation $X/G$.

Note that
$$|\pi(x)-\pi(y)|\le |x-y|\eqlbl{eq:pi(xy)=<(xy)}$$ 
for any $x,y\in X$;
i.e. $\pi\:X\to X/G$ is a distance non-expanding map.

\parbf{Examples:}
\begin{itemize}
\item $\ZZ$ acts on $\RR$ by shifts $n\cdot x\df 2\cdot\pi\cdot n+x$; 
in this case $\RR/\ZZ$ is isometric to $\SS^1\subset \RR^2$ with the induced length metric.
\item The set of all isometries of $\RR$ which can be presented as a composition 
of a finite number of shifts $x\mapsto x+1$ and reflections $x\mapsto -x$ forms a group, 
say $G$;
in this case $\RR/G$ is isometric to the interval $[0,\tfrac12]$.
\item $\SS^1$ acts on $\RR^2$ by rotations which fix the origin;
in this case $\RR^2/\SS^1$ is isometric to the ray $[0,\infty)$.
\end{itemize}


\begin{thm}{Proposition}\label{prop:length/G}
Let $X$ be a proper length space such that a group $G$ acts on $X$ by isometries and has closed orbits.
Then  $X/G$ a proper length space.
\end{thm}

\parit{Proof.}???
Given $\bar x$ and $\bar y$ in $X/G$, choose an arbitrary $x\in X$ such that $\pi(x)=\bar x$.
Further, since $X$ is proper and the orbits are closed,
we can choose $y$ such that $\pi(y)=\bar y$ and 
$$|x-y|=|\bar x-\bar y|.$$

I.e. for any $\bar y\in \bar B_r(\bar x)$ there is $y\in \bar B_r(x)$ such that $\pi(y)=\bar y$,
or equivalently $\pi(\bar B_r(x))=\bar B_r(\bar x)$ for any $r>0$.
Since $X$ is proper, $\bar B_r(x)$ is compact;
hence $\bar B_r(\bar x)\subset X/G$ is compact for any $r>0$ because it is the continuous image of a compact set,
and therefore $X/G$ is proper.

It remains to show that $X/G$ is a length space.
Let $\gamma$ be a curve from $x$ to $y$
such that 
$$\length\gamma<|x-y|_X+\eps$$
Consider curve $\bar \gamma=\pi\circ\gamma$ in $X/G$.
Note that $\bar\gamma$ is a curve from $\bar x$ to $\bar y$
and from \ref{eq:pi(xy)=<(xy)}, we get
\begin{align*}
\length\bar\gamma\le\length\gamma
\end{align*}
Hence
$$\length\bar \gamma<|\bar x-\bar y|_{X/G}+\eps$$
\qedsf


\begin{thm}{Proposition}\label{prop:CBB/G}
Let $X\in \CBB{}{0}$ such that a group $G$ acts on $X$ by isometries and has closed orbits.
Then  $X/G\in\CBB{}{0}$  
\end{thm}

\parit{Proof.}
Applying Proposition~\ref{prop:length/G},
we get that $X/G$ is a proper and length space.
By Theorem~\ref{thm:defs_of_alex}\ref{SHORT.1+3}, it remains to show that 
$$\angk{}{\bar p}{\bar x}{\bar y}+\angk{}{\bar p}{\bar y}{\bar z}+\angk{}{\bar p}{\bar z}{\bar x}
\le 2\cdot\pi.
\eqlbl{eq:1+3-X/G}$$
holds for any $\bar p,\bar x,\bar y, \bar z\in X/G$.


Choose arbitrary $p\in X$ such that $\pi(p)=\bar p$.
Since $X$ is proper and orbits are closed, 
we can choose $x$, $y$ and $z\in X$ such that
\begin{align*}
\pi(x)&=\bar x,
&
|p-x|_X&=|\bar p-\bar x|_{X/G},
\\
\pi(y)&=\bar y, 
&
|p-y|_X&=|\bar p-\bar y|_{X/G},
\\
\pi(z)&=\bar z,
&
|p-z|_X&=|\bar p-\bar z|_{X/G}.
\end{align*}
By the definition of metric in $X/G$,
we have 
\begin{align*}
|x-y|_X&\ge|\bar x-\bar y|_{X/G},
&
|y-z|_X&\ge|\bar y-\bar z|_{X/G},
&
|z-x|_X&\ge|\bar z-\bar x|_{X/G}.
\end{align*}
Taking all this into account, we get
$$\begin{aligned}
\angk{}pxy&\ge\angk{}{\bar p}{\bar x}{\bar y},
&
\angk{}pyz&\ge\angk{}{\bar p}{\bar y}{\bar z},
&
\angk{}pzx&\ge\angk{}{\bar p}{\bar z}{\bar x},
\end{aligned}
\eqlbl{eq:X/G=<X}
$$
Since $X\in\CBB{}{0}$,
we have 
$$\angk{}pxy+\angk{}pyz+\angk{}pzx\le 2\cdot\pi.$$
This inequality plus \ref{eq:X/G=<X} implies \ref{eq:1+3-X/G}.
\qeds






\section{Number of isolated fixed-point orbits}

Let $X\in\CBB{}{0}$.
A point $p\in X$ is called \emph{extremal}\index{extremal point} if $\mangle\hinge pxy\le\tfrac\pi2$ 
for any hinge $ \hinge pxy$ in $X$.

For example if $X\iso [0,1]$ then both ends $0$ and $1$ are extremal points and the remaining points are not extremal.
Further in a $n$-dimensional cube, each of $2^n$ vertices is an extremal point and the remaining points are not extremal.
In a regular triangle, the vertices are the only extremal points, 
In a regular pentagon there are no extremal points.

Let $G$ act by isometries on $\RR^n$.
According to Proposition~\ref{prop:CBB/G},
$\RR^n/G\in\CBB{}{0}$.
Assume that the action of $G$ on $\RR^n$ is \emph{properly discontinuous}\index{properly discontinuous};
i.e., given a compact set $K\subset \RR^n$ and $x\in\RR^n$, 
there are only finitely many elements $g\in G$ such that $g\cdot x\in K$.

In this case $\RR^n/G$ is a polyhedral space;
it is not hard to prove, but we will not give a proof here.
In particular, we can talk about volume and dimension of $\RR^n/G$.

It tuns out that a point $\bar x\in \RR^n/G$ is extremal if and only if
one (and therefore any) point $x\in \RR^n$ such that $\pi(x)=\bar x$
is an isolated fixed point for some subgroup of $G$.
More precisely, if $G_x$ is the subset\footnote{The subset $G_x$ is in fact a subgroup of $G$ and it is called the \emph{stabilizer}\index{stabilizer} of $x$.} of all isometries $g$ in $G$ 
such that $g\cdot x=x$, then for any $y\ne x$ there is $g\in G_x$ such that $g\cdot y\ne y$.
Therefore counting the number of such $G$-orbits is equivalent to the counting extremal points in $\RR^n/G$.
All this means that the following theorem implies some nontrivial information about the $G$-action.

\begin{thm}{Theorem}
Suppose $G$ acts by isometries on $\RR^n$ and this action is properly discontinuous.
Then $\RR^n/G$ has at most $2^n$ extremal points.
\end{thm}

Before going into proofs, let us consider a couple of examples which show that $\RR^n/G$ can have exactly $2^n$ extremal points.

Consider the set of all isometries which can be presented as a composition of parallel translations $x\mapsto x+v$ with a vector $v$ with all integer coordinates, and all the reflections in the coordinate hyperplanes.
This defines a group action, say $G$ on $\RR^n$ for which the quotient $\RR^n/G$ is isometric to the cube $[0,\tfrac12]^n$.
Note that each vertex of the cube is an extremal point.










