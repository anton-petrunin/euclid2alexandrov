\chapter{??? Curvature bounded above}\label{chap:cba}

\section{Definitions}

\begin{thm}{Definition}\label{def:cba}
A proper length space $X$ has non-negative curvature in the sense of Alexandrov (briefly $X\in\CAT{}{0}$%
\footnote{$\CAT{}{0}$ stays for ``curvature bounded above by $0$''. 
If in the definition of model triangle, one exchanges the Euclidean plane with a sphere or Lobachevsky plane of constant curvature $k$, then one gets the definition of \emph{spaces with curvature $\le k$ in the sense of Alexandrov}, which denoted as $\CAT{}{k}$. We will only consider the case $k=0$.})
if the following inequality
$$|z-p|_X\le |\tilde z-\tilde p|_{\RR^2}
\eqlbl{eq:def:cba}$$
holds for any triangle $[xyz]$ in $X$, its model triangle $[\tilde x\tilde y\tilde z]=\modtrig{}(xyz)$,
any point $p\in\left]xy\right[$ and the corresponding point $\tilde p\in \left]\tilde x\tilde y\right[$.
\end{thm}

\begin{thm}{Unique geodesics}\label{lem:cat-unique} 
In a $\Cat{}{0}$ space, pairs of points  are joined by unique geodesics, and these geodesics depend continuously on their endpoint pairs.
\end{thm}

\parit{Proof.} 
Let $\spc{U} \in \Cat{}{0}$, $p^1,p^2\in \spc{U}$.
Suppose $p^1_n\to p^1$, $p^2_n\to p^2$ as $n\to\infty$.
Let $z_n$ be the midpoint of a geodesic $[p^1_n p^2_n]$ and $z$ be the midpoint of a geodesic $[p^1p^2]$.  

It suffices to show that 
\[\dist{z_n}{z}{}\to0
\ \ \text{as}\ \ 
n\to\infty.
\eqlbl{eq:z_n->z}\]

By the triangle inequality, the $z_n$ are approximate midpoints of $p^1$ and $p^2$. 
Apply (2+2)-point comparison (\ref{def:2+2}) to the quadruple $p^1,p^2,z_n,z$. 
For $p=p^1$ or $p=p^2$, we see that $\angk{0} {p}{z_n}{z}$ is arbitrarily small when $n$ is sufficiently large.  
By the law of cosines, \ref{eq:z_n->z} follows.
\qeds

\begin{thm}{Theorem}
\label{thm:defs_of_cat} 
Let  $\spc{U}$ be a geodesic space.  Then
\begin{subthm}{cat-ineq} 
$\spc{U}\in\Cat{}{0}$
\end{subthm}
if and only if 
one of the following conditions holds for all $p,x,y\in \spc{U}$:

\begin{subthm}{cat-2-sum} (adjacent-angles comparison\index{comparison!adjacent-angles comparison}) for any geodesic $[x y]$ and $z\in \left]x y\right[$, we have
\[\angk0 z p x
+\angk0 z p y\ge \pi.\]
\end{subthm}

\begin{subthm}{cat-monoton}
(point-on-side comparison\index{comparison!point-on-side comparison}) 
for any geodesic $[x y]$ and $z\in \left]x y\right[$, we have
\[\angk0 x p y\ge\angk0 x p z,\]
or equivalently, 
\[\dist{\tilde p}{\tilde z}{}\ge \dist{p}{z}{},\]
where $\trig{\tilde p}{\tilde x}{\tilde y}=\modtrig0(p x y)$, 
$\tilde z\in\left] \tilde x\tilde y\right[$, $\dist{\tilde x}{\tilde z}{}=\dist{x}{z}{}$.
\end{subthm}

\begin{subthm}{cat-hinge}(%hinge 
angle comparison\index{% hinge comparison})
for any hinge $\hinge x p y$, the angle 
$\mangle\hinge x p y$ exists and
\[\mangle\hinge x p y\le\angk0 x p y.\]
\end{subthm}
%SBA:  a ``hinge'' in English is a ``movable mechanism''.  It must MOVE. BBI uses it correctly:  if you look up ``hinge'' in BBI index, it sends you to monotonicity.
\end{thm}


\parbf{Remark.}
\label{22remark}
In the following proof, the part (\ref{SHORT.cat-ineq})$\Rightarrow$(\ref{SHORT.cat-2-sum})
only requires that the (2+2)-point comparison (\ref{def:2+2}) hold for any quadruple, and does not require the existence of geodesics. 
The same is true of the parts (\ref{SHORT.cat-2-sum})$\Leftrightarrow$(\ref{SHORT.cat-monoton}) and
(\ref{SHORT.cat-monoton})$\Rightarrow$(\ref{SHORT.cat-hinge}).  Thus the conditions (\ref{SHORT.cat-2-sum}), (\ref{SHORT.cat-monoton}) and (\ref{SHORT.cat-hinge}) are valid for any metric space (not necessarily intrinsic) which satisfies (2+2)-point comparison (\ref{def:2+2}). 
The converse does not hold; for example, all these conditions are 
vacuously true in a 
totally disconnected space, while 
(2+2)-point comparison is not.

\parit{Proof. (\ref{SHORT.cat-ineq}) $\Rightarrow$ (\ref{SHORT.cat-2-sum})}. 
By Alexandrov's lemma (\ref{lem:alex}), 
\[\angk0 p z x +\angk0 p z y  < \angk{0} p x y \text{\,\,\, or \,\,\,} \angk0 z p x  +\angk0 z p y  =\pi.\]
In the former case, (2+2)-point comparison (\ref{def:2+2}) applied to the quadruple $p, z, x, y$ implies
\[\angk0 z p x  +\angk0 z p y  \ge \angk{0} z x y =\pi.\]

\parit{(\ref{SHORT.cat-2-sum}) $\Leftrightarrow$ (\ref{SHORT.cat-monoton})}. Follows directly from  Alexandrov's lemma (\ref{lem:alex}).

\parit{(\ref{SHORT.cat-monoton}) $\Rightarrow$ (\ref{SHORT.cat-hinge}).} 
By (\ref{SHORT.cat-monoton}), for $\bar p\in\left]x p\right]$ and $\bar y\in\left]x y\right]$ the function $(\dist{x}{\bar p}{},\dist{x}{\bar y}{})\mapsto\angk0 x{\bar p}{\bar y}$ is nondecreasing in each argument.
In particular, 
$\mangle\hinge x p y\z=\inf\angk0 x{\bar p}{\bar y}$.
Thus $\mangle\hinge x p y$ exists and is
at most $\angk0 x p y$. 

\parit{(\ref{SHORT.cat-hinge}) $\Rightarrow$ (\ref{SHORT.cat-2-sum}).} 
By (\ref{SHORT.cat-hinge}) and the triangle inequality for angles (\ref{claim:angle-3angle-inq}),
\[\angk0 z p x
+\angk0 z p y \ge \mangle\hinge z p x
+\mangle\hinge z p y \ge \pi.\]

\begin{wrapfigure}{r}{25mm}
\begin{lpic}[t(0mm),b(0mm),r(0mm),l(0mm)]{pics/cat-monoton-ineq(0.45)}
\lbl[rb]{23,30;$\tilde p^1$}
\lbl[lt]{22,1;$\tilde p^2$}
\lbl[rb]{3,20;$\tilde x^1$}
\lbl[lt]{48,20;$\tilde x^2$}
\lbl[tl]{23,19;$\tilde q$}
\end{lpic}
\end{wrapfigure}

\parit{(\ref{SHORT.cat-monoton}) $\Rightarrow$ (\ref{SHORT.cat-ineq}).}
Given a quadruple  $p^1,p^2,x^1,x^2$, we must verify (2+2)-point comparison (\ref{def:2+2}).
Construct the model triangles  $\trig{\tilde p^1}{\tilde p^2}{\tilde x^1} = \modtrig0(p^1 p^2 x^1 )$ 
and $\trig{\tilde p^1}{\tilde p^2}{\tilde x^2}= \modtrig0(p^1 p^2 x^2)$, lying on either side of a common segment $[\tilde p^1 \tilde p^2]$.
We may suppose 
\[\angk{0} {p^1}{p^2}{x^1}+\angk{0} {p^1}{p^2}{x^2}
\le
\pi
\ \ \text{and}\ \ 
\angk{0}{p^2}{p^1}{x^1}+\angk{0} {p^2}{p^1}{x^2}
\le 
\pi,\] 
since otherwise (2+2)-point comparison holds trivially.  
Then $[\tilde p^1 \tilde p^2]$ and $[\tilde x^1 \tilde x^2]$ intersect, say at $\tilde q$.  

By assumption, there is a geodesic $[p^1 p^2]$.
Choose $q\in[p^1 p^2]$ corresponding to $\tilde q$; 
that is, $\dist{p^1}{q}{}=\dist{\tilde p^1}{\tilde q}{}$.
Then 
\[\dist{x^1}{x^2}{} \le \dist{x^1}{q}{} + \dist{q}{x^2}{} \le \dist{\tilde x^1}{\tilde q}{} + \dist{\tilde q}{\tilde x^2}{} = \dist{\tilde x^1}{\tilde x^2}{},\]
where the second inequality follows from (\ref{SHORT.cat-monoton}). 
Since increasing one side of a planar triangle increases the opposite angle,
\begin{align*}
\angk{0} {p^1}{x^1}{x^2} \le  \mangle\hinge{ \tilde p^1}{ \tilde x^1}{ \tilde x^2}
= \angk{0} {p^1}{p^2}{x^1} + \angk{0} {p^1}{p^2}{x^2}.
\end{align*}
\qedsf



\section{Reshetnyak's gluing theorem}

\begin{thm}{Reshetnyak's Gluing Theorem}\label{thm:reshetnyak}
Let $X$ be a proper geodesic space which contains two convex subsets
$X_1$ and $X_2$ such that $X_i\in \CAT{}{0}$ and $X=X_1\cup X_2$ then $X\in\CAT{}{0}$
\end{thm}

This theorem is mostly used to construct examples
of $\CAT{}{0}$ spaces. One can take two $\CAT{}{0}$ spaces $X_1$ and $X_2$
with convex sets $C_i \subset X_i$ 
and an isometry $f\: C_1 \to C_2$.
Then attach these spaces together along the isometry $f$.
It is easy to check that the induced map of $X_i$ in the glued space, say $X$, 
is distance preserving and from ??? it follows that  
$X$ is geodesic. 
Therefore, according to above theorem the resulting space $X$ is a $\CAT{}{0}$ space.

For example, one can glue two copies of $\RR^n$ along an isometry between closed unit balls in them,
the resulting space will be $\CAT{}{0}$.


\parit{Proof.}
Set $C= X_1\cap X_2$.
Since both $X_i$ are convex, we have that so is $C$.

We need to show that any given triangle $[xyz]$ in $X$ is thin.
The comparison trivially holds if all the vertices lie in one of $X_i$.
Hence, without loss of generality, we may assume that $x\in X_1$ and $y,z\in X_2$.

Then there are points 
$\bar y,\bar z\in C$ such that
$\bar y\in [xy]$
and $\bar z\in [xz]$. 
Decompose triangle $[xyz]$ into three triangles $[x\bar y\bar z]$, $[\bar y\bar z z]$ and $[\bar y z y]$.
Applying Inheritance for thin triangles (\ref{lem:inherit-angle}) twice,
first for the triangle $[x\bar y z]$ decomposed into $[x\bar y\bar z]$, $[\bar y\bar z z]$
and then for  the $[x y z]$ decomposed into $[x\bar y z]$, $[\bar y y z]$,
we get that $[x y z]$ is thin. \qeds

\section*{Exercises}

\begin{pr}
Let $P$ be a two-dimensional polyhedral space.
Show that $P\in\CAT{}{0}$ 
if and only if any two points in $P$ are joined by unique geodesic. 

Does the same hold for higher dimensional polyhedral spaces?
\end{pr}
