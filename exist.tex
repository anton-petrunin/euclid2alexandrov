\chapter{Alexandrov's existence theorem}\label{chap:exist}

\section{The goal}

In this chapter we prove the following theorem.

\begin{thm}{Alexandrov's existence theorem}\label{thm:alex-exist}
Let $P$ be a polyhedral space which is homeomorphic to $\SS^2$.
If $P$ has non-negative curvature at each point,
then there is a convex polyhedron $K$ whose surface is isometric to $P$.

(The convex polyhedron in the theorem can degenerate to a convex polygon.
In this case its surface is defined as its doubling; 
i.e., two copies of the polygon glued along the boundary. 
See Section~\ref{sec:suface}.)
\end{thm}

The proof of this theorem relies on Alexandrov's uniquencess theorem (\ref{thm:alexandrov-uni'}).
Together with Alexandrov's uniquencess theorem, Alexandrov's existence theorem 
give an ``if and only if'' characterization of surfaces of convex polyhedra.
Namely the following result holds.

\begin{thm}{Alexandrov's theorem}\label{thm:alexandrov}
A metric space $P$ is isometric to the surface of a convex polyhedron in $\RR^3$ %???
if and only if $P$ is a polyhedral space which is homeomorphic to $\SS^2$,
and the curvature of $P$ is non-negative at each point.

Moreover, the polyhedron associated to $P$ is unique up to isometry of $\RR^3$.
\end{thm}

\parbf{The plan.}
In Section~\ref{sec:polhedra} 
we consider the space of convex polyhedra in $\RR^3$.
In Section~\ref{sec:poly-space} we consider the space of polyhedal spaces.
In these two sections we introduce a number of notations, 
and prove that if we fix the number $k$ of vertices,
then after removing neglectable sets,
both of the spaces become $(3\cdot k-6)$-dimensional manifolds.

In Section~\ref{sec:outline},
we use the introduced notation to give an outline of the proof of the existence theorem.
After that, it remains to prove three lemmas:
the Open lemma, the Closed lemma and the Connecting lemma.
The Closed lemma and the Open lemma are proved in Section~\ref{closed+open}.
The proof of the Connecting lemma (\ref{lem:Phi(P)-connect}) occupies two sections
\ref{sec:deformation}
and \ref{sec:connect}.
In Section \ref{sec:deformation} we give a couple more definitions and introduce Alexandrov's patching construction, which is a key ingredient in the proof.


\section{Space of convex polyhedra}\label{sec:polhedra}

In this section, we consider the space $\mathbf{K}_k$
formed by the congruence classes of
convex polyhedra in $\RR^3$ with exactly $k$ vertices.
The main result of this section is Lemma~\ref{lem:P-mnld-nosimm}.

Assume $K$ is a convex polyhedron in $\RR^3$ with exactly $k$ vertices.
Let us denote by $[K]$ its congruence class;
i.e., if $K'$ is another convex polyhedron in $\RR^3$ with exactly $k$ vertices, then $[K]=[K']$ if and only if $K$ is congruent to $K'$.

The set of all congruence class of convex polyhedra in $\RR^3$ with exactly $k$ vertices 
will be denoted by $\mathbf{K}_k$.
This set is equipped with the metric
$$|[P]-[P']|_{\mathbf{K}_k}
\df
\inf_{\iota}\{|P-\iota(P')|_{\mathcal{H}(\RR^3)}\},$$
where the infimum is over all isometries $\iota\:\RR^3\to\RR^3$.
(In other words, $\mathbf{K}_k$ is the subspace of $\mathcal{H}(\RR^3)/_{\cong}$ formed by congruence classes of convex polyhedra with exactly $k$ vertices,
see Exercise~\ref{ex:hausdorff-upto-cong}.)

The statement \emph{$K$ is a convex polyhedron with exactly $k$ vertices in $\RR^3$} can be written as $[K]\in \mathbf{K}_k$, 
but we often will write it as $$K\in\left]\mathbf{K}_k\right[,$$ formally here $\left]\mathbf{K}_k\right[$ stands for the union of all congruence classes in $\mathbf{K}_k$.

We say that $K\in\left]\mathbf{K}_k\right[$ \emph{has a symmetry} if there is an isometry $\iota\:\RR^3\to\RR^3$, which is distinct from identity,
such that $\iota(K)=K$.
If there is no such isometry, we say that  $K$ has \emph{no symmetry}.

Note that any segment or triangle in $\RR^3$ has a symmetry.
On the other hand, for any $k\ge 4$,
a generic $K\in\left]\mathbf{K}_k\right[$ 
has no symmetry.

\begin{thm}{Exercise}\label{ex:nosym-K}
Prove that the set of congruence classes of polyhedra with no symmetry form an open dense subset of $\mathbf{K}_k$ for any $k\ge 4$.
\end{thm}


\begin{thm}{Lemma}\label{lem:P-mnld-nosimm}
Fix an integer $k\ge 4$.
Assume $K\in\left]\mathbf{K}_k\right[$ has no symmetry.
Then $[K]$ admits 
an open neighborhood in $\mathbf{K}_k$ 
which is homeomorphic 
to an open domain in $\RR^{3\cdot k-6}$.
\end{thm}

In the proof, we will use the following technical statement.
Roughly it states that if two polyhedra are close in the sense of Hausdorff, then their vertex sets are also close.
(Do not be scared by the long formulation;
this statement is much easier to prove than it is to formulate.)

\begin{thm}{Exercise}\label{ex:eps-delta}
Let $K$ be a convex polyhedron in $\RR^3$ 
with $k$ vertices,
say $w_1$, $w_2,\dots,w_k$.
Then for any $\eps>0$ there is $\delta>0$ 
such that if $K'$ is another convex polyhedron 
with $k$ vertices
such that 
$|K-K'|_{\mathcal{H}(\RR^3)}<\delta$ 
then the vertices of 
$K'$
can be labeled by $w_1'$, $w_2',\dots,w_k'$
in such a way that 
$$|w_i-w_i'|<\eps$$ 
for each $i$.

Moreover for the unique orientation preserving isometries $\iota,\iota'\:\RR^3\to \RR^3$
such that 
\begin{align*}
\iota(w_1)&=(0,0,0)
&
\iota'(w_1')&=(0,0,0),
\\
\iota(w_2)&=(x_2,0,0),
&
\iota'(w_2')&=(x_2',0,0),
\\
\iota(w_3)&=(x_3,y_3,0),
&
\iota'(w_2')&=(x_3',y_3',0)
\end{align*}
with $x_2,x_2',y_3,y_3'>0$
we have 
$$|\iota(w_i)-\iota'(w_i')|<\eps$$ 
for each $i$.
\end{thm}

\parit{Proof of Lemma~\ref{lem:P-mnld-nosimm}.}
Let $K$ be a convex polyhedron with vertices $w_1,w_2,\dots,w_k$.

Let us choose orthogonal coordinates $(x,y,z)$ in $\RR^3$
so that
\begin{align*}
 w_1&=(0,0,0),
\\
w_2&=(x_2,0,0),
\\
w_3&=(x_3,y_3,0),
\\
w_4&=(x_4,y_4,z_4),\\
&\vdots\\
w_k&=(x_k,y_k,z_k)
\end{align*}
and $x_2,y_3> 0$.

Note that one can choose small enough $\eps>0$
such that if   
$$\begin{aligned}
x_i'&\lg x_i\pm\eps,&
y_i'&\lg y_i\pm\eps,&
z_i'&\lg z_i\pm\eps
\end{aligned}
\eqlbl{eq:xyz}$$
for all $i$ then
 the points
$$\begin{aligned}
w_1'&=(0,0,0),
\\
w_2'&=(x_2',0,0),
\\
w_3'&=(x_3',y_3',0),
\\
w_4'&=(x_4',y_4',z_4'),\\
&\vdots\\
w_k'&=(x_k',y_k',z_k').
\end{aligned}
\eqlbl{eq:000x}
$$
form vertices of a convex polyhedron, say $K'$.


Since $K$ has no symmetry and $\eps>0$ is small,
the polyhedron $K'$ also has no symmetry.

\begin{thm}{Exercise}
Prove the last statement.
\end{thm}

Therefore 
$$(x_2',x_3',y_3',x_4',y_4',z_4',\dots,x_k',y_k',z_k')
\mapsto 
[K']$$ 
describes a continuous injective map 
$f\:U\to\mathbf{K}_k$,
where
\begin{align*}
 U=(x_2-\eps,x_2+\eps)\times (x_3-\eps,x_3+\eps)&\times (y_3-\eps,y_3+\eps)\times
\dots
\\
&\dots\times(z_k-\eps,z_k+\eps)\subset\RR^{3\cdot k-6}.
\end{align*}

It remains to show that $f(U)$ contains a small neghborhood of the congruence class $[K]$ in $\mathbf{K}_k$.
In other words, given a convex polyhedron $K'$ 
with exactly $k$ vertices such that $|K-K'|_{\mathcal{H}(\RR^3)}$ is sufficiently small,
we need to find an isometry of $\RR^3$ which moves $K'$ into a new position, say $K''$,
such that its vertices could be described as in \ref{eq:000x} and \ref{eq:xyz} holds 
for all $i$.
This follows from the Exercise~\ref{ex:eps-delta}.
\qeds



\section{Space of polyhedral spaces}\label{sec:poly-space}

The main result of this section is Lemma~\ref{lem:poly-mnfld}.

Let $P$ be a polyhedral space 
with nonnegative curvature that is
homeomorphic to $\SS^2$.
A point $v\in P$ will be called a
\emph{vertex of $P$}\index{vertex of polyhedral space}
if $v$ has positive curvature.

Fix $k\ge 4$.
Let us denote by $\mathbf{P}_k$
the subspace of $\mathcal{M}$
formed by the isometry classes of polyhedral spaces
with nonnegative curvature,
with exactly $k$ vertices,
and which are
homeomorphic to $\SS^2$.
 

For $\mathbf{P}_k$ we will use conventions similar to the ones made for $\mathbf{K}_k$.
Namely given a polyhedral space $P$, 
we denote by $[P]$ its isometry class.
The statement 
\emph{$P$ is a polyhedral space with nonnegative curvature, with exactly $k$ vertices, and which is homeomorphic to $\SS^2$} 
can be written as $[P]\in \mathbf{P}_k$, 
but we often will write it as 
$$P\in\left]\mathbf{P}_k\right[,$$ 
formally here $\left]\mathbf{P}_k\right[$ 
stays for the union of all isometry classes in $\mathbf{P}_k$.

A space $P\in \left]\mathbf{P}_k\right[$ is called
\emph{realizable}\index{realizable}
if there is $K\in \left]\mathbf{K}_k\right[$ such that $P$ is isometric to the surface of $K$.
So the Alexandrov's existence theorem can be stated as 
\emph{any space in $\left]\mathbf{P}_k\right[$ is realizable}.


\begin{thm}{Lemma}\label{lem:poly-mnfld}
Let $P\in\left]\mathbf{P}_k\right[$ be a realizable polyhedral with no symmetry. 
Then $[P]$ has a neighborhood in $\mathbf{P}_k$ which is homeomorphic to an open domain in $\RR^{3\cdot k-6}$.
\end{thm}

In the proof we will use the following technical statement.

\begin{thm}{Exercise}\label{ex:P,eps-delta}
Let $P, P'\in\left] \mathbf{P}_k\right[$ and $\eps>0$.
Assume $P$ admits a triangulation $\mathcal{T}$ 
with $k$ vertices 
such that each edge is formed by the unique geodesic between its endpoints.
Then there is $\delta>0$ such that if $f\:P'\to P$ is a $\delta$-isometry then

\begin{subthm}{}
The vertices of $P$ and $P'$ can be labeled 
as $v_1,v_2,\dots,v_k$ and $v_1',v_2',\dots,v_k'$ correspondingly
in such a way that $|f(v_i')-v_i|<\eps$ for any $i$.
\end{subthm}

\begin{subthm}{}
$P'$ admits a triangulation  $\mathcal{T}'$
such that a pair of vertices $(v_i',v_j')$ 
is connected by an edge if and only if 
the corresponding pair $(v_i,v_j)$ 
is connected by an edge in $\mathcal{T}$ 
and each edge of $\mathcal{T}'$ is the necessarily unique geodesic.
\end{subthm}
\end{thm}



\parit{Proof of Lemma~\ref{lem:poly-mnfld}.}
Since $P$ is realizable, 
we can identify $P$ with the surface of a convex polyhedron, 
say $K$.  Since $P$ has no symmetry,
Exercise~\ref{pr:K-P-simmetry} implies that $K$ has no symmetry.

Note that one can triangulate each face of $K$ in
a such a way that only vertices of the face are the vertices of the triangulation.
This follows from Exercise~\ref{pr:tringulation-of-poly},
but it is much easier since the face is convex;
see the comment in the solution to this exercise.

These triangulations together give a triangulation, 
say $\mathcal{T}$,
of $P$ with vertices only at the vertices of $P$.
In particular $\mathcal{T}$ has exactly $k$ vertices;
by  Euler's formula (\ref{thm:euler}), 
$\mathcal{T}$ has $3\cdot k-6$ edges.
Note that each edge of $\mathcal{T}$ is a line segment in $\RR^3$.
Therefore each edge of $\mathcal{T}$ is the necessarily unique geodesic between its endpoints in $P$.
(This property is not achieved  if $K$ degenerates to a plane polygon, but since $P$ has no symmetry, it cannot be degenerate.)

Note that the triangulation $\mathcal{T}$ together with the lengths of its edges, 
say $a_1$, $a_2,\dots,a_l$, 
describe $P$ up to isometry.
Indeed, the side lengths describe each triangle in $\mathcal{T}$
and $P$ can be obtained by gluing these triangles according the rule encoded in $\mathcal{T}$.

The same construction can be performed 
if we change the lengths of edges a bit.
Say, fix a sufficiently small $\eps>0$ 
and make the lengths of edges to be 
$a_1'$, $a_2',\dots, a_l'$,
such that 
$$a_i'\lg a_i\pm\eps$$ 
for each $i$.
If $\eps$ is small enough,
the sides of each triangle of $\mathcal{T}$ satisfy the strict triangle inequality.
Therefore one can cut such triangles from the plane and glue them 
together according the rule encoded in $\mathcal{T}$ and obtain a polyhedral space $P'$ which is homeomorphic to $\SS^2$.
Further note that the angle of a triangle depends continuously on its sides.
Hence for small enough $\eps>0$, 
the curvature of any vertex in $P'$ is still positive;
i.e., $P'\in\left]  \mathbf{P}_k\right[$.

Clearly if $\eps>0$ is small enough
then so is $|P-P'|_{\mathcal{M}}$.
I.e.,
$$f\:(a_1', a_2',\dots, a_l')
\mapsto 
[P']$$ 
give a continuous map 
$f\:U\to\mathbf{P}_k$,
where
\begin{align*}
 U=(a_1-\eps,a_1+\eps)\times (a_2-\eps,a_2+\eps)
\times\dots\times
(a_l-\eps,a_l+\eps)
\subset
\RR^l.
\end{align*}
It remains to establish the following two claims. 
\begin{clm}{}\label{clm:injective}
$f$ is injective. 
\end{clm}
 
\begin{clm}{}\label{clm:surjective} $f(U)$ contains a neighborhood of $[P]$ in $\mathbf{P}_k$.
\end{clm}

Assume $f$ is not injective for all $\eps>0$.
I.e.,
there is a collection of numbers $(a_1',a_2',\dots, a_l')$ arbitrarily close to $(a_1,a_2,\dots, a_l)$ such that $K'=f(a_1',a_2',\dots, a_l')$ has a symmetry.
Note that a symmetry of $K'$ has to give a nontrivial permutation of the vertexes of $K'$.
Passing to the limit as $a_i'\to a_i$ for each $i$,
we get that $K$ has a symmetry,
a contradiction.

The condition \ref{clm:surjective} follows from Exercise~\ref{ex:GH=>eps-isom}.
\qeds







\section{Outline of the proof}\label{sec:outline}

Let us denote by $\mathbf{R}_k$ the subset of $\mathbf{P}_k$
formed by isometry classes of realizable spaces.
As was already mentioned, it is sufficient to prove that
\begin{clm}{}
$\mathbf{R}_k=\mathbf{P}_k$ for any integer $k\ge 3$.
\end{clm}

In the proof, we use the following lemmas.

\begin{thm}{Closed lemma}\label{lem:Phi(P)-closed}
$\mathbf{R}_k$ is a closed subset of $\mathbf{P}_k$.
\end{thm}

Let us denote by $\mathbf{P}_k'$ the subspace of $\mathbf{P}_k$
formed by the isometry classes of spaces without symmetry.

\begin{thm}{Exercise}\label{ex:open-dense}
$\mathbf{P}'_k$ is an open dense%
\footnote{i.e., the closure of $\mathbf{P}'_k$ is $\mathbf{P}_k$.}
 subspace of $\mathbf{P}$.
\end{thm}

Further, 
set $\mathbf{R}_k'=\mathbf{R}_k\cap \mathbf{P}_k'$;
i.e. $P\in \left]\mathbf{R}_k'\right[$ if $P$  is a realizable space without symmetry from $\left]\mathbf{P}_k\right[$. 

\begin{thm}{Open lemma}\label{lem:Phi(P)-open}
$\mathbf{R}_k'$ is an open subset of $\mathbf{P}_k'$.
\end{thm}


\begin{thm}{Connecting lemma}\label{lem:Phi(P)-connect}
???Any two points in the space $\mathbf{P}_{k}$ can be connected by a path.
Moreover, any path in $\mathbf{P}_{k}$ with endpoints in $\mathbf{P}_{k}'$
can be deformed into a path in $\mathbf{P}_{k}'$

??? $\mathbf{P}'_{k}$ is connected;
i.e., $\emptyset$ and whole $\mathbf{P}'_{k}$
are the only subsets of $\mathbf{P}_{k}'$ which are open and closed at the same time.
\end{thm}


\parit{Proof of Existence theorem modulo the lemmas above.}
From the Closed and Open lemmas, 
it follows that  $\mathbf{R}_k'=\mathbf{R}_k\cap \mathbf{P}_k'$ is a subset of $\mathbf{P}_{k}'$ which is closed and open at the same time.
Clearly $\mathbf{R}_k'$ is nonempty.
Therefore, by the Connecting lemma, $\mathbf{R}_k'=\mathbf{P}_{k}'$.
Finally, applying Exercise~\ref{ex:open-dense} and the Closed lemma again, we get $\mathbf{R}_k=\mathbf{P}_{k}$.
\qeds

In the next section we prove the Closed and Open lemmas.
The Connecting lemma is proved in sections \ref{sec:deformation} and \ref{sec:connect}.
The proof of the connecting lemma goes by induction on $k$;
it is the hardest part of the remaining part of the proof.








\section{Closed and open lemmas}\label{closed+open}

\parit{Proof of Closed lemma.}
We need to  show that $P_n$ 
is a sequence of realizable spaces from $\left]\mathbf{P}_k\right[$
which is converging in the sense of Gromov--Hausdorff to a space $P_\infty\in \left]\mathbf{P}_k\right[$
then $P_\infty$ is realizable.

\begin{thm}{Exercise}
The diameter of convex polyhedron does not exceed the diameter of its surface.
\end{thm}

For each $P_n$, consider $K_n\in \left]\mathbf{K}_k\right[$
such that the surface of $K_n$ is isometric to $P_n$.
Note that diameter of all $P_n$ is bounded above by some real constant $R$.
According to the exercise, 
we may assume that all $K_n$ lie in a fixed ball of radius $R$ in $\RR^3$.
Therefore by Blaschke's compactness theorem (\ref{thm:compact+Hausdorff}) we may choose a subsequence of $K_n$
which converge in the sense of Hausdorff to say $K_\infty$.
Clearly $K_\infty$ is a polyhedron with at most $k$ vertices.
According to Problem~\ref{pr:H>GH-boundary}, the surface of $K_\infty$ is isometric to $P_\infty$. 
In particular, by ???, $K_\infty$ has exactly $k$ vertices and hence $P_\infty$ is realizable.
\qeds

The proof of Open lemma relies on lemmas \ref{lem:P-mnld-nosimm}, \ref{lem:poly-mnfld}
and the following theorem;
its rigorous proof of it given in Section \ref{sec:domain-invariance}.

\begin{thm}{Domain invariance theorem}\label{thm:domain-invariance}
If $\Omega$ is an open subset of $\RR^n$ 
and $f\:\Omega \to \RR^n$ is an injective continuous map,
then the image $f(\Omega)$ is open.
\end{thm}

Note that from Exercise~\ref{ex:compact-homeo} it follows that $f$ is a homeomorphism from $\Omega$ to $f(\Omega)$.
So this theorem might seem trivial, 
since a homeomorphic mapping of one space onto
another always takes open sets into open sets by the definition of homeomorphism.
However, what was said above implies only that the image of an open set is open
as a subset of $f(\Omega)$, but not as a subset of $\RR^n$. For example, under the identity mapping $[0,\infty)\to \RR$ the image is not an open subset of $\RR$, although it is open as a subset of $[0,\infty)$.


\parit{Proof of Open lemma.} 
Consider map $\Phi_k\:\mathbf{K}_k\to\mathbb{P}_k$
which sends congruence class $[K]\in \mathbf{K}_k$
to the isometry class of its surface.

According to Exercise ???, $\Phi_k$ is continuous
and by Alexandrov uniqness theorem $\Phi_k$ is injective.
If $[K]\in \mathbf{K}_k'$ then according to Problem ???, $\Phi_k[K]\in \mathbb{P}_k'$.
By Lemma ???, we can choose small neighborhoods $U\ni [K]$ in $\mathbf{K}_k'$ and $V\ni [K]$ in $\mathbb{P}_k'$
such that both $U$ and $V$ are homeomorphic to open sets in $\RR^{3\cdot n-6}$ and $\Phi(U)\subset V$.
By  Domain invariance theorem (\ref{thm:domain-invariance}),
$\Phi_k(U)$ is open subset of $V$.
In particular, the image $\Phi_k(\mathbf{K}_k')$ is open.
\qeds




 
 
\section{Deformation by patching}\label{sec:deformation}

In this section we describe a construction which will be used in the next section to prove Connecting lemma (\ref{lem:Phi(P)-connect}).


\parbf{Deformation of metric.}
Let $P$ be a compact metric space space;
as usual, denote by $|{*}-{*}|$ the metric on $P$.

A family of metrics $|{*}-{*}|_t$,
$t\in[0,1]$ on $P$
will be called \emph{deformation of $P$}
if $|{*}-{*}|_0=|{*}-{*}|$ 
and 
$$(t,x,y)\mapsto |x-y|_t$$
is a continuous function on
$[0,1]\times P\times P$.

Set $P_t=(P,|{*}-{*}|_t)$;
i.e., $P_t$ is a metric space with underlying space $P$ and the metric $|{*}-{*}|_t$, in particular $P_0=P$.  

Note that the family of metric spaces $P_t$ is continuous in the sense of Gromov--Hausdorff;
i.e., for any $t_0\in[0,1]$ we have 
$$t\to t_0\ \ \Rightarrow\ \ P_t\GHto P_{t_0}.$$

\parbf{Deformations and continuous families.}
To make a deformation from
a one parameter family of metric spaces $P_t$  
one has to produce a homeomorphism 
$h_t\:P_0\to P_t$ for each $t$,
such that
$$(t,x,y)\mapsto |h_t(x)-h_t(y)|_{P_t}$$
is a continuous function on  $[0,1]\times P_0\times P_0$.

Thus one may think of deformation as about 
a family of metric spaces $P_t$ with a family of homeomorphisms $h_t\:P_0\to P_t$ as above.

Note that not any one parameter family of metric spaces which is continuous in the sense of Gromov--Hausdorff forms a deformation.

\medskip

Now assume in addition that $P_t\in\left]\mathbf{P}_k\right[$ 
for any $t\in [0,1]$.
Fix a homeomorphism $\SS^2\to P_0$;
so we may think that $P_t=(\SS^2,|{*}-{*}|_t)$ 
for all $t$.
Note that in this case, one can label vertices of $P_t$,
say as $v^1_t, v^2_t,\dots,v^k_t$
in such a way that $t\mapsto v^i_t$ forms a path in $\SS^2$.
Moreover this labeling is unique up to permutation of the upper indexes.

\begin{thm}{Exercise}
Prove the last two statements. 
\end{thm}

This labeling gives a bijection between the vertices of $P_0$ and $P_t$, which will be referred further as the \emph{vertex correspondence of the deformation}.
We also will say that the vertex  $v^i_t\in P_t$
\emph{corresponds} to vertex $v^i\in P$.

\medskip

Let us use the introduced notation to formulate a slightly stronger version 
of Connecting Lemma (\ref{lem:Phi(P)-connect}), 
which turns out to be easier to prove using induction by $k$. 

Further we say that a 
\emph{deformation $(P,|{*}-{*}|_t)$ is from $P$ to $P'$}%
\index{deformation from $P$ to $P'$}
if $P\z\iso (P,|{*}-{*}|_0)$ 
and $P'\iso (P,|{*}-{*}|_1)$.
 


\parit{Patching construction.}\index{patching construction}
Choose any two vertices $p$ and $q$ in $P$  denote their curvatures by $\omega_p$ and $\omega_q$.

\begin{wrapfigure}{r}{50mm}
\begin{lpic}[t(-5mm),b(-0mm),r(0mm),l(0mm)]{pics/Pt(0.5)}
\lbl[rt]{9,15;$\tfrac{\omega_p}{2}$}
\lbl[lt]{82,14;$\tfrac{\omega_q}{2}$}
\lbl[rb]{16,17,-10;$\hat p=p$}
\lbl[lb]{77,17,10;$q=\hat q$}
\lbl[rb]{46,47;$p_t$}
\lbl[lb]{66,41;$q_t$}
\end{lpic}
\end{wrapfigure}

Cut $P$ along an arbitrary geodesic $[pq]$. 
In the obtained hole,
we will glue a one-parameter family of patches $A_t$; 
this way we obtain a deformation $P_t$ of $P$.  

Consider a one parameter family of convex quadrilaterals 
$[\hat{p}p_tq_t\hat{q}]$ in the plane
such that
$$\begin{aligned}
\mangle p_t\hat{p}\hat{q}
&=\tfrac {\omega_q}2,
&
\mangle\hat{p}\hat{q}q_t
&=\tfrac {\omega_q}2
\end{aligned}
\eqlbl{eq:<pqq_t}
$$
and
$|\hat{p}-\hat{q}|_{\RR^2}=|p-q|_P$.
We assume that the points $p_t$ and $q_t$ depend continuously on the parameter $t\in [0,1]$
and $p_0=\hat p$, $q_0=\hat q$;
so at $t=0$ the quadrilateral degenerates to the segment $[\hat p\hat q]$.

To construct the patch $A_t$,
take two copies of $[\hat{p}p_tq_t\hat{q}]$
and identify them along corresponding points on the sides 
$[\hat{p}p_t]$,
$[p_tq_t]$,
$[q_t\hat{q}]$.
Now  glue $A_t$ in the hole of $P$ 
so that 
$\hat p$ is glued to $p$ and $\hat q$ to $q$.
Denote by $P_t$ the obtained space.

\begin{thm}{Exercise}
Prove that $P_t$ is a deformation of $P$
by constructing the needed one parameter family of homeomorphisms 
$P\z\to P_t$.
\end{thm}

The polyhedral space $P_t$ has vertices at the points $p_t$ and $q_t$,
it also inherits every vertex of $P$ except $p$ and $q$. 
To see this, 
notice that 
a neighborhood of $[pq]$ in $P$ 
is isometric to two copies 
of the green quadrilateral on the picture, with identified the corresponding points 
on the sides marked by blue and black.
The patch $A_t$ is isometric to two copies 
of the purple quadrilateral, with identified corresponding points 
on the sides marked by red.
After cutting $P$ along $[pq]$ (the black edge on the picture) and gluing in the patch $A_t$, 
it will look like two copies of the union of two quadrilaterals, green and purple,
with the corresponding points 
on the sides marked by blue and red identified.

\begin{wrapfigure}{r}{50mm}
\begin{lpic}[t(-3mm),b(-0mm),r(0mm),l(0mm)]{pics/Pt-3(0.5)}
\lbl[rt]{9,15;$\tfrac{\omega_p}{2}$}
\lbl[lt]{82,14;$\tfrac{\omega_q}{2}$}
\lbl[rb]{16,17,-10;$\hat p=p$}
\lbl[lb]{77,17,10;$q=\hat q$}
\lbl[r]{54,56;$p_1=q_1$}
\end{lpic}
\end{wrapfigure}

\parit{Case of small curvature.}
If in addition 
$$\omega_p+\omega_q<2\cdot \pi$$
then one can choose $|p_1-q_1|$ arbitrary small and even equal zero.

In the later case the quadrilateral degenerates to a triangle;
so $P_1$ has $k-1$ vertices,
one less than $P_0$.

\medskip

Now we will use the patching construction to prove the following Lemma.
Briefly, it states that if $k\ge 4$, 
then any space $P_0$ with isometry class in $\mathbf{P}_k$ 
and two marked vertices $p_0,q_0$
can be deformed in $\mathbf{P}_{k}$
so that the the vertices corresponding to $p_0$ and $q_0$ get much closer than any other pair of vertices of  and their curvatures getting small.

\section{Connecting lemma; first part}\label{sec:connect}

In this section we use the patching construction described above to prove the following lemma.

\begin{thm}{Lemma}\label{lem:cont+}
Given $P, P'\in\left] \mathbf{P}_k\right[$, 
there is a deformation $P_t$ from $P$
to $P'$
such that 
 $P_t\in\left]\mathbf{P}_k\right[$ for all $t$.

Moreover:

\begin{subthm}{}
Any bijection between vertices of $P$ 
and $P'$ can be realized as the vertex correspondence  of the deformation $P_t$.
\end{subthm}

\begin{subthm}{}
???
The space $P_t$ has no symmetry for any $t\ne0,1$.
\end{subthm}


\end{thm}

First we prove the following.

\begin{thm}{Lemma}\label{lem:k>=4}
Let $k\ge 4$, $\eps>0$,
$P_0\in\left] \mathbf{P}_k\right[$ 
and $p_0,q_0$ be two vertices of $P_0$.
Then there is a deformation $P_t\in\left] \mathbf{P}_{k}\right[$
from $P_0$ to $P_1$ such that 
\begin{subthm}{}
The corresponding vertices $p_1$ and $q_1$ lie on the distance at least 10 times smaller than the distance between any other pair of vertices in $P_1$
\end{subthm}
\begin{subthm}{}
The curvatures of $P_1$ at $p_1$ and $q_1$ are smaller than $\eps$.
\end{subthm}
\end{thm}

\parit{Proof.} Note that the patching construction makes possible to a move any portion of curvature from one vertex to an other
as far as at each vertex the curvature is in the range $(0,2\cdot\pi)$.
I.e., given two vertices $x_0$ and $y_0$ in $P_0$ we can construct a deformation $P_t$ with corresponding vertices $x_t$ and $y_t$
such that the curvatures $\omega_{x_1}$ and $\omega_{y_1}$ take any values in $(0,2\cdot\pi)$ such that
$$\omega_{x_1}+\omega_{y_1}=\omega_{x_0}+\omega_{y_0}.$$



I claim that almost all curvature of $P_1$ 
can be moved to a given two vertices.
This can be easily arranged by applying the patching construction recursively;
i.e.,
\begin{itemize}
\item Start with $P_0$; 
choose a pair of vertices and construct a deformation $P_t$.
\item Choose an other pair of vertices in $P_1$ and apply the construction again to obtain a deformation from $P_1$ to $P_2$ and $t\in [1,2]$.
\item Repeat this as many times as necessary.
\item After $n$ times you get a ``long'' deformation $P_t$ with $t\in [0,n]$, which can be linearly  reparametrized by $[0,1]$.
\end{itemize}

Therefore, if $k\ge 4$,
then almost all curvature can be moved to the vertices different form $p_1$ and $q_1$,
making the curvatures at $p_1$ and $q_1$ arbitrary small.
Once it is done we may apply the case of small curvature of the patching construction 
to meet the condition on the distance
$|p_1-q_1|_{P_1}$.
\qeds






 





\parit{Proof of Lemma~\ref{lem:cont+}.}
I will apply induction on $k$.

\parit{Base case, $k=3$.}
Let $u$, $v$ and $w$ be the vertices of $P$
and $u'$, $v'$ and $w'$ be the vertices of $P'$.
Assume we want to construct a deformation $P_t$ for which 
$u$ corresponds to $u'$, 
$v$ corresponds to $v'$
and
$w$ corresponds to $w'$.

Cut $P$ by geodesics $[uv]$, $[vw]$ and $[wu]$.
Note that according to Exercise~\ref{ex:poly+geod}, these geodesics
intersect only at the common vertices.
Therefore after the cutting,
we obtain two congruent flat triangles.
It follows that $P$ is completely determined 
by three distances $a=|u-v|_P$, $b=|v-w|_P$ and $c=|w-u|_P$.

Repeat the same procedure for $P'$, 
we obtain three numbers $a'$, $b'$ and $c'$.

Set
\begin{align*}
a_t
&=(1-t)\cdot a+t\cdot a'
&
b_t
&=(1-t)\cdot b+t\cdot b'
&
c_t
&=(1-t)\cdot c+t\cdot c'.
\end{align*}
The both triples $a,b,c$ and $a',b',c'$
satisfy the strict triangle inequality.
Therefore the same holds for the triple $a_t,b_t,c_t$ for any $t\in[0,1]$.

Now, for each $t\in[0,1]$, 
glue the space $P_t$ from two copies of triangle with sides $a_t,b_t$ and $c_t$.
denote by $u_t$, $v_t$ and $w_t$ the corresponding vertices of $P_t$.

To make a deformation from the family $P_t$,
choose a homeomorphism $P\to P_t$ which sends linearly triangle to triangle and such that $u\mapsto u_t$, $v\mapsto v_t$ and $w\mapsto w_t$.

\parit{Induction step.}
Fix $k\ge 4$
and assume that the lemma is already proved if the number of vertices is strictly less then $k$.

Applying Lemma \ref{lem:k>=4},
we may assume that $P$ has two vertices
$p$ and $q$ with very small curvatures and on distance at least 10 times smaller than the distance between any other pair of vertices in $P$.
Analogously, we may assume in $P'$ the corresponding (by the given bijection) vertices 
$p'$ and $q'$ satisfy the same condition.

Let us apply the case of small curvature of the patching construction for $P$.
I.e.,
let us cut $P$ along $[pq]$ and glue in the hole a patch glued from two congruent triangles. 
This way we obtain a new polyhedral space, say $Q$,
which isometry class belongs to $\mathbf{P}_{k-1}$.
In the obtained space $Q$, the points $p$ and $q$ become regular, instead we get an other vertex which will be denoted by $z$.
Note that there are unique geodesics $[zp]$ and $[zq]$ and they cut at $z$ two equal angles.
 
Let $Q'\in\left]\mathbf{P}_{k-1}\right[$ be the result 
of the same construction for $P'$;
we denote by $z'$ the vertex which appear in $Q'$ instead of $p'$ and $q'$.

Applying the induction hypothesis, we can find a deformation $Q_t$ from $Q$ to $Q'$ 
such that $z$ corresponds to $z'$
and and the vertex correspondence for the rest of the vertices taken the same as the correspondence for $P$ and $P'$.
Denote by $z_t$ the vertex of $Q_t$ which corresponds to $z\in Q$.

Now we will use the obtained deformation $Q_t$
to construct the needed deformation $P_t$.
Note that we can choose points $p_t, q_t\in Q_t$ such that 
\begin{enumerate}[(i)]
\item $p_0=p$, $p_1=p'$, $q_0=q$, $q_1=q'$.
\item For any $t$, each distance $|z_t-p_t|_{Q_t}$ and $|z_t-q_t|_{Q_t}$ is positive
but smaller then the distance between any pair of vertices of $Q_t$
\item The (necessary unique) geodesics $[z_tp_t]$ and $[z_tq_t]$ cut two equal angles at $z_t$.
\item $p_t$, $q_t$ depend continuously on $t$; 
i.e., for the one parameter family of homeomorphisms $h_t\:Q\to Q_t$ of the deformation,
the maps $t\mapsto h^{-1}_t(p_t)$ and $t\mapsto h^{-1}_t(q_t)$ are both continuous paths in $Q$.
\end{enumerate}

Note that these conditions imply that the existence of two geodesics,
say $\gamma_1$ and $\gamma_2$
from $p_t$ to $q_t$ which form a digon in $Q_t$ containing $z_t$ inside. 
Cut $Q_t$ along the digon, 
in the remaining part (the one without $z_t$)
glue the corresponding points on $\gamma_1$ and $\gamma_2$.
We obtain a polyhedral space $P_t$;
the points $p_t$ and $q_t$ become vertices in $P_t$.

By construction, $P_0= P$ and $P_1= P'$.
It remains to construct a homeomorphism $P\to P_t$ for each $t$ which makes a deformation from the family $P_t$.
The later is left to the reader.
\qeds






\section{???Braking the symmetry}

It remains to show that ???.
Note that any symmetry of $P_t$ 
induce a nontrivial permutation of its vertices.
In particular, if the curvatures of all vertices are different then the space has no symmetry.
If both $P$ and $P'$ have no symmetry, it is easy to modify the above deformation in such a way that all curvatures are different
for all values with exception for finite number of values $t_1,t_2,\dots,t_n$
and at each $t_i$ there is exactly one pair of vertices with the same curvature.
If $P_{t_i}$ has a symmetry then the induced permutation only changes the places of these two vertices.



\section{Remarks}

The rest of the proof presented here is nearly the same as in original one.

There is an alternative, very interesting proof due to Volkov,
???.

\parbf{Connecting lema.}
The patching construction described below was introduced by Alexandrov in ???;
it was much later than his publiction of Existence theorem ???. 
It seems that I was the first to realize that this construction simplifies the the proof of Alexandrov existence theorem.

The Connecting lemma has a simple but not elementary proof using Teichm\"uller theory.

Here is the idea.
Let $P$ be a polyhedral space with exactly $k$ vertices.
Note that $P$ with all its vertices removed admits natural complex structure.
It describes a map, say $\Psi_k$ from $\mathbf{P}_k$ to the Teichm\"uller space $\mathcal{T}_k$ which is the space of that the space of conformal structures on sphere with $k$ points removed.
(The map consists of forgetting the metric, and
remembering only the conformal structure.)

The space $\mathcal{T}_k$ is homeomorphic to the space of configurations of $k$ points in the $\SS^2$
up to a conformal map $\SS^2\to \SS^2$.
It is easy to see that $\mathcal{T}_k$ is connected.

Further the preimage $\Psi_k^{-1}(T)\subset \mathbf{P}_k$ is also connected for any $T\in \mathcal{T}_k$.
To prove the later statement, one has to observe that $P$ can be uniquely recovered if one knows 
$T=\Psi_k(P)$,
the curvatures at each vertex 
and the total area of $P$.



\section*{Exercises}

\begin{pr}
Consider a regular octahedra $H$,
with vertices $a$, $a'$, $b$, $b'$, $c$, $c'$
and assume that the pairs $(a,a')$, $(b,b')$ and $(c,c')$
are opposite. 

Cut from $H$ a pyramid with vertex $a$ by a plane $\Pi$ parallel to the plane containing $b$, $b'$, $c$ and $c'$.
Denote by $P$ the remaining surface of $H$;
it is bounded by a broken line $xyx'y'$ which bounds a square in $\Pi$.

Consider the patch $R_\alpha$, which is a planar rhombus with the same side length as $xyx'y'$ and with angle $\alpha$ at one vertex.
Glue $R_\alpha$ to $P$ along $xyx'y'$ by a length-preserving map of its boundary vertex-to-vertex.
Denote the obtained space as $P_\alpha$.

\begin{enumerate}
\item For which $\alpha$ does the space $P_\alpha$ has non-negative curvature?
For such $\alpha$ denote by $K_\alpha$ the convex polyhedron with surface isometric to $P_\alpha$.

\item For which pairs of $\alpha, \alpha'$ are the polyhedra $K_\alpha$ and $K_{\alpha'}$ congruent?

\item Construct another polygonal patch (not a rhombus) which gives a space with non-negative curvature.

\item   Characterize all such patches.
\end{enumerate}
\end{pr}

\begin{pr}
Let $P$ be a non-negatively curved polyhedral metric on $\SS^2$. 
Cut a triangle $\Delta$ from $P$ along geodesics and equip it with the induced length metric.
Show that $\Delta$ is isometric to a planar triangle if and only if the sum of its angles is $\pi$.
\end{pr}

Hint: Use Problem~\ref{pr:sum>=pi} to prove the ``only if'' part. To prove the ``if'' part, construct a distance preserving map explicitly.





\begin{pr}\label{pr:sum>=pi}
Let $P$ be a non-negatively curved polyhedral space homeomorphic to a sphere
and let $\Delta$ be a triangle in $P$ bounded by 3 geodesics.
Denote by $\alpha, \beta$ and $\gamma$ the angles of $\Delta$.
Show that $\alpha+\beta+\gamma-\pi$  is equal to the sum of curvatures of all points in the interior of $\Delta$. 

In particular, the sum of the angles of any triangle in $P$ is at least $\pi$.
\end{pr}

Hint: Pass to the doubling of $\Delta$ and apply \ref{ex:sum=2pi} together with Exercise~\ref{ex:poly+geod}.

\begin{pr}  
Let $P$ be the surface of a regular tetrahedron.
Find a periodic local geodesic%
\footnote{see Definition~\ref{def:local-geod}} in $P$.
Show that any two distinct vertices can be joined by arbitrary long local geodesic.   
\end{pr}





\begin{pr}
Let $P$ be a non-negatively curved polyhedral space homeomorphic to a sphere with exactly 4 vertices
$a$, $b$, $c$ and $d$.
Let us draw on $P$ a geodesic between each pair of vertices $[ab]$,
$[bc]$, $[cd]$ and $[da]$.
Show that either these geodesics intersect only at the common ends, 
or exactly two of them intersect at an interior point.

In the latter case, show that $P$ is isometric to the doubling of a convex quadrilateral. 

(You may use Alexandrov's theorem, doing this problem directly is harder.)
\end{pr}


\begin{pr}
Show that $\mathbf{K}_k$ is connected.%
\footnote{Note that Alexandrov's theorem implies that $\mathbf{P}_k$ is also connected.}
\end{pr}




???\parbf{Base.} 
$\mathbf{R}_3=\mathbf{P}_3$.

\parit{Proof.}
Assume that $P\in \left]\mathbf{P}_3\right[$ has exactly three vertices, say $u$, $v$, and $w$. 
It is sufficient to show that 
\begin{clm}{}\label{clm:doule-trig}
$P$ is isometric to a
doubling of a planar triangle.
i.e., the surface of this triangle in $\RR^3$.
\end{clm}

Choose geodesics  $[uv]$, $[vw]$ and $[wu]$ between each pair of points.
According to Exercise~\ref{ex:poly+geod}
these geodesics do not intersect each other at the interior points. 

Cut $P$ along the geodesics $[uv]$, $[vw]$ and $[wu]$.
As a result we get two congruent flat triangles $\Delta$ and $\Delta'$.
Hence \ref{clm:doule-trig} follows.
\qeds 





   

