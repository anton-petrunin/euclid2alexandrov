\chapter{Appetizers}\label{chap:metr}

\section{Metric spaces}

\begin{thm}{Definitions}
\label{def:metric-space}
A metric space is a pair $(X,\Dist)$ where $X$ is a set and 
$$\Dist \: X \times X \rightarrow [0,\infty)$$
is a function such that 

\begin{subthm}{def:metric-space:zero}
$\Dist(x,y) = 0$ if and only if $x=y$;
\end{subthm}

\begin{subthm}{}
$\Dist(x,y) = \Dist(y,x)$ for any $x, y \in X$;
\end{subthm}

\begin{subthm}{triangle-inq}
$\Dist(x,z)\le \Dist(x,y)+\Dist(y,z)$ for any $x, y, z \in X$.
\end{subthm}

\end{thm}

The condition (\ref{SHORT.triangle-inq}) above is called \emph{triangle inequality}.
The set $X$ is called \emph{underlying set}\index{underlying set} of the metric space $(X,\Dist)$
and its elements are called \emph{points} of the metric space.
The function $\Dist\: X \times X \rightarrow [0,\infty)$ is called a \emph{metric}\index{metric}, 
the value $\Dist(x,y)\ge 0$ is called the \emph{distance}\index{distance} from $x$ to $y$.

%Unless it is stated otherwise, the metric will be always denoted $\Dist$.
Often we will we will say ``consider metric space $X$'' meaning that on the set $X$ is equipped with a metric  which is denoted as $\Dist$.
The distance $\Dist(x,y)$ between $x$ and $y$ in a metric space $X$ will be further also denoted as 
$$|x-y|=|x-y|_X=\Dist(x,y).$$
We will write $|x-y|_X$ to emphasize that the points $x$ and $y$ belong to the metric space $X$.

It should be noted that the expression ``$x-y$'' for two points in a metric space makes no sense
and $|x-y|$ should be read as \emph{distance from $x$ to $y$}\index{distance}.

\section*{Examples}
\begin{itemize}
\item Discrete metric. For any set $X$, the discrete metric on $X$ is defined by 
$$\Dist(x,y)=\left[
\begin{aligned}
&1&&\text{if}&x&\not=y
\\
&0&&\text{if}&x&=y
\end{aligned}
\right.$$

\item Euclidean space. 
The set is formed by arrays of $n$ real numbers
$$\bm{x}=(x_1,x_2,\dots,x_n)\in\RR^n$$
and the distance function defined as 
$$\Dist(\bm{x},\bm{y})
\df
\lVert\bm{x}-\bm{y}\rVert_2,$$
where 
$$\lVert\bm{x}\rVert_2
\df
\sqrt{x^2_1+x^2_2+\dots+x_n^2}.
$$
Further, this distance will be also denoted as $|x-y|_{\RR^n}$.
\end{itemize}

\begin{thm}{Exercise}\label{ex:Euclidean-is-metric}
Show that Euclidean space is a metric space.
\end{thm}

\begin{itemize}
\item \label{manhattan-metric} Manhattan metric on the plane. The set is formed by all pairs $\bm{x}\z=(x_1,x_2)\in\RR^2$ and the metric is defined as 
$$\Dist(\bm{x},\bm{y})
\df
\lVert\bm{x}-\bm{y}\rVert_1,$$
where 
$$\lVert\bm{x}\rVert_1
\df
|x_1|+|x_2|.
$$
\item %???
Space of functions with sup-norm.
Given a set $X$,
consider the set $\mathcal{F}(X)$ of all bounded functions 
$f\: X\to\RR$ with the metric defined as
$$\Dist(f,g)
\df
\lVert f-g\rVert_\infty,$$
where
$$\lVert f\rVert_\infty
\df 
\sup_{x\in X}|f(x)|.
$$

\item Unit sphere.
The set of unit vectors in $\RR^{n+1}$ can be equipped with the angle metric 
$$\Dist(u,v)=\arccos\langle u,v\rangle,$$ 
where $\langle {*},{*}\rangle$ denotes the scalar product.
The obtained metric is called $n$-dimensional unit sphere and it will be denoted by $\SS^n$.
\end{itemize}

\section*{Distance preserving maps}

Given an arbitrary subset $A\subset X$ of a metric space $X=(X,\Dist)$,
one can give $A$ the metric defined as the restriction of $\Dist$ to $A\times A\z\subset X\times X$. 
In this situation, the metric space $(A,\Dist|_{A\times A})$ is called a \emph{subspace}\index{subspace} $A$ of $X$.

\begin{thm}{Definition}
Let $X$ and $Y$ be metric spaces.
\begin{subthm}{}
A map $f\:X\to Y$ is \emph{distance preserving}\index{distance preserving map} if
$$|f(x)-f(x')|_Y=|x-x'|_X$$
for any $x,x'\in X$.
\end{subthm}

\begin{subthm}{}
A distance preserving bijection $f\:X\to Y$ is called an \emph{isometry}\index{isometry}.
\end{subthm}

\begin{subthm}{}
The spaces $X$ and $Y$ are called \emph{isometric}\index{isometric spaces} (briefly $X\iso Y$)
 if there is an isometry  $f\:X\to Y$.
\end{subthm}

\end{thm}

Note that 
\begin{itemize}
\item ``$\iso$'' is an equivalence relation on the class of metric spaces.
\item Distance preserving map is necessarily injective.
\item The existence of a distance preserving map $X\to Y$ is equivalent to the existence of subset of $Y$ which is isometric to $X$.
\end{itemize}

\section{Calculus}

\begin{thm}{Definition}
 Let $X$ be a metric space.
A sequence of points $x_1, x_2, \ldots$ in $X$ is called \emph{convergent}\index{convergent}
if there is 
$x_\infty\in X$ such that $|x_\infty -x_n|\to 0$ as $n\to\infty$.  
That is, for every $\eps > 0$, there is a natural number $N$ such that for all $n \geq N$, we have $$|x_\infty-x_n| < \eps.$$

In this case we say that the sequence $(x_n)$ converges to $x_\infty$ 
or $x_\infty$ is the limit of the sequence $(x_n)$
and write 
$x_\infty=\lim_{n\to\infty} x_n$
or $x_n\to x_\infty$ as $n\to\infty$.
\end{thm}

Note that any converging sequence has unique limit.

\begin{thm}{Definition}
Let $X$ and $Y$ be metric spaces.
A map $f\:X\to Y$ is called continuous if for any convergent sequence $x_n\to x_\infty$ in $X$,
the sequence $y_n\z=f(x_n)$ converges to $y_\infty=f(x_\infty)$ in $Y$.

Equivalently, $f\:X\to Y$ is continuous if for any $x\in X$ and any $\eps>0$
there is $\delta>0$ such that 
$$|x-x'|_X<\delta\ \Rightarrow\ |f(x)-f(x')|_Y<\eps.$$

\end{thm}

It is not hard to see that two definitions of continuity given above are equivalent;
try to prove it.

\section*{Open and closed sets}

\begin{thm}{Definition}
A subset $A$ of a metric space $X$ is called \emph{closed}\index{closed set} if whenever a sequence $(x_n)$ of points from $A$ converges in $X$, we have that $\lim_{n\to\infty} x_n \in A$.

A set $\Omega \subset X$ is called \emph{open}\index{open set} if the complement $X \setminus \Omega$ is a closed set.
Equivalently, $\Omega \subset X$ is open if for any $z\in \Omega$ 
there is $\eps>0$ such that the \emph{ball}\index{ball}
$$B_\eps(z)=\set{x\in X}{|x-z|<\eps}$$
is contained in $\Omega$.
\end{thm}


The proof of equivalence of the two definitions of an open set given above is left to the reader.

Note that whole space as well as the empty set are both open and closed at the same time.
Also, the half-closed interval $[0,1)$ is neither open nor closed as a subset of the real line.


\section*{Closure, interior and boundary}

Note that the intersection of an arbitrary number of closed sets is closed.
It follows that for any set $Q$ in a metric space $X$, there is a minimal closed set which contains $Q$;
this set is called the \emph{closure of $Q$}\index{closure} and is denoted as $\Closure Q$.
The closure of $Q$ can be obtained as the intersection of all closed sets $A\supset Q$.
It is also equal to the set of all limit points for all sequences in $Q$.


Similarly, the union of an arbitrary number of open sets is open.
It follows that for any set $Q$ in a metric space, there is a maximal open set which is contained in $Q$;
this set is called the \emph{interior of $Q$}\index{interior} and is denoted as $\Interior Q$.
The interior of $Q$ can be obtained as the union of all open sets in $Q$.
The interior of $Q$ can be also characterized as the set of all point $p\in Q$ such that $B_\eps(p)\subset Q$ for some $\eps>0$.

Further the set-theoretic difference
$$\Closure Q\backslash \Interior Q$$
is called the \emph{boundary}\index{boundary} of $Q$ and is denoted as $\partial Q$ or $\partial_X Q$ if we want to emphasise that $Q$ is a subset of $X$.
Clearly, $\partial Q$ is a closed set.
A point $p$ lies in the boundary of $Q$ if and only if for any $\eps>0$, there are points $q\in Q$ and $q'\notin Q$ such that $|p-q|,|p-q'|<\eps$.

\section*{Completeness}

\begin{thm}{Definition}
Let $X$ be a metric space.
A sequence of points $x_1, x_2, x_3, \ldots$ in $X$ is called \emph{Cauchy}\index{Cauchy sequence}, 
if for every $\eps>0$, 
there is an integer $N$ such that 
$$|x_m - x_n| < \varepsilon,$$
for any $m, n > N$.
\end{thm}

\begin{thm}{Definition}
A metric space $X$ is called \emph{complete}\index{complete space} if any Cauchy sequence in $X$ is convergent.
\end{thm}

\parbf{Examples:}
\begin{itemize}
\item The real line as well as Euclidean space are complete.
\item The subspace of real line formed by the rational numbers is not complete.
\end{itemize}

\begin{thm}{Exercise}\label{ex:close-complete}
Let $X$ be a complete metric space and $A\subset X$,
then the subspace formed by $A$ is complete if and only if $A$ is a closed subset of $X$.
\end{thm}



\parbf{Completion.}
For any metric space $X$, 
there is a canonical construction of a complete metric space $\bar X$, 
which contains $X$.
The space $\bar X$ is called the \emph{completion}\index{completion} of $X$ and it is constructed as 
a set of equivalence classes of Cauchy sequences in $X$. 

For any two Cauchy sequences $\bm{x}=(x_n)$ and $\bm{y}=(y_n)$ in $X$.
Note that from the triangle inequality, 
we get that $\ell_n=|x_n-y_n|$ is a Cauchy sequence.
Set
$$|\bm{x}-\bm{y}| = \lim_{n\to\infty} |x_n-y_n|.$$
This limit exists because 
the space of real numbers is complete. 

The defined function $|\bm{x}-\bm{y}|$ satisfies all axioms of metric but \ref{def:metric-space:zero};
i.e., two different Cauchy sequences may have the distance $0$ 
(the functions of that type are called \emph{pseudometrics}\index{pseudometric}). 
But ``having distance 0'' is an equivalence relation on the set of all Cauchy sequences, and the set of equivalence classes, which we denote $\bar X$, is a metric space called the \emph{completion} of $X$.  

The original space is embedded in this space via the identification of an element $x$ of $X$ with the equivalence class of the sequence with constant value $x$.  
This defines a distance preserving map $X\to \bar X$, as required. 

It is staightforward to verify all claims made about $\bar X$ and prove that the distance function given is well-defined and is indeed a metric on $\bar X$.







\section{Topology}

\begin{thm}{Definition}
Let $X$ and $Y$ be metric spaces.
A map $f\:X\to Y$ is called a \emph{homeomorphism}\index{homeomorphism}
if it is a continuous bijection and the inverse $f^{-1}\:Y\to X$ is also continuous.

Two metric spaces $X$ and $Y$ are called \emph{homeomorphic}\index{homeomorphic spaces} (briefly $X\hom Y$) if there is homeomorphism $f\:X\to Y$.
\end{thm}

It is straightforward to check that $\hom$ is an equivalence relation.

Every isometry is a homeomorphism,
but not every homeomorphism is an isometry.
The homeomorphism class of metric space is much larger than its isometry class.

\begin{thm}{Exercise}\label{ex:cont+biject}
Give an example of a continuous bijection between metric spaces that is not a homeomorphism.
\end{thm}

Let $\rho$ and $\rho'$ be two metrics on the same set $X$.
We say that $\rho$ is \emph{equivalent} to $\rho'$ if the identity map 
$(X,\rho)\to (X,\rho')$ is a homeomorphism.
In other words, $\rho$ is equivalent to $\rho'$ if and only if
for any sequence of points $(x_n)$ and a point $x_\infty$ in $X$, we have
$$\rho(x_n,x_\infty)\to 0\ \ \Leftrightarrow\ \ \rho'(x_n,x_\infty)\to 0.$$

The class of equivalent metrics on a fixed set $X$
is described completely by all open subsets of $X$;
i.e., a metric $\rho$ is equivalent to a metric $\rho'$
if and only if any set which is open with respect to the metric $\rho$ is open with respect to the
metric $\rho'$ and visa versa.
\begin{thm}{Exercise}
Prove the last statement. 
\end{thm}

The set of open sets in $X$ is called the \emph{topology}\index{topology} of $X$.
A topology has to satisfy the following axioms:
\begin{enumerate}
\item the union of an arbitrary number of open sets is open,
\item the intersection of a finite number of open sets is open
\item the empty set and $X$ are both open sets
\end{enumerate}
A set with chosen topology is called \emph{topological space}.

Many of concepts described above can be formulated in terms of toplogy.
For example a map $f\: X\to Y$ is continuous if and only if for any open set $U\subset Y$,
the preimage $f^{-1}(U)$ is an open set in $X$.
Compactness also can be described entirely in topological terms;
see Theorem~\ref{thm:compact+covering}.
On the other hand, completeness is not a topological property,
for example the space $[0,1)$ is homeomorphic%
\footnote{say $x\mapsto \tfrac x{1-x}$ is a homeomorphism}
to $[0,\infty)$, but $[0,1)$ is not complete whereas $[0,\infty)$ is.

Given a topological space $X$, 
we may say ``consider a metric $\rho$ on $X$'' 
meaning that $\rho$ is a metric which describe the topology of $X$.
In the case that $X$ is a metric space,
we mean that $\rho$ is equivalent to the original metric on $X$. 

\begin{thm}{Definition}
A map $f\:X\to Y$ is called an \emph{embedding}\index{embedding}
if $f$ gives a homeomorphism of $X$ to its image $f(X)$ in $Y$.
\end{thm}

Note that any distance preserving map is an embedding,
but the converse is not true. 

\parbf{Remarks.}
These definitions are a starting point of so called \emph{point-set topology};
but this is as far as we need to go in this direction.
We will only use the fact that any metric space is naturally a topological space.

A topological space which can appear this way is called \emph{metrizable}%
\index{metrizable topological space}.
Not every topological space is metrizable;
i.e., one can find a set $X$ with a collection of \textit{open sets}
which satisfies above 3 axioms, but there is no metric on $X$ for which these are the open sets; 
we will not consider such monsters in these lectures.



\section{Compactness}

\begin{thm}{Definition}
A metric space $X$ is \emph{compact}\index{compact space} if any sequence of points in $X$ contains a convergent subsequence.

If a subset $A$ in a metric space forms a compact subspace then we say that $A$ is a \emph{compact subset}\index{compact subset} of $X$.
\end{thm}


The proofs of the following properties are left to the reader;
these proofs can be also found in many books,
for example in \cite{rudin}
(which is an excellent book).

\parbf{Properties:} %??? remove those we do not use???
\begin{itemize}
\item (Heine--Borel theorem.) A subset of Euclidean space is compact if and only if it is both closed and bounded.
\item Any closed subset of a compact space is compact.
\item\label{compact=>closed} Any compact subset of a metric space is closed.
\item Any compact metric space is complete.
\item The Cartesian product of two compact spaces $X\times Y$ equipped with the metric 
$$|(x_0,y_0)-(x_1,y_1)|\df \max\{|x_0-x_1|,|y_0-y_1|\}$$
is compact.
\item \label{ex:compact-image}
If $f\: X \to Y$ is continuous map between metric spaces and $X$ is compact, then the image $f(X)$ is a compact subset of $Y$.
\item\label{ex:compact-homeo}
If $f\: X \to Y$ is a continuous bijection between metric spaces and $X$ is compact, then the inverse map $f^{-1}\: Y \to X$ is continuous.
\item\label{ex:EVT}
(Extreme Value Theorem) Prove that if $X$ is a compact metric space, then any continuous function $f\: X \to \mathbb{R}$ attains a global maximum value at some point of $X$.  That is, there exists $x \in X$ such that $f(y) \leq f(x)$ for all $y \in X$.
(Here, it is understood that the metric we are considering on $\mathbb{R}$ is the usual Euclidean metric.)
\end{itemize}

\begin{thm}{Definition}
A subset $A$ of a metric space $X$ is called \emph{bounded}\index{bounded set} if for one (and therefore any) point $x$ there is a constant $D<\infty$ such that 
$$|x-a|_X\le D$$ for any $a\in A$.
\end{thm}


\begin{thm}{Definition}\label{def:proper}
A metric space $X$ is called \emph{proper}\index{proper space} if any bounded, 
closed set in $X$ is compact.
\end{thm}


\subsection*{Nets and packings}

\begin{thm}{Definition}
Let $X$ be a metric space and let $\eps>0$.
A set $A\subset X$ is called an \emph{$\eps$-net of $X$}\index{net} if for any point $x\in X$ there is a point $a\in A$
such that $|x-a|\le\eps$.
\end{thm}

\begin{thm}{Definition}
Let $X$ be a metric space and let $\eps>0$.
A set of points $x_1,x_2,\dots,x_n\in X$ such that $|x_i-x_j|>\eps$ for all $i\not=j$ is called an \emph{$\eps$-packing}\index{packing}.

The supremum of all integers $n$ for which 
there is a $\eps$-packing $x_1,x_2,\dots,x_n\z\in X$ 
is denoted as $\pack_\eps X$.
($\pack_\eps$ takes integer value or $\infty$.)

If $n=\pack_\eps X$ is finite,
then an $\eps$-packing $x_1,x_2,\dots,x_n\in X$ 
is called a \emph{maximal $\eps$-packing}\index{maximal $\eps$-packing}.
\end{thm}

\begin{thm}{Exercise}\label{ex:packing=net}
Any maximal $\eps$-packing is an $\eps$-net.
\end{thm}

\begin{thm}{Exercise}\label{ex:net=2packing}
Let $\{a_1,a_2,\dots,a_n\}$ be a finite $\eps$-net in a metric space $X$.
Show that $\pack_{2\cdot\eps}X\le n$.
\end{thm}

\begin{thm}{Theorem}\label{thm:finite_pack=compact}
Let $X$ be a complete metric space.
Then the following conditions are equivalent:

\begin{subthm}{thm:finite_pack=compact:compact}
$X$ is compact;
\end{subthm}

\begin{subthm}{thm:finite_pack=compact:pack}
$\pack_\eps X$ is finite for any $\eps>0$;
\end{subthm}

\begin{subthm}{thm:finite_pack=compact:net}
$X$ is \emph{totally bounded}\index{totally bounded space};
i.e., for any $\eps>0$ there is a finite $\eps$-net in $X$.
\end{subthm}

\end{thm}

\parit{Proof; (\ref{SHORT.thm:finite_pack=compact:compact})$\Rightarrow$(\ref{SHORT.thm:finite_pack=compact:net}).}
Assume $X$ has no finite $\eps$-net.
Then for any point array $z_1,z_2,\dots,z_{n-1}$ in $X$
there is a $z_n\in X$
such that $|z_i-z_n|>\eps$ for any $i<n$.

Applying the above statement inductively, we can construct an infinite sequence $(z_n)$ in $X$
such that $|z_i-z_j|>\eps$ for all $i\not=j$.
Therefore $(z_n)$ has no convergent subsequence,
a contradiction.

\parit{(\ref{SHORT.thm:finite_pack=compact:net})$\Leftrightarrow$(\ref{SHORT.thm:finite_pack=compact:pack}).}
Follows from exercises \ref{ex:net=2packing} and \ref{ex:packing=net}.

\parit{(\ref{SHORT.thm:finite_pack=compact:net})$\Rightarrow$(\ref{SHORT.thm:finite_pack=compact:compact}).}
Set $\eps_k=\frac1{2^k}$.
For each $k$, 
let $\{z_{1,k},z_{2,k},\dots,z_{n_k,k}\}$ be an $\eps_k$-net.
Note that for each $k$, the collection of balls
$$\bar B_{\eps_k}(z_{i,k})=\set{x\in X}{|x-z_i|\le \eps_k}$$
cover all of $X$.

To show $X$ is compact, given an infinite sequence $x_1,x_2,\dots$ in $X$, we must find a convergent subsequence.
We shall apply a diagonal process to choose a subsequence
$x_{n_1},x_{n_2},\dots$  with the following property:
\textit{for each $k$ there is $i_k$ such that
$x_{n_m}\z\in\bar  B_{\eps_k}(z_{i_k,k})$ for all $m\ge k$}.

For the first step, note that since the finite collection of balls $\bar B_{\eps_1}(z_{i,1})$ cover $X$, there must be an index $i_1$ such that $\bar B_{\eps_1}(z_{i_1,1})$ contains infinitely many terms of our sequence $(x_n)$.  Let $F_1 = \bar B_{\eps_1}(z_{i_1,1})$. Choose an integer $n_1$ such that $x_{n_1} \in F_1$.

For the second step, since the balls $\bar B_{\eps_2}(z_{i,2})$ cover $X$, in particular they cover $F_1$.  Since $F_1$ contains infinitely many points of our sequence $(x_n)$, there must be an index $i_2$ such that $F_1 \cap \bar B_{\eps_2}(z_{i_2,2})$ also contains infinitely many of the $x_n$.  Let $F_2 = F_1 \cap \bar B_{\eps_2}(z_{i_2,2})$ and choose $n_2 > n_1$ such that $x_{n_2} \in F_2$.

Proceeding inductively at the $k$-th step, we know that the balls $\bar B_{\eps_k}(z_{i,k})$ cover $X$ and hence $F_{k-1}$, which contains infinitely many terms of our sequence $(x_n)$.  So there must exist $i_k$ so that $F_{k-1} \cap \bar B_{\eps_k}(z_{i_k,k})$ also contains infinitely many of the $x_n$.  Let $F_k = F_{k-1} \cap \bar B_{\eps_k}(z_{i_k,k})$ and choose $n_k > n_{k-1}$ with $x_{n_k} \in F_k$.

Note that $F_{k+1} \subseteq F_k$, so that for all $m \geq k$, $x_{n_m} \in F_k \subseteq B_{\eps_k}(z_{i_k,k})$ as desired.  It follows that $(x_{n_k})$ is a Cauchy sequence.
Since $X$ is complete, 
$(x_{n_k})$ converges, and we have proven that $X$ is compact.
\qeds

\section*{Lebesgue's number}

The results in this section will not be used directly further in the lectures.
We present it only to give a connection to standard definition of comapct space.
In addition, a similar argument will be used once in the proof of Hopf--Rinow theorem (\ref{thm:Hopf-Rinow}).

\medskip

Let $X$ be a metric space, a collection of open subsets $\{U_\alpha\}_{\alpha\in\mathcal A}$ is called \emph{open cover}\index{open cover} of $X$ if 
$$X=\bigcup_{\alpha\in A}U_\alpha.$$


\begin{thm}{Lebesgue's number lemma}\label{lem:lebesgue-number}
Let $\{U_\alpha\}_{\alpha\in\mathcal A}$ be an open cover of a compact metric space $X$.
Then there is an $\eps>0$ (it is called a \emph{Lebesgue number}\index{Lebesgue number} of the cover)
such that for any $x\in X$ the ball $B_\eps(x)\subset U_\alpha$ for some $\alpha\in\mathcal{A}$.
\end{thm}

\parit{Proof.}
Given $x\in X$, denote by $\rho(x)$ the maximal value $R>0$ such that $B_R(x)\subset U_\alpha$ for some $\alpha\in\mathcal{A}$.
Clearly $\rho(x)>0$ for any $x\in X$.

Without loss of generality, we may assume $\rho(x)<\infty$ for one (and therefore any) $x\in X$.
Otherwise the conclusion of the lemma holds for arbitrary $\eps>0$.

Note  $$|\rho(x)-\rho(y)|\le |x-y|$$ for any $x,y\in X$;
in particular $\rho$ is continuous.
Then the conclusion of the lemma holds for 
$$\eps=\tfrac12\cdot\min_{x\in X}\{\rho(x)\}.$$
\qedsf


As a corollary of Lebesgue's number lemma, 
we obtain an alternative definition of compact metric space using open coverings.
This definition is the standard definition of compact spaces.
Since it use only the notion of open sets, it can be generalized to so called topological spaces.


\begin{thm}{Theorem}\label{thm:compact+covering}
A metric space $X$ is compact if and only if any open cover of $X$ contains a finite subcover.

I.e., for any open cover $\{U_\alpha\}_{\alpha\in\mathcal A}$ of $X$ there is a finite set 
$\{\alpha_1,\alpha_2,\dots,\alpha_n\}\subset \mathcal A$ such that 
$$X=\bigcup_{i=1}^nU_\alpha.$$
 
\end{thm}

\parit{Proof; ``if''-part.}
First let us show that $X$ is complete.
Assume contrary;
i.e., there is a Cauchy sequence $(x_n)$ which is not converging.
Set $r_n=\sup_{m\ge n}\{|x_n-x_m|_X\}$ and
$U_n=X\backslash \bar B_{r_n}(x_n)$.
Since $x_n$ does not converge, we have $$\bigcap_{n=1}\bar B_{r_n}(x_n)=\emptyset,$$
or equivalently $\{U_n\}_{n=1}^\infty$ is a cover of $X$.
On the other hand it is easy to see that any finite sub-collection of $\{U_n\}_{n=1}^\infty$ does not contain $x_n$ for all large $n$, a contradiction. 


Fix $\eps>0$ and consider cover of $X$ by open balls $\{B_\eps(x)\}_{x\in X}$.
Note that if $\{B_\eps(x_i)\}_{i=1}^n$ is a finite subcover then $\{x_1,x_2,\dots,x_n\}$ forms an $\eps$-net in $X$. Apply Theorem~\ref{thm:finite_pack=compact}.

\parit{``only if''-part.}
Let $\eps>0$ be a Lebesgue's number of the covering.
Choose a finite $\tfrac\eps2$-net $\{x_1,x_2,\dots,x_n\}$ of $X$.
Clearly 
$$\bigcup_{i=1}^n B_\eps(x_i)=X.\eqlbl{eq:balls-cover-X}$$
For each $x_i$ choose $U_{\alpha_i}$ such that $U_{\alpha_i}\supset B_\eps(x_n)$.
From \ref{eq:balls-cover-X}, 
$$\bigcup_{i=1}^n U_{\alpha_i}=X.$$
\qedsf


\section{Length structure}

Recall that \emph{real interval} is an arbitrary convex set of real line;
i.e. a set $\II$ such that $a,b\in \II$ and $a<x<b$ implies $x\in\II$.
For example, given two real numbers $a<b$ we may consider the following intervals
\begin{align*}
[a,b]
&\df
\set{x\in \RR}{a\le x\le b},
&
[a,b)
&\df
\set{x\in \RR}{a\le x< b},
\\
(a,b]
&\df
\set{x\in \RR}{a< x\le b},
&
(a,b)
&\df
\set{x\in \RR}{a< x< b}.
\end{align*}
In addition to the bounded intervals described above, there are unbounded intervals
\begin{align*}
[a,\infty)
&\df
\set{x\in \RR}{a\le x},
&
(a,\infty)
&\df
\set{x\in \RR}{a< x},
\\
(-\infty,a]
&\df
\set{x\in \RR}{a\ge x},
&
(-\infty,a)
&\df
\set{x\in \RR}{a> x}.
\end{align*}
Finally the whole real line is also an interval
$$(-\infty,\infty)=\RR.$$

\begin{thm}{Definition}\label{def:curve}
A \emph{curve}\index{curve} is a continuous mapping $\alpha\:\II\to X$,
where $\II$ is a real interval and $X$ is a metric space. 

If $\II=[a,b]$ and $$\alpha(a)=p,\ \ \alpha(b)=q,$$
we say that $\alpha$ is a \emph{curve from $p$ to $q$}\index{curve from $p$ to $q$}.
\end{thm}

\begin{thm}{Definition}\label{def:length}
Let $\alpha\:\II\to X$ be a curve. Define \emph{length}\index{length of curve} of $\alpha$ as
\begin{align*}
\length \alpha
&= 
\sup \{|\alpha(t_0)-\alpha(t_1)|+|\alpha(t_1)-\alpha(t_2)|+\dots
\\
&\ \ \ \ \ \ \ \ \ \ \ \ \ \ \ \ \ \ \ \ \ \ \ \ \ \ \ \ \ \ \ \ \dots+|\alpha(t_{k-1})-\alpha(t_k)|\}. 
\end{align*}
where the supremum is taken over all $k$ and all sequences $t_0 < t_1 < \cdots < t_k$ in $\II$.

A curve is called \emph{rectifiable}\index{rectifiable curve} if its length is finite.
\end{thm}

\begin{thm}{Exercise}\label{ex:nonrectifiable-curve}
Construct a curve $\alpha\:[0,1]\to\RR^2$ which is not rectifiable.
\end{thm}


\begin{thm}{Semicontinuity of length}\label{thm:length-semicont}
Length is a lower semi-continuous functional on the space of curves
$\alpha\:\II\to X$ with respect to point-wise convergence. 

In other words: assume that a sequence
of curves $\alpha_n\:\II\to X$ converges point-wise 
to a curve $\alpha_\infty\:\II\to X$;
i.e., for any fixed $t\in\II$, we have $\alpha_n(t)\to\alpha_\infty(t)$ as $n\to\infty$. 
Then 
$$\liminf_{n\to\infty} \length\alpha_n \ge \length\alpha_\infty.\eqlbl{eq:semicont-length}$$

\end{thm}



\begin{wrapfigure}{r}{33mm}
\begin{lpic}[t(-5mm),b(0mm),r(0mm),l(0mm)]{pics/stairs(0.3)}
%\lbl[lb]{162,180;$\alpha_0$}
%\lbl[lb]{82,150;{\small $\alpha_1$}}
%\lbl[lb]{42,130;{\tiny $\alpha_2$}}
\end{lpic}
\end{wrapfigure}

Note that the inequality \ref{eq:semicont-length} might be strict.
For example the diagonal of unit square say $\alpha_\infty$ (red on the picture)
can be  approximated by a sequence of stairs-like
polygonal curves $\alpha_n$
with sides parallel to the sides of the squre,
$\alpha_6$ is black the picture.
In this case
$\length\alpha_\infty=\sqrt{2}$
and $\length\alpha_n=2$ for all $n$.

\parit{Proof.}
Fix $\eps > 0$ and choose a sequence $t_0<t_1<\dots<t_k$ in $\II$
such that 
\begin{align*}
\length\alpha_\infty-
(|\alpha_\infty(t_0)-\alpha_\infty(t_1)|&+|\alpha_\infty(t_1)-\alpha_\infty(t_2)|+\dots
\\
&\dots+|\alpha_\infty(t_{k-1})-\alpha_\infty(t_k)|)<\eps
\end{align*}


Set 
\begin{align*}\Sigma_n
&\df
|\alpha_n(t_0)-\alpha_n(t_1)|+|\alpha_n(t_1)-\alpha_n(t_2)|+\dots
\\
&\ \ \ \ \ \ \ \ \ \ \ \ \ \ \ \ \ \ \ \ \ \ \ \ \ \ \ \ \ \ \ \ \dots+|\alpha_n(t_{k-1})-\alpha_n(t_k)|.
\\
\Sigma_\infty
&\df
|\alpha_\infty(t_0)-\alpha_\infty(t_1)|+|\alpha_\infty(t_1)-\alpha_\infty(t_2)|+\dots
\\
&\ \ \ \ \ \ \ \ \ \ \ \ \ \ \ \ \ \ \ \ \ \ \ \ \ \ \ \ \ \ \ \ \dots+|\alpha_\infty(t_{k-1})-\alpha_\infty(t_k)|.
\end{align*}
Note that $\Sigma_n\to \Sigma_\infty$ as $n\to\infty$
and $\Sigma_n\le\length\alpha_n$ for each $n$.
Hence
$$\liminf_{n\to\infty} \length\alpha_n \ge \length\alpha_\infty-\eps.$$
Since $\eps>0$ is arbitrary, we get \ref{eq:semicont-length}.\qeds

\section*{Length spaces}%???Change to length spaces; do we need almost-mid-point property???

\begin{thm}{Definition}
A metric space $X$ is called \emph{length space}\index{length space} if for any two points $x,y\in X$ and any $\eps>0$, there is a curve $\gamma$ from $x$ to $y$ such that
$$\length \gamma<|x-y|_X+\eps.$$

\end{thm}

\parbf{Examples.} The real line as well as higher dimensional Euclidean spaces are length spaces.
A discrete space (with at least two points) is not a length space.
Also, a circle in the plane forms a subspace of a length space which is not a length space.


\parbf{Induced length metric.}
Given a complete metric space $(X,\rho)$, 
consider the function $\hat \rho\:X\z\times X\to\RR$ defined as
$$\hat \rho(x,y)\df\inf_\alpha\{\length\alpha\}$$
where the infimum is taken along all the curves $\alpha$ from $x$ to $y$.

It is straightforward to see that $\hat \rho\:X\times X\to\RR$ satisfies all conditions of the metric if $\hat \rho(x,y)<\infty$ for all $x,y\in X$.
In this case, the metric $\hat \rho$ will be called the \emph{induced length metric}\index{induced length metric} of $\rho$.

\begin{thm}{Exercise}\label{ex:length-is-not-homeo}
Construct a metric space $(X,\rho)$ such that the length  metric $\hat \rho$ is finite
but $(X,\rho)$ is not homeomorphic to $(X,\hat \rho)$.
\end{thm}


 
\section*{Geodesics}

\begin{thm}{Definition}\label{def:geodesic}
 A curve $\alpha\:\II\to X$ is called a \emph{geodesic}\index{geodesic}%
\footnote{formally our ``geodesic'' should be called ``unit-speed minimizing geodesic'',
and the term ``geodesic'' is reserved for curves which \emph{locally} satisfy the identity
$$|\alpha(t_0)-\alpha(t_1)|_X=\Const\cdot|t_0-t_1|$$
for some $\Const\ge 0$.}
 if it is a distance preserving map;
i.e., if 
$$|\alpha(t_0)-\alpha(t_1)|_X=|t_0-t_1|$$
for any $t_0,t_1\in\II$.

The metric space $X$ is called \emph{geodesic}\index{geodesic space}
if any two points in $X$ can be joined by a geodesic. 
\end{thm}





\begin{thm}{Exercise}\label{ex:proper=>geodesic}
Show that any proper length space is geodesic.
\end{thm}

\begin{thm}{Definition}
A metric space $X$ is called \emph{locally compact}\index{locally compact space} if for any $x\in X$ there is an $\eps>0$ such that the closed ball
$$\bar B_\eps(x)=\set{y\in X}{|x-y|_X\le \eps}$$
is compact.
\end{thm}

Note that any proper metric space is locally compact (see Definition~\ref{def:proper}).
The converse does not hold in general.
For example, any infinite set equipped with the discrete metric is locally compact, but not proper.

\begin{thm}{Exercise} \label{ex:lc-not-complete}
Give an example of metric space which is locally compact, but not complete. 
\end{thm}


\begin{thm}{Hopf--Rinow theorem}\label{thm:Hopf-Rinow}
Any complete, locally compact length space is proper.
\end{thm}

\parit{Proof.}
Let $X$ be a locally compact length space.
Given $x\in X$, denote by $\rho(x)$ the supremum of all $R>0$ such that
the closed ball $\bar B_R(x)$ is compact.
Since $X$ is locally compact 
$$\rho(x)>0\ \ \text{for any}\ \ x\in X.\eqlbl{eq:rho>0}$$
It is sufficient to show that $\rho(x)=\infty$ for some (and therefore any) point $x\in X$.

Assume contrary; i.e. $\rho(x)<\infty$.
Let us show that the closed ball $W=\bar B_{\rho(x)}(x)$ is compact.
To prove this claim, notice that the closed ball $\bar B_{\rho(x)}(x)$
is a closed set,
therefore it  forms a complete subspace of $X$ (see Exercise~\ref{ex:close-complete}).
Further, since $X$ is a length space, for any $\eps>0$, the set $\bar B_{\rho(x)-\eps}(x)$ is a compact $\eps$-net in $\bar B_{\rho(x)}(x)$;
it remains to apply Problem~\ref{pr:compact-net}.

Next we will reapeat the argument in the Lebesgue number lemma (\ref{lem:lebesgue-number}).

Note that if $\rho(x)+|x-y|<\rho(y)$, then 
$\bar B_{\rho(x)+\eps}(x)$ is a closed subset of $\bar B_{\rho(y)}(y)$ for some $\eps>0$.
In this case compactness of $\bar B_{\rho(y)}(y)$ implies compactness of $\bar B_{\rho(x)+\eps}(x)$, a contradiction.
Applying the same observation again switching $x$ and $y$, we get 
$$|\rho(x)-\rho(y)|\le |x-y|_X;$$
in particular $\rho$ is a continuous function.
Set $\eps=\min_{y\in W}\{\rho(y)\}$; 
the minimum is defined since $W$ is compact.
From \ref{eq:rho>0}, we have $\eps>0$.

Choose a finite $\tfrac\eps{10}$-net $\{a_1,a_2,\dots,a_n\}$ in $W$.
The union $U$ of the closed balls $\bar B_\eps(a_i)$ is compact.
Clearly $\bar B_{\rho(x)+\frac\eps{10}}(x)\subset U$.
Therefore $\bar B_{\rho(x)+\frac\eps{10}}(x)$ is compact
or $\rho(x)>\rho(x)$, 
a contradiction.\qeds








\section{Polyhedral spaces}

\parbf{Simplex.}
Let $\{v_0,v_1,\dots,v_m\}$ be a set of points in $\RR^N$ 
such that the $m$ vectors 
$$v_1-v_0, v_2-v_0,\dots,v_m-v_0$$ 
are linearly independent (in particular $N\ge m$).
Then the convex hull 
$$\Delta^m=\Conv(v_0,v_1,\dots,v_m)$$
is called an \emph{$m$-dimensional simplex}\index{simplex}.

So,
a 0-dimensional simplex is a one-point set; 
a 1-dimensional simplex is a line segment;
a 2-dimensional simplex is a triangle;
a 3-dimensional simplex is a tetrahedron.

If $\Delta^m$ as above
then the convex hull of any $(k+1)$-point subset of $\{v_0,v_1,\dots,v_m\}$ also forms a $k$-dimensional simplex which will be called \emph{face}\index{face} of $\Delta^m$.


\parbf{Barycentric coordinates.}
Let $\Delta^m=\Conv(v_0,v_1,\dots,v_m)$ be an $m$-dimensional simplex.
Note that any point $x\in \Delta^m$ can be uniquely presented as
$$x=\lambda_0\cdot v_0+\lambda_1\cdot v_1+\dots+\lambda_n\cdot v_m$$
where $\lambda_0,\lambda_1,\dots,\lambda_m\ge 0$ and
$\lambda_0+\lambda_1+\dots+\lambda_n=1$.
The array $(\lambda_0,\lambda_1,\dots,\lambda_m)$ will be called the \emph{barycentric coordinates}\index{barycentric coordinates} of the point $x$.

\begin{thm}{Exercise}\label{ex:convex-hull}
Verify the claim that $$\Conv(v_0,v_1,\dots,v_m) = \set{\sum_{i=0}^m \lambda_i \cdot v_i}{\lambda_i \ge 0 \textrm{ and } \sum_{i=0}^m \lambda_i = 1}.$$

\end{thm}




\section*{Simplicial complexes}

A \emph{geometric simplicial complex}\index{geometric simplicial complex} is defined as a finite collection $\mathcal{K}$
of simplices in $\RR^n$ that satisfies the following conditions:
\begin{itemize}
\item Any face of a simplex from $\mathcal{K}$ is also in $\mathcal{K}$.
\item The intersection of any two simplices $\Delta_1$ and $\Delta_2\in \mathcal{K}$ is either an empty set or it 
is a common face in $\Delta_1$ and $\Delta_2$.
\end{itemize}

Note that to describe a geometric simplicial complex $\mathcal{K}$ in $\RR^n$ 
it is sufficient to give a finite set of vertices $V=\{v_1,\dots, v_k\}$ of all the simplices in $\mathcal{K}$
and the set of subsets $\mathcal{S}$ of $V$ 
such that $X\in \mathcal{S}$ if and only if there is a simplex of $\mathcal{K}$ with vertices in $X$.
The information encoded in $\mathcal{S}$ is called \emph{abstract simplicial complex}.

More formally, an \emph{abstract simplicial complex} 
is a family $\mathcal{S}$ of subsets in a finite set $V$
such that for every set $X$ in $\mathcal{S}$, 
and every subset $Y\subset  X$, 
$Y$ also belongs to $\mathcal{S}$
(in particular $\emptyset\in\mathcal{S}$).

As it was noted above, 
a geometric simplicial complex $\mathcal{K}$
defines an abstract simplicial complex on the set of the vertices of $\mathcal{K}$.
On the other hand, 
given an abstract simplicial complex $\mathcal{S}$ on finite set $V=\{v_1,\dots, v_k\}$ 
one can construct geometric simplicial complex
in $\RR^k$, by taking identifying $V$ with a  basis of $\RR^k$.

\parbf{Underlying set.}
We say that a point $x$ belongs to geometric simplicial complex $\mathcal{K}$ if it belongs to one of its simplices.
The set of all points of  $\mathcal{K}$ is called \emph{underlying set} of $\mathcal{K}$;
it will be denoted as $|\mathcal{K}|$.

If $\mathcal{K}$ is a simplicial complex
and $\{v_1,v_2,\dots,v_k\}$ is the set of all its vertices
then any point $x$ in $\mathcal{K}$ can be described 
through barycentric coordinates $(\lambda_1,\lambda_2,\dots,\lambda_n)$, so $\lambda_1+\lambda_2+\dots+\lambda_n=1$,
$\lambda_i\ge 0$ for any $i$ and 
for any $x$ the set $X=\set{i}{\lambda_i>0}$
belong to the corresponding abstract complex. 

\parbf{Dimension.}
The dimension  of geometric simplicial complex is defined as the maximal dimension of all of its simplices.

\section*{Polytopes}

A subset $P$ of $\RR^n$ is called \emph{polytope} if it can be presented as 
underlying set of some geometric simplicial complex.
Given a polytope $P$,
any simplicial complex $\mathcal{K}$ with the underlying set $P$ is called \emph{triangulation}%
\footnote{The term is a bit misleading; 
the triangulation may contain simplexes of arbitrary dimension}
of $P$.
Note that if polytope $P$ contains infinite number of points then it has many distinct triangulations.

\begin{thm}{Exercise}
Let $P$ be a polytope.
Show that dimension of any triangulation of $P$ has the same dimension.
\end{thm}

The exercise above makes possible to define
the dimension of polytope is defined 
as the dimension of its triangulation.

With a slight abuse of notation we may say that a simplicial complex $\mathcal{K}$ is homeomorphic to simplicial complex $\mathcal{K}'$
if their underlying spaces are homeomorphic.
A metric (or topological) space $Q$ is called \emph{topological polytope} if it admits a homeomorphsm from a polytope say $P$.
Such a homeomorphism together with triangulation of $P$
is called \emph{topological triangulation} of $Q$.  

\begin{thm}{Exercise}
Find a topological triangulation of sphere%
\footnote{The unit $n$-sphere is $\SS^n = \set{(x_1, x_2, \ldots, x_{n+1}) \in \mathbb{R}^{n+1}}{x_1^2 + x_2^2 + \ldots + x_{n+1}^2 = 1} $.} $\SS^2$ with minimal number of triangles.
\end{thm}

You will need at least 14 triangles 
to do topological triangulation the torus $\TT^2=\SS^1\times\SS^1$.
Finding such triangulation might be interesting, but proving its minimality is not fun.

The dimension of topological polytope is
 also defined as the maximal dimension of the 
simplices in its triangulation.
This value also the same for any triangulation,
but the proof requires Domain Invariance Theorem (\ref{thm:domain-invariance}).

\section*{Polyhedral spaces}

Informally speaking, a \emph{polyhedral space} is a length space which admit a triangulation such that each simplex is isometric to a simplex in Euclidean space.

More formally.
Let $\Delta=\Conv(v_0,v_1,\dots,v_k)$ be a $k$-dimensional simplex. 
A map $f\:\Delta\to\RR^m$ is called \emph{linear} if
$$f(x)
=
\lambda_0\cdot f(v_0)+\lambda_1\cdot f(v_1)+\dots+\lambda_k\cdot f(v_k)$$
for any point $x\in \Delta$ with barycentric coordinates $(\lambda_0,\lambda_1,\dots,\lambda_k)$.

A metric $\rho$ on simplex $\Delta$ is called \emph{linear} 
if there is a liner map $f\:\Delta\to\RR^m$ such that
$$\rho(x,y)=|f(x)-f(y)|_{\RR^m}$$

A metric on simplicial complex is called linear if its restriction to each simplex is linear.

A length space $P$ is called \emph{polyhedral space} it is isometric to a simplicial complex $\mathcal{S}$ with linear metric $d$.
In this case an isometry is $\iota\:(\mathcal{S},d)\to P$ is called \emph{triangulation of polyhedral sapce}.

\begin{thm}{Exercise}
 Show that any convex nondegenerate polyhedron%
\footnote{i.e., a convex hull of finite number of points which has nonempty interior.}
 in $\RR^m$ is an $m$-dimensional polyhedral space.
\end{thm}

\begin{thm}{Exercise}\label{ex:bry-is-poly}
Show that boundary any convex nondegenerate polyhedron%
\footnotemark[\value{footnote}] in $\RR^m$ equipped with its length metric is an $(m-1)$-dimensional polyhedral space.
\end{thm}


\begin{thm}{Exercise}\label{ex:dim-poly}
All triangulations of a given polyhedral space
have the same dimension.
\end{thm}

As it follows from the above exercise,
the dimension of polyhedral space is defined as dimension of its triangulation.







\section*{Exercises}

\begin{pr}\label{pr:nonisometry}
Give an example of a metric space $X$ with a distance preserving map $f\:X\z\to X$ which is not a bijection. 
\end{pr}

\begin{pr}\label{pr:non-contracting=>isometry}
Let $K$  be a compact metric space and
$f\:K\to K$ be a non-contracting map;
i.e., 
$$|f(x)-f(y)|_K\ge |x-y|_K$$
for any $x,y\in K$.
Prove that $f$ is an isometry.
\end{pr}



\begin{pr}\label{ex:kuratowski}
Let $X$ be a metric space.
Given $z\in X$, denote by $\Dist_z$ the function $X\to\RR$ defined as $\Dist_z(x)=|x-z|_X$.

Fix $x\in X$.
Show that the map 
$$K_x\: X\to\mathcal{F}(X)$$ 
defined by 
$$K_x(z) = \Dist_z-\Dist_x$$
is distance preserving;
i.e. 
$$|K_x(y)-K_x(z)|_{\mathcal{F}(X)}=|y-z|_X$$
for any $y,z\in X$.
\end{pr}

{\it The map $K_x\: X\to\mathcal{F}(X)$ is called the \emph{Kuratowski embedding with base $x$}\index{Kuratowski embedding}.

If $X$ has bounded diameter, i.e., if there is a constant $D<\infty$,
such that $|z-y|\le D$ for any $z,y\in X$,
then $\Dist_x$ is bounded for any $x\in X$
and the map 
$$K\:X\to\mathcal{F}(X)$$ given by 
$$K(x) = \Dist_x $$ 
is also distance preserving.
The latter map is also called the \emph{Kuratowski embedding}\index{Kuratowski embedding}.}

\begin{pr}\label{pr:compact->F_N}
Show that any compact space is isometric to a subset of $\mathcal{F}(\NN)$;
i.e., the space of bounded sequences with the metric induced by sup-norm.
\end{pr}

\begin{pr}\label{pr:complition-kuratowski}
Show that completion of any metric space is isometric to the closure of its image under Kuratowski embedding.
\end{pr}

\begin{pr}\label{pr:almost-min}
Let $X$ be a complete metric space and $\rho\:X\to \RR$ be a continuous positive function.

Show that there is $x\in X$ such that
$$\rho(y)>\tfrac{99}{100}\cdot\rho(x)$$ 
for any $y\in B_{\rho(x)}(x)$.
\end{pr}

\begin{pr}\label{pr:compact-net}
Show that a complete metric space $X$ is compact if for any $\eps>0$ there is a compact $\eps$-net in $X$.  
\end{pr}






\begin{pr}\label{pr:trig-inq=>interval}
Assume that for any three points $x$, $y$ and $z$ of a compact length metric space $X$ we have 
$$|x-z|_X\ge |x-y|_X, |y-z|_X\ \Longrightarrow\  |x-z|_X= |x-y|_X+ |y-z|_X$$
Show that $X$ is isometric to a closed real interval.
\end{pr}


\begin{pr}\label{pr:1000}
Show that there is a topological triangulation of $\SS^3$
with $1000$ vertices such that each pair of vertices is connected by an edge. 
\end{pr}

\begin{pr}\label{pr:tringulation-of-poly} Let $P$ be a (possibly nonconvex) polygon
 equipped with the induced length metric.
Show that $P$ admits a triangulation
such that the set of vertices of the triangulation is the set of vertices of $P$. 
\end{pr}

\begin{pr}
\label{pr:tringulation-of-poly-3D}
Show that the analogous statement for a polyhedron 3-dimensional space does not hold.
\end{pr}



\begin{pr}\label{pr:2-triangulations}
Let $\mathcal{A}$ and $\mathcal{B}$ be two triangulations of a polyhedral space.
Show that there is a triangulation $\mathcal{C}$ such that each triangle of $\mathcal{A}$ and $\mathcal{B}$ is a union of triangles in $\mathcal{C}$.
\end{pr}


\begin{pr}\label{ex:acute-triangulation}
Show that any triangle admits a triangulation into acute triangles.
\end{pr}

\begin{pr}\label{pr:delaunay.triangulation}
Let $P$ be a convex hull of the finite set of points $\{x_1,x_2,\dots,x_n\}$ in $\RR^2$.
Assume that positive reals $r_1,r_2,\dots,r_n$ are chosen in such a way that the balls 
$B_i=B_{r_i}(x_i)$ cover $P$
and moreover each side of $P$ is covered by the balls centered on this side.

Show that $P$ admits a triangulation with vertices at $x_i$ such that each triangle is covered by the three balls centered at its vertices. 
\end{pr}


\begin{pr}\label{pr:polytope=local.cone}
Let $P$ be a compact subset of Euclidean space.
Show that $P$ is a polytope for every point $x\in P$
there is a \emph{cone%
\footnote{A cone with tip $x$ is a set formed by union of a set of rays starting at $x$.} $K_x$ with tip at $x$} and $\eps>0$
such that 
$$B_\eps(x)\cap P
=
B_\eps(x)\cap K_x.$$
 
\end{pr}

\begin{pr}
A 1-dimensional polyhedral space $T$ such that any two points in $T$ are joint by unique geodesic is called \emph{metric tree}.

Let $F$ be a metric space with finite number of points.
Show that $F$ is isometric to a subset of metric tree if and only if
for any four points $x_0,x_1,y_0,y_1$ in $F$
we have 
$$|x_0-x_1|_F+|y_0-y_1|_F
\le
|x_0-y_0|_F+|x_0-y_1|_F+|x_1-x_0|_F+|x_1-y_1|_F.$$
\end{pr}


%???+broken lines
