\chapter{Convex surfaces}\label{chap:unique}

\section{Surface of convex polyhedron}\label{sec:suface}

Let $K$ be a  convex polyhedron in $\RR^3$.

Assume $K$ is \emph{non-degenerate}; i.e., it has nonempty interior.
In this case the boundary of $K$ equipped with the induced length metric will be called the \emph{surface of $K$}\index{surface}\label{def:surface}.
According to Exercise~\ref{ex:bry-is-poly}, the surface of a polyhedron is a polyhedral space.

In the \emph{degenerate} case,
i.e., if $K$ is a flat polygon, the surface of $K$ is defined differently.
You have to imagine that you are living in $\RR^3$
and can walk on a flat polygon made from rigid material;
so to get from one of its sides to the other, you have to travel over a boundary edge.

More formally, we define the surface of a convex polygon $K$ as its \emph{doubling}\index{doubling};
i.e., two copies of $K$ glued along the corresponding points of their boundaries.

We shall be interested in when an abstract polyhedral space $P$ can be realized as the surface of a convex polyhedron in $\RR^3$.  
The proposition below 
says that a necessary condition for this is that $P$ must be homeomorphic to $\SS^2$.

\begin{thm}{Proposition}\label{prop:P-is-S^2}
Let $K$ be a non degenerate convex polyhedron or a convex polygon.
Then the surface of $K$ is homeomorphic to $\SS^2$.
\end{thm}

\parit{Proof.}
Any convex flat polygon,
is homeomorphic to disc.
Therefore its doubling is homeomorphic to a sphere. 

If $K$ is nondegenerate polyhedron, 
we may assume that the origin $0\in\RR^3$ lies in the interior of $K$.
Consider the map $f\:P\to \SS^2$ defined by $f(x) = x/|x|$;
in other words, a point $x\in P$ is mapped to the intersection of the ray $[0x)$ with the unit sphere $\SS^2\subset \RR^3$.

Since $K$ is convex and the origin $0$ is an interior point, 
for any $z\neq 0$,
the ray $[0z)$ intersects $P$ at exactly one point.
It follows that, $f\:P\to\SS^2$ is a bijection. 

Since $0\notin P$, the map $f\:P\to\SS^2$ is continuous.
Given $y\in \SS^2$,
denote by $h(y)$ the length of the intersection of the ray $[0y)$ with $P$. 
Then it is easy to see that $f^{-1}(y)=h(y)\cdot y$;
since $h$ is continuous function on $\SS^2$, so is $f^{-1}$.
(Alternatively, you may apply
 Exercise~\ref{ex:compact-homeo}.)
Therefore $f\:P\to\SS^2$ is a homeomorphism.
\qeds

\section{Surface of convex body}

\begin{thm}{Theorem}
Let $K$ be a nondegenerate convex body
then its surface has nonnegative curvature in the sense of Alexandrov.
\end{thm}



\section{Curvature}


By Proposition~\ref{prop:P-is-S^2},
if $P$ which be realized as the surface of a nondegenerate convex polyhedron or polygon in $\RR^3$ then it is homeomorphic to $\SS^2$;
in particular $P$ is a polyhedral surface.
Now we will formulate another necessary condition on $P$.

\begin{thm}{Exercise}\label{ex:curv-is-nonneg}
Let $P$ be the surface of a convex polyhedron $K$. Show that the curvature of any point $p\in P$ is non-negative. 
\end{thm}

Further in this section, 
we study geometry of polyhedral surfaces with non-negative curvature.
We start with an exercise.

\begin{thm}{Exercise}\label{ex:poly+geod}
Let $P$ be a non-negatively curved polyhedral surface.

\begin{subthm}{}
Show that if two geodesics in $P$ intersect at two points, 
then these are the end points for both geodesics.
\end{subthm}

\begin{subthm}{}
Show that a geodesic in $P$ cannot pass through a vertex of $P$.
\end{subthm}

\end{thm}

Here is the main theorem in the section;
it gives a global geometric property of 
non-negatively curved polyhedral surface.

\begin{thm}{Theorem}\label{thm:poly-cbb}
Let $P$ be a polyhedral surface.
Assume $P$ has non-negative curvature at each point.
Then 
\[\mangle\hinge pxy\ge\angk{}pxy\]
for any hinge $\hinge pxy$ in $P$.
\end{thm}

\begin{thm}{Corollary}
Let $P$ be a polyhedral surface.
Then $P$ has non-negative curvature at each point if and only if 
\[\angk{}pxy+\angk{}pyz+\angk{}pzx
\le 
2\cdot\pi\]
for any hinge quadruple of points $p,x,y,z$ in $P$
such that $p$ is distinct from each of $x$, $y$ and $z$.
\end{thm}



\parit{Proof.}
Let $[pxy]$ be a triangle in $P$ and let $[\tilde p\tilde x\tilde y]$ be the model triangle of $[pxy]$.
Set $\ell=|x-y|_P=|\tilde x-\tilde y|_{\RR^2}$.

Denote by $\gamma(t)$ the geodesic $[xy]$ parametrized by length starting from $x$
and let $\tilde \gamma(t)$ be the geodesic $[\tilde x\tilde y]$ parametrized by length starting from $\tilde x$.
It is sufficient to show that 
$$| p- \gamma(t)|\le|\tilde p-\tilde \gamma(t)| 
\eqlbl{eq:comp-gamma}$$
for any $t$ in $[0,\ell]$.

A point $p$ in $P$ will be calles \emph{regular}%
\index{regular point} if $p$ is not a vertex
 of $P$ or equivalently, the curvature of $P$ vanish at $p$.
Since any vertex can be approximated by regular points, 
we may assume that $p$ is a regular point.


From the cosine law, we get that the function 
$$\tilde f(t)=|\tilde p-\tilde \gamma(t)|^2-t^2$$
is linear.
Consider the function
$$f(t)=|p- \gamma(t)|^2-t^2.$$
Note that 
\begin{align*}
f(0)&=\tilde f(0),\\
f(\ell)&=\tilde f(\ell).
\end{align*}
Note that the inequality~\ref{eq:comp-gamma} is equivalent to
$$f(t)\ge \tilde f(t).
\eqlbl{eq:comp-f}$$


According to Jensen's inequality, 
to prove \ref{eq:comp-f} it is sufficient to show that $f$ is a concave function.
The latter follows once we prove the following:
\begin{clm}{}\label{clm:h>=f}
For any $t_0\in\left]0,\ell\right[$ there is a \emph{supporting linear function}\index{supporting linear function};
i.e., a function $h(t)$ such that 
 $$h(t_0)=f(t_0)\ \  \text{and} \ \ h(t)\ge f(t)$$
for any $t\in(t_0-\eps,t_0+\eps)$ and some fixed $\eps>0$.
\end{clm}



Note that according to Exercise~\ref{ex:poly+geod}, 
$\gamma(t_0)$ is regular.
Since $p$ is regular,
a geodesic $[p\gamma(t)]$ contains only regular points.
Therefore for small enough $\eps>0$,
 the $\eps$-neighborhood of $[p\gamma(t)]$, say $\Omega$ contains only regular points. 
We may assume that $\Omega$ is homeomorphic to a disc;
in this case there is a locally distance preserving embedding $\iota\:\Omega\to\RR^2$.
Note the image $\iota([p\gamma(t)])$ is a line segmentthat 
and $\iota(\Omega)$ is the $\eps$-neighborhood of $\iota([p\gamma(t)])$ in $\RR^2$;
in particular $\iota(\Omega)$ is convex.
Thus $\iota(\Omega)$ contains a triangle with  base $\iota([\gamma(t_0-\eps)\ \gamma(t_0+\eps)])$  and vertex $\iota(p)$.

Clearly, for any $t\in[t_0-\eps,t_0+\eps]$ 
we have 
$$|\iota(p)-\iota(\gamma(t))|\ge|p-\gamma(t)|.$$
Note that
the function
$$h(t)= |\iota(p)-\iota(\gamma(t))|^2-t^2$$
is linear and it 
satisfies the condition \ref{clm:h>=f}.
\qeds

If in the above proof the geodesic from $p$ to $\gamma(t_0)$ is not unique,
then the inequality $h(t)\ge f(t)$ might be strict for $t$ arbitrary close to $t_0$.


\section{Cauchy's theorem}

The first step in the proof of Alexandrov's theorem (\ref{thm:alexandrov}) 
is Alexandrov's uniqueness theorem (\ref{thm:alexandrov-uni'}) 
which in turn generalizes Cauchy's theorem formulated below.
We start with the proof of Cauchy's theorem and then modify it to prove Alexandrov's uniqueness theorem.
 
\begin{thm}{Cauchy's theorem}\label{thm:cauchy} Let $K$ and $K'$ be two non-degenerate convex polyhedra in $\RR^3$;
denote their surfaces%
\footnote{Their boundaries equipped with the induced length metric.} 
by $P$ and $P'$.
If there is an isometry $P\to P'$ which sends each face of $K$ to a face of $K'$,
then $K$ is congruent to $K'$.
\end{thm}


First I break the proof into two parts, ``local'' and ``global'',
which will be proved in the following sections.

\parit{Outline of the proof.} 
Consider the graph $\Gamma$ formed by the edges of $K$ 
(the edges of $K'$ form the same graph).
 
For an edge $e$ in $\Gamma$, 
\begin{itemize}
\item denote by $\alpha_e$ the corresponding dihedral angle of $K$;
\item denote by $\alpha'_e$ the corresponding dihedral angle of $K'$.
\end{itemize}
Mark an edge $e$ of $\Gamma$ with 
$({+})$ if $\alpha_e < \alpha'_e$ and with $({-})$ if $\alpha_e > \alpha'_e$.

Now remove from $\Gamma$ everything which was not marked;
i.e., leave only the edges marked by $(+)$ or $(-)$ and their endpoints.
The statement of Cauchy's theorem is equivalent to the fact that $\Gamma$ is an empty graph.
Let us assume the contrary and try to arrive at a contradiction.

\medskip

Note that $\Gamma$ is embedded into $P$, which is homeomorphic to $\SS^2$ (see Proposition~\ref{prop:P-is-S^2}).
In particular, the edges coming from one vertex have a natural cyclical order. 
Given a vertex $v$ of $\Gamma$, we can count the \emph{number of sign changes} around $v$;
i.e., the number of pairs of adjacent edges which are marked by different signs. 

\begin{thm}{Local lemma}\label{lem:local}
For any vertex of $\Gamma$ the number of sign changes is at least $4$.
\end{thm}

In other words, the local lemma states that at each vertex of $\Gamma$, one can choose 4 edges  marked by $(+)$, $(-)$, $(+)$ and $(-)$ which are in the same cyclical order.

Once the Local lemma is proved, we get a contradiction by applying the following lemma.

\begin{thm}{Global lemma}\label{lem:global}
Let $\Gamma$ be a nonempty sub-graph of the graph formed by the edges of a convex polyhedron.  Then it is impossible to mark all of the edges of $\Gamma$ by $(+)$ or $(-)$ 
such that the number of sign changes around each vertex of $\Gamma$ is at least $4$.
\end{thm}
\qedsf

\begin{wrapfigure}{r}{27mm}
\begin{lpic}[t(-12mm),b(-5mm),r(0mm),l(-3mm)]{pics/penta(0.15)}
\end{lpic}
\end{wrapfigure}

\begin{thm}{Exercise}
Assume that we glue one pentagon and 10 triangles in $\RR^3$ along the rule shown in the picture.
Assume that it forms a part of a surface of a convex polyhedron and each vertex is a vertex of the polyhedron.

Use the local lemma to show that this configuration is rigid;  say one can not fix the position of the pentagon and continuously move the remaining 5 vertices in a new position so that each triangle moves by a one parameter family of isometries of $\RR^3$.
\end{thm}

\section{Arm lemma and Local lemma}

To prove the Local lemma, we will need the following.

\begin{thm}{Arm lemma}\label{lem:arm}
Assume that $A=[a_0 a_1\dots a_n]$ is a convex polygon in $\RR^2$
and $A'=[a'_0 a'_1\dots a'_n]$ be a closed broken line in $\RR^3$
such that 
$$|a_i-a_{i+1}|=|a'_i-a'_{i+1}|$$ for any $i\in\{0,\dots,n-1\}$
and 
$$\angle a_i\le \angle a'_i$$ 
for each $i\in\{1,\dots,n-1\}$.
Then 
$$|a_0-a_n|\le |a'_0-a'_n|$$
and equality holds if and only if $A$ is congruent to $A'$.
\end{thm}

One may view the broken lines $[a_0a_1\dots a_n]$ and $[a'_0a'_1\dots a'_n]$ as a robot's arm in two positions.
The arm lemma states that when the arm opens, 
the distance between the ``shoulder'' and ``tips of the fingers'' increases. 

\begin{thm}{Exercise}
Show that the arm lemma does not hold if 
instead of the convexity,
one only the local convexity;
i.e.,  if you go along the broken line $a_0 a_1\dots a_n$, then you only turn left.
\end{thm}

In the proof, we will use the following exercise. Equivalently, one can think of this as of triangle inequality on the unit sphere $\SS^2$ with the induced length metric.  

\begin{thm}{Exercise}\label{ex:angle-triangle}
Let $w_1,w_2,w_3$ be unit vectors in $\RR^3$.
Denote by $\theta_{i,j}$ the angle between the vectors $v_i$ and $v_j$.
Then 
$$\theta_{1,3}\le \theta_{1,2}+\theta_{2,3}$$
and in case of equality, the the vectors $w_1,w_2,w_3$ lie in a plane.
\end{thm}




\parit{Proof.}
We will view $\RR^2$ as the $xy$-plane in $\RR^3$, 
so that both $A$ and $A'$ lie in $\RR^3$.
Let $a_m$ be the vertex of $A$ which has maximal distance to the line $(a_0a_n)$.

Let us shift indexes of $a_i$ and $a'_i$ down by $m$,
so that 
\begin{align*}
a_{-m}&:=a_0,
&
a'_{-m}&:=a'_0,
\\
a_{-(m-1)}&:=a_1,
&
a'_{-(m-1)}&:=a'_1,
\\
&\vdots&&\vdots
\\
a_{0}&:=a_m,
&
a'_{0}&:=a'_m,
\\
&\vdots&&\vdots
\\
a_k&:=a_n&a'_k&:=a'_n,
\end{align*}
where $k=n-m$.
The symbol ``$:=$'' means \emph{an assignment} as in programming 
(the order of variables in an assignment statement is important: $a:=b$ means that both $a$ and $b$ take the value $b$
and $b:=a$ means that both take the value $a$).

Without loss of generality, we may assume that
\begin{itemize}
\item $a_0=a'_0$ and they both coinside with the origin $(0,0,0)\in\RR^3$;
\item all $a_i$ lie in the $xy$-plane and the $x$-axis is parallel to the line $(a_{-m}a_k)$;
\item the angle $\angle a'_0$ lies in $xy$-plane and contains the angle $\angle a_0$ inside
and the directions to $a'_{-1}$,$a_{-1}$, $a_{1}$ and $a'_{1}$ from $a_0$ appear in the same cyclic order.
\end{itemize}

\begin{wrapfigure}{r}{58mm}
\begin{lpic}[t(-20mm),b(-20mm),r(0mm),l(-3mm)]{pics/arm(0.3)}
\lbl[tr]{69,67;$a_0$}
\lbl[tl]{135,107;$a_1$}
\lbl[t]{135,67;\small{$x_1$}}
\lbl[l]{167,142;$a_2$}
\lbl[t]{168,67;\small{$x_2$}}
\lbl[l]{158,209;$a_3$}
\lbl[t]{156,67;\small{$x_3$}}
\lbl[l]{22,155;$a_{-1}$}
\lbl[t]{22,67;\small{$x_{-1}$}}
\lbl[lt]{46,206;$a_{-2}$}
\lbl[t]{46,67;\small{$x_{-2}$}}
\lbl{88,74;\small{$\sigma_1$}}
\lbl{117,76;$\sigma_2$}
\lbl{186,80;$\sigma_3$}
\end{lpic}
\end{wrapfigure}

Denote by $x_i$ and $x'_i$ the projections of $a_i$ and $a'_i$ to $x$-axis.
We can assume in addition that $x_k\ge x_{-m}$.
In this case 
$$|a_k-a_{-m}|=x_k-x_{-m}$$
and
$$|a'_k-a'_{-m}|\ge x'_k-x'_{-m}.$$
The latter follows because projection is a distance non-expanding map.  

Therefore it is sufficient to show
that 
$$x'_k-x'_{-m}\ge x_k-x_{-m}.$$
The later holds if
$$x'_i-x'_{i-1}\ge x_i-x_{i-1}.\eqlbl{eq:|bb|=<|aa|}$$
for each $i$.

It remains to prove \ref{eq:|bb|=<|aa|};
in the proof we assume that $i>0$, 
the case $i\le 0$ is similar.
Let us
\begin{itemize}
\item denote by $\sigma_i$ the angle between the vector $w_i=a_{i}-a_{i-1}$ and the $x$-axis;
\item denote by $\sigma'_i$ the angle between the vector $w'_i=a'_{i}-a'_{i-1}$ and the $x$-axis.
\end{itemize}
Note that
$$\begin{aligned}
x_i-x_{i-1}&=|a_i-a_{i-1}|\cdot\cos\sigma_i,
\\
x'_i-x'_{i-1}&=|a_i-a_{i-1}|\cdot\cos\sigma'_i
\end{aligned}
\eqlbl{eq:proj}$$
for each $i>0$.
By construction $\sigma_1\ge \sigma'_1$.
Note that $\angle (w_{i-1},w_i)=\pi -\angle a_i$.
From convexity of $[a_1 a_1\dots a_i]$, we have
$$\sigma_i=\sigma_1+(\pi-\angle a_1)+\dots+(\pi-\angle a_i)$$
 for any $i>0$.
Since $\angle (w'_{i-1},w'_i)=\pi -\angle a'_i$.
Applying Exercise~\ref{ex:angle-triangle} $i-1$ times,
we get
$$\sigma'_i\le\sigma'_1+(\pi-\angle a'_1)+\dots+(\pi-\angle a'_i).$$
Since $\angle a_j\le \angle a'_j$ for each $j$, summing it we get
$$\sigma_i\ge \sigma'_i.$$
Applying \ref{eq:proj}, we get \ref{eq:|bb|=<|aa|}.

In the case of equality, we have $\sigma_i=\sigma'_i$ for each $i$,
that implies $\angle a_i=\angle a'_i$ for each $i$.
This also implies that all $a'_i$ lie in $xy$-plane.
The latter easily follows from the equality case in the Exercise~\ref{ex:angle-triangle}.
\qeds
 
\parit{Proof of Local lemma (\ref{lem:local}).}
Assume that the Local lemma does not hold at the vertex $v$ of $\Gamma$.
Let cut from $P$ a small pyramid $\Delta$ 
with vertex $v$.
Then one can choose two points $a$ and $b$ on the base of $\Delta$
so that on one side of the segments $[va]$ and $[vb]$ we have only $(+)$'s, 
and on the other side only $(-)$'s.

%\begin{wrapfigure}{r}{53mm}
%\begin{lpic}[t(-15mm),b(-12mm),r(0mm),l(0mm)]{pics/4-changes(0.25)}
%\lbl[wtl]{110,153;$v$}
%\lbl[tr]{70,88;$a_0=a$}
%\lbl[tr]{14,128;$a_1$}
%\lbl[br]{19,181;$a_2$}
%\lbl[br]{73,228;$a_3$}
%\lbl[bl]{139,222;$a_4$}
%\lbl[bl]{170,195;$b=a_5$}
%\lbl[w]{60,143;\tiny{$(+)$}}
%\lbl[w]{90,190;\tiny{$(+)$}}
%\lbl[w]{155,127;\tiny{$(-)$}}
%\end{lpic}
%\end{wrapfigure}

The base polygon is formed by two broken lines with ends at $a$ and $b$.
Assume that 
$$a=a_0,\ a_1,\ \dots,\ a_n=b$$ 
form the broken line along the side marked with $(+)$'s.
Denote by 
$$a'=a_0',\  a_1',\ \dots,\ a'_n=b'$$ 
the corresponding points in $P'$.
Since each marked edge passing through $a_i$ has a $(+)$ on it or nothing, 
we have $$\angle a_{i-1}a_ia_{i+1}\le\angle  a'_{i-1}a'_ia'_{i+1}$$
for each $i$.
\begin{thm}{Exercise}
Prove the last statement. 
\end{thm}
By construction we have $|a_i-a_{i-1}|=|a'_i-a'_{i-1}|$ for all $i$.
By the Arm lemma (\ref{lem:arm}), 
we get 
\begin{clm}{}\label{clm:ab<ab}
$|a-b|\le |a'-b'|$ and equality holds if no edge from $v$ is marked with a $(+)$.
\end{clm}
Repeating the same construction exchanging the places of $K$ and $K'$ gives  
\begin{clm}{}\label{clm:ab>ab}
$|a-b|\ge |a'-b'|$ and equality holds no edge from $v$ is marked with a $(-)$.
\end{clm}

The claims 
\ref{clm:ab<ab}  and \ref{clm:ab>ab} 
together imply $|a-b|=|a'-b'|$ 
and it follows that no edge at $v$ is marked;
i.e., $v$ is not a vertex of $\Gamma$,
a contradiction.
\qeds

\section{Global lemma}

Before going into the proof, we suggest to do the following.

\begin{thm}{Exercise}
Try to mark the edges of an octahedron
by $(+)$'s and $(-)$'s
such that there would be 4 sign changes at each vertex.

Show that this is impossible.
\end{thm}

The proof of the Global lemma is based on counting the sign changes 
in two ways;
the first is as one moves around each vertex of $\Gamma$ 
and the second is as one moves around each of the regions separated by $\Gamma$
on the surface $P$. 
If two edges are adjacent at a vertex,
then they are also adjacent in moving around the region to whose boundary
they belong. 
The converse is true as well. 
Therefore, both of the ways of counting give the same number.




\parit{Proof of \ref{lem:global}.}
We can assume that $\Gamma$ is connected;
that is, one can get from any vertex to any other vertex by walking along edges.
(If not, pass to any connected component of $\Gamma$.)
Denote by $k$ and  $l$ the number of vertices and edges respectively in $\Gamma$.
Denote by $m$ the number of \emph{regions} which $\Gamma$ cuts from $P$.
Since $\Gamma$ is connected, each region is homeomorphic to an open disc.%
\begin{thm}{Exercise}
Prove the last statement.
\end{thm}
Now we can apply Euler's formula
$$k-l+m=2.
\eqlbl{eq:cauchy:euler}$$

Denote by $s$ the total number of sign changes in $\Gamma$ for all vertices. 
By the Local lemma (\ref{lem:local}), we have 
$$ 4\cdot k\le s.\eqlbl{eq:S>=4k}$$

Now let us get an upper bound on $s$ by counting the number of sign changes when you go around
each region. 
Denote by $m_n$ the number of regions which are bounded by $n$ edges;
if an edge appear twice when you go around the regions it is counted twice.
Note that each region is bounded by at least $3$ edges;
therefore
$$m=m_3+m_4+m_5+\dots\eqlbl{eq:3-4-5}$$
Counting edges and using the fact that each edge belongs to exactly two regions, we get
$$2\cdot l=3\cdot m_3+ 4\cdot m_4+5\cdot m_5+\dots.$$
Combining this with Euler's formula (\ref{eq:cauchy:euler}), we get
$$4\cdot k=8+2\cdot m_3+4\cdot m_4+6\cdot m_5+\dots
\eqlbl{eq:k=2+}$$
Observe that the number of sign changes in $n$-gon regions has to be even number which is at most $n$.
Therefore
$$s \le 2\cdot m_3 + 4\cdot m_4 + 4\cdot m_5 + 6\cdot m_6+\dots
\eqlbl{eq:23-44-45}$$
Clearly,  \ref{eq:S>=4k} and \ref{eq:23-44-45} contradict \ref{eq:k=2+}.
\qeds


\section{Alexandrov's uniqueness theorem}

Alexandrov's uniqueness theorem states that the conclusion of Cauchy's theorem (\ref{thm:cauchy}) still holds if one removes the phrase ``which sends each face of $K$ to a face of $K'$'' from it.
For your convenience we repeat the formulation here:

\begin{thm}{Alexandrov's uniqueness theorem}\label{thm:alexandrov-uni'}
Any two convex polyhedra in $\RR^3$ with isometric surfaces are congruent.
\end{thm}

The proof is along the same lines as the proof of Cauchy's theorem.
We will only describe the necessary modifications.

Let $\iota\:P\to P'$ be an isometry.
Mark in $P$ all the edges of $K$ and all the $\iota$-preimages of edges of $K'$, which will further be called fake edges.
These lines divide $P$ into convex polygons, say $\{Q_i\}$, and the restriction of $\iota$ to each $Q_i$ is a rigid move.
These polygons will play the role of faces in the proof of Cauchy's theorem.

\begin{center}
\begin{lpic}[t(-35mm),b(-30mm),r(0mm),l(0mm)]{pics/fake(0.35)}
\lbl[r]{10,120;$a$}
\lbl[r]{116,120;$a'$}
\lbl[rb]{34,170;$b$}
\lbl[rb]{143,170;$b'$}
\lbl[bl]{86,145;$c$}
\lbl[tl]{202,167;$c'$}
\lbl[tl]{79,108;$d$}
\lbl[tl]{158,108;$d'$}
\lbl[rt]{54,139;$v$}
\lbl[lt]{162,143;$v'$}
\lbl[]{100,86;A fake vertex $v\in K$ and the corresponding point $v'\in K'$.}
\end{lpic}
\end{center}

A vertex of $Q_i$ can be a vertex of $K$ or it can be a fake vertex;
i.e., lie on intersection of an edge and fake edge.
For the first type of vertex, the Local lemma can be proved in exactly the same way. 
For a fake vertex $v$, it is easy to see that both parts of the edge coming through $v$ are marked with $(+)$
while both of the fake edges at $v$ are marked with $(-)$.
Therefore, the Local lemma holds for the fake vertices as well.

The remainder of the proof needs no further modifications.





\section{Comments}

In the Euclid's Elements, 
solids were called equal if the same holds for their faces, but no proof was given.
Legendre   became   interested   in   it
towards   the   end   of   the   18th
century   and
talked to his colleague Lagrange
about it,  who  in  turn  suggested  this problem to  Cauchy in 1813
who soon proved it.
In 1950, Alexandrov understood that the condition on the equality of faces can be exchanged by much weaker condition on the surface of polyhedra.

We present a proof which has only minor modifications of Alexandrov's original proof in \cite{alexandrov}.
Two alternative proofs due to Pogorelov 
and Senkin--Zalgaller 
are nicely discussed in Pak's lecture notes \cite{pak},
both of these proofs are fun to read.

\parit{Arm lemma.}
The original Cauchy's proof (if you read French, see \cite{cauchy})
also used Arm lemma, 
but its proof contained an error 
which was corrected in 1928 by Steinitz.

The proof of Arm Lemma which we present is due to  Zaremba.
This and yet couple other beautiful proofs can be found in the letters between Schoenber and Zaremba published in \cite{schoenberg-zaremba}.

The following variation of Arm Lemma 
which makes sense for nonconvex shperical polygons.

\begin{thm}{Zalgaller's Arm Lemma}
Let $A=a_1a_2\dots a_n$ and $A'=a'_1a'_2\dots a'_n$ be two spherical $n$-gons (not nesessury convex).
Assume that $A$ lies in a half-sphere,
corresponding sides of $A$ and $A'$ are equal
and each angle of $A$ is bigger or equal to the corresponding angle in $A'$.
Then $A$ is congruent to $A'$.   
\end{thm}

See the original paper in Russian \cite{zalgaller} or its translation in ???.
This lemma is used in the Senkin--Zalgaller proof of the uniqueness theorem mentioned above.


\parit{Global lemma.} 
Alexandrov's book includes an other proof of Global lemma in addition to the one given here.
This proof require no calculations, it is easier to explain but harder to write down.



\section*{Exercises}

\begin{pr}\label{pr:sum>1pi}
Show that the sum of the exterior angles of any closed broken line in $\RR^3$ is at least $2\cdot\pi$.
\end{pr}

\noi Hint: Use induction on the number of edges.

Use the previous problem to solve the next one.

\begin{pr}\label{pr:aa+2} 
Consider a convex polygons $A=[a_1a_2\dots a_n]$ in $\RR^2$ 
and a closed broken line
$A'=[a'_1a'_2\dots a'_n]$ in $\RR^3$.
Let us enumerate the vertices in an $n$-periodic way;
i.e., set $a_{n+k}=a_k$ and $a'_{n+k}=a'_k$ for any $k$.

Assume $|a_i-a_{i-1}|=|a'_i-a'_{i-1}|$ and $|a_i-a_{i-2}|\le |a'_i-a'_{i-2}|$ for all $i$.
Show that $A$ is congruent to $A'$.
\end{pr}

\begin{pr}\label{pr:arm}
Give a complete proof of the Arm lemma (\ref{lem:arm}) using the following plan.

Use induction on $n$.
Prove the base case $n=2$.

For a point $b$ on the broken line $[a_0a_1\dots a_n]$, 
denote by $b'$ the corresponding point on $[a'_0a'_1\dots a'_n]$;
i.e., if $b\in[a_{i-1}a_i]$ then $b'\in[a'_{i-1}a'_i]$ and $|a_i-b|=|a'_i-b'|$.

Assume  $|a_0-a_n|\ge |a'_0-a'_n|$.
Applying the induction hypothesis,  $|a_i-a_j|\le |a'_i-a'_j|$ if $|i-j|<n$.
Moreover for any point $b$ on the broken line $[a_0a_1\dots a_n]$, we have
$|a_i-b|\le |a'_i-b'|$ if $0<i<n$.

Choose $b\in [a_{n-1}a_n]$ to be the closest point to $a_n$ auch that
$|a_0-b|=|a'_0-b'|$.

Show that the case $b= a_{n-1}$ can be reduced to the case $n=2$.

In the remaining case,  apply the previous problem to the broken lines
$[a_0a_1\z\dots a_{n-1}b]$ and $[a'_0a'_1\z\dots a'_{n-1}b']$.
\end{pr}

\begin{pr}\label{pr:tetrahedron} 
Assume that the surface of a nonregular tetrahedron $T$ has curvature $\pi$ at each of it vertices.
Show that 

\begin{enumerate}[i)]
\item all faces of $T$ are congruent;
\item the line passing through midpoints of opposite edges of $T$ itersects these edges at right angles. 
\end{enumerate}
 
\end{pr}



\begin{pr}\label{pr:K-P-simmetry}
Let $K$ be a convex polyhedron in $\RR^3$;
denote by $P$ its surface.
Show that each isometry $\iota\:P\z\to P$,
can be extended to an isometry of $\RR^3$. 
\end{pr}

The following problem is a quantitative version of Exercise \ref{ex:poly+geod}

\begin{pr}\label{pr:poly+geod+}
Let $P$ be a non-negatively curved polyhedral space homeomorphic to a sphere.

Assume two vertices in $P$ are jointed by two geodesics,
denote by $m_1$ and $m_2$ their midpoints.
Show that
$$|m_1-m_2|_P> \tfrac1{100}\cdot\ell\cdot\eps$$
where $\eps$ is the minimum of curvatures of the vertices of $P$
and $\ell$ is the minimal distance between pairs of vertices of $P$.
\end{pr}


\begin{wrapfigure}{r}{43mm}
\begin{lpic}[t(-10mm),b(-15mm),r(0mm),l(0mm)]{pics/cauchy-zalgaller(0.2)}
\end{lpic}
\end{wrapfigure}

\begin{pr}\label{pr:zalgaller}  
Assume that we glue in $\RR^3$
four regular pentagons 
and 22 equilateral triangles 
along the rule shown on the picture
such that they form a part of surface of a convex polyhedron.

Use the Local lemma to show that this configuration has 
a rotational symmetry with axis passing through the midpoint of the marked edge.
\end{pr}


