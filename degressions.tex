\chapter{Digressions into discrete geometry}\label{chap:degressions}

This complete chapter is beautiful and useless 




\section{Bezdek--Connelly theorem}

This is yet another relaxing section%???
.

\begin{thm}{Proposition}
Let $R>0$ and $a_1,a_2,\dots,a_n$ and $b_1,b_2,\dots,b_n$ be two collections of points in $\RR^m$ such that 
$$|a_i-a_j|\ge |b_i-b_j|$$
for all $i$ and $j$.
Then
$$\bigcap_{i=1}^n B_R(a_i)\not=\emptyset\ \ \Longrightarrow\ \ \bigcap_{i=1}^n B_R(b_i)\not=\emptyset.$$
\end{thm}

\parit{Proof.}
Applying  Kirszbraun's theorem (\ref{thm:kirszbraun}),
we get a distance non-contracting map $f\:\RR^m\to\RR^m$ such that $f(a_i)=b_i$.
Choose $x\in\bigcap_{i=1}^n B_R(a_i)$.
Since $f$ distance non-contracting, 
$$|f(x)-b_i|\le |x-a_i|\le R$$ 
for each $i$.
In particular, $f(x)\in B_R(b_i)$ for each $i$;
hence the result. 
\qeds

Given a set $A \subset \RR^m$, let $\vol_m(A)$ denote $m$-dimensional volume of $A$.  For example $\vol_1$ measures length in $\RR^1$, $\vol_2$ measures area in $\RR^2$, and $\vol_3$ measures the usual notion of volume in $\RR^3$.

\begin{thm}{Conjecture} \label{conj:volumesOfBalls}
Let $R>0$ and $a_1,a_2,\dots,a_n$ and $b_1,b_2,\dots,b_n$ be two collections of points in $\RR^m$ such that 
$$|a_i-a_j|\ge |b_i-b_j|$$
for all $i$ and $j$.
Then
$$\vol_m \left(\bigcap_{i=1}^n B_R(a_i)\right)\le \vol_m \left(\bigcap_{i=1}^n B_R(b_i)\right)\eqlbl{eq:n}$$
and
$$\vol_m \left(\bigcup_{i=1}^n B_R(a_i)\right)\ge \vol_m \left(\bigcup_{i=1}^n B_R(b_i)\right).
\eqlbl{eq:u}$$

\end{thm}

\begin{thm}{Exercise}
Prove this conjecture in case $m=1$.
\end{thm}


The inequality \ref{eq:u} was conjectured by Poulsen in 1954 and Kneser in 1955 and Hadwiger in 1956.
The inequality \ref{eq:n} was conjectured much later, it appears in list of problems of Klee and
Wagon published in 1991.
Both inequalities are trivial in the case $m=1$;
both are open problems for $m>2$.
Both were proved in case $m=2$ by Bezdek and Connelly in 2002.
Here we will describe ideas in their proof without going into details.

\parit{Not quite working idea.}
Assume one can construct $n$ smooth curves $\alpha_i\:[0,1]\z\to\RR^2$ such that  
$$\text{$\alpha_i(0)=a_i$, $\alpha_i(1)=b_i$ and}\ \ell_{i,j}(t)=|\alpha_i(t)-\alpha_j(t)|\ \text{is nonincreasing}.
\eqlbl{eq:nonincreasing}$$
In this case one can consider functions
$$v(t)=\area \left(\bigcap_{i=1}^n B_R(\alpha(t))\right)
\ \ \text{and}\ \ 
V(t)=\area \left(\bigcup_{i=1}^n B_R(\alpha(t))\right).$$
In order to prove \ref{eq:n}, it is sufficient to show that $v'(t)\ge 0$ for all $t$.
Similarly, to prove \ref{eq:u}, it is sufficient to show that $V'(t)\le 0$ for all $t$.
The latter is a calculus problem; it is technical, but straightforward.

\medskip

As one can see by the following exercise,
this approach cannot lead to a solution directly. 

\begin{thm}{Exercise}
Construct points $a_1,a_2,a_3,a_4,b_1,b_2,b_3,b_4\in\RR^2$ such that 
$$|a_i-a_j|\ge |b_i-b_j|$$
for all $i$ and $j$, but there are no curves $\alpha_1,\alpha_2,\alpha_3,\alpha_4\:[0,1]\to\RR^2$ which satisfy \ref{eq:nonincreasing}.
\end{thm}

In their proof, Bezdek and Connelly found a work around which uses the following theorems.

 
\begin{thm}{Alexander's theorem}\label{thm:alexander}
Let $a_1,a_2,\dots,a_n$ and $b_1,b_2,\dots,b_n$ be two collections of points in $\RR^m$.
Viewing $\RR^{2\cdot m}$ as $\RR^m \times \RR^m$, we shall consider $\RR^m = \RR^m \times \{0\}$ as a coordinate subspace of $\RR^{2\cdot m}$.
Then there is a choice of curves $\alpha_i\:[0,1]\to \RR^{2\cdot m}$ such that
$\alpha_i(0)=a_i=(a_i,0)$, $\alpha_i(1)=b_i=(b_i,0)$ and 
the function $\ell_{i,j}(t)=|\alpha_i(t)-\alpha_j(t)|$ is monotonic for each $i$ and $j$.
\end{thm}

\parit{Proof.} Straightforward calculations show that conclusion of the theorem hold for
$$
\alpha_i(t) = \left(\frac{a_i + b_i}{2} + 
\cos(\pi\cdot t)\cdot \frac{a_i - b_i}2,\  
\sin(\pi\cdot t)\cdot \frac{a_i - b_i}2\right). 
$$
\qedsf

\begin{thm}{Archimedes' theorem}
Consider the unit sphere $\SS^2\subset \RR^3$.
Denote by $\Pi\:\SS^2\to\RR$ a coordinate projection.
Then for any subinterval interval $[a,b]\z\subset [-1,1]$,
we have 
$$\area\left[\Pi^{-1}([a,b])\right]=2\cdot\pi\cdot(b-a).$$
In other words, the area of the unit sphere which lies between two cutting parallel planes of distance $h$ is equal to $2\cdot\pi\cdot h$.  
\end{thm}

\parit{Proof.} 
The set $\Pi^{-1}([a,b])$ is a surface of revolution of function 
$$f(x)=\sqrt{1-x^2}$$ 
restricted to the interval $[a,b]$.
Therefore
\begin{align*}
 \area\left[\Pi^{-1}([a,b])\right]
&=2\cdot\pi\cdot \int\limits_a^b f(x)\cdot\sqrt{1+(f'(x))^2}\cdot dx
\\
&=2\cdot\pi\cdot (b-a).
\end{align*}
\qedsf


Let us denote by $\BB^m$ the unit ball in $\RR^m$ 
and $\vol_m$ the $m$-dimensional volume.  Note that when we write $\vol_m \SS^m$, we are considering the $m$-dimensional analogue of surface area of the unit sphere, not the $(m+1)$-dimensional volume of the region it bounds in $\RR^{m+1}$.
Archimedes' theorem admits the following generalization in higher dimensions.
The proof goes along the same lines.
The case $m=1$ coincides with the original Archimedes' theorem;
we will use only the case $m=2$.


\begin{thm}{Generalized Archimedes' theorem}
Let $m$ be a positive integer.
Consider the unit sphere $\SS^{m+1}\subset \RR^{m+2}$.
Denote by $\Pi\:\SS^{m+1}\to \RR^m$ a coordinate projection;
clearly $\Pi(\SS^{m+1})=\BB^m$.
Then for any domain $\Omega\subset \BB^m$,
we have 
$$\vol_{m+1}\left[\Pi^{-1}(\Omega)\right]=2\cdot\pi\cdot\vol_m\Omega.$$
In particular
$$\vol_{m+1}\SS^{m+1}=2\cdot\pi\cdot\vol_m\BB^m.$$

\end{thm}

\parit{The working idea of Conjecture~\ref{conj:volumesOfBalls} for $m=2$.} According to Alexander's theorem we can construct curves $\alpha_i\:[0,1]\to\RR^4$ which satisfy \ref{eq:nonincreasing}.
Since we are considering $\RR^2$ as a subspace of $\RR^4$,
given $x\in \RR^2 \subset \RR^4$ we need to distinguish the notions of balls centered at $x$ in $\RR^2$ and balls centered at $x$ in $\RR^4$.
We will denote them $B_R(x;\RR^2)$ and $B_R(x;\RR^4)$ respectively.
From the Generalized Archimedes' theorem, we get that 
$$
\begin{aligned}
\area \left(\bigcap_{i=1}^n B_R(a_i;\RR^2)\right)
&=\tfrac{1}{2\cdot\pi\cdot R}\cdot\vol_3\left(\partial \bigcap_{i=1}^n B_R(a_i;\RR^4)\right)
\\
\area \left(\bigcap_{i=1}^n B_R(b_i;\RR^2)\right)
&=\tfrac{1}{2\cdot\pi\cdot R}\cdot\vol_3\left(\partial \bigcap_{i=1}^n B_R(b_i;\RR^4)\right)
\end{aligned}
\eqlbl{eq:nab}$$
Now consider the function 
$$w(t)=\tfrac{1}{2\cdot\pi\cdot R}\cdot\vol_3\left(\partial \bigcap_{i=1}^n B_R(\alpha_i(t);\RR^4)\right).$$
If $w(t)$ nonincreasing for all $t$, then together with \ref{eq:nab} it would imply \ref{eq:n}.

To show that $w(t)$ nonincreasing one needs to calculate its derivative and show that it is nonpositive.
The latter follows since 
$$w'(t)=-\sum_{i<j} \ell_{i,j}'(t)\cdot \theta_{i,j}(t),
$$ 
for some nonnegative values $\theta_{i,j}(t)$.
In fact for any fixed $t$, the value $\theta_{i,j}(t)$ can be expressed using only the values $\ell_{i,j}(t)$ and $R$.
(The inequality $\theta_{i,j}\ge 0$ follows from the geometric interpretation of this number as the area of certain sets in  $\bigcap_{i=1}^n B_R(\alpha_i(t);\RR^4)$, which we do not discuss here.) 


The inequality \ref{eq:u} is proved in a similar way.
We have to define 
$$W(t)=\tfrac{1}{2\cdot\pi\cdot R}\cdot\vol_3\left(\partial \bigcup_{i=1}^n B_R(\alpha_i(t);\RR^4)\right)$$
and prove that 
$$W'(t)=\sum_{i<j} \ell_{i,j}'(t)\cdot \Theta_{i,j}(t),
$$
for some nonnegative values $\Theta_{i,j}(t)$.
Then, apply the Generalized Archimedes theorem to get
$$\area \left(\bigcup_{i=1}^n B_R(a_i;\RR^2)\right)=W(0),$$
$$\area \left(\bigcup_{i=1}^n B_R(b_i;\RR^2)\right)=W(1).$$
Hence \ref{eq:u}.

\begin{thm}{Exercise}\label{ex:try3dim}
Try to understand why the same idea does not work in the case $m=3$.
\end{thm}

\begin{thm}{Exercise}\label{ex:try-a4}
Construct points $a_1,a_2,a_3,a_4,b_1,b_2,b_3,b_4\in\RR^2$ such that 
$$|a_i-a_j|\ge |b_i-b_j|$$
for all $i$ and $j$, but
$$\length \left(\partial\bigcup_{i=1}^4 B_R(a_i)\right)
<
\length \left(\partial\bigcup_{i=1}^4 B_R(b_i)\right).$$

\end{thm}

\begin{thm}{Exercise}\label{ex:try-R-infty}
Apply inequality \ref{eq:u} for $R\to \infty$ to show that 
if $a_1$, $a_2,\z\dots,a_n$, $b_1$, $b_2,\z\dots,b_n\in\RR^2$ such that 
$$|a_i-a_j|\ge |b_i-b_j|$$
for all $i$ and $j$ then
$$\length \left(\partial\Conv(a_1,a_2,\dots,a_n)\right)\ge \length \left(\partial\Conv(b_1,b_2,\dots,b_n)\right).$$

\end{thm}





