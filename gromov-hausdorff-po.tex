\chapter{The space of spaces}\label{chap:gromov-hausdorff}

Here we introduce convergence of metric spaces,
which was introduced by Gromov in \cite{gromov-polynomial-growth}.
In a couple of years this convergence spread into all branches of geometry and yet further.
In fact today I have difficulty to understand  how one could do geometry without this type of convergence.

To define convergence we introduce a metric on the set of isometry classes of compact metric spaces.
Our definition differs from the standard one, but it defines the same convergence.

Usually the Gromov--Hausdorff metric is defined using the following idea.
The Gromov--Hausdorff distance between subspaces in
the same metric space 
is assumed to be no greater than the Hausdorff distance between
them. 
In other words, if two subspaces of the same space are close to each
other in the sense of Hausdorff distance in the ambient space, they must be
close to each other as abstract metric spaces. 
Second, we assume that
the distance between isometric spaces to be zero. 
Then the Gromov--Hausdorff distance is 
defined as maximum distance satisfying these two requirements.

\section{Almost isometries}\label{sec:alm-isom}

Here we introduce a class of maps called $\eps$-isometries.
Here a positive value $\eps$ measure how much the maps in this class remind isometries.
The metric spaces which admit $\eps$-isometries from one to an other should be consideres to be ``close'' to eachother.


\begin{thm}{Definition} Let $X$ and $Y$ be metric spaces and $\eps > 0$. 
A  map $f\: X \z\to Y$ is called an $\eps$-isometry 
if 
$$|f(x)-f(x')|_Y\lege |x-x'|_X\pm\eps$$
for any $x,x'\in X$ 
and if $f(X)$ is an $\eps$-net in $Y$.
\end{thm}

Note that we do not assume that $f$ is continuous.

For example, if $\eps\ge\diam X,\diam Y$ then any map $f\:X\to Y$ is an $\eps$-isometry.

\begin{thm}{Exercise}\label{ex:alm-isom:compositon}
Let $f\:X\to Y$ and $g\:Y\to Z$ be two $\eps$-isometries.
Show that $g\circ f\: X\to Z$ is a $(3\cdot\eps)$-isometry.
\end{thm}


\begin{thm}{Exercise}\label{ex:alm-isom:inverse}
 Assume $f\: X \z\to Y$ is an $\eps$-isometry.
Show that there is a $(3\cdot\eps)$-isometry 
$g\: Y\to X$.
More over the $(3\cdot\eps)$-isometry $g$ can be chousen in such a way that
$$|g\circ f(x)-x|_X,\ |f\circ g(y)-y|_Y\le 2\cdot \eps$$
for any $x\in X$ 
and $y\in Y$.
\end{thm}

\begin{thm}{Exercise}\label{ex:alm-isom-limit}
Assume $X$ and $Y$ be compact metric spaces and
and for each natural $n$ there is an $\tfrac1n$-isometry
$f_n\:X\to Y$.
Then a subsequence of $f_n$ converges to an isometry $f_\infty\:X\to Y$.  
\end{thm}


\section{A partial order}




Given two metric spaces $X$ and $Y$, we will write $X\preccurlyeq Y$ if there is a noncontracting map $f\:X\to Y$;
i.e., if 
$$ |x-x'|_X\le|f(x)-f(x')|_Y$$
for any $x,x'\in X$.

Further, given $\eps>0$, we will write $X\preccurlyeq Y+\eps$
if there is a map $f\:X\to Y$ such that 
$$|x-x'|_X\le|f(x)-f(x')|_Y+\eps$$
for any $x,x'\in X$.
The next proposition follows directly from the definition of ``$\preccurlyeq$''. 

\begin{thm}{Proposition}\label{prop:po-eps}
If for given three metric spaces $X$, $Y$ and $Z$
we have 
$X\preccurlyeq Y+\eps$ and 
$Y\preccurlyeq Z+\delta$ 
then
$$X\preccurlyeq Z+\eps+\delta.$$

\end{thm}


\section{Gromov--Hausdorff metric}

Now we will cook up a metric space out of metric spaces.
More precisely, we will define the so called  Gromov--Hausdorff metric on the set of \emph{isometry classes} of compact metric spaces.
(Being isometric is an equivalence relation on the class of metric spaces, 
and an isometry class is an equivalence class with respect to this equivalence relation.)

Given two metric spaces, the Gromov--Hausdorff distance from the isometry class of $X$ 
to the isometry class of $Y$ will be denoted as $d_{GH}(X,Y)$;
but we will often say (not quite correctly) 
``$d_{GH}(X,Y)$ is the Gromov--Hausdorff distance from  $X$ 
to  $Y$''.
As you will see further, $d_{GH}(X,Y)=0$ if and only if $X$ is isometric to $Y$.

In other words, from now on, if I say ``metric space'',
you should guess from the context did I mean ``metric space'' 
or ``isometry class of this metric space'' (sorry for notation abuse).


\begin{thm}{Definition}\label{def:GH}
Let $X$ and $Y$ be compact metric spaces. 
The Gromov--Hausdorff distance $d_{GH}(X, Y )$
 between $X$ and $Y$ is defined as infimum of all $\eps>0$ such that $X\preccurlyeq Y+\eps$ and $Y\preccurlyeq X+\eps$.
\end{thm}

\begin{thm}{Theorem}\label{thm:GH-is-a-metric}
The set of isometry classes of compact metric spaces 
equipped with Gromov--Hausdorff metric forms a metric space.
\end{thm}

This metric space will be denoted further as $\mathcal{M}$\index{$\mathcal{M}$}; named for ``metric space''.
Before proving this theorem we will prove the following.

\begin{thm}{Theorem}\label{thm:GH=>eps-isom}
Given a compact space $X$, and $\eps>0$
there is $\delta>0$ such that
if $d_{GH}(X,Y)<\delta$ then there is an $\eps$-sometry 
$X\to Y$.
\end{thm}

To prove this theorem we will need the following lemma.

\begin{thm}{Lemma}\label{lem:GH=>eps-isom}
Let $X$ be a compact space then for any $\eps>0$ there is $\delta>0$
such that if a map $h\:X\to X$ satisfies 
$$|x-x'|\le|h(x)-h(x')|+\delta $$ 
for any $x,x'\in X$
then $h$ is an $\eps$-isometry.
\end{thm}

The above lemma is a quantative version of Problem~\ref{pr:non-contracting=>isometry}.
We suggest to read the first solution of this problem before reading the following proof. 

\parit{Proof.}
Choose a maximal $\tfrac\eps{10}$-packing $A=\{a_1,a_2,\dots,a_k\}$.
Choose sufficiently small $\delta>0$
(any $\delta$ such that $k^2\cdot \delta<\tfrac{\eps}{10}$ and $|a_i-a_j|>\tfrac\eps{10}+\delta$ for all $i\ne j$ will do).

Let  $h_n\:X\to X$ be a sequence of maps
such that 
$$|x-x'|\le|h_n(x)-h_n(x')|+\tfrac\delta{2^n}\eqlbl{dist=<dist+delta}$$
for any $x,x'\in X$ and each $n$.
Arguing by contradiction, 
assume that each $h_n$ fails to be an $\eps$-isometry.

Note that for each $n$,
the set $A_n=h_n\circ h_{n-1}\circ\dots\circ h_1(A)$
is a maximal $\tfrac\eps{10}$-packing of $X$.
In particular $A_n$ and therefore $h_n(X)$ both form $\tfrac\eps{10}$-nets in $X$. 

Since $h_n$ is not an $\eps$-isometry there are points $x,y\in X$ such that
$$|x-y|+\eps
<
|h_n(x)-h_n(y)|.$$
Choose $x',y'\in A_{n-1}$ 
such that $|x-x'|, |y-y'|\le\tfrac\eps{10}$;
clearly
$$|x'-y'|+\tfrac\eps{2}
<
|h_n(x')-h_n(y')|.
\eqlbl{dist<dist-eps}$$

Set $$L_n=\sum_{x,y\in A_n}|x-y|.$$
Since $\delta$ is sufficiently small, 
\ref{dist<dist-eps} together with \ref{dist=<dist+delta} imply
$$L_{n-1}+\tfrac\eps{10}<L_n.$$
On the other hand 
$$L_n\le k^2\cdot\diam X<\infty$$
for any $n$,
a contradiction.
\qeds

\parit{Proof of Theorem~\ref{thm:GH=>eps-isom}.}
Assume $d_{GH}(Y,X_n)\to 0$ as $n\to\infty$.
It means that there are two sequences of maps $f_n\:X_n\to Y$ and $g_n\:Y\to X_n$
such that 
$$ |x-x'|_{X_n}\le|f_n(x)-f_n(x')|_{Y}+\delta_n\eqlbl{eq:f_n}$$
and
$$|y-y'|_Y\le|g_n(y)-g_n(y')|_{X_n}+\delta_n\eqlbl{eq:g_n}$$
for any $x,x'\in X_n$ and $y,y'\in Y$ and some sequence $\delta_n\to 0+$.

Fix $\eps>0$.
Let us show that the maps $g_n$ are $\eps$-isometries for all large  $n$.

First note that from \ref{eq:f_n}, \ref{eq:g_n} and Lemma \ref{lem:GH=>eps-isom},
it follows that 
$$h_n=f_n\circ g_n\:Y\to Y$$ 
is an $\tfrac{\eps}2$-isometry for all large $n$.

It follows that $h_n(Y)$ and therefore $f_n(X_n)$ forms an $\tfrac{\eps}2$-net in $Y$ for all large $n$.
In particular, given $x\in X_n$ one can find 
 $y\in Y$ such that
$|g_n(x)-h_n(y)|\le \tfrac{\eps}2$.
Hence $|x-f_n(y)|\le \tfrac {\eps}2+\delta_n$.
In particular, $g_n(Y)$ is a $\eps$-net for all large $n$.

Further, since  $h_n$ 
is an $\tfrac{\eps}2$-isometry, we have
$$|y-y'|\lege |h_n(y)-h_n(y')|\pm\tfrac{\eps}2.$$
Together with \ref{eq:f_n} and \ref{eq:g_n}, it implies that
$$|g_n(y)-g_n(y')|\lege |y-y'|\pm\eps$$
for all large $n$.
\qeds

\parit{Proof.}
Let $X$, $Y$ and $Z$ be arbitrary  compact metric spaces.
We need to check the following (see Definition~\ref{def:metric-space}).
\begin{enumerate}[{\it (i)}]
\item\label{GH-1} $d_{GH}(X,Y)\ge 0$;
\item\label{GH-2} $d_{GH}(X,Y)=0$ if and only if $X$ is isometric to $Y$;
\item\label{GH-3} $d_{GH}(X,Y)=d_{GH}(Y,X)$;
\item\label{GH-4} $d_{GH}(X,Y)+d_{GH}(Y,Z)\ge d_{GH}(X,Z)$.
\end{enumerate}


Note that {\it (\ref{GH-1})}, {\it(\ref{GH-3})} and ``if''-part of {\it(\ref{GH-2})} follow directly from the definition of Gromov--Hausdorff metric (\ref{def:GH}).
Further {\it(\ref{GH-4})} follows from Proposition~\ref{prop:po-eps}.
Finally {\it (\ref{GH-3})} follows from Teorem~\ref{thm:GH=>eps-isom} and Exercise~\ref{ex:alm-isom-limit}
\qeds




\section{Gromov--Hausdorff convegence}

The Gromov--Hausdorff defined above give rise to the convergence of compact metric spaces;
we say that a sequence of compact metric spaces $X_n$ converges to a compact metric space $X_\infty$ 
(briefly $X_n\GHto X_\infty$)
if
$d_{GH}(X_n,X_\infty)\to0$ as $n\to 0$.

In fact this convergence is the only important  outcome from everything above in this chapter.
(No one cares about value of Gromov--Hausdorff distance between particular pair of spaces,
but many care whether a particular sequence of spaces converges.)

From Theorem~\ref{thm:GH=>eps-isom} and Exercise~\ref{ex:alm-isom:inverse} we get the following characterization of Gromov--Hausdorff convergence.

\begin{thm}{Proposition}
A sequence of compact metric spaces $X_n$ converges to a compact metric space $X_\infty$ 
if and only if one of the following equivalent conditions hold.
\begin{subthm}
For some sequence $\eps_n\to 0+$ 
there is an $\eps_n$-isometry $f_n\:X_\infty\to X_n$ for each $n$.
\end{subthm}

\begin{subthm}
For some sequence $\eps_n'\to 0+$ 
there is an $\eps_n'$-isometry $g_n\:X_n\to X_\infty$ for each $n$.
\end{subthm}
\end{thm}




\section{Gromov's compactness theorem}

The following theorem is analogous to Blaschke's compactness theorems (\ref{thm:compact+Hausdorff}).

\begin{thm}{Gromov's compactness theorem}\label{thm:gromov-compactness}%
Let $\mathcal{Q}$ be a closed subset of $\mathcal{M}$.
Show that $\mathcal{Q}$ is compact if and only if there is a sequence of positive numbers $\eps_1,\eps_2,\dots$ such that $\eps_n\to 0$ and 
$$\pack_{\eps_n}X\le n\eqlbl{eq:pack<n}$$
for any space $X$ in $\mathcal{Q}$.
\end{thm}

Note that the conclusion of theorem does not hold if the inequality holds only for $n\ge 2$.
For example take $\mathcal{Q}$ to be the set of isometry classes of all one- and two-point metric spaces .

\begin{thm}{Lemma}
The space $\mathcal{M}$ is complete.
\end{thm}

\parit{Proof.}
Let $(X_n)$ be a Cauchy sequence in $\mathcal{M}$,
we need to show that $X_n$ converges in $\mathcal{M}$.
Note that it is sufficient to show that $(X_n)$ has a converging subsequence.

Passing to a subsequence, we may assume that 
$d_{GH}(X_n,X_{n+1})< \tfrac1{2^n}$.
Let $f_n\:X_n\to X_{n+1}$ and $g_n\:X_{n+1}\to X_n$ be the maps such that
$$???.\eqlbl{eq:GH-f-g}$$

First we need to construct a new metric space $X_\infty$, the limit of $(X_n)$,
from the material we have in our hands.

Consider all sequences $\mathcal{W}$ of points $x_n\in X_n$ such that $x_n=g_n(x_{n+1})$ for each $n$.
Given two  two sequences $\bm{x}=(x_1,x_2,\dots)$ and $\bm{y}=(y_1,y_2,\dots)$ like that
set $\ell_n=|x_{n}-y_{n}|_{X_n}$.
Note that \ref{eq:GH-f-g} implies that 
$$\ell_{n+1}\le \ell_n+\tfrac1{2^n}.$$
Clearly, $\ell_n\ge 0$ for all $n$;
therefore $\ell_n$ converges as $n\to \infty$.
Set 
$$\Dist(\bm{x},\bm{y})
\df
\lim_{n\to\infty} \ell_n.$$
The function $\Dist\:\mathcal{W}\times\mathcal{W}\to[0,\infty)$
satisfies all the conditions in the definition of metric space 
except the condition \ref{def:metric-space:zero}.
Consider relation $\sim$ on $\mathcal{W}$ such that
$\bm{x}\sim \bm{y}$ 
if and only if $\Dist(\bm{x},\bm{y})=0$.
Set $X_\infty'$ to be the set of $\sim$-equivalence clsses of $\mathcal{W}$
with the metric induced by $\Dist$.
Note that the space $X_\infty'$ is a genuine %???
metric space.
Set $X_\infty$ to be the completion of $X_\infty'$
It remains to show that $X_n\GHto X_\infty$ as $n\to \infty$. 

First we need to show that $X_\infty$ is compact.
...

Now 
\qeds

\parit{Proof.}
Let $(X_n)$ be a Cauchy sequence in $\mathcal{M}$.
Passing to a subsequence if necessary, we can assume that there is an 
$\tfrac1{2^n}$-isometry $\iota_n\: X_n\to X_{n+1}$ for each $n$.
Let us call a sequence of points $x_n\in X_n$ ``nice'' if 
$$|\iota_n(x_n) - x_{n+1}|_{X_{n+1}}\le \tfrac3{2^{n}}$$ 
for each $n$.

Note that if $\bm{x}=(x_1,x_2,\dots)$ and $\bm{y}=(y_1,y_2,\dots)$ are nice 
then 
$$\ell_n=|x_n-y_n|_{X_n}$$ 
is a Cauchy sequence of real numbers.
Therefore the following limit is well defined:
$$\Dist(\bm{x},\bm{y})\df\lim_{n\to\infty}|x_n-y_n|_{X_n}.$$ 
Consider the equivalence relation 
$\bm{x}\sim\bm{y}$ $\Leftrightarrow$ $\Dist(\bm{x},\bm{y})=0$ on the set of nice sequences.
Then $\Dist$ induces a metric on the set of $\sim$-equivalence classes of nice sequences.
The obtained metric space will be denoted as $X_\infty$.

Clearly for each $x_n\in X_n$ there is a nice sequence with $n$-th element $x_n$.
Choosing one such a sequence for each $x_n\in X_n$ defines a map $X_n\to X_\infty$ which is $\tfrac{10}{2^n}$-isometry.
I.e., $X_n$ converges to $X_\infty$ in the sense of Gromov--Hausdorff.
\qeds

\parit{Proof of \ref{thm:gromov-compactness}; ``only if'' part.}
First note that any compact set has bounded diameter.
Further $\pack_\eps X\le 1$ if and only if $\diam X\le \eps$,
Therefore there is finite value $\eps_1$ such that $\pack_{\eps_1} X\le 1$ for any $X\in\mathcal{Q}$.

If there is no sequence $\eps_n\to0$ as described in the problem, then for a fixed fixed $\delta>0$
there is a sequence of spaces $X_n\in\mathcal{Q}$ such that $$\pack_\delta X_n\to\infty\ \ \text{as}\ \  n\to\infty.$$
Since $\mathcal{Q}$ is compact, 
this sequence has a partial limit say $X_\infty\in\mathcal{Q}$.
It is easy to see that $\pack_{\delta/10} X_\infty=\infty$;
the later contradicts Theorem~\ref{thm:finite_pack=compact}.

\parit{``If'' part.}
Let us fix the sequence $\eps_n\to 0$ as in the problem and consider the set $\hat{\mathcal{Q}}$ of all (isometry classes of all) metric spaces $X$ such that
$\pack_{\eps_n} X\le n$ for any $n$. 
According to Exercise~\ref{ex:pack-GH}, $\hat{\mathcal{Q}}$ is closed in $\mathcal{M}$.
Clearly $\mathcal{Q}\subset\hat{\mathcal{Q}}$.
Therefore it is sufficient to prove that $\hat{\mathcal{Q}}$ is compact.

Given positive integer $n$ consider set of all metric spaces $\mathcal{W}_n$
with number of points at most $n$ and diameter $\le \eps_1$.
Note that $\mathcal{W}_n$ is compact for each $n$.
Further a maximal $\eps_n$-packing of any $X\in\hat{\mathcal{Q}}$ forms a subspace from $\mathcal{W}_n\cap\hat{\mathcal{Q}}$.
Therefor $\mathcal{W}_n$ forms an $\eps_n$-net in  $\hat{\mathcal{Q}}$.
Exercise~\ref{ex:compact-net} implies that $\hat{\mathcal{Q}}$ is compact.
\qeds






\section*{Exercises}

\begin{pr}
Let $X$ and $Y$ be compact subspaces of a metric space $Z$.
Show that 
$$d_{GH}(X,Y)\le 2\cdot d^Z_H(X,Y).$$
Give an example of two subspaces $X$ and $Y$ in $Z$
such that
$$d_{GH}(X,Y)= 2\cdot d^Z_H(X,Y).$$
\end{pr}


\begin{pr}\label{pr:GH1}
Let $X=\{x,y,z\}$ be a three point subset of Euclidean plane with distances
$$|x-y|=|y-z|=|z-x|=1.$$
\begin{enumerate}[(i)]
\item Find the minimal Hausdorff distance from $X$ to a one-point subset of the plane.
\item Find the Gromov--Hausdorff distance from $X$ to the one-point metric space. 
\end{enumerate}
\end{pr}

\begin{pr}\label{pr:GH2}
Let $X$ and $Y$ be a compact metric spaces which have isometric $\eps$-nets.
Show that 
$$d_{GH}(X,Y)\le 2\cdot\eps.$$
Is it allways true that 
$$d_{GH}(X,Y)\le \eps?$$
\end{pr}




\def\rad{\mathop{\rm rad}}
\begin{pr}\label{pr:GH3}
Define the \emph{radius of a metric space}\index{radius of a metric space} $X$ as 
$$\rad X=\inf_x\left\{\sup_y\{|x-y|_X\}\right\}.$$
Equivalently, 
$$\rad X=\inf\set{R>0}{\text{there is}\ x\in X\  \text{such that}\ B_R(x)\supset X}.$$
 
\begin{enumerate}[(i)]
\item Show that for any compact metric space $X$ we have
$$\tfrac12\cdot\diam X\le \rad X\le \diam X.$$
\item Show that for any compact metric spaces $X,Y$ we have
$$|\rad X-\rad Y|\le 2\cdot d_{GH}(X,Y).$$
\end{enumerate}
\end{pr}

\begin{pr}\label{ex:point-diam}
Let $P$ be a one-point metric space. 
Prove that 
$$d_{GH} (X, P ) = \frac{\diam X}2$$ for any compact metric space $X$.
\end{pr}

\begin{pr}\label{ex:d_GH-and-diam}
 Let $X$ and $Y$ be two compact metric spaces.
Prove that 
$$|\diam X - \diam Y |\le 2\cdot d_{GH} (X, Y ).$$
In other words, $\diam$ is a $2$-Lipschitz function on $\mathcal{M}$.
\end{pr}

\begin{pr}\label{ex:pack-GH}
Assume $X_n$ be a sequence of compact metric spaces which converges to a compact metric space $X_\infty$
in the sense of Gromov--Hausdorff.
Show that for any $\eps>0$
$$\pack_\eps X_n\ge\pack_\eps X_\infty$$ 
for all large enough $n$.
In particular, $\pack_\eps$ is a lower semicontinuous function on $\mathcal{M}$.
\end{pr}

