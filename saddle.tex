\chapter{Saddle surfaces}

\section{Polyhedral saddle surfaces}

%???+PIC

A 2-dimensional polytope in $\RR^3$ is called 
\emph{polyhedral disc} if it is homeomorphic to a the unit disc
\[\DD=\set{(x,y)\in\RR^2}{x^2+y^2\le 1}.\]

A vertex of a triangulation of a polyhedral disc 
is called interior if it lies in the \emph{interior vertex} of $\DD$ 
and \emph{boundary vertex} if it lies on the boundary of $\DD$.
The boundary and interior vertices can be defined in a more combinatorial fashion.
The link of any vertex is formed by a broken line.
For interior vertex the link is a closed line,
and for boundary vertex it is open; 
i.e., it has the beginning and the end.
This equivalence as well as the correctness of this definition folows from domain invariance;
see \ref{thm:domain-invariance}.

A polyhedral disc $P$ in $\RR^3$
is called \emph{saddle} if any interior vertex of $P$ lies in the convex hull of its link.

\begin{thm}{Proposition}\label{prop:neg-poly}
Let $P$ be a saddle polyhedral disc in $\RR^3$
equipped with the induced length metric. 
Then $P$ has non-positive curvature;
that is the total angle around any point in the interior of $P$ is at least $2\cdot\pi$.
\end{thm}

The proof relies on the following lemma

\begin{thm}{Hemisphere lemma}
\label{lem:hemisphere}
Any closed curve of length $<2\cdot \pi$  in $\SS^2$ lies in an open hemisphere. 
\end{thm}

\parit{Proof.} 
Let $\alpha$ be a closed curve in $\SS^2$ of length $2\cdot\ell$.
Assume $\ell<\pi$.

Let us dived $\alpha$ in two subarcs, 
say $\alpha_1$ and $\alpha_2$ with length $\ell$ each. 
Denote by $p$ and $q$ their endpoints. 
Since $\dist{p}{q}{}\z\le\ell<\pi$, there is a unique geodesic $[pq]$ in $\SS^2$.  
Let $z$ be the midpoint of $[pq]$.  

We claim that $\alpha$ lies in the open hemisphere centered at $z$.  
If not, $\alpha$ intersects the boundary  great circle,
say at the point $x$.
Without loss of generality we may assume that $x\in\check\alpha_1$. 

Consider rotation of $\SS^2$ 
by angle $\pi$ with fixed point $z$.
Arc $\alpha_1$ together with its rotation form a closed curve of length $2\cdot \ell$ that passes through $x$ and its antipodal point $x'=-x$.
Thus 
\begin{align*}
\ell
&=\length \alpha_1\ge
\\
&\ge \dist{x}{x'}{\SS^2}=
\\
&=\pi,
\end{align*}
where $\dist{x}{x'}{\SS^2}$ denotes the distance from $x$ to $x'$ in the length metric on $\SS^2$, contradiction.
\qeds

\parit{Proof of Proposition~\ref{prop:neg-poly}.}
Let $x$ be an interior vertex of the saddle surface $\Sigma$, denote by $\theta$ the total angle around $x$.

Choose sufficiently small $r>0$;
it should be smaller that the distance from $x$ to any simplex of $\Sigma$ which does not contain $x$.
Note that the intersection of $\Sigma$ with
a sphere of radius $r$ centered at $x$ forms a closed curve, say $\alpha$.

Since $\Sigma$ is saddle,
each equator in the sphere intersects $\alpha$. 
Therefore by Hemisphere lemma, 
the $\length\alpha\ge 2\cdot\pi\cdot r$.
On the other hand,
$\length\alpha=\theta\cdot r$.
Hence the result.
\qeds

\section{Thin triangles} \label{sec:thin-triangle} 
 
\begin{thm}{Definition of thin triangles
}\label{def:k-thin}
Let $\trig{x^1}{x^2}{x^3}$ be a triangle in a metric space.
Consider its model triangle
$\trig{\tilde x^1}{\tilde x^2}{\tilde x^3}=\modtrig0({x^1}{x^2}{x^3})$ 
and the  \emph{natural map}\index{natural map} $\trig{\tilde x^1}{\tilde x^2}{\tilde x^3}\to \trig{x^1}{x^2}{x^3}$ 
that sends a point $\tilde z\in[\tilde x^i\tilde x^j]$ to the corresponding point $z\in[x^ix^j]$
(i.e. such that $\dist{\tilde x^i}{\tilde z}{}=\dist{x^i}{z}{}$ and therefore $\dist{\tilde x^j}{\tilde z}{}=\dist{x^j}{z}{}$).

We say the triangle $\trig{x^1}{x^2}{x^3}$ is \emph{thin}\index{thin} if the natural map $\trig{\tilde x^1}{\tilde x^2}{\tilde x^3}\to \trig{x^1}{x^2}{x^3}$ is distance nonexpanding.
\end{thm}


\begin{thm}{Inheritance lemma for thin triangles}
\label{lem:inherit-angle} 
In a metric space, consider a triangle $\trig p x y$ that \emph{decomposes}\index{decomposed triangle} 
into two triangles $\trig p x z$ and $\trig p y z$;
that is, $\trig p x z$ and $\trig p y z$ have common side $[p z]$, and the sides $[x z]$ and $[z y]$ together form the side $[x y]$ of $\trig p x y$.

If  both triangles $\trig p x z$ and $\trig p y z$ are thin, then triangle $\trig p x y$ is  thin.
\end{thm} 

We shall need the following model-space lemma.

\begin{thm}{Lemma}\label{lem:quadrangle}
Let $\trig{\tilde p}{\tilde x}{\tilde y}$ be a triangle in $\RR^2$ and $\tilde z\in[\tilde x\tilde y]$.
Set $\tilde D=\Conv\trig{\tilde p}{\tilde x}{\tilde y}$.  
Construct  points $\dot p, \dot x, \dot y, \dot z\in \RR^2$ such that 
$\dist{\dot p}{\dot x}{}=\dist{\tilde p}{\tilde x}{}$, 
$\dist{\dot p}{\dot y}{}=\dist{\tilde p}{\tilde y}{}$,
$\dist{\dot x}{\dot z}{}=\dist{\tilde x}{\tilde z}{}$, 
$\dist{\dot y}{\dot z}{}=\dist{\tilde y}{\tilde z}{}$,
$\dist{\dot p}{\dot z}{}\le \dist{\tilde p}{\tilde z}{}$
and points $\dot x$ and $\dot y$ lie on either side of $[\dot p\dot z]$.
Set $\dot D=\Conv\trig {\dot p}{\dot x}{\dot z}\cup \Conv\trig {\dot p} {\dot y} {\dot z}$.

Then there is a distance nonexpanding map $F\:\tilde D\to \dot D$ that maps $\tilde p$, $\tilde x$, $\tilde y$ and $\tilde z$ to $\dot p$, $\dot x$, $\dot y$ and $\dot z$ respectively.
\end{thm}

\begin{wrapfigure}{r}{27mm}
\begin{lpic}[%draft,
t(-5mm),b(0mm),r(0mm),l(0mm)]{pics/resh(0.3)}
\lbl[t]{47,9;$\tilde p$}
\lbl[l]{82,88;$\tilde x$}
\lbl[r]{2,74;$\tilde y$}
\lbl[b]{37,82;$\tilde z$}
\lbl[tr]{33,62;$\tilde z_x$}
\lbl[tl]{46,63;$\tilde z_y$}
\end{lpic}
\end{wrapfigure}

\parit{Proof.} 
By Alexandrov's lemma (\ref{lem:alex}), 
there are nonoverlapping triangles 
$\trig{\tilde p}{\tilde x}{\tilde z_y}\iso\trig {\dot p}{\dot x}{\dot z}$ 
and 
$\trig{\tilde p}{\tilde y}{\tilde z_x}\iso\trig {\dot p}{\dot y}{\dot z}$
 inside triangle $\trig{\tilde p}{\tilde x}{\tilde y}$.

Connect points in each pair
$(\tilde z,\tilde z_x)$, 
$(\tilde z_x,\tilde z_y)$ 
and $(\tilde z_y,\tilde z)$ 
with arcs of circles centered at 
$\tilde y$, $\tilde p$, and $\tilde x$ respectively. 
Define $F$ as follows.
\begin{itemize}
\item Map  $\Conv\trig{\tilde p}{\tilde x}{\tilde z_y}$ isometrically onto  $\Conv\trig {\dot p}{\dot x}{\dot y}$;
similarly map $\Conv \trig{\tilde p}{\tilde y}{\tilde z_x}$ onto $\Conv \trig {\dot p}{\dot y}{\dot z}$.
\end{itemize}

\begin{itemize}
\item If $x$ is in one of the three circular sectors, say at distance $r$ from center of the circle, let $F(x)$ be the point on the corresponding segment 
$[p z]$, 
$[x z]$ 
or $[y z]$ whose distance from the lefthand endpoint of the segment is $r$.
\item Finally, if $x$ lies in the remaining curvilinear triangle $\tilde z \tilde z_x \tilde z_y$, 
set $F(x) = z$. 
\end{itemize}
By construction, $F$ satisfies the conditions of the lemma. 
\qeds


\parit{Proof of Inheritance for thin triangles (\ref{lem:inherit-angle}).}
Construct model triangles $\trig{\dot p}{\dot x}{\dot z}\z=\modtrig0(p x z)$ 
and $\trig {\dot p} {\dot y} {\dot z}=\modtrig0(p y z)$ so that $\dot x$ and $\dot y$ lie on opposite sides of $[\dot p\dot z]$.

\begin{wrapfigure}{r}{28mm}
\begin{lpic}[t(2mm),b(0mm),r(0mm),l(0mm)]{pics/cat-monoton-ineq(0.45)}
\lbl[b]{23,29;$\dot z$}
\lbl[lt]{20,1;$\dot p$}
\lbl[r]{0,20;$\dot x$}
\lbl[l]{51,20;$\dot y$}
\lbl[tl]{23,19;$\dot w$}
\end{lpic}
\end{wrapfigure}

Suppose
\[\angk0{z}{p}{x}+\angk0{z}{p}{y}
<
\pi.\]
Then for some point $\dot w\in[\dot p\dot z]$, we have \[\dist{\dot x}{\dot w}{}+\dist{\dot w}{\dot y}{}
<
\dist{\dot x}{\dot z}{}+\dist{\dot z}{\dot y}{}=\dist{x}{y}{}.\]
Let $w\in[p z]$ correspond to $\dot w$; i.e. $\dist{z}{w}{}=\dist{\dot z}{\dot w}{}$. 
Since $\trig p x z$ and $\trig p y z$ are thin, we have 
\[\dist{x}{w}{}+\dist{w}{y}{}<\dist{x}{y}{},\]
contradicting the triangle inequality. 

Thus 
\[\angk0{z}{p}{x}+\angk0{z}{p}{y}
\ge
\pi.\]
By Alexandrov's lemma (\ref{lem:alex}), this is equivalent to 
\[\angk0 x p z\le\angk0 x p y.
\eqlbl{eq:for|pz|}\]

Let $\trig{\tilde  p}{\tilde  x}{\tilde  y}=\modtrig0 (p x y)$ 
and $\tilde  z\in[\tilde  x\tilde  y]$ correspond to $z$; i.e. $\dist{x}{z}{}=\dist{\tilde  x}{\tilde  z}{}$.
Inequality~\ref{eq:for|pz|} is equivalent to $\dist{ p}{ z}{}\le \dist{\tilde  p}{\tilde  z}{}$.
Hence  Lemma~\ref{lem:quadrangle} applies.  Therefore 
there is a distance nonexpanding map $F$ that  sends 
$\trig{\tilde  p}{\tilde  x}{\tilde  y}$ to $\dot D=\Conv\trig {\dot p}{\dot x}{\dot z}\cup \Conv\trig {\dot p} {\dot y} {\dot z}$ 
in such a way that 
$\tilde p\mapsto \dot p$,
$\tilde x\mapsto \dot x$,
$\tilde z\mapsto \dot z$
and
$\tilde y\mapsto \dot y$.

By assumption, the natural maps $\trig {\dot p} {\dot x} {\dot z}\to\trig p x z$ and $\trig {\dot p} {\dot y} {\dot z}\to\trig p y z$ are distance nonexpanding.  
By composition,  the natural map from $\trig{\tilde  p}{\tilde  x}{\tilde  y}$ to $\trig p y z$ is distance nonexpanding, as claimed.
\qeds

\section{Polyhedral discs with non-positive curvature}

\begin{thm}{Lemma}
In a polyhedral disc non-positive curvature
any two points are joint by unique geodesic.

Moreover the unique geodesic depends continuously from its end points.
\end{thm}

\parit{Proof}
Let $P$ be a polyhedral disc non-positive curvature and $x,y\in P$ and $|x-y|=\ell$.
Since $P$ is compact,
there is a geodesic from $x$ to $y$.

Assume that there are two geodesics $\alpha,\beta\:[0,\ell]\to P$ from  from $x$ to $y$.
Note that the intersection of $\alpha$ as well as $\beta$
with any triangle in a triangulation of $P$ is a line segment.


Without loss of generality, we may assume that 
$\alpha(t)= \beta(t)$ if and only if $t=0$ or $l$.
Otherwise choose a maximal with respect to inclusion open interval 
$(a,b)\subset [0,\ell]$ such that $\alpha(t)\ne \beta(t)$ for any $t\in (a,b)$ and set $x=\alpha(a)$ and $y=\alpha(b)$.

According to Jordan curve theorem (\ref{thm:jordan}), 
$\alpha$ and $\beta$ cut from $P$ a polyhedral disc $P'$.
Denote by $\phi$ and $\psi$ the angles of this disc at $x$ and $y$.
Note that turn of the boundary is $\pi-\phi$ at $x$,
$\pi-\psi$ at $y$ and at the remaining boundary points it is nonpositive. 

According to \ref{thm:poly-disc+gauss},\[\mathrm{T}+\Omega=2\cdot\pi.\]
where $\mathrm{T}$ and $\Omega$ 
are correspondingly 
the turn of boundary 
and the curvature of interior point of $P'$.

From above we have that $\mathrm{T}\le 2\cdot\pi -\phi-\psi$
and $\Omega\le 0$.
Hence $\phi=\psi=0$;
in particular, the geodesics $\alpha$ and $\beta$ coincide at the beginning edge,
a contradiction.

It remains to prove the second part of lemma.
Let $x_n\to x_\infty$ and $y_n\to y_\infty$ be two sequences of points in $P$.
Let $\ell_n=|x_n-y_n|$,
denote by $\alpha_n\:[0,\ell_n]\to P$ 
the geodesic from $x_n$ to $y_n$.
Note that any converging subsequence of $\alpha_n$ converges to the unique geodesic from $x$ to $y$.
On the other hand, since $P$ is compact,
any sequence of geodesics has a converging subsequence (see ???).
Hence the secont part of lemma follows. 
\qeds


\begin{thm}{Proposition}
In a polyhedral disc with non-positive curvature
any triangle is thin.
\end{thm}


\section{Shefel's theorem}

\begin{thm}{Shefel's theorem}\label{thm:shefel}
Let $P\subset \RR^2$ be a convex polygon
and $f\:P\to \RR$ a Lipschitz function.
Assume that the graph 
\[\Gamma_f=\set{(x,y,z)\in\RR^3}{z=f(x,y),\ \ (x,y)\in P}\]
equipped with length metric has nonpositive curvature.
\end{thm}

\parit{Proof.}
Fix small $\eps>0$.
Note that there is a piecewise linear function $h\:P\to \RR$ such that 
$$h(x,y)\lg f(x,y)\pm\eps$$
for any $(x,y)\in P$.
In other words, the graph $\Gamma_h$ of $h$ lies between 
two graphs $\Gamma_{f+\eps}$
 and  $\Gamma_{f-\eps}$.

Now we start to cut hats from below of $\Gamma_h$.
Cutting a hat is the following procedure...

Note that since $\Gamma_{f}$ and therefore $\Gamma_{f+\eps}$ has no hats.
Hence by applying cutting a hat to any graph which lies below $\Gamma_{f+\eps}$, we get a graph which is still below $\Gamma_{f+\eps}$.
In particular, after cutting all hats,
we get a graph of a function which lies below $\Gamma_{f+\eps}$.

\begin{clm}{}\label{clm:shefel-poly}
After cutting all hats from $\Gamma_h$,
we get a graph of a piecewise linear function.
\end{clm}



From \ref{clm:shefel-poly} it follows that for any $n$
there is a saddle piecewise linear function 
$h_n\:P\to\RR$ such that $h(x)\lg f(x)\pm\tfrac1n$
for any $x\in P$.

Denote by $|x-y|_n$ the ???
and $|x-y|_\infty$ the ???.
It is sufficient to show that
\[|x-y|_n\to|x-y|_\infty\ \text{as}\ n\to\infty\]
for any $x,y\in P$,


\qeds


