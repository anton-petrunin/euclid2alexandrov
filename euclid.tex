

\section{Geodesics, triangles and hinges}

Let $X$ be a metric space.

\parbf{Geodesics.}
Given a pair of points $x,y\in X$,
we will denote by $[xy]$ the image of a geodesic from $x$ to $y$. 

In general, a geodesic between $x$ and $y$ need not exist and if it exists, it need not be unique.  However,  once we write $[x y]$ we mean that we made a choice of a geodesic between $x$ and $y$.

Also we will use the following notational short-cuts:
\begin{align*}
\left] x y \right[&=[xy]\backslash\{x,y\},
&
\left] x y \right]&=[xy]\backslash\{x\},
&
\left[ x y \right[&=[xy]\backslash\{y\}.
\end{align*}

\parbf{Triangles.}
For a triple of points $x,y,z\in X$, 
a choice of a triple of geodesics $([x y], [y z], [z x])$ will be called a \emph{triangle}\index{triangle} and we will use the short notation 
$\trig x y z \z=([x y], [y z], [z x])$.

Given a triple $x,y,z\in \spc{X}$ there may be no triangle 
$\trig x y z$ simply because one of the pairs of these points cannot be joined by a geodesic, and also there may be many different triangles with these vertices, any of which can be denoted by $\trig x y z$.
Once we write $\trig x y z$, it means that we made a choice of such a triangle, 
i.e. a choice of each $[x y], [y z]$ and $[z x]$.

\section{Model triangles and angles}

\parbf{Model triangles.}
Given 
$x,y,z\in \spc{X}$. 
Let us define its \emph{model triangle}\index{model triangle} $\trig{\tilde x}{\tilde y}{\tilde z}$ 
(briefly, 
$\trig{\tilde x}{\tilde y}{\tilde z}=\modtrig0(x y z)$%
\index{$\modtrig0$!$\modtrig0({*}{*}{*})$}) to be a triangle in the Euclidean plane  such that
\[\dist{\tilde x}{\tilde y}{\RR^2}=\dist{x}{y}{X},
\ \ \dist{\tilde y}{\tilde z}{\RR^2}=\dist{y}{z}{X},
\ \ \dist{\tilde z}{\tilde x}{\RR^2}=\dist{z}{x}{X}.\]
Note that a model triangle exists and is unique up to congruence.

In this case, a point $\tilde p\in [{\tilde x}{\tilde y}]$ is said to be \emph{corresponding} to the point $p\in [xy]$
if $\tilde p$ divides $[\tilde x\tilde y]$ in the same ratio as $p$ divides $[xy]$.
(Equivalently, $\dist{\tilde x}{\tilde p}{\RR^2}\z=\dist{x}{p}{X}$ or 
 $\dist{\tilde y}{\tilde p}{\RR^2}\z=\dist{y}{p}{X}$.)



\parbf{Model angles.}
Given $x,y,z \in X$ such that
$$\dist{x}{y}{},\dist{x}{z}{}>0,$$ the angle measure of the model triangle
$\trig{\tilde x}{\tilde y}{\tilde z}=\modtrig0(x y z)$ at $\tilde  x$ will be called the \emph{model angle}\index{model angle} of the triple $x$, $y$, $z$ and it will be denoted by
$\angk0 x y z$.

\parbf{Fat and thin triangles.}
Let $\trig{x_1}{x_2}{x_3}$ be a triangle in a metric space
and 
$\trig{\tilde x_1}{\tilde x_2}{\tilde x_3}=\modtrig0({x_1}{x_2}{x_3})$ be its model triangle.
For any point $\tilde z$ $\trig{\tilde x_1}{\tilde x_2}{\tilde x_3}$,
one can consider the corresponding point $z$ on the sides of the original $\trig{x_1}{x_2}{x_3}$;
i.e., if $\tilde z\in [\tilde x_i\tilde x_j]$ then $z\in[x_ix_j]$ and it divides $[x_ix_j]$ in the same ratio as $\tilde z$ divides $[\tilde x_i\tilde x_j]$.
This construction defines so called \emph{natural map}\index{natural map} $\tilde z\mapsto z$ from the model triangle
$\trig{\tilde x_1}{\tilde x_2}{\tilde x_3}=[{\tilde x_1}{\tilde x_2}]\cup[{\tilde x_2}{\tilde x_3}]\cup[{\tilde x_3}{\tilde x_1}]$ 
to the original triangle $\trig{x_1}{x_2}{x_3}= [{x_1}{x_2}]\cup[{x_2}{x_3}]\cup[{x_3}{x_1}]$.

\begin{thm}{Definition}\label{def:fat-thin}
We say the triangle $\trig{x_1}{x_2}{x_3}$ in a metric space is 
\emph{fat}\index{fat} (respectively \emph{thin}\index{thin}) 
if the natural map $\trig{\tilde x_1}{\tilde x_2}{\tilde x_3}\to \trig{x_1}{x_2}{x_3}$ is distance non-contracting (respectively distance non-expanding).
\end{thm}
For example, in Euclidean space any triangle is both fat and thin.
In fact, it is true that any proper length space with this property is isometric to a convex subset of a Hilbert space.

The following is a statement in plane geometry,
although its formulation uses fancier terms.

%???+PIC

\begin{thm}{Alexandrov's lemma}
\index{Alexandrov's lemma}
\index{Alexandrov's lemma}
\label{lem:alex}  
Let $x,y,z,p$ be distinct points in a metric space such that $p\in \left]x z\right[$.
Then the following expressions have the same sign:
\begin{subthm}{lem-alex-difference}
$\angk0 x y p
-\angk0 x y z$,
\end{subthm} 

\begin{subthm}{lem-alex-angle}
$\pi - \angk0 p y x
-\angk0 p y z$.
\end{subthm}

\end{thm}

\parit{Proof.} 
Consider the model triangle $\trig{\tilde x}{\tilde y}{\tilde p}=\modtrig0 x y p$.
Take 
a point $\tilde z$ on the extension of 
$[\tilde x \tilde p]$ beyond $\tilde p$ so that $\dist{\tilde x}{\tilde z}{}=\dist{x}{z}{}$ (and therefore $\dist{\tilde p}{\tilde z}{}=\dist{p}{z}{}$). 
 
Since increasing a side in a planar triangle increases the opposite angle, 
the following expressions have the same sign:
\begin{enumerate}[(i)]
\item $\mangle\hinge{\tilde x}{\tilde y}{\tilde z}-\angk0{x}{y}{z}$;
\item $\dist{\tilde y}{\tilde z}{}-\dist{y}{z}{}$;
\item $\mangle\hinge{\tilde p}{\tilde y}{\tilde z}-\angk0{p}{y}{z}$.
\end{enumerate}
Since 
\[\mangle\hinge{\tilde x}{\tilde y}{\tilde z}=\mangle\hinge{\tilde x}{\tilde y}{\tilde p}=\angk0{x}{y}{p}\]
and
\[ \mangle\hinge{\tilde p}{\tilde y}{\tilde z}
=\pi-\mangle\hinge{\tilde p}{\tilde x}{\tilde y}
=\pi-\angk0{p}{x}{y},\]
the statement follows.
\qeds


\section{Hinges and angles}

\parbf{Hinges.}
Let $p,x,y\in \spc{X}$ be a triple of points such that $p$ is distinct from $x$ and $y$.
A pair geodesics $([p x],[p y])$ will be called a \emph{hinge}\index{hinge} and it will be denoted by 
$\hinge p x y=([p x],[p y])$\index{$\hinge{{*}}{{*}}{{*}}$}.


\parbf{Angles.}
Given a hinge $\hinge p x y$, we define its \emph{angle}\index{angle} as 
\index{$\mangle$!$\mangle\hinge{{*}}{{*}}{{*}}$}
\[\mangle\hinge p x y
\df
\lim_{\bar x,\bar y\to p} \angk0 p{\bar x}{\bar y},\]
where $\bar x\in\left ]p x\right]$ and $\bar y\in\left]p y\right]$, if the limit exists.

\begin{thm}{Exercise}
Given an example of a hinge in a proper length space for which the angle is not defined, i.e. the above limit does not exist.
\end{thm}



\begin{thm}{Triangle inequality for angles}
\label{claim:angle-3angle-inq}
Let  $[px]$, $[py]$ and $[pz]$  
be three geodesics in a metric space.
If all of the angles $\alpha=\mangle\hinge p {x}{y}$, $\beta=\mangle\hinge p {y}{z}$ and $\gamma=\mangle\hinge p {x}{z}$ are defined, then they satisfy the triangle inequality:
\[\gamma\le \alpha+\beta.\]

\end{thm}

\begin{wrapfigure}[7]{r}{30mm}
\begin{lpic}[t(-0mm),b(0mm),r(0mm),l(0mm)]{pics/s-choice(0.33)}
\lbl[rb]{45,100;$t$}
\lbl[rt]{45,31;$\tau$}
\lbl[W]{50,65;$s\ \ $}
\lbl[l]{18,60,-25;$=\beta+\eps$}
\lbl[l]{18,69,24;$=\alpha+\eps$}
\end{lpic}
\end{wrapfigure}

\parit{Proof.} 
Since $\gamma\le\pi$, we can assume that $\alpha+\beta< \pi$.
Parametrize $[px]$, $[py]$, and $[pz]$
by arc-length starting from $p$ and denote the obtained curves
 by $\sigma_x$, $\sigma_y$ and $\sigma_z$.
Given any $\eps>0$, for all sufficiently small $t,\tau,s\in\RR_+$ we have
\begin{align*}
\dist{\sigma_x(t)}{\sigma_z(\tau)}{}
\le 
&\,\dist{\sigma_x(t)}{\sigma_y(s)}{}+\dist{\sigma_y(s)}{\sigma_z(\tau)}{}\\
<
&\,\sqrt{t^2+s^2-2\cdot t\cdot  s\cdot \cos(\alpha+\eps)}+
\\
&+\sqrt{s^2+\tau^2-2\cdot s\cdot \tau\cdot \cos(\beta+\eps)}
\\
\intertext{Below we define $s(t,\tau)$ so that for $s=s(t,\tau)$, this chain continues}
\le
&\,\sqrt{t^2+\tau^2-2\cdot t\cdot \tau\cdot \cos(\alpha+\beta+2\cdot \eps)}.
\end{align*}
Thus for any $\eps>0$, 
\[\gamma\le \alpha+\beta+2\cdot \eps.\]
Hence the result.

To define $s(t,\tau)$, consider three rays $\tilde \sigma_x$, $\tilde \sigma_y$, $\tilde \sigma_z$ in the Euclidean plane starting at one point, such that $\mangle(\tilde \sigma_x,\tilde \sigma_y)=\alpha+\eps$, $\mangle(\tilde \sigma_y,\tilde \sigma_z)=\beta+\eps$ and $\mangle(\tilde \sigma_x,\tilde \sigma_z)=\alpha+\beta+2\cdot \eps$.
We parametrize each ray by length from the common end.
Given two positive numbers $t,\tau\in\RR_+$, let $s=s(t,\tau)$ be %a 
the 
number such that 
$\tilde \sigma_y(s)\in[\tilde \sigma_x(t)\ \tilde \sigma_z(\tau)]$. Clearly $s\le\max\{t,\tau\}$, % i.e. if $t$ and $\tau$ are both sufficiently small then so is $s$.
so $t,\tau,s$ may be taken sufficiently small.
\qeds


\section*{Exercises}

\begin{pr}\label{pr:1-st-var}%
\footnote{Hint: Apply the definition of angle and triangle inequality
$$|z-\bar y|\ge|\bar z-\bar y|+ |\bar z -z|$$ 
for $\bar z\in  \left]xz\right]$.}
Let $X$ a metric space with hinge $\hinge x y z$.
Assume the angle $\alpha=\mangle\hinge x y z$ is defined.
Show that
$$|z-\bar y|\le |z-x|-|x-\bar y|\cdot\cos\alpha+o(|x-\bar y|)$$
for $\bar y\in \left]xy\right]$.
\end{pr}

\begin{pr} Prove that the sum of adjacent angles is at least $\pi$.

More precisely: let $\spc{X}$ be a geodesic space and $p,x,y,z\in \spc{X}$.
If $p\in \left] x y \right[$, then 
\[\mangle\hinge pxz+\mangle\hinge pyz\ge \pi\]
whenever  each angle on the left-hand side is defined.
\end{pr}


