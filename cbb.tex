\chapter{Curvature bounded below}\label{chap:cbb}

\section{Definitions}

\begin{thm}{Definition}\label{def:cbb}
A proper length space $X$ has non-negative curvature in the sense of Alexandrov (briefly $X\in\CBB{}{0}$%
\footnote{$\CBB{}{0}$ stays for ``curvature bounded below by $0$''. 
If in the definition of model triangle, one exchanges the Euclidean plane with a sphere or Lobachevsky plane of constant curvature $k$, then one gets the definition of \emph{spaces with curvature $\ge k$ in the sense of Alexandrov}, which denoted as $\CBB{}{k}$. We will only consider the case $k=0$.}%
)
if any triangle in $X$ is fat (see Definition~\ref{def:fat-thin}).
\end{thm}

Note that Euclidean space, as well as any convex set in the Euclidean space form a $\CBB{}{0}$ space.

The following theorem gives a number of equivalent ways to define $\CBB{}{0}$ spaces.

\begin{thm}{Theorem}
\label{thm:defs_of_alex} 
Let $\spc{X}$ be a proper length space. 
Then the following are equivalent.

\begin{subthm}{3-sum} $\spc{X}\in\CBB{}{0}$.
\end{subthm}
%if and only the following conditions holds for all $p,x,y,z\in \spc{X}$:

\begin{subthm}{1+3}((1+3)-point comparison) if $p$ is distinct from $x$, $y$, and $z$ then
\[\angk0 p{x}{y}
+\angk0 p{y}{z}
+\angk0 p{z}{x}\le 2\cdot\pi.\]
\end{subthm}


\begin{subthm}{2-sum} 
(adjacent angle comparison\index{comparison!adjacent angle comparison}) 
for any geodesic $[x y]$ and $p\in \left]x y\right[$, $z\not=p$ we have
\[\angk0 pyz
+\angk0 pzx\le \pi.\]
\end{subthm}

\begin{subthm}{angle}(hinge comparison\index{hinge comparison})
for any hinge $\hinge x y z$, the angle 
$\mangle\hinge x y z$ is defined and 
\[\mangle\hinge x y z\ge\angk0 x y z.\]
Moreover, if $p\in\left]x y\right[$, $z\not=p$ then 
\[\mangle\hinge p y z + \mangle\hinge p z x\le\pi\]
for any two hinges $\hinge p z y$ and $\hinge p z x$ with common side $[pz]$.%
\footnote{It is not known if the last inequality is necessary, even for spaces homeomorphic to $\SS^2$.}
\end{subthm}
\end{thm}

\begin{wrapfigure}[5]{r}{30mm}
\begin{lpic}[t(0mm),b(-10mm),r(0mm),l(0mm)]{pics/lem_alex1(0.35)}
\lbl[br]{17,59;$x$}
\lbl[r]{1,2;$y$}
\lbl[l]{86,13;$z$}
\lbl[lb]{67,32;$p$}
\end{lpic}
\end{wrapfigure}


\parit{Proof of Theorem~\ref{thm:defs_of_alex}.}
Note that $X$ is a geodesic space%
\footnote{I.e., any two points in $X$ can be joined by a geodesic.}%
,
as it is proper and length (see Exercise~\ref{ex:proper=>geodesic}).

\parit{(\ref{SHORT.1+3}) $\Rightarrow$ (\ref{SHORT.2-sum})}. Since $p\in \left]x y\right[$, we have $\angk0 p x y=\pi$. 
Thus, (1+3)-point comparison
\[\angk0 p x y
+\angk0 p y z
+\angk0 p z x\le2\cdot\pi\]
implies
\[\angk0 p y z
+\angk0 p z x
\le\pi.\]

\parit{(\ref{SHORT.2-sum}) $\Leftrightarrow$ (\ref{SHORT.3-sum})}. 
Follows directly from Alexandrov's lemma (\ref{lem:alex}).

\parit{(\ref{SHORT.2-sum}) $+$ (\ref{SHORT.3-sum}) $\Rightarrow$ (\ref{SHORT.angle}).} 
From Proposition~\ref{prop:angle-mono}, 
we get that for $\bar y\in\left]xy\right]$ and $\bar z\in\left]xz\right]$ the function $(\dist{x}{\bar y}{},\dist{x}{\bar z}{})\mapsto\angk0 x{\bar y}{\bar z}$ is nonincreasing in each argument.
In particular, 
$\mangle\hinge x y z\z=\sup\{\angk0 x{\bar y}{\bar z}\}$.
Thus, $\mangle\hinge x y z$ is defined and it is
at least $\angk0 x y z$.

From above and (\ref{SHORT.2-sum}), it follows that $\mangle\hinge p y z + \mangle\hinge p z x\le\pi$.

\begin{wrapfigure}[10]{r}{30mm}
\begin{lpic}[t(0mm),b(0mm),r(0mm),l(0mm)]{pics/angle-comp(0.30)}
\lbl[r]{0,35;$z$}
\lbl[l]{65,41;$p$}
\lbl[tr]{48,38;$w$}
\lbl[lb]{77,96;$x$}
\lbl[lt]{85,4;$y$}
\end{lpic}
\end{wrapfigure}

\parit{(\ref{SHORT.angle}) $\Rightarrow$ (\ref{SHORT.1+3}).}
Consider a point $w\in \left] p z \right[$ close to $p$.
From (\ref{SHORT.angle}), it follows that 
\[\mangle\hinge w x z+ \mangle\hinge w x{p}\le\pi\ \ \text{and}\ \ \mangle\hinge w y z + \mangle\hinge w y{p}\le\pi.\]
By the triangle inequality for angles (see \ref{claim:angle-3angle-inq}), we have $\mangle\hinge w x y\le \mangle\hinge w x p +\mangle\hinge w y{p}$.
Therefore we get 
\[\mangle\hinge w x z+ \mangle\hinge w y z +\mangle\hinge w x y
\le
2\cdot\pi.\]
Applying the first inequality in (\ref{SHORT.angle}), we obtain
\[\angk0 w x z
+ \angk0 w y z 
+\angk0 w x y
\le
2\cdot\pi.\]
Passing to the limits  $w\to p$, we obtain 
\[\angk0 p x z 
+ \angk0 p y z 
+\angk0 p x y
\le
2\cdot\pi.\]
\qedsf

\begin{thm}{Proposition}\label{prop:GH-lim(CBB)}
Let $(X_n)$ be a sequence of compact $\CBB{}{0}$ spaces 
which converge to a compact space $X_\infty$ in the sense of Gromov--Hausdorff.

Then $X_\infty\in \CBB{}{0}$. 
\end{thm}

\parit{Proof.}
According to Exercise~\ref{ex:length=closed-in-GH},
$X_\infty$ is a length space.

For each natural number $n$, choose $f_n\:X_n\to X_\infty$ to be an $\eps_n$-isometry 
for some sequence $\eps_n\to0$ as $n\to\infty$. 
Take arbitrary points 
$p_\infty,x_\infty,y_\infty,z_\infty\in X_\infty$
and choose points $p_n,x_n,y_n,z_n\in X_n$ such that
$f_n(p_n)\to p_\infty$, 
$f_n(x_n)\to x_\infty$, 
$f_n(y_n)\to y_\infty$,
$f_n(z_n)\to z_\infty$ 
as $n\to\infty$.

Since $X_n\in\CBB{}{0}$, from \ref{1+3} we have 
\[\angk0 {p_n}{x_n}{y_n}
+\angk0 {p_n}{y_n}{z_n}
+\angk0 {p_n}{z_n}{x_n}\le 2\cdot\pi.\]

Clearly, 
$$\angk0 {p_n}{x_n}{y_n}\to \angk0 {p_\infty}{x_\infty}{y_\infty},
\ \angk0 {p_n}{y_n}{z_n}\to \angk0 {p_\infty}{y_\infty}{z_\infty},\ 
\angk0 {p_n}{z_n}{x_n}\to \angk0 {p_\infty}{z_\infty}{x_\infty}$$ 
as $n\to\infty$.
Therefore
\[\angk0 {p_\infty}{x_\infty}{y_\infty}
+\angk0 {p_\infty}{y_\infty}{z_\infty}
+\angk0 {p_\infty}{z_\infty}{x_\infty}\le 2\cdot\pi.\]
The proposition follows from \ref{1+3}.
\qeds



\section{???Surface of convex body}

The Theorem \ref{thm:poly-cbb} together with Exercise~\ref{ex:curv-is-nonneg} implies that that surface of any convex polyhdron is a $\CBB{}{0}$ space.
Further, applying Problem~\ref{pr:H>GH-boundary} and Proposition~\ref{prop:GH-lim(CBB)}, we get the following.

\begin{thm}{Proposition}
The surface of a convex body in $\RR^3$ is a $\CBB{}{0}$ space.
\end{thm}

In this section we prove the following theorem,
which is a convese of the above proposition.

\begin{thm}{Theorem}\label{thm:body}
A metric space $X$ is isometric to the surface of a convex body%
\footnote{We allow a convex body to degenerate to a planar figure but not to a segment.
As in the case of convex polyhedra, the surface of a planar figure is defined as its doubling.}
if and only if 
$X$ is a $\CBB{}{0}$-space which is homeomorphic to $\SS^2$.
\end{thm}

The following proposition gives the main 
step in the proof of the theorem above. 

\begin{thm}{Proposition}\label{prop:approx-by-poly}
Given a $\CBB{}{0}$ space $X$ homeomorphic to a sphere,
there is a non-negatively curved polyhedral space  $\tilde X$ homeomorphic to $\SS^2$
for which the Gromov--Hausdorff distance $|X-\tilde X|_{\mathcal{M}}$ arbitrarily small. 
\end{thm}

\parit{Proof with cheating.}
In the proof we will use two claims without proof;
the proofs are not hard but tedious.

\begin{clm}{}\label{clm:triangulation}
Given $\eps>0$, 
there is a triangulation $\mathcal T$ of $X$ with edges formed by geodesics and the diameter of each triangle is less than $\eps$.
\end{clm}

Fix small $\eps>0$ and choose a triangulation $\mathcal T$ of $X$ provided by the claim \ref{clm:triangulation}.

\begin{thm}{Exercise}\label{ex:sum=<2pi}
The sum of the angles at one of the vertices of $\mathcal T$ is $\le 2\cdot\pi$. 
\end{thm}

For each triangle in $\mathcal T$, construct a model triangle and glue them together the same way as the corresponding triangles in $X$.
Denote the obtained polyhedral space by $\tilde X$, and given a vertex $v_i$ of $\mathcal T$,
we will denote by $\tilde v_i$ the corresponding point in $\tilde X$.
Applying the hinge comparison (\ref{angle})
we have that the sum of the angles around a vertex $\tilde v_i$ of $\tilde X$
is at most as big as the sum of the angles  around the corresponding vertex $v_i$ in $X$.
Hence $\tilde X$ is non-negatively curved.

Now to prove the proposition,
it is sufficient to show that if $\eps>0$ is small enough 
then $\tilde X$ is sufficiently close to $X$.

First note that the set of vertices $\{v_i\}$
forms an $\eps$-net in $X$
and the set of vertices $\{\tilde v_i\}$
forms an $\eps$-net in $\tilde X$.
Therefore it is sufficient to show that the inequalities
$$|\tilde v_i-\tilde v_j|_{\tilde X}\le |v_i-v_j|_X\eqlbl{eq:=<}$$
$$|v_i-v_j|_X-4\cdot\pi\cdot\eps\le |\tilde v_i-\tilde v_j|_{\tilde X}\eqlbl{eq:>=}$$
hold for any $i$ and $j$.

To prove \ref{eq:=<}, consider a geodesic $[v_iv_j]$ in $X$.
Let $v_i=x_0,x_1,\dots,x_n=v_j$ be the points of intersection of $[v_iv_j]$ with the edges of triangulation
listed in the order from $v_i$ to $v_j$.%
\footnote{Note that according problem~\ref{pr:angle=0} and \ref{pr:angle+angle=pi}, if a geodesic $[v_iv_j]$ intersect an edge at two points then it contains this edge. In this case one can take as $x$ any point on this edge. Taking this into account, we have that $n$ is finite.}

Fix $k\in\{1,\dots,n\}$.
Let $[pqr]$ be the triangle in $\mathcal T$ which contains $[x_{k-1}x_k]$ inside.
Without loss of generality,
we can assume that $x_{k-1}\in[pq]$ and $x_k\in[pr]$.
Applying definition of $\CBB{}{0}$ spaces twice,
first for the triangle $[pqr]$ and $x_k\in[pr]$
and then for the triangle $[pqx_{k}]$ and $x_{k-1}\in[pq]$
we get that
$$|\tilde x_k-\tilde x_{k-1}|_{\tilde X}\le |x_k-x_{k-1}|_X$$
holds for each $k$.
Summing up, we get \ref{eq:=<}.

The proof of \ref{eq:>=} is similar,
but in this part we use yet the following claim without proof.

\begin{clm}{}
There is a constant $\rho=\rho(X)>0$ such that for all small $\eps>0$,
the Gromov--Hausdorff distance from $\tilde X$ 
to any real segment is at least $\rho$.   
\end{clm}

Consider a geodesic $[\tilde v_i\tilde v_j]$ in $\tilde X$.
Let $\tilde v_i=\tilde y_0,\tilde y_1,\dots,\tilde y_m=\tilde v_j$ be the points of intersection of $[\tilde v_i\tilde v_j]$ with the edges of triangulation
listed in the order from $\tilde v_i$ to $\tilde v_j$.
Assume $[\tilde v_i\tilde v_j]$ crosses  twice one edge
of $\mathcal{T}$.
Say $[\tilde v_i\tilde v_j]$ crosses edge $\ell$ at $y_k$ and $y_m$.
Since the length of  $\ell$ is smaller than $\eps$
and $[\tilde v_i\tilde v_j]$ is a geodesic, we get that $|y_k-y_m|_{\tilde X}<\eps$.
Hence the geodesic $[y_ky_m]$ 
together with the segment of $\ell$ 
from $y_k$ to $y_m$ forms a closed curve on $\tilde X$ of length $<2\cdot\eps$.

 

 
According to Claim~\ref{clm:cheat2}, $[\tilde v_i\tilde v_j]$ intersects each triangle at most once.

Fix $k\in\{1,\dots,n\}$.
Let $[\tilde p\tilde q\tilde r]$ be the triangle in $\tilde X$ which contains $[\tilde y_{k-1}\tilde y_k]$ inside.
Without loss of generality,
we can assume that $\tilde y_{k-1}\in[\tilde p\tilde q]$ and $\tilde y_k\in[\tilde p\tilde r]$.
Set
\begin{align*}
\alpha&=\mangle\hinge p{q}{r}, 
&
\beta&=\mangle\hinge {q}p{r}, 
&
\gamma&=\mangle\hinge {r}p{q},
\\ 
\tilde\alpha&=\angk{}p{q}{r}
&
\tilde\beta&=\angk{}{q}p{r}
&
\tilde\gamma&=\angk{}{r}p{q}
\end{align*}
By hinge comparison, (\ref{angle}) 
we have
\begin{align*}
\alpha&\ge \tilde\alpha, 
&
\beta&\ge\tilde\beta, 
&
\gamma&\ge\tilde\gamma
\end{align*}
For the triangle $\Delta=[pqr]$
we define its \emph{curvature} as  
$$\kappa(\Delta)=\alpha+\beta+\gamma-\pi=\alpha+\beta+\gamma-(\tilde\alpha+\tilde\beta+\tilde\gamma).$$
From the above $\kappa(\Delta)\ge 0$.
Together with the rule of cosines, straightforward estimates give the following: 
\begin{align*}
|y_{k}-y_{k-1}|_{X}
&\le \sqrt{|p-y_{k-1}|^2+|p-y_{k}|^2-2|p-y_{k-1}|\cdot|p-y_{k}|\cdot\cos\alpha}
\\
&\le |\tilde y_{k}-\tilde y_{k-1}|_{\tilde X}+\eps\cdot(\alpha-\tilde\alpha)
\\
&\le|\tilde y_{k}-\tilde y_{k-1}|_{\tilde X}+\eps\cdot\kappa(\Delta).
\end{align*}

Summing it up, we get
\begin{align*}
|v_i-v_j|_{X}
&\le \sum_{k=1}^m|y_{k}-y_{k-1}|_{X}
\\
&\le \sum_{k=1}^m|\tilde y_{k}-\tilde y_{k-1}|_{\tilde X}+\eps\cdot\sum_{\Delta}\kappa(\Delta).
\end{align*}
where the last sum is taken over all triangles $\Delta$ in $\mathcal T$.
Hence \ref{eq:>=} boils down to the inequality
$$\sum_{\Delta\ \text{in}\ \mathcal T}\kappa(\Delta)\le 4\cdot\pi.$$

\begin{thm}{Exercise}
Prove the last inequality.\footnote{Hint: Use   Exercise~\ref{ex:sum=<2pi}
 and Euler's formula 
the same way as Exercise~\ref{ex:sum=2pi}.}
\end{thm}
\qedsf

Theorem~\ref{thm:body} and Proposition~\ref{prop:poly-in-cbb} will be proved next week.






\section*{Proof of Theorem~\ref{thm:body}}

\parit{``Only if'' part.}
Assume $X$ is a surface of a convex body $B$
($B$ might degenerate to a flat figure, but not a line segment).
The same argument as in Proposition~\ref{prop:P-is-S^2}, 
shows that $X$ is homeomorphic to $\SS^2$.
The convex body $B$ can be approximated by a sequence of convex polyhedra $K_n$ 
in the sense of Hausdorff.
The proof of the last statement is the same as in Lemma~\ref{lem:between-poly}.

Denote by $P_n$ the surface of $K_n$.
According to Problem~\ref{pr:H>GH-boundary}, 
$P_n$ converge to $X$ in the sense of Gromov--Hausdorff.
Applying Propositions~\ref{prop:GH-lim(CBB)} and~\ref{prop:poly-in-cbb}, we get that $X\in \CBB{}{0}$.

\parit{``If'' part.}
Assume $X$ is a $\CBB{}{0}$ space which is homeomorphic to a sphere.
According to Proposition~\ref{prop:approx-by-poly},
there is a sequence of non-negatively curved polyhedral spaces $P_n$
which converge to $X$ in Gromov--Hausdorff sense and such that each $P_n$
is homeomorphic to $\SS^2$.

Applying Alexandrov's existence theorem (\ref{thm:alex-exist}), we obtain a sequence of convex polyhedra $K_n$
such that the surface of $K_n$ is isometric to $P_n$ for each $n$.
Note that for all $n$ the diameter of $K_n$ is bounded by the diameter of $P_n$.
Since $P_n\to X$ in the sense of Gromov--Hausdorff we have that 
$\diam P_n\to \diam X$ as $n\to\infty$ because $\diam$ is a continuous function (see Exercise~\ref{ex:d_GH-and-diam}.)
In particular, $\diam P_n\le C$ for some fixed constant $C$ and all $n$.

Without loss of generality we may assume that each $K_n$ contains the origin  $0\in\RR^3$.
Therefore $K_n$ lies in a fixed bounded region for all large $n$.
Applying Blaschke's compactness theorem (\ref{thm:compact+Hausdorff}),
we can pass to a Hausdorff-converging subsequence of $K_n$.
Denote by $B$ its limit.

According to Problem~\ref{pr:H>GH-boundary}, $X$ is isometric to the surface of $B$. 
\qeds

\section{Comments}

The Theorem \ref{thm:poly-cbb} admits the following generalization to higher dimensions.

\begin{thm}{Theorem}
Let $P$ be $m$-dimensional polyhedral space.
Then $P\in\CBB{}{0}$ if and only if each of the following conditions hold.
\begin{subthm}{}
Any simplex in $P$ is a face in an $m$-dimensional simplex.
\end{subthm}

\begin{subthm}{}
Any $(m-1)$-dimensional simplex in $P$ is a face of one or two an $m$-dimensional simplecies.
\end{subthm}

\begin{subthm}{}
The link of any simplex of dimension $\le m-2$ is connected.
\end{subthm}

\begin{subthm}{}
The angle around any simplex of dimension $m-2$ is $\le 2\cdot\pi$.
\end{subthm}



\end{thm}


Globalization theorem.

Applications in Riemannian geometry.



\section*{Exercises}

\begin{pr}\label{pr:angle=0}
Let $X\in \CBB{}{0}$ and $\hinge x y z$ be a hinge in $X$.
Assume $\mangle\hinge x y z=0$.
Show that either $[xy]\subset [xz]$ or $[xz]\subset [xy]$.
\end{pr}


\begin{pr}\label{pr:geod+3}
Let $X\in \CBB{}{0}$.
Show that given three distinct points $x$, $y$ and $p$ in $X$,
there is at most one geodesic from $x$ to $y$ which passes through $p$.
\end{pr}

\begin{pr}\label{pr:angle+angle=pi} Let $X\in \CBB{}{0}$ and $\hinge z x p$ and $\hinge z y p$ be two hinges with common side $[zp]$ in $X$.
Assume that points $p$, $x$, $y$ and $z$ are distinct and $z\in[xy]$.
Show that 
\[\mangle\hinge z p y + \mangle\hinge z p x=\pi. \]

\end{pr}

\begin{pr}\label{pr:belt}
Let $X$ be a $\CBB{}{0}$ space with is homeomorphic to $\SS^2$.
Assume that $\diam X= D$ and $X$ contains a closed simple curve of length $\eps$ which cuts $X$ into two domains of diameter at least $R$ each.
Set $\II=[0,D]$. Prove that 
$$|X-\II|_{\mathcal{M}}\le \tfrac{D}{R}\cdot\eps.$$  
\end{pr}

















